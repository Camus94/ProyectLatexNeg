\documentclass[12pt,letterpaper]{report}

\usepackage[
backend=biber,
style=apa,
sortcites,
natbib=true,
isbn=false]{biblatex}
\addbibresource{otros/library.bib}
\usepackage[dvipsnames]{xcolor}
\usepackage{fontspec}
\usepackage[spanish]{babel}
\usepackage[margin=2.54cm]{geometry}
% \usepackage{float}
\usepackage{multirow}
\usepackage{setspace}
\onehalfspacing
\usepackage[Conny]{fncychap}

\begin{document}

% \section*{Lista de abreviaturas}

% Tabla 'Abreviaturas' Orden alfabético
\begin{tabular}{llll}
    \textsc{a.} & Ergativo & \textsc{ref.} & Refetente conocido \\
    \textsc{comp.} & Completivo &  \textsc{refl.} & Reflexivo \\
    \textsc{abs.} & Absolutivo & \textsc{rel.} & Relativo \\
    \textsc{af.} & Afirmativo & \textsc{s.} & Singular \\
    \textsc{b.} & Absolutivo & \textsc{trm.} & Terminativo \\
\textsc{cloc.} & Cislocativo \\
\textsc{cl.} & Clítico \\
\textsc{cn.} & Clasificador nominal \\
\textsc{cnum.} & Clasificador numeral \\
\textsc{deit.} & Deíctico \\
\textsc{dem.} & Demostrativo \\
\textsc{det.} & Determinante \\
\textsc{dur.} & Durativo \\
\textsc{enf.} & Enfático \\
\textsc{est.} & Estativo \\
\textsc{exist.} & Existencial \\
\textsc{foc.} & Foco \\
\textsc{hab.} & Habitual \\
\textsc{incomp.} & Incompletivo \\
\textsc{inten.} & Intentivo \\
\textsc{irr.} & Irreal \\
\textsc{neg.} & Negación \\
\textsc{opt.} & Optativo \\
\textsc{pac.} & Paciente \\
\textsc{pfv.} & Perfectivo \\
\textsc{pl.} & Plural \\
\textsc{p.} & Posicional \\
\textsc{pos.} & posesivo \\
\textsc{pot.} & Potencial \\
\textsc{prednv.} & Predicado no verbal \\
\textsc{prep.} & Preposición \\
\textsc{pro.} & Pronombre \\
\textsc{progr.} & Progresivo \\

\end{tabular}


\part{Ejemplos sobre el fenómeno de la negación}
\chapter{Lenguas de México}
\section*{Acateco de la frontera sur}

\noindent El Akateko es una lengua indígena mexicana que se habla en la frontera sur del país, principalmente en el estado de Chiapas. Pertenece al subgrupo k'anjobalano junto con otras lenguas como el K'anjobal, el Chuj y el Tojolabal. \vspace{0.3cm}


{\noindent \setmainfont{Charis SIL}
% Ejemplo 1
\begin{tabular}{lllll}
(1) & \textbf{maː} & ø-aw-ʔab'=k'al & y-ʔek' & tiempo \\
& \textsc{\textbf{neg}.comp} & \textsc{abs3-erg2sg}-sentir\textsc{=dur} & \textsc{erg3}-pasar & tiempo \\
& \multicolumn{4}{l}{“No sentiste cómo pasó el tiempo” (pág. 62)}
\end{tabular} \vspace{0.25cm}

% Ejemplo 2
\noindent \begin{tabular}{llllll}
(2) & \textbf{man} & mimam-ø-ox & teʔ & naː & tiʔ \\
 & \textsc{\textbf{neg}.pred} & grande-\textsc{abs3-irr} & \textsc{cn}:madera & casa & esta \\
 & \multicolumn{5}{l}{“Esta casa no es grande” (pág. 81)}
\end{tabular} \vspace{0.25cm}

% Ejemplo 3
\noindent \begin{tabular}{lllllll}
(3) & nax & weto & \textbf{k'am} & ci-ø-toː & nax & yekal \\
& \textsc{nc}:hombre & Beto & \textsc{\textbf{neg}.incomp} & \textsc{incomp-abs3}-ir & \textsc{pro}:hombre & mañana \\
& \multicolumn{6}{l}{“Beto no se va a ir mañana” (pág. 73)}                         
\end{tabular} \vspace{0.25cm}

% Ejemplo 4
\noindent \begin{tabular}{llll}
(4) & \textbf{k'am}-ø & b'ey & c-in-toː=an \\
& \textsc{\textbf{neg}.exist-abs3} & \textsc{prep}:donde & \textsc{incomp-abs.1sg}-ir=\textsc{cl.1sg} \\
& \multicolumn{3}{l}{“No hay lugar donde me vaya” (Pág. 73)}  
\end{tabular} \vspace{0.25cm}

% Ejemplo 5
{\footnotesize
\noindent \begin{tabular}{llllllllll}
(5) & \textbf{ʔamax} & mitaʔ & ø-y-ʔaʔ & meter & s-b'a & nax & y-in & xun & ȼetal \\
& \textsc{\textbf{neg}.foc}  & acaso & \textsc{abs3-erg3}-poner & meter & \textsc{erg3-refl} & \textsc{pro}:hombre & \textsc{erg3-prep}:en & una & cosa \\
& \multicolumn{9}{l}{“Él no se puede meter en esta cosa” (pág. 75)}                                                         
\end{tabular} \vspace{0.25cm}
}

% Ejemplo 6
\noindent \begin{tabular}{llllll}
(6) & \textbf{man} & ø-w-ʔoːtax & max & čekel & š-ø-xul=an \\
& \textsc{\textbf{neg}.irr} & \textsc{abs3-erg1sg}-saber & quién & quién & \textsc{compl-abs3}-venir=\textsc{cl1s} \\
& \multicolumn{5}{l}{“No sé quién vino” (pág. 181)}                       
\end{tabular} \vspace{0.25cm}

% Ejemplo 7
\noindent \begin{tabular}{lllll}
(7) & \textbf{man} & lalan-ø-ox & s-wey & nax \\
& \textsc{\textbf{neg}.progr} & \textsc{progr-abs3-irr} & \textsc{erg3}-dormir & \textsc{pro}:hombre \\
& \multicolumn{4}{l}{“(Él) no está durmiendo” (pág. 149)}        
\end{tabular} \vspace{0.3cm}

}

Para \textcolor{MidnightBlue}{
\citet{acateco}} es una lengua de marcación ergativa-absolutiva al igual que el resto de lenguas mayas. Posee un sistema explícito de categorización léxica tanto para humanos, animales, plantas y productos de madera; incluso posee clasificadores numerales.

En cuanto a la negación, los ejemplos (1) al (7) indican que siempre aparece como una forma léxica independiente, precediendo —técnicamente en la totalidad de lo casos— al verbo y aplicable a distintas nociones como la existencia y a características pragmáticas como la focalización de algún elemento dentro de la construcción.
\section*{Chatino de Zacatepec}
\addcontentsline{toc}{section}{Chatino de Zacatepec}

\noindent El Chatino es una lengua mexicana que pertenece a la familia lingüística oto-mangue. Se clasifica dentro de la subfamilia Zapotecana y tiene diversas variedades dialectales que difieren en ciertos aspectos, como la pronunciación, el vocabulario y la gramática. El Chatino de Zacatepec se habla en la pequeña comunidad de San Marcos Zacatepec en el Estado de Oaxaca. \vspace{1cm} %%%%%%%%%% Abreviaturas %%%%%%%%%%%
\footnote{H: habitual, P: potencial, ENF: enfático}

{\setmainfont{Charis SIL}

% Ejemplo 8
\begin{tabular}{lllll}
(8) & \textbf{a3} & lakan3 & natan3 & ndolo7o3 \\
& \textsc{\textbf{neg}} & \textsc{h}.ser.\textsc{1sg} & gente & \textsc{h}.saber \\
& \multicolumn{4}{l}{“No soy un profesor” (pág. 91)}
\end{tabular} \vspace{0.5cm}

% Ejemplo 9
\begin{tabular}{lll}
(9) & \textbf{a3} & tilyon30 \\
& \textsc{\textbf{neg}} & gordo.\textsc{1sg} \\
& \multicolumn{2}{l}{“No estoy gordo” (pág. 91)} \\
\end{tabular} \vspace{0.5cm}

% Ejemplo 10
\begin{tabular}{llllll}
(10) & kwa31 & \textbf{a3} & ki7ya3 & 7a21 & kyo3 \\
& ya & \textsc{\textbf{neg}} & \textsc{p}.caer & \textsc{enf} & lluvia \\
& \multicolumn{5}{l}{“Ya, no lloverá mucho” (pág. 91)}                
\end{tabular} \vspace{1cm}

} 

\textcolor{MidnightBlue}{\citet{chatino}} describe esta lengua como variedad muy conservadora del chatino ya que conserva las sílabas penúltimas de las raíces bisilábicas, a diferencia de la mayoría de las otras variedades. Presenta un inventario tonal inusualmente grande de 8 categorías tonales y un sistema aspectual verbal muy complejo, intercambiabilidad de rasgos gramaticales y una fonología con sonidos poco comunes tipológicamente. El orden básico de palabras es VSO, pero se encuentran otros órdenes con motivaciones pragmáticas específicas.

La negación en el chatino de Zacatepec se expresa a través del elemento libre a3, que siempre precede al predicado que niega. La forma \textit{a3} puede encontrarse al inicio o en medio de la oración. Puede negar predicados verbales (8) o no verbales (9). Se aplica a eventos que han ocurrido (aspecto completivo), que ocurren habitualmente (aspecto habitual), o que ocurrirán en el futuro (aspecto potencial) (10).

No parece haber otras partículas de negación mencionadas en la gramática además de la forma \textit{a3}.
\section*{Chichimeco Jonaz}

\noindent Chichimeco y chichimeca son términos utilizados para referirse tanto a un conjunto de lenguas o variedades lingüísticas, así como para denominar a un grupo de pueblos indígenas nómadas que habitaban en el centro y norte de México. El Chichimeco Jonaz pertenece a la subfamilia otopame de la familia otomangue. Esta variante se localiza en el municipio de San Luis de la Paz, en el noreste del estado de Guanajuato. \vspace{0.25cm}

{\setmainfont{Doulos SIL}
    % Ejemplo 11
    \begin{tabular}{llll}
        (11) & PRES       & ṹβ̃ã e-pã́s-βʷ           & “Le pone los zapatos”          \\
             & PAS.REM    & ṹβ̃ã u-pã́s-βʷ           & “Le puso los zapatos”          \\
             & PAS.REC    & ṹβ̃ã ku-pã́s-βʷ          & “Le puso los zapatos”          \\
             & PAS INM    & ṹβ̃ã su-pã́s-βʷ          & “Le puso los zapatos”          \\
             & FUT        & ṹβ̃ã a-pã́s-βʷ           & “Le va a poner los zapatos”    \\
             &            &                        &                                \\
             & PRES       & ṹβ̃ã sa-pã́s-umé         & “No le pone los zapatos”       \\
             & PAS.REM    & ṹβ̃ã sa-pã́s-umé         & “No le puso los zapatos”       \\
             & PAS.REC    & ṹβ̃ã su-pã́s-umé         & “No le puso los zapatos”       \\
             & PAS.INM    & ṹβ̃ã sa-pã́s-umé         & “No le puso los zapatos”       \\
             & FUT        & siʔá̤n3 ṹβ̃ã sa- pã́s-umé & “No le va a poner los zapatos” \\
             &            &                        &                                \\
             & PRES       & ṹβ̃ã ra-pã́s-βʷ          & “Le pondría los zapatos”       \\
             & PAS.REM    & ṹβ̃ã ma-pã́s-βʷ          & “Le habría puesto los zapatos” \\
             & PAS.REC    & ṹβ̃ã ma-pã́s-βʷ          & “Le habría puesto los zapatos” \\
             & PAS.INM    & ṹβ̃ã ma-pã́s-βʷ          & “Le habría puesto los zapatos” \\
             & FUT        & ṹβ̃ã a-βã́s-βʷ           & “Le pondría los zapatos”       \\
             & (pág. 145) &                        &                                \\
    \end{tabular}
} \vspace{0.25cm}

\textcolor{MidnightBlue}{\citet{chichimeco}} nos dice que tipológicamente es una lengua aglutinante con características polisintéticas. Presenta una gran riqueza morfológica de naturaleza concatenativa y no-concatenativa. En la morfología verbal hay una compleja expresión de persona y número, a través de distintos paradigmas de morfemas que incluyen prefijos, sufijos, cambios internos (mutaciones consonánticas, cambios vocálicos), alternancias tonales y verbales.

El morfema \textit{-umé} tiene la función de marcar la negación en el verbo. Aparece junto con el prefijo \textit{sa-} en las formas negativas del modo afirmativo y su presencia parece causar la elisión del sufijo de objeto {\setmainfont{Charis SIL}\textit{-βw}}. Esta distribución \textit{—sa-} + \textit{-umé—} parece estar restringida a las formas negativas del modo afirmativo en presente y pasado (remoto, reciente, inmediato) ya que en las formas negativas de futuro se requiere el uso la partícula {\setmainfont{Charis SIL}\textit{siʔá̤n}.}

\section*{Chinanteco de Lealao}

\noindent La lengua chinanteca pertenece a la familia lingüística otomange y es hablada y es hablada por el pueblo chinanteco en el Estado de Oaxaca, en México. Es una de las lengua más habladas en la región de la Sierra Norte de Oaxaca y se estima que cuenta con unos 90 000 hablantes. Algunas de las variantes más importantes son el chinanteco de de Valle Nacional, el chinanteco de Usila, el chinanteco de Quiotepec y el chinanteco de Sam Juan Lealao. Este útimo está ubicado en la parte noreste del estado de Oaxaca en el distrito de Choapan. \vspace{1cm}

{\setmainfont{Doulos SIL}
    % Ejemplo 12
    \begin{tabular}{lll}
        (12) & ʔa\textsuperscript{L}2ʔe\textsuperscript{M}     & maʔ\textsuperscript{L}-líʔ\textsuperscript{L}i \\
             & \textsc{neg}                                    & \textsc{trm}-recordar.\textsc{1sg}              \\
             & \multicolumn{2}{l}{``No me acuerdo'' (pág. 32)}
    \end{tabular} \vspace{0.5cm}

    % Ejemplo 13
    \begin{tabular}{lllll}
        (13) & duʔ\textsuperscript{M}                                           & ʔa\textsuperscript{L}-ʔi\textsuperscript{L}-kuʔ\textsuperscript{M}i & miː\textsuperscript{L} & diáʔ\textsuperscript{L} \\
             & para                                                             & \textsc{neg}-\textsc{inten}-comer.\textsc{c}3                       & 3\textsc{refl}         & \textsc{pl}             \\
             & \multicolumn{4}{l}{``... para que no se los comiera'' (pág. 32)}
    \end{tabular} \vspace{0.5cm}

    % Ejemplo 14
    \begin{tabular}{lllllll}
        (14) & ʔa\textsuperscript{L}-ʔi̧\textsuperscript{M} & zïː\textsuperscript{M} & nï\textsuperscript{M} & zïː\textsuperscript{M}nuː\textsuperscript{M} & ba\textsuperscript{H} & nï\textsuperscript{M} \\
        & \textsc{neg-deit}:ese (animado) & perro & \textsc{pausa} & coyote & \textsc{af} & \textsc{pausa} \\
        & \multicolumn{6}{l}{``Ese no es un perro, es un coyote'' (pág. 32)}
    \end{tabular} \vspace*{0.5cm}
        
%	Ejemplo 15
\begin{tabular}{llllllll}
	(15) & ʔa\textsuperscript{L}-ha̧\textsuperscript{M}	& ʔi\textsuperscript{M} & ké\textsuperscript{L} & ʔi\textsuperscript{M} & kiú̧ʔ\textsuperscript{H}u\textsuperscript{M} & ba\textsuperscript{H} & nï\textsuperscript{M} \\
	& \textsc{neg.deit}:ese (inanimado) & \textsc{rel} & mío & \textsc{rel} & de.\textsc{2sg} & af & \textsc{af} \\
	& \multicolumn{7}{l}{``Ese no es mío, es tuyo más bien'' (pág. 33)}	
\end{tabular} \vspace{1cm}
}

Existen dos estrategias principales para expresar negación: 1) el prefijo {\setmainfont{Doulos SIL}ʔaL-}, y 2) la palabra negativa {\setmainfont{Doulos SIL}ʔaLʔeM}. La distribución de estas estrategias depende del tipo de negación y construcción: La palabra negativa funciona como predicado intransitivo negando la oración. «Se usa cuando la negación es completa o cierta» \textcolor{MidnightBlue}{\citep[pág. 32]{chinanteco}}. El prefijo «se reserva para situaciones negativas menos ciertas o no cumplidas», como en el intentivo. Con los deícticos animado e inanimado, el prefijo niega la identidad de un referente nominal. En oraciones relativas se utiliza el prefijo en lugar de la palabra negativa.
\section*{Chontal de san Pedro Huamelula, Sierra baja de Oaxaca}

\noindent El chontal de Oaxaca pertenece a la familia lingüística tequistlateca o tequistlatecana, que tiene algunos cientos de hablantes en la sierra y costas del estado de Oaxaca, en México.

Se trata de una lengua seriamente amenazada y en grave peligro de extinción. Según datos de 1990, tenía alrededor de 900 hablantes de la variedad de la sierra baja. Pero para 2011 se estima que su número se ha reducido drásticamente a tan sólo 100 personas, todos ellos adultos mayores de 70 años. \vspace{0.5cm}

{\setmainfont{Charis SIL}

% Ejemplo 16
\begin{tabular}{llll}
(16) & \textbf{maa}=ya'& tes & 'oo-daɡu-o' \\
& \textsc{\textbf{neg}}=\textsc{1sg.agt} & qué & amar-\textsc{sjntv.sg-2sg.pac} \\
& \multicolumn{3}{l}{``yo no te quiero'' (pág. 53)}
\end{tabular} \vspace{0.5cm}

% Ejemplo 17
\begin{tabular}{lllll}
(17) & \textbf{maa} & o-tayɡi & kwa & naa=sa \\
& \textsc{\textbf{neg}} & \textsc{2sg.pos}-palabra & dice & \textsc{ref=dem} \\
& \multicolumn{4}{l}{``tú no sabes hablar'' (pág. 54)}
\end{tabular} \vspace{0.5cm}

% Ejemplo 18
\begin{tabular}{llllll}
(18) & l-i-pekwe’ & naa=sa & \textbf{maa} & xolay-’uy & may-pa \\
& \textsc{det-3sg.pos}-esposo & \textsc{ref=dem} & \textsc{\textbf{neg}} & existir-\textsc{dur.sg} & ir-\textsc{pfv.sg} \\
\end{tabular} \vspace{0.5cm}

%Ejemplo 19
\begin{tabular}{lll}
(19) & \textbf{maa} & čo-lyu-pa \\
& \textsc{\textbf{neg}} & levantarse-\textsc{cloc-pfv.sg} \\
& \multicolumn{2}{l}{``No se levantó'' (pág.67)}
\end{tabular} \vspace{0.5cm}

%Ejemplo 20
\begin{tabular}{lll}
(20) & \textbf{maa} & šaxko-pa \\
& \textsc{\textbf{neg}} & encontrar-\textsc{pfv.sg} \\
& \multicolumn{2}{l}{``No la encontró'' (pág. 70)}
\end{tabular} \vspace{0.5cm}

}

Tipológicamente, el chontal de Oaxaca es una lengua sintética \textcolor{MidnightBlue}{\citep{ChontalOaxaca}}, es decir que emplea bastantes afijos y morfemas que se agregan a las palabras para expresar diversas modificaciones y significados gramaticales. Su morfología es principalmente prefijante en la clase nominal, mientras que en la flexión verbal predomina más bien los sufijos. El orden de palabras es bastante flexible.

En cuanto a la negación, utiliza la forma independiente \textit{maa} a la cual pueden adherirse clíticos (16). No parece existir alguna restricción con respecto a su posición dentro de una oración.

\section*{Chontal de Tabasco}
\addcontentsline{toc}{section}{Chontal de Tabasco}

\noindent El chontal de San Carlos Macuspana es una lengua que pertenece a la familia lingüística maya, específicamente al grupo cholano-tzeltalano. Se habla en el municipio de Macuspana, en el Estado de Tabasco, México.
%%%%%%%%%%%%%%%%%%%%%%%% Abreviaturas %%%%%%%%%%%%%%%%%%
\footnote{A: ergativo (juego A), B: absolutivo (juego B), EST: estativo, CLNUM: clasificador numeral, PI: plural inclusivo}
%%%%%%%%%%%%%%%%%%%%%%%%%%%%%%%%%%%%%%%%%%%%%%%%%%%%%%%
\vspace{0.5cm}

{\setmainfont{Charis SIL} 

% Ejemplo 21
\begin{tabular}{llllll}
(21) & \textbf{mach} & u-chäp-ka n& ni & un-p'e & kwa' \\
& \textsc{\textbf{neg}} & \textsc{a3}-cuece-\textsc{est} & \textsc{det} & uno-\textsc{clnum} & cosa \\
& \multicolumn{5}{l}{``No se cuece nada (pág. 73)''} \\
\end{tabular} \vspace{0.3cm}

% Ejemplo 22
\begin{tabular}{lllll}
(22) & si & \textbf{mach'an} & kä-yok  & patan=t'oko'\\
& si & \textsc{\textbf{neg}} & \textsc{a1}-chico & trabajo=\textsc{pi}\\
& \multicolumn{4}{l}{``Si no tenemos nuestro trabajito (pág. 73)''} \\
\end{tabular} \vspace{0.3cm}

% Ejemplo 23
\begin{tabular}{lllll}
(23) & \textbf{mame'} & x-ik-et & ti' & k'ak' \\
& \textsc{\textbf{neg}} & ir-\textsc{opt-b2} & boca & fuego\\
& \multicolumn{4}{l}{``No vayas a la orilla del fuego (pág. 74)''} \\
\end{tabular} \vspace{0.3cm}

% Ejemplo 24
\begin{tabular}{lllll}
(24) & \textbf{machme'} & x-ik-et & ti' & k'ak' \\
& \textsc{\textbf{neg}} & ir-\textsc{opt-b2} & boca & fuego\\
& \multicolumn{4}{l}{``No vayas a la orilla del fuego (pág. 74)''} \\
\end{tabular} \vspace{0.3cm}

% Ejemplo 25
\begin{tabular}{llll}
(25) & \textbf{moni'} & u-k'än -ka-ø & xan' \\
& \textsc{\textbf{neg}} & \textsc{a3}-usar-\textsc{est-b3} & gusano \\
& \multicolumn{3}{l}{``Ya no se usa gusano (pág. 42)''}\\
\end{tabular} \vspace{0.3cm}

% Ejemplo 26
\begin{tabular}{llll}
(26) & \textbf{mani'} & u-k'än -ka-ø & bojte' \\
& \textsc{\textbf{neg}} & \textsc{a3}-usar-\textsc{est-b3} & palo encestado \\
 & \multicolumn{3}{l}{``Ya no se usa árbol encestado''} \\
\end{tabular} \vspace{0.5cm}
}

La negación se expresa mediante partículas que se anteponen al verbo, como ocurre en otras lenguas mayas. \textcolor{MidnightBlue}{\citet{ChontalTabasco}} señala señala la existencia de 4 de estas partículas: \textit{mach} «no» (21), \textit{mach'an / ma'an} «no hay» (22), negación de advertencia (23 - 24) \textit{machme' / mame} «no sea que, no vaya a ser que» y negación enfática (25 - 26) \textit{moni'/mani'} «ya no, no es».
\section*{Huave de San Mateo del Mar}

\noindent El huave de San Mateo del Mar es una lengua indígena que se habla en el estado de Oaxaca, México. Es una lengua que constituye por sí misma una familia lingüística, es decir, no está relacionada genéticamente con ninguna otra lengua.

Tiene un alineamiento nominativo-acusativo en cuanto al sujeto, mientras que para el objeto muestra un alineamiento neutro excepto en la marca de plural que concuerda con el objeto. En la flexión verbal se distingue morfológicamente el aspecto (completivo, incompletivo, puntual), el tiempo (presente, futuro) y la persona.
%%%%%%%%%%%%%%% Abreviaturas %%%%%%%%%%%%%%%%%
\footnote{TV: vocal temática, SB: subordinador, CMPL: completivo, INDEF: pronombre indefinido}
%%%%%%%%%%%%%%%%%%%%%%%%%%%%%%%%%%%%%%%%%%%%%%%%% 
\vspace{0.5cm}

{\setmainfont{Charis SIL}

% Ejemplo 27
\begin{tabular}{lllll}
 (27) & s'ik & \textbf{ⁿɡo}-nambiy & a & puty\\
& yo & \textsc{\textbf{neg}}-matar & \textsc{det} & perro \\
& \multicolumn{4}{l}{``Yo no maté al perro (pág. 54)''}
\end{tabular}\vspace{0.5cm}

% Ejemplo 28
\begin{tabular}{lll}
(28) & \textbf{ⁿɡo}-membyol & s'ik \\
& \textsc{\textbf{neg}}-ayudar.2 & yo \\
& \multicolumn{2}{l}{``No me ayudas (pág. 55)''} \\
\end{tabular} \vspace{0.5cm}

% Ejemplo 29
\begin{tabular}{lllllll}
(29) & aj & ka & kuchil & ka & \textbf{ⁿɡo}-ma-a-jior & u-xinɡ \\
& \textsc{det} & \textsc{det} & cuchillo & \textsc{det} & \textsc{\textbf{neg}-sb-tv}-tener & \textsc{pos.1}-nariz \\
& \multicolumn{6}{l}{``Este cuchillo no tiene filo (pág. 124)''} \\
\end{tabular} \vspace{0.5cm}

% Ejemplo 30
\begin{tabular}{lll}
(30) & \textbf{ⁿɡo}-na-ⁿɡal & \textbf{ni}-kʷa-\textbf{hiⁿd} \\
& \textsc{\textbf{neg}}-1-comprar & \textsc{pro.\textbf{neg}}-qué-\textsc{\textbf{neg}.indef} \\
& \multicolumn{2}{l}{``No compré nada (pág. 216)''} \\
\end{tabular} \vspace{0.5cm}

% Ejemplo 31
\begin{tabular}{lll}
(31) & \textbf{ni}-kʷa-\textbf{hiⁿd} & ta-ⁿɡal-as \\
& \textsc{\textbf{neg}}-qué-\textsc{\textbf{neg}.indef} & \textsc{cmpl}-comprar-\textsc{cmpl.1} \\
& \multicolumn{2}{l}{``Nada compré (pág. 2016)''} \\
\end{tabular} \vspace{0.3cm}

}

En cuanto a la negación, utiliza el prefijo {\setmainfont{Charis SIL} \textit{ⁿɡo}-} como parte de la morfología verbal (27)(28)(29). \textcolor{MidnightBlue}{\citet{Huave}} describe una serie de pronombres negativos que pueden funcionar como negadores dentro de la cláusula siempre y cuando se encuentren en posición preverbal (31). En el caso de que estos pronombres negativos ocupen una posición postverbal, es obligatorio que el verbo lleve el prefijo de negación {\setmainfont{Charis SIL} \textit{ni-}} (30). En ambos casos, el pronombre negativo va siempre acompañado del sufijo {\setmainfont{Charis SIL} -\textit{hiⁿd}}, considerado una forma dislocada con valor negativo.
\section*{Huichol (Wixárika)}
\addcontentsline{toc}{section}{Huichol}

\noindent El wixárika, también conocido como huichol, pertenece a la rama meridional de la amplia familia lingüística uto-azteca, una de las más extensas de América en número de lenguas y hablantes. Se habla en comunidades indígenas huicholas asentadas en la Sierra Madre Occidental de México. \vspace{0.5cm}
%%%%%%%%%%%%%%%%%% Abreviaturas %%%%%%%%%%%%%%%%%%%%%%%%%%%%%%%%
\footnote{AS1: primera asercion, IPFV: imperfectivo, NARR: narrativo, SBJ: sujeto}
%%%%%%%%%%%%%%%%%%%%%%%%%%%%%%%%%%%%%%%%%%%%%%%%%%%%%%%%%%%%%%%%
{\setmainfont{Charis SIL} 

% Ejemplo 32
\begin{tabular}{llll}
(32) & 'ikɨ & ki & pi-\textbf{ka}-hekwa \\
& \textsc{dem} & casa & \textsc{as1-\textbf{neg}-ser.nueva} \\
& \multicolumn{3}{l}{``Esta casa no es nueva'' (pág. 103)}
\end{tabular} \vspace{0.5cm}

% Ejmplo 33
\begin{tabular}{lll}
(33) & ne-'iwa-ma & pi-\textbf{ka}-'ane-kai \\
& \textsc{1sg}-hermano-\textsc{pl} & de.esta.manera-\textsc{\textbf{neg}}-ser-\textsc{ipfv} \\
& \multicolumn{2}{l}{``Mis hermanos no era así'' (pág. 110)}
\end{tabular} \vspace{0.3cm}

% Ejemplo 34
\begin{tabular}{lll}
(34) & kwitɨ & \textbf{ka}-pan-ku-ke-ni \\
& rápido & \textsc{\textbf{neg}-x-x}-levantarse-\textsc{narr} \\
& \multicolumn{2}{l}{``No se para rápido'' (pág. 141)}
\end{tabular} \vspace{0.3cm}

% Ejemplo 35
\begin{tabular}{lll}
(35) & hauki & pu-\textbf{mawe} \\
& no.saber & \textsc{as1-\textbf{neg}.exist} \\
& \multicolumn{2}{l}{``No sé, no está...'' (pág. 112)}
\end{tabular} \vspace{0.3cm}

% ejemplo 36
\begin{tabular}{lllll}
(36) & kumu & ne-maine-hepaɨ & 'ukara-tsi & pu-\textbf{mawe}-kai \\
& como & \textsc{1sg.sbj}-decir-como & mujer-\textsc{pl} & \textsc{as1-\textbf{neg}.exist-ipfv} \\
& \multicolumn{4}{l}{``... como estoy diciendo, no había mujeres'' (pág. 114)}
\end{tabular} \vspace{0.3cm}

}

\textcolor{MidnightBlue}{\citep{Huichol}}

La negación es parte de la morfología verbal por medio del prefijo {\setmainfont{Charis SIL} \textit{ka}-} (32), (33) y (34). Existe también una negación existencial bajo la forma verbal supletiva {\setmainfont{Charis SIL} \textit{mawe}} (35) y (36).
\section*{Lacandon}

\noindent La lengua lacandona pertenece a la familia lingüística maya. Es una lengua en peligro de extinción que conserva rasgos arcaicos del proto-maya que se han perdido en otros idiomas mayances modernos.

 Tiene una estructura ergativa en la morfología del verbo. Hace distinciones entre posesión alienable e inalienable mediante afijos posesivos distintos para cada categoría. Asimismo, tiene un sistema prolífico para derivar neologismos a partir de raíces mayances.

Esta lengua es hablada principalmente en el área selvática de Chiapas y en partes de la selva del norte de Guatemala, específicamente en la región histórica conocida como Las Cañadas. Sus hablantes son los pueblos lacandones, grupos mayances que lograron aislarse durante siglos en la espesura de la selva y pudieron preservar su idioma y tradiciones.

Hoy en día sólo quedan unos pocos cientos de hablantes lacandones dispersos en pequeñas comunidades. La lengua ha sufrido desplazamiento por el español y está clasificada oficialmente como «en peligro de extinción». Sin embargo, existen esfuerzos tanto de académicos como de activistas lacandones para documentarla y revitalizar su uso. \vspace{0.5cm}

{\setmainfont{Charis SIL} 
% Ejemplo 37
\begin{tabular}{ll}
(37) & \textbf{maʔ}  \\
& \textsc{\textbf{neg}} \\
& ``No'' (pág. 82)
\end{tabular} \vspace{0.5cm}

% Ejemplo 38
\begin{tabular}{llll}
(38) & tin & t'an & \textbf{maʔ} \\
& \ yo & decir & \textsc{\textbf{neg}} \\
& \multicolumn{3}{l}{``Yo diría que no'' (pág. 82)}
\end{tabular} \vspace{0.5cm}

% Ejemplo 39
\begin{tabular}{lllll}
(39) & heʔ & + & \textbf{maʔ} & → hemaʔ \\
& eso & & \textsc{\textbf{neg}} & \\
& \multicolumn{4}{l}{``Eso no'' (Pág. 82)}
\end{tabular} \vspace{0.5cm}

% Ejemplo 40
\begin{tabular}{lllll}
(40) & kooč & + & \textbf{maʔ} & → maʔkoč \\
& ancho & & \textsc{\textbf{neg}} & \\
& \multicolumn{4}{l}{``Estrecho''}
\end{tabular} \vspace{0.5cm}
}

La negación funciona de manera análoga al español \textcolor{MidnightBlue}{\citep{lacandon}}, es decir, es un elemento libre con la forma {\setmainfont{Charis SIL} \textit{maʔ}} (37) (38). También puede utilizarse para formar compuestos con otros elementos (39) (40).
\section*{Mixe de Ayutla}
\addcontentsline{toc}{section}{Mixe de Ayutla}

\noindent La lengua mixe de Ayutla pertenece a la rama mixe de la familia lingüística mixe-zoque. Se habla principalmente en la región montañosa del Estado de Oaxaca, México, en una pequeña comunidad llamada San Pedro y San Pablo Ayutla. 
%%%%%%%%%%%%%%%%%%%%% Abreviaturas %%%%%%%%%%%%%%%%%%%%%
\footnote{A: sujeto de verbo transitivo, ASSOC: asociativo, CMPLZ: complementador, DEP: (aspecto) dependiente, PERF: perfecto, S: sujeto de verbo intransitivo, VBLZ: verbalizador}
%%%%%%%%%%%%%%%%%%%%%
\vspace{0.5cm}
{\setmainfont{Charis SIL} 

% Ejemplo 41
\begin{tabular}{lllll}
(41) & \textbf{ka't} & ëëts & xë'n & n-tun-t \\
& \textsc{\textbf{neg}} & \textsc{1pl} & como & \textsc{1a-hacer-pl.dep} \\
& \multicolumn{4}{l}{``No hicimos nada'' (pág. 448)}
\end{tabular} \vspace{0.3cm}

% Ejemplo 42
\begin{tabular}{llllll}
(42) & jëts & ja+tu'uk & kääjp & \textbf{ka't} & n-uk-ex-ät-n \\
& y & otro & pueblo & \textsc{\textbf{neg}} & \textsc{1a-x-ver-vblz-perf.dep} \\
& \multicolumn{5}{l}{``Y otros pueblos que no conozco'' (pág. 448)}
\end{tabular} \vspace{0.3cm}

% Ejemplo 43
\begin{tabular}{llllll}
(43) & kuu & anä'äjk & \textbf{ka't} & tëjk & t-ex-päät-y \\
& \textsc{cmplz} & joven.gente & \textsc{\textbf{neg}} & casa & \textsc{3a-ver}-encontrar-\textsc{dep} \\
& \multicolumn{5}{l}{``(Se dice) que los jovenes no valoran la casa'' (pág. 448)}
\end{tabular} \vspace{0.3cm}

% Ejemplo 44
\begin{tabular}{lll}
(44) & \textbf{ni}-pëën & y-\textbf{ka}-tän-y \\
& \textsc{\textbf{neg}}-quien & \textsc{3s-\textbf{neg}}-estarse-\textsc{dep} \\
& \multicolumn{2}{l}{``Nadie estaba'' (pág. 448)}
\end{tabular} \vspace{0.5cm}

% Ejemplo 45
\begin{tabular}{llll}
(45) & \textbf{ni}-tii & jä'äy & t-\textbf{ka}-mëët \\
& \textsc{\textbf{neg}}-qué & gente & \textsc{3a-\textbf{neg}-assoc} \\
& \multicolumn{3}{l}{``Las personas no tienen nada'' (pág. 448)}
\end{tabular} \vspace{0.5cm}

}

La negación se expresa por medio de la partícula independiente {\setmainfont{Charis SIL} \textit{ka't}}, la cual siempre precede al verbo. Su uso implica la aparición de una «marca de dependencia aspectual» \textcolor{MidnightBlue}{\citep{mixe}} en el verbo de manera obligatoria (41) (42) (43). Asimismo, existe también el prefijo {\setmainfont{Charis SIL} \textit{ni}-} que se une a palabras interrogativas (44) (45) para otorgar un sentido negativo. El uso de este prefijo desencadena una marcación obligatoria en el verbo por medio del prefijo {\setmainfont{Charis SIL} \textit{ka}-}.
\section*{Mixteco de San Andrés Yutatío}

\noindent La lengua mixteca es parte de la familia otomangue. Se hablaba principalmente en Oaxaca, Puebla y Guerrero. La variante estudiada pertenece al municipio de Tezoatlán, San Andrés Yutatío, Oaxaca.
%%%%%%%%%%%%%%%%%%%% Abreviaturas %%%%%%%%%%%%%%%%%%%%
\footnote{NO:REAL: no realizado, AF: afirmación}
%%%%%%%%%%%%%%%%%%%
\vspace{0.2cm}

{\setmainfont{Charis SIL} 

% Ejemplo 46
\noindent \begin{tabular}{lllllllllll}
(46) & kátoó & nda̱'o & de̱'e di'íi & ̱ndiko & xi & tído & \textbf{ko̱} & tí'a & \textbf{ta'on} & xi \\
& le:gusta & mucho & [hija:de-mí] & molerá & ella & pero & \textsc{\textbf{neg}} & sabe & \textsc{\textbf{neg}} & ella \\
& \multicolumn{10}{l}{``A mi hija le gusta mucho moler, pero no sabe'' (pág. 139)}
\end{tabular} \vspace{0.2cm}

% ejemplo 47
\noindent \begin{tabular}{lllllllllll}
(47) & \textbf{ko̱} & kóni & \textbf{ta'on} & xi & kandía & xi & ña̱ & [ni ka'in & ̱xí'ín & xí] \\
& \textsc{\textbf{neg}} & quiere & \textsc{\textbf{neg}} & él & aceptará & él & lo:que & dije-yo & a & él \\
& \multicolumn{10}{l}{``No quiere aceptar lo que le dije'' (pág.169)}
\end{tabular} \vspace{0.2cm}

% Ejemplo 48
\noindent \begin{tabular}{lllll}
(48) & \textbf{k\underline{o}} & díkó & ná & yá'\underline{a} \\
& \textsc{\textbf{neg}} & venden & ellos & chile \\
& \multicolumn{4}{l}{``Ellos no venden chile'' (pág. 187)}
\end{tabular} \vspace{0.2cm}

% Ejemplo 49
\noindent \begin{tabular}{llllll}
(49) &  \textbf{k\underline{o}} & díkó & \textbf{ta'on} & ná & yá'\underline{a} \\
& \textsc{\textbf{neg}} & venden & \textsc{\textbf{neg}} & ellos & chile \\
& \multicolumn{5}{l}{``Ellos no venden chile'', (pág. 187)}
\end{tabular} \vspace{0.2cm}

% Ejemplo 50
{\small
\noindent \begin{tabular}{llllllllllll}
 (50) & \textbf{\underline{o} d\underline{u}ú} & ta & Juan & ní & sa'\underline{a}n & yúk\underline{u} & koni & ta & Beto & va & n\underline{i} sa'\underline{a}n \\
 & \textsc{\textbf{neg}} & él (joven) & Juan & \textsc{no:real} & fue & monte & ayer & él (joven) & Beto & \textsc{af} & fue \\
 & \multicolumn{11}{l}{``No fue Juan quien fue al monte ayer, sino Beto'' (pág. 176)}
\end{tabular} \vspace{0.2cm}
}

% Ejemplo 51
\noindent \begin{tabular}{lllllll}
(51) & \textbf{\underline{o} d\underline{u}ú} & ña & nda\underline{a} & kí\underline{a}n & ka'\underline{a}n & n\underline{a} \\
& \textsc{\textbf{neg}} & ella (cosa) & verdadera & es-ella (cosa) & habla & él \\
& \multicolumn{6}{l}{``No es verdad lo que dice él'' (pág. 176)}
\end{tabular} \vspace{0.2cm}

% Ejemplo 52
\noindent \begin{tabular}{llllll}
(52) & iin & ích\underline{i} & \textbf{k\underline{o}} & vá'a & kíán \\
& un & camino & \textsc{\textbf{neg}} & bueno & es-él \\
& \multicolumn{5}{l}{``Es un camino malo (lit. no bueno)'' (pág.181)}
\end{tabular} \vspace{0.25cm}

}

La negación en este lengua toma diferentes formas: puede ser por medio de adverbios negativos {\setmainfont{Charis SIL} \textit{k\underline{o}} y \textit{ta'on}}, los cuales suelen aparecer de manera conjunta rodeando al verbo (46) (47). De acuerdo con \textcolor{MidnightBlue}{\citet{Mixteco}}, el adverbio principal y obligatorio para la negación es {\setmainfont{Charis SIL} \textit{k\underline{o}}}, mientras que {\setmainfont{Charis SIL} \textit{ta'on}} parece dar énfasis a la negación (48) (49). Contrucción de frases sustantivas negativas por medio de la expresión {\setmainfont{Charis SIL} \underline{o} \textit{d\underline{u}ú}} que se presenta siempre al inicio de la oración (50) (51). Una frase adjetiva negativa por medio del adverbio {\setmainfont{Charis SIL} \textit{k\underline{o}}} precediendo al adjetivo que modifica (52).
\section*{Oluteco}

\noindent El oluteco es una lengua indígena que pertenece a la familia mixezoque y se habla en la comunidad de Oluta, Veracruz, México. Representa una variedad conservadora del proto-mixezoque por lo que su estudio permite reconstruir la proto-lengua.

Es una lengua ergativa con marcación en el núcleo. Distingue tres personas gramaticales en singular y cuatro en plural. Los pronombres personales son proclíticos que anteceden al verbo. Tiene un sistema muy complejo de aspecto con distinciones entre completivo, incompletivo e irrealis. El aspecto interactúa con el sistema de inverso propio de lenguas ergativas.
%%%%%%%%%%%%%%%%%%%%%%%%%%% Abreviaturas %%%%%%%%%%%%%%%%%%%%%%%
\footnote{A: marcador de persona del juego \textsc{a}, B: marcador de persona del juego \textsc{b}, AN: clítico para animados, APL3: aplicativo asociativo y comitativo, EV: evidencial, C: marcador de persona del juego c, CAUS: causativo,  INCD incompletivo de dependientes, INV: inverso, INCI.I: incompletivo de independientes intransitivo, INCI.T: incompletivo de independientes transitivo, NMZR: nominalizador, RLTVZR: relativizador}
%%%%%%%%%%%%%%%%%%%%%%%%%%%%%%%%%%%%%%%%%%%%%%%%%%%%%%%%%%%%%%%%
\vspace{0.2cm}

{\setmainfont{Charis SIL}

{\footnotesize
% Ejemplo 53
\noindent \begin{tabular}{lllllll}
(53) & naʔkxej=xü=k & tuk & ta=\textbf{kaː}=moːyʔ-e & jamaj=k & muːxi-nak & kay-e \\
& cuando=\textsc{ev=an} & uno & \textsc{c3\textbf{neg}}=dar=\textsc{incd} & aquel=\textsc{an} & pájaro-\textsc{dim} & comer-\textsc{nmzr} \\
& \multicolumn{6}{l}{``Cuando uno no le da comida a aquel pajarito'' (pág. 351)}
\end{tabular} \vspace{0.2cm}
}

% Ejemplo 54
\noindent \begin{tabular}{lllll}
(54) & \textbf{kaː} =seme & ʔit-ü-pa=k & meːnyu & ʔi=tükaw \\
& \textsc{\textbf{neg}}=mucho & existir=\textsc{inv-inci.i=an} & dinero & \textsc{a3(posd)}=papá \\
& \multicolumn{4}{l}{``No tiene mucho dinero su papá de él'' (pág. 354)}
\end{tabular} \vspace{0.2cm}

% Ejemplo 55
\noindent \begin{tabular}{llll}
(55) & jaʔmej & mi=\textbf{kaː}=müː-nükx-ü-pa & jaːmu \\
& así & \textsc{b2=\textbf{neg}=apl3}-ir-\textsc{inv-inci.i} & viento \\
& \multicolumn{3}{l}{``De esa manera no te lleva el viento'' (pág. 361)}
\end{tabular} \vspace{0.2cm}

% Ejemplo 56
\noindent \begin{tabular}{lll}
(56) & ʔi=\textbf{kaː}=yak-pot-pe=k & tüpx-i \\
& \textsc{a3=\textbf{neg}=caus}-reventar-\textsc{inci.t=an} & torcer-\textsc{nmzr}\\
& \multicolumn{2}{l}{``No revienta la reata'' (pág. 362)}
\end{tabular} \vspace{0.2cm}

% Ejemplo 57
\noindent \begin{tabular}{lll}
(57) & ta & tax=\textbf{kaː}=tzum-pa \\
& \textsc{cond} & \textsc{c1(local)=\textbf{neg}}=amarrar-inci.i \\
& \multicolumn{2}{l}{``Si no te amarro'' (pág. 365)}
\end{tabular} \vspace{0.2cm}

% Ejemplo 58
{\small
\noindent \begin{tabular}{lllllll}
(58) & pero & jamaj & jaykak & ʔi=\textbf{kaː}=ʔix+kap-pe & jaʔ & pün-ʔa=jeʔ \\
& pero & aquel & gente & \textsc{a3=\textbf{neg}=}conocer-\textsc{inci.t} & él & quien-rltvzr=ese \\
& \multicolumn{6}{l}{``Pero aquella gente no sabía quien era ese'' (pág. 377)}
\end{tabular} \vspace{0.25cm}
}

}

El verbo en oluteco es una palabra polimorfémica \textcolor{MidnightBlue}{\citep{Oluteco}} y contiene la información de los argumentos centrales de la oración. La negación es por medio de un proclítico de la forma {\setmainfont{Charis SIL} \textit{kaː}} y ocupa la penúltima posición más alejada de la raíz verbal en presencia de otros clíticos y afijos.
\section*{Otomí de San Ildefonso Tultepec}

\noindent El otomí es un lengua de la familia Otomangue de la rama otomangue occidental. El otomí de San Ildefonso Tultepec es la variante que se utiliza en la localidad rural del mismo nombre en el sur del Estado de Querétaro, México.

Es una lengua con morfología tanto acumulativa (afijos y clíticos) como no acumulativa (mutaciones consonánticas y cambios tonales). Los procesos de derivación y flexión se realizan sobre todo mediante afijos y clíticos. Hay varianza morfológica condicionada fonológica, gramatical y léxicamente. \vspace{0.5cm}

{\setmainfont{Charis SIL} 

% Ejemplo 59
\begin{tabular}{llll}
(59) & ko & \textbf{hinɡi} & ø=hand-i \\
& porque & \textsc{\textbf{neg}} & \textsc{3.pres=ver-fl} \\
& \multicolumn{3}{l}{``Porque no ve'' (pag. 59)}
\end{tabular} \vspace{0.5cm}

% Ejemplo 60
\begin{tabular}{llll}
(60) & ya & \textbf{hinɡi} & ø=hand-ø-i \\
& \textsc{adv} & \textsc{\textbf{neg}} & \textsc{3.pres}=ver-\textsc{3obj-fl} \\
& \multicolumn{3}{l}{``Ya no la ve'' (pág. 228)}
\end{tabular} \vspace{0.5cm}

% Ejemplo 61
\begin{tabular}{lllllll}
(61) & pero & nu=yu̠ & ya & \textbf{hinɡi} & ø=ne-ø=r & tsibi \\
& pero & \textsc{def=dem.pl} & \textsc{adv} & \textsc{\textbf{neg}} & \textsc{3.pres}=querer-\textsc{3obj=sg} & lumbre \\
& \multicolumn{6}{l}{``Pero esas no quieren lumbre'' (pág. 113)}
\end{tabular} \vspace{0.5cm}

% Ejemplo 62
\begin{tabular}{lll}
(62) & \textbf{him}=bi & <d>in-ø-i \\
& \textsc{\textbf{neg}=3.psd} & <\textsc{tnp}>encontrar-\textsc{3obj-fm} \\
& \multicolumn{2}{l}{``No la encontraron'' (pág. 55)}
\end{tabular} \vspace{0.5cm}

% Ejemplo 63
\begin{tabular}{llll}
(63) & \textbf{hi}=mí & 'ba̠-i & ya'pu̠ \\
& \textsc{\textbf{neg}=3.imp} & pararse-\textsc{fm} & lejos \\
& \multicolumn{3}{l}{``No estaba parado lejos'' (pág. 59)}
\end{tabular} \vspace{0.5cm}

% Ejemplo 64
\begin{tabular}{llll}
(64) & pwes & \textbf{hin}=da & ñünɡ-a=n'a \\
& pues & \textsc{\textbf{neg}=3irr} & comer.\textsc{ajmf-depd}=uno\\
& \multicolumn{3}{l}{``Uno no come'' (pág. 107)}
\end{tabular} \vspace{0.5cm}

}

Esta lengua utiliza como marcador de polaridad negativa la palabra {\setmainfont{Charis SIL} \textit{hinɡi}} (59) - (61). Este marcador se ajusta morfofonológicamente cuando funciona como anfitrión de un clitico \textcolor{MidnightBlue}{\citep{Otomi}}. Esta forma adoptada es {\setmainfont{Charis SIL} \textit{hi}} más una consonante nasal que toma el punto de articulación de la consonante inmediata al clítico (62) - (64).
\section*{Sierra Popoluca}
\addcontentsline{toc}{section}{Sierra Popoluca}

\noindent El popoluca es una lengua mixe-zoqueana hablada en el sur del estado de Veracruz, México, por aproximadamente 28,000 personas. Pertenece a la rama mixeana de la familia mixe-zoqueana.
%%%%%%%%%%%%% Abreviaturas %%%%%%%%%%%%%%%%%%%%%
\footnote{ABS: absolutivo, INC: incompletivo, LOC\textsubscript{\emph{applic}}: instrumento, IPSR: poseedor inclusivo de primera persona, OPT: optativo, PLU\textsubscript{\emph{nonsap}}: plural del participante fuera del acto de habla, PLu\textsubscript{\emph{sap}}: plural del participante del acto de habla}
%%%%%%%%%%%%%%%%%%%%%%%%%%%%%%%%%%%%%%%%%%%%%%%% 
\vspace{0.5cm}

{\setmainfont{Charis SIL} 

% Ejemplo 65
\begin{tabular}{llll}
(65) & \textbf{ʔotʔoy} & ø+mɨɨch-kaʔ-taʔm-ɨʔ & woonyi \\
& \textsc{\textbf{neg}} & \textsc{3abs}+jugar-\textsc{loc\textsubscript{\emph{applic}}-pl-imp} & niña \\
& \multicolumn{3}{l}{``No juegues (juegos) con las niñas'' (pág.478)}
\end{tabular} \vspace{0.5cm}

% Ejemplo 66
\begin{tabular}{lll}
(66) & \textbf{ʔotʔoy} & ʔaʔm=seet-taʔm-ɨ \\
& \textsc{\textbf{neg}} & mirar=regreso-\textsc{plu\textsubscript{\emph{sap}}-imp} \\
& \multicolumn{2}{l}{``No mires atrás''}
\end{tabular} \vspace{0.5cm}

% Ejemplo 67
\begin{tabular}{llll}
(67) & tan+manɨk & \textbf{ʔotʔoy} & ø+mɨɨch-yaj-ʔiny \\
& 1.\textsc{ipsr+}niño & \textsc{\textbf{neg}} & \textsc{3abs+}jugar-\textsc{plu\textsubscript{nonsap}-opt} \\
& \multicolumn{3}{l}{``Nuestros hijos no deberían jugar (allí)'' (pág. 478)}
\end{tabular} \vspace{0.5cm}

% Ejemplo 68
\begin{tabular}{lll}
(68) & \textbf{dya} & ʔa+ʔɨks.i=juʔy-pa \\
& \textsc{\textbf{neg}} & \textsc{abs}+grano.de.maíz=comprar-\textsc{inc} \\
& \multicolumn{2}{l}{``No compramos maíz'' (pág. 608)}
\end{tabular} \vspace{0.5cm}

% Ejemplo 69
\begin{tabular}{lll}
(69) &  \textbf{dya} & ø+kamam \\
& \textsc{\textbf{neg}} & \textsc{3abs+}duro \\
& \multicolumn{2}{l}{``No está duro'' (pág. 609)}
\end{tabular} \vspace{0.5cm}

% Ejemplo 70
\begin{tabular}{lllll}
(70) & \textbf{dya} & ʔi+kuʔt-pa & jeʔm & kaʔnpu \\
& \textsc{\textbf{neg}} & \textsc{3erg}+comer-\textsc{inc} & \textsc{dem} & huevo \\
& \multicolumn{4}{l}{``Ella no se comió el huevo'' (pág. 610)}
\end{tabular} \vspace{0.5cm}

}

La negación en esta lengua es por medio de dos partículas \textcolor{MidnightBlue}{\citep{Popoluca}}. Para el caso de imperativos (65) - (66) y el modo optativo (67) se utilza la forma independiente {\setmainfont{Charis SIL} \textit{ʔotʔoy}}. Para el resto de contrucciones se utiliza la partícula {\setmainfont{Charis SIL} \textit{dya}} (68) - (70).
\section*{Seri}

\noindent La lengua seri es una lengua indígena que se habla en el noroeste de México. Es una lengua aislada que constituye su propia familia lingüística. Posee una enorme cantidad de alomorfía dentro de la morfología verbal. Es decir, aún después de aplicar las reglas fonológicas existentes, muchas veces quedan dos, tres, cuatro o más alomorfos supletivos para ciertos morfemas, condicionados por una variedad de factores. Es especialmente frecuente encontrar alomorfía condicionada por la transitividad superficial de la cláusula.

En cuanto a los sustantivos, estos no están marcados para el caso gramatical. Hay numerosos artículos definidos en Seri que históricamente se derivan de verbos y que indican la posición o dirección de movimiento del ítem nominal.

En Seri existe un sistema de referencia cruzada o cambio de sujeto entre cláusulas, el cual indica un cambio de sujeto entre ciertos tipos de cláusulas. \vspace{0.5cm}

{\setmainfont{Charis SIL} 

% Ejemplo 71
\begin{tabular}{ll}
(71) & t-\textbf{m}-afp \\
& \textsc{rl}-\textsc{\textbf{neg}}-llegar \\
& ``¿No llegó él?'' (pág. 21)
\end{tabular} \vspace{0.5cm}

% Ejemplo 72
\begin{tabular}{lll}
(73) & ik-oː-ʔit & iʔ-t-\textbf{m}-amšo-ʔo \\
& \textsc{inf-d}-comer & \textsc{1sg.suj-rl-\textbf{neg}-}querer-\textsc{x} \\
& \multicolumn{2}{l}{``NO quiero comer'' (pág. 21)}
\end{tabular} \vspace{0.5cm}

% Ejemplo 73
\begin{tabular}{ll}
(73) & im-χó-\textbf{m}-aː \\
& hacia-\textsc{enf-\textbf{neg}}-moverse \\
& ``Él no viene'' (pág. 22)
\end{tabular} \vspace{0.5cm}

% Ejemplo 74
\begin{tabular}{lllll}
(74) & piest & ʔant & s-\textbf{m}-iːx & k-e-ya  \\
& fiesta & tierra & \textsc{irr-\textbf{neg}}-ubicar & \textsc{nmzr}-decir/\textsc{d-inter} \\
& \multicolumn{4}{l}{``¿No habrá una fiesta?'' (pág. 24)}
\end{tabular} \vspace{0.5cm}

% Ejemplo 75
\begin{tabular}{lll}
(75) & ʔim-íp-aːɬ & i-\textbf{m}-ataχ-iʔa \\
& \textsc{1sg.obj-irr}-acompañar & \textsc{nmzr-\textbf{neg}-}ir-\textsc{decl}\\
& \multicolumn{2}{l}{``Él no fue conmigo'' (pág. 25)}
\end{tabular} \vspace{0.5cm}

}

El seri es una lengua de estructura Sujeto-Objeto-Verbo (\textsc{svo}) \textcolor{MidnightBlue}{\citep{Seri}}. Utiliza una serie de prefijos dentro de la frase verbal para expresar diversas nociones como persona y número del sujeto u objeto. Entre estos prefijos tenemos tenemos el prefijo negativo {\setmainfont{Charis SIL} \textit{m}-}. Al parecer es el prefijo más cercano a la raiz verbal (71) - (75).
\section*{Tarahumara (Rarámuri)}
\addcontentsline{toc}{section}{Tarahumara}

\noindent El choguita rarámuri es una variedad del rarámuri que es parte de la familia de lenguas yuto-aztecas.
%%%%%%%%%%%%% Abreviaturas %%%%%%%%%%%%%%%%%%%%%
\footnote{CL: partícula de final de cláusula, COP: Cópula, DUB: dubitativo, IMPF: imperfectivo, PASS: pasivo, PROH: prohibitivo, VBLZ: verbalizador}
%%%%%%%%%%%%%%%%%%%%%%%%%%%%%%%%%%%%%%%%%%%%%%%%
\vspace{0.5cm}

{\setmainfont{Charis SIL} 

% Ejemplo 76
\begin{tabular}{llllll}
(76) & \textbf{ke}, & \textbf{ke} & me & o'báta & a'lé \\
& \textsc{\textbf{neg}} & \textsc{\textbf{neg}} & casi & feroz.\textsc{pl} & \textsc{dub}\\
& \multicolumn{5}{l}{``No, casi no son bravos'' (pág. 515)}
\end{tabular} \vspace{0.5cm}

% Ejemplo 77
\begin{tabular}{llllllll}
(77) & \textbf{'ka't͡ʃè} & kai'nâ-ma & a'lé & \textbf{ke} & naʔ'pô-suwa & 'ká & ba \\
& \textsc{\textbf{neg}} & cosecha-\textsc{fut.sg} & \textsc{dub} & \textsc{\textbf{neg}} & maleza-\textsc{cond.pass} & \textsc{cop.irr} & \textsc{cl} \\
& \multicolumn{7}{l}{``No habrá ninguna cosecha si no se realiza el deshierbe'' (pág.519)}
\end{tabular} \vspace{0.5cm}

% Ejemplo 78
\begin{tabular}{lllll}
(78) & \textbf{'ka't͡ʃè}=ko & waʔlu-'bê & buʔu-'rú-i & t͡ʃa'bèi=ko  \\
& \textsc{\textbf{neg}}.porque=\textsc{emph} & grande-más & camino-\textsc{vblz-impf} & antes=\textsc{emph} \\
& \multicolumn{4}{l}{``Porque antes no había camino grande'' (pág. 520)}
\end{tabular} \vspace{0.5cm}

% Ejemplo 79
\begin{tabular}{llllll}
(79) & \textbf{ke'tâsi} & pe & \textbf{ke} & 't͡ʃó & na'wà-li \\
& \textsc{\textbf{neg}} & apenas & \textsc{\textbf{neg}} & ya & llegar-\textsc{psd} \\
& \multicolumn{5}{l}{``No, todavía no llega'' (pág. 516)}
\end{tabular} \vspace{0.5cm}

% Ejemplo 80
\begin{tabular}{lll}
(80) & \textbf{'kíti} & na'là-ka \\
& \textsc{\textbf{neg} (proh)} & llorar-\textsc{imp.sg}\\
& \multicolumn{2}{l}{``¡No llores!'' (pág. 528)}
\end{tabular} \vspace{0.5cm}

}

En esta lengua existen una serie de formas libres con valores negativos \textcolor{MidnightBlue}{\citep{tarahumara}}. La forma {\setmainfont{Charis SIL} \textit{ke}} funciona tanto como una negación exclamativa como el marcador de negación clausal (76). La forma {\setmainfont{Charis SIL} \textit{'ka't͡ʃè}} es el marcador de negación clausal por excelencia (77), y en algunos casos sirve para introducir cláusulas explicativas con valor negativo (78). la forma {\setmainfont{Charis SIL} \textit{ke'tâsi}} también funciona como negador clausal y como una interjección (79). Finalmente, la forma {\setmainfont{Charis SIL} \textit{'kíti}} es un imperativo negativo, al que algunos autores suelen llamar un \textit{prohibitivo} (80).
\section*{Tepehua de Huehuetla}
\addcontentsline{toc}{section}{Tepehua de Huehuetla}

\noindent El tepehua es una lengua indígena que se habla en algunos pueblos de los Estados mexicanos de Hidalgo, Puebla y Veracruz. Pertenece a la familia lingüística totonaca. El tepehua tiene tres variedades: el tepehua de Huehuetla, el de Pisaflores y el de Tlachichilco. La variedad de Huehuetla es la que se describe en el documento y se habla en el pueblo de Huehuetla en Hidalgo, así como en algunas comunidades cercanas.
%%%%%%%%%%%%% Abreviaturas %%%%%%%%%%%%%%%%%%%%%
\footnote{COMP: complementador, IMPFV: imperfectivo, PFV: perfectivo, RPT: \textit{Reported speech} (evidencial), SUB: sujeto}
%%%%%%%%%%%%%%%%%%%%%%%%%%%%%%%%%%%%%%%%%%%%%%%%

\vspace{0.5cm}

{\setmainfont{Charis SIL} 

% Ejemplo 81
\begin{tabular}{llllll}
(81) & maa & \textbf{jaantu} & laa-y & 7alin & s-7asqat'a-7an \\
& \textsc{rpt} & \textsc{\textbf{neg}} & poder-\textsc{impfv} & haber & \textsc{3pos}-hijo-\textsc{pl.pos} \\
& \multicolumn{5}{l}{``Él/ella no puede tener hijos'' (pág. 578)}
\end{tabular} \vspace{0.4cm}

% Ejemplo 82
\begin{tabular}{lll}
(82) & \textbf{jaantu} & lapanak \\
& \textsc{\textbf{neg}} & persona \\
& \multicolumn{2}{l}{``Él no era una persona (humano)'' (pág. 580)}
\end{tabular} \vspace{0.4cm}

% Ejemplo 83
\begin{tabular}{lllllll}
(83) & 7ix-jun-niita & juu & lapanak & maa & \textbf{jaantu} & lhuu \\
& \textsc{psd}-ser-\textsc{pfv} & \textsc{det} & persona & \textsc{rpt} & \textsc{\textbf{neg}} & muchos \\
& \multicolumn{6}{l}{``La multitud no era muy numerosa'' (pág. 580)}
\end{tabular} \vspace{0.4cm}

% Ejemplo 84
\begin{tabular}{lllll}
(84) & \textbf{jaantu} & k-lakask'in & nii & 7a-miilhpa-t'i \\
& \textsc{\textbf{neg}} & \textsc{1sub}-querer & \textsc{comp} & \textsc{irr}-cantar-\textsc{2sg.sub.pfv}\\
& \multicolumn{4}{l}{``No quiero que tu cantes'' (pág. 582)}
\end{tabular} \vspace{0.4cm}

% Ejemplo 85
\begin{tabular}{lll}
(85) & \textbf{jaantu} & tu7u7 \\
& \textsc{\textbf{neg}} & algo \\
& \multicolumn{2}{l}{``Nada'' (pág.583)}
\end{tabular} \vspace{0.4cm}

% Ejemplo 86
\begin{tabular}{lll}
(86) & \textbf{jaantu} & laqlhuu \\
& \textsc{\textbf{neg}} & caro \\
& \multicolumn{2}{l}{``Barato, no caro'' (pág. 584)}
\end{tabular} \vspace{0.4cm}

}

En esta lengua, la partícula independiente {\setmainfont{Charis SIL} \textit{jaantu}} es utilizada para negar tanto cláusulas como frases \textcolor{MidnightBlue}{\citep{Tepehua}}. Parece no tener ningún tipo de restricción y basta con su colocación antes del elemento al que negará (81) - (86).
\section*{Tepehuano del sur}

\noindent El Tepehuano del Sureste es una lengua tepimán de la familia yuto-azteca que se habla en la Sierra Madre de Durango, México. La lengua tepehuana no tiene una clara distinción entre una cláusula compleja y una cláusula simple, ya que muestra varios tipos de cláusulas incrustadas o dependientes, como cláusulas de complemento y de relativo, así como cláusulas no incrustadas, como la coordinación. Además, la lengua tepehuana también exhibe frases verbales yuxtapuestas formadas por la combinación de dos verbos sin marcadores de subordinación o coordinación explícitos. \vspace{0.5cm}

{\setmainfont{Charis SIL} 

% Ejemplo 87
\begin{tabular}{llllll}
(87) & \textbf{cham tu’} & tu’ & ja’tkam & ja’pi & xi’-xbulhi-k  \\
& \textsc{\textbf{neg}} & algo & persona & pero & \textsc{red:pl}-remolino-\textsc{pnct}\\
& \multicolumn{5}{l}{``Esos no eran humanos, eran remolinos'' (pág. 110)}
\end{tabular} \vspace{0.5cm}

% Ejemplo 88
\begin{tabular}{llllll}
(88) & \textbf{cham tu’} & kɨʼn & ya’ & ja’p & jim-iñ-ji \\
& \textsc{\textbf{neg}} & \textsc{posp}:con & \textsc{dir} & \textsc{dir} & ir-\textsc{1sg.suj-cd} \\
& \multicolumn{5}{l}{``No vine para molestarte con eso'' (pág. 111)}
\end{tabular} \vspace{0.5cm}

% Ejemplo 89
\begin{tabular}{llllll}
(89) & na=ch & \textbf{cham} & agren & mui’ & tu-ma-mar-ka’ \\
& \textsc{sb=1pl.suj} & \textsc{\textbf{neg}} & a propósito & un montón de & \textsc{dur-red:pl}-hijo-\textsc{est} \\
& \multicolumn{5}{l}{``Para no tener una gran cantidad de hijos'' (pág. 112)}
\end{tabular} \vspace{0.5cm}

% Ejemplo 90
\begin{tabular}{lllll}
(90) & dai & na=m & \textbf{cham} & pensar-ka’ \\
& solo & \textsc{sb=3pl.suj} & \textsc{\textbf{neg}} & pesar-\textsc{est} \\
& \multicolumn{4}{l}{``Solo que no estaban pensando'' (pág. 112)}
\end{tabular} \vspace{0.5cm}

% Ejemplo 91
{\small
\begin{tabular}{llllllll}
(91) & \textbf{Cham} & mat-am & \textbf{cham} & mat-am & ma’n & na=m-jax & kai’ch \\
& \textsc{\textbf{neg}} & saber-\textsc{3pl.suj} & \textsc{\textbf{neg}} & saber-\textsc{3pl.suj} & único & \textsc{sb=3pl.suj-adv} & decir \\
& \multicolumn{7}{l}{``Ellos no saben, no saben, dicen lo mismo'' (pág. 112)}
\end{tabular} \vspace{0.5cm} }

}

La marcación de la negación en esta lengua es por medio de la partícula {\setmainfont{Charis SIL} \textit{cham tu’}} (87) (88) o por medio de su forma simple {\setmainfont{Charis SIL} \textit{cham}} (89) - (91) \textcolor{MidnightBlue}{\citep{Tepehuano}}. La posición de la partícula muestra si se trata de negación clausal o frasal, ya que esta se antepone al elemento que negará.
\section*{Tlahuica de San Juan Atzingo}
\addcontentsline{toc}{section}{Tlahuica de San Juan Atzingo}

\noindent El tlahuica es una lengua otopame hablada en San Juan Atzingo y otras comunidades del Municipio de Ocuilan, en el Estado de México. Se caracteriza por tener diferentes clases verbales indicadas por medio de pautas de prefijación en los verbos. Estos prefijos contienen información sobre la persona, el tiempo, el aspecto y el modo, así como también sobre la transitividad. Las clases verbales del tlahuica tienen motivación morfológica y semántica.
%%%%%%%%%%%%% Abreviaturas %%%%%%%%%%%%%%%%%%%%%
\footnote{GI: grupo I, GII: grupo II, TR: transitivo}
%%%%%%%%%%%%%%%%%%%%%%%%%%%%%%%%%%%%%%%%%%%%%%%%


\vspace{0.5cm}

{\setmainfont{Charis SIL} 

% Ejemplo 92
\begin{tabular}{ll}
(92) & \textbf{tét}-kuntu-ts’ája \\
& \textsc{\textbf{neg}-2pl.pres.tr.gi}-enojar \\
& ``(Ustedes) no lo hacen enojar'' (pág. 197)
\end{tabular} \vspace{0.3cm}

% Ejemplo 93
\begin{tabular}{ll}
(93) & \textbf{té}-lu-tǔhnɡi \\
& \textsc{\textbf{neg}-1sg.pres.tr.gii}-esconder \\
& ``No lo escondo'' (pág. 197)
\end{tabular} \vspace{0.3cm}

% Ejemplo 94
\begin{tabular}{lll}
(94) & kǎkɨ & \textbf{te}-lu-nantli \\
& \textsc{pro.1sg} & \textsc{\textbf{neg}-1sg.pres.tr}-ser mamá\\
& \multicolumn{2}{l}{``Yo no soy mamá'' (pág. 199)}
\end{tabular} \vspace{0.3cm}

% Ejemplo 95
\begin{tabular}{llll}
(95) & kǎkɨ & \textbf{te}-lu-lǒ & pa-t\textsuperscript{h}ó \\
& \textsc{pro.1sg} & \textsc{\textbf{neg}-1sg.pres.tr}-estar & \textsc{1sg.pos}-casa \\
& \multicolumn{3}{l}{``Yo no estoy en mi casa'' (pág. 199)}
\end{tabular} \vspace{0.3cm}

% Ejemplo 96
\begin{tabular}{ll}
(96) & \textbf{nó}-kilu-nú \\
& \textsc{\textbf{neg}-1sg.fut.tr.gi}-despertar \\
& ``No lo voy a despertar'' (pág. 204)
\end{tabular} \vspace{0.3cm}

% Ejemplo 97
\begin{tabular}{ll}
(97) & \textbf{nó}-kilu-nu-hə́ \\
& \textsc{\textbf{neg}-1sg.fut.tr.gi}-despertar-\textsc{pl} \\
& ``No lo vamos a despertar'' (pág. 204)
\end{tabular} \vspace{0.5cm}

}

\textcolor{MidnightBlue}{\citet{Tlahuica}} describe dos prefijos verbales para la negación en esta lengua: {\setmainfont{Charis SIL} \textit{té-,tét-}}. Menciona claramente que se desconoce la motivación de la aparición de uno u otro (92) - (95). Hay un cambio de los prefijos de negación que pueden predecirse debido a un cambio en el \textsc{tam}. Estos prefijos no guardan relación alguna en forma por lo cual los llama «prefijos \textit{portmanteu}» \textcolor{MidnightBlue}{\citep[pág. 200]{Tlahuica}}. En el caso del tiempo futuro el prefijo {\setmainfont{Charis SIL} \textit{nó-}} es el utilizado para la negación (96) (95).
\section*{Tlapaneco (me’phaa de Zilacayotitlán)}

\noindent El me’phaa es una lengua perteneciente a la familia lingüística otomangue. Forma parte de la familia Tlapaneco-Subtiaba del grupo Tlapaneco-Mangueano. La comunidad de Zilacayotitlán se encuentra en el estado de Guerrero, México en las regiones de la montaña y la costa chica. 
% Está ubicada a 13 kilómetros de la cabecera municipal del municipio de Atlamajalcingo del Monte.

El me'phaa es una lengua con verbos al inicio de la oración en la cual las relaciones gramaticales se indican en el núcleo. Es una lengua altamente flexiva que se caracteriza por tener elementos morfológicos adheridos al tema verbal que indican información sintáctica. Estos elementos incluyen flexión de persona, animacidad, polaridad y cambio de voz. Sin embargo, hasta el momento no se han documentado afijos derivativos en esta lengua. \vspace{0.5cm}

{\setmainfont{Charis SIL} 

% Ejemplo 98
\begin{tabular}{ll}
(98) & \textbf{ta¹ga³}-ta³-ɡa¹yaa³²-’\\
& \textsc{\textbf{neg}.comp-2sg}-correr.\textsc{2sg-pah-sg}\\
& ``No corriste'' (pág. 36)
\end{tabular} \vspace{0.5cm}

% Ejemplo 99
\begin{tabular}{ll}
(99) & \textbf{ta¹ga³}-gra¹k[aa²]-uun³ \\
& \textsc{\textbf{neg}.comp}-caerse-\textsc{1sg} \\
& ``No me caí'' (pág. 66)
\end{tabular} \vspace{0.5cm}

}

La negación se marca por medio de prefijos y está convinada con el aspecto completivo por lo tanto la negación es parte de la flexión verbal \textcolor{MidnightBlue}{\citep{Tlapaneco}}. El prefijo utilizado es {\setmainfont{Charis SIL} \textit{ta¹ga³-}} (98) - (99.)
\section*{Totonaco de Tuxtla}

\noindent  El totonaco es parte de la familia de lenguas totonaco-tepehua, que se habla en los Estados de Veracruz Y Puebla. Las variantes principales son el totonaco de la sierra (central del sur) y el totonaco de la costa. Es una lengua tonal, lo que significa que se utiliza el tono para diferenciar e indicar diversos valorores gramaticales. Es una lengua aglutinante, ya que las palabras pueden estar formadas por múltiples elementos que se añaden para expresar diferentes significados. \vspace{0.5cm}

{\setmainfont{Charis SIL} 

% Ejemplo 100
\begin{tabular}{ll}
(100) & \textbf{ni}-xa-k-xkuli-y \\
& \textsc{\textbf{neg}-psd-1sg.suj}-fumar-\textsc{incomp} \\
& ``Yo no fumaba'' (pág. 78)
\end{tabular} \vspace{0.5cm}

% Ejemplo 101
\begin{tabular}{lll}
(101) & \textbf{nitu} & xa-k-wi \\
& \textsc{\textbf{neg}} & \textsc{psd-1sg.suj}-estar.sentado \\
& \multicolumn{2}{l}{``Yo no estaba sentado''} (pág. 78)
\end{tabular} \vspace{0.5cm}

% Ejemplo 102
\begin{tabular}{lll}
(102) & \textbf{ni}-chixku & kit \\
& \textsc{\textbf{neg}}-hombre & yo \\
& \multicolumn{2}{l}{``Yo no soy hombre''} (pág. 78)
\end{tabular} \vspace{0.5cm}

% Ejemplo 103
\begin{tabular}{lll}
(103) & \textbf{nitu} & k-laqapas-∅ \\
& \textsc{\textbf{neg}} & \textsc{1sg.suj}-conocer-\textsc{incomp}\\
& \multicolumn{2}{l}{``Yo no conozco nada''} (pág. 79)
\end{tabular} \vspace{0.5cm}

% Ejemplo 104
\begin{tabular}{llll}
(104) & \textbf{nitu} & k-maxki-qoo-nit=i & liwat \\
& \textsc{\textbf{neg}} & \textsc{1sg.suj}-dar-\textsc{3pl-pftv=jf} & comida \\
& \multicolumn{3}{l}{``No les he dado comida a ellos''} (pág. 79)
\end{tabular} \vspace{0.5cm}

}

\textcolor{MidnightBlue}{\citet{Totonaco}} describe dos marcas para la negación en totonaco: {\setmainfont{Charis SIL} \textit{ni-, nitu'}}. La primera es un prefijo que precede a la marca de tiempo/modo (100), mientras que la segunda es una partícula que precede a todo el verbo (101). Los predicados nominales se niegan únicamente con {\setmainfont{Charis SIL} \textit{ni-}} (102). {\setmainfont{Charis SIL} \textit{nitu'}} se utiliza con verbos transitivos y ditransitivos, ocurre con el aspecto incompletivo y con objetos de de tercera persona singular (103) - (104). En el resto de los casos, como con verbos dinámicos, se recurre a {\setmainfont{Charis SIL} \textit{ni-}}.
\section*{Triqui de Copala}

\noindent El triqui es una lengua de la familia oto-mangue del grupo mixtecano. Se divide en tres variedades principales: Copala, San Andrés Chicahuaxtla y San Martín Itunyoso, las cuales son bastante diferentes en sonidos y significado de las palabras. También existen diferencias en las clases de palabras, como pronombres y conjunciones. Las tres variantes se encuentran en el Estado de Oaxaca, México.
%%%%%%%%%%%%% Abreviaturas %%%%%%%%%%%%%%%%%%%%%
\footnote{( - ): El autor indica que {\setmainfont{Charis SIL} \textit{ma'3}} es una partícula que expresa modo y que esta simpre acompaña a construcciones negativas}
%%%%%%%%%%%%%%%%%%%%%%%%%%%%%%%%%%%%%%%%%%%%%%%%
\vspace{0.4cm}

{\setmainfont{Charis SIL} 

% Ejemplo 105
\begin{tabular}{llllll}
(105) & \textbf{ne³} & Guaá⁴ & me³ & so'³ & ma'³ \\
& \textsc{\textbf{neg}} & Juan & ser & él & ( - ) \\
& \multicolumn{5}{l}{``No es Juan'' (pág. 74)}
\end{tabular} \vspace{0.2cm}

% Ejemplo 106
\begin{tabular}{llllll}
(106) & \textbf{nuveé⁴} & Guaá⁴ & me³ & so'³ & ma'³ \\
& \textsc{\textbf{neg}} & Juan & ser & él & ( - ) \\
& \multicolumn{5}{l}{``No es Juan'' (pág. 74)}
\end{tabular} \vspace{0.2cm}

% Ejemplo 107
\begin{tabular}{llllll}
(107) & \textbf{nuveé⁴} & Guaá⁴ & ca'anj³² & Macáá⁵ & ma'³ \\
& \textsc{\textbf{neg}} & Juan & fue & México & ( - ) \\
& \multicolumn{5}{l}{``No fue Juan el que fue a México'' (pág. 73)} 
\end{tabular} \vspace{0.2cm}

% Ejemplo 108
\begin{tabular}{llllll}
(108) & \textbf{ne³} & náán⁵ & cha³na̱¹ & yatzíj⁵ & ma'³ \\
& \textsc{\textbf{neg}} & lavar & mujer &  ropa & ( - ) \\
& \multicolumn{5}{l}{``La mujer no lava la ropa'' (pág. 114)}
\end{tabular} \vspace{0.2cm}

% Ejemplo 109
\begin{tabular}{lllll}
(109) & \textbf{ne³} & cane̱² & so'³ & ma'³ \\
& \textsc{\textbf{neg}} & bañarse & él & ( - ) \\
& \multicolumn{4}{l}{``Él no se bañó'' (pág. 114)}
\end{tabular} \vspace{0.2cm}

% Ejemplo 110
\begin{tabular}{lllll}
(110) & \textbf{se²} & cane³² & so'³ & ma'³ \\
& \textsc{\textbf{neg}} & barñarse.\textsc{fut} & él & ( - ) \\
& \multicolumn{4}{l}{``Él no se va a bañar'' (pág. 115)}
\end{tabular} \vspace{0.3cm}

}

En esta lengua, existen dos partículas negativas para formar una frase nominal negativa \textcolor{MidnightBlue}{\citep{Triqui}}. Las partículas son {\setmainfont{Charis SIL} \textit{ne³, nuveé⁴}}. Cuando las construcciones cuentan con la cópula {\setmainfont{Charis SIL} \textit{me³}} los hablantes utilizan ambas partículas en una aparente variación libre (105) y (106). Se utiliza {\setmainfont{Charis SIL} \textit{nuveé⁴}} cuando se quiere negar la veracidad de lo dicho por otro (107). Por su parte, la negación clausal recurre al adverbio {\setmainfont{Charis SIL} \textit{ne³}}, tanto como para tiempo presente (108), como para pasado (109). En futuro se utiliza {\setmainfont{Charis SIL} \textit{se²}} (110). Las construcciones negativas suelen ser acompañadas de una partícula que da énfasis a la negación.
\section*{Tsotsil}
\addcontentsline{toc}{section}{Tsotsil}

\noindent El tsotsil es una lengua maya del grupo tseltalano hablada en el Estado de Chiapas, México. Tiene una morfología altamente prefijal y sufijal, y reduplicación parcial.
%%%%%%%%%%%%% Abreviaturas %%%%%%%%%%%%%%%%%%%%%
\footnote{NT: Aspecto neutral (\textit{neutral aspect})}
%%%%%%%%%%%%%%%%%%%%%%%%%%%%%%%%%%%%%%%%%%%%%%%%
\vspace{0.5cm}

{\setmainfont{Charis SIL}

% Ejemplo 111
\begin{tabular}{lllll}
(111) & \textbf{mu} & vinik-\textbf{uk} & li & Petul-e \\
& \textsc{\textbf{neg}} & hombre-\textsc{\textbf{neg}} & \textsc{det} & \textsc{cl} \\
& \multicolumn{4}{l}{``Petul no es un hombre (aún)'' (pág. 13)}
\end{tabular} \vspace{0.5cm}

% Ejemplo 112
\begin{tabular}{lll}
(112) & \textbf{mu} & p'ij-\textbf{uk} \\
& \textsc{\textbf{neg}} & pequeño-\textsc{\textbf{neg}} \\
& \multicolumn{2}{l}{``Él no es pequeño'' (pág. 13)}
\end{tabular} \vspace{0.5cm}

% Ejemplo 113
\begin{tabular}{lll}
(113) & \textbf{mu} & x-7abtej \\
& \textsc{\textbf{neg}} & \textsc{nt}-trabajar \\
& \multicolumn{2}{l}{``Él no va a trabajar'' (pág. 13)}
\end{tabular} \vspace{0.5cm}

}

\textcolor{MidnightBlue}{\citet{Tzotzil}} explica que la negación se hace por medio de la partícula {\setmainfont{Charis SIL} \textit{mu}} en combinación con el sufijo {\setmainfont{Charis SIL} \textit{-uk}}. Esta combinación se cumple en predicados nominales (111) y adjetivales (112). Para la negación clausal solo se manifiesta la partícula {\setmainfont{Charis SIL} \textit{nu}} (113).
\section*{Zapoteco de Zoochina}

\noindent El zapoteco es una lengua de la familia oto-mangue. El zapoteco de Zoochina se habla en la comunidad de San Jerónimo Zoochina en el municipio de San Baltasar Yatzahi el Bajo dentro del distrito de Villa Alta en el Estado de Oaxaca. Es una lengua con complejidad laríngea en el nivel fonológico y posee una serie de fenómenos que han sido denominados «morfología asociada a la predicación». \vspace{0.5cm}

{\setmainfont{Charis SIL}

% Ejemplo 114
\begin{tabular}{lll}
(114) & ya & tø=køx-’oy-tam-a \\
& \textsc{neg} & \textsc{1suji}=comer-\textsc{ap-pl.pah-incomp.neg} \\
& \multicolumn{2}{l}{``No comemos'' (pág. 167)}
\end{tabular} \vspace{0.5cm}

% Ejemplo 115
\begin{tabular}{lll}
(115) & ya & ’øn=ker-tsøk-ø \\
& \textsc{neg} & \textsc{agt}=creer-hacer-incomp.neg \\
& \multicolumn{2}{l}{``No lo creo'' (pág. 167)}
\end{tabular} \vspace{0.5cm}

% Ejemplo 116
\begin{tabular}{llll}
(116) & tøx & ya & tø=min-wø \\
& \textsc{1pro} & \textsc{neg} & \textsc{1suji}=venir-compi.neg \\
& \multicolumn{3}{l}{``Yo no vine'' (pág. 169)}
\end{tabular} \vspace{0.5cm}

% Ejemplo 117
\begin{tabular}{lllll}
(117) & ya & ’øn=mux-wø & ti & ’øy=nøji \\
& \textsc{neg} & \textsc{1agt}=saber-\textsc{compi.neg} & qué & \textsc{3posd}=nombre \\
& \multicolumn{4}{l}{``No supe cómo se llama'' (pág. 170)}
\end{tabular} \vspace{0.5cm}





}

Esta lengua cuenta con tres marcas negativas: {\setmainfont{Charis SIL} \textit{ya, yampa, 'u}} \textcolor{MidnightBlue}{\citep{zapoteco}}. La forma {\setmainfont{Charis SIL} \textit{ya}} niega oraciones con aspecto incompletivo (114) - (115) y completivo (116) - (117).
\section*{Zoque de San Miguel Chimalapa}
\addcontentsline{toc}{section}{Zoque de San Miguel Chimalapa}

\noindent El zoque es una de las siete agrupaciones linguísticas que conforman a la familia mixe-zoque y pertenece a la rama zoqueana. Se habla en los Estados mexicanos de Oaxaca y Chiapas. El zoque de Chimalapa se habla en el municipio de San Miguel Chimalapa al Este del Estado de Oaxaca.

Se trata de una lengua ergativa para la primera y tercera persona, mientras que para la segunda persona muestra un comportamiento nominativo-acusativo. Es polisentética y con marcación en el núcleo.\footnote{S.I: sujeto independiente, AP: antipasivo, PL.PAH: plural de participante del acto de habla, ICP: incompletivo, A: agente, CMP.I: completivo independiente, S.D: sujeto dependiente, S: sujeto, ICP.D: incompletivo dependiente, OPT: Optativo, SBR: subordinador}  \vspace{0.5cm}

{\setmainfont{Charis SIL}

% Ejemplo 119
\begin{tabular}{lll}
(119) & \textbf{ya} & tø=køx-’oy-tam-\textbf{a }\\
& \textsc{\textbf{neg}} & \textsc{1s.i}=comer-\textsc{ap-pl.pah-icp.\textbf{neg}} \\
& \multicolumn{2}{l}{``No comemos'' (pág. 167)} 
\end{tabular} \vspace{0.5cm}

% Ejemplo 120
\begin{tabular}{llll}
(120) & tøx & \textbf{ya} & tø=min-\textbf{wø} \\
& \textsc{1pro} & \textsc{\textbf{neg}} & \textsc{1s.i}=venir-\textsc{cmp.i.\textbf{neg}} \\
& \multicolumn{3}{l}{``Yo no vine'' (pág. 169)}
\end{tabular} \vspace{0.5cm}

% Ejemplo 121
\begin{tabular}{llll}
(121) & pe & \textbf{yampa} & Ø=yets-\textbf{a} \\
& pero & \textsc{\textbf{neg}} & \textsc{3s.d}=llegar-\textsc{perf.pres.\textbf{neg}}\\
& \multicolumn{3}{l}{``Pero no ha llegado'' (pág. 171)}
\end{tabular} \vspace{0.5cm}

% Ejemplo 122
\begin{tabular}{lll}
(122) & \textbf{’u} & ’øm=tsaxø-\textbf{wø} \\
& \textsc{\textbf{neg}.imp} & \textsc{2s}=avergonzar-\textsc{icp.d} \\
& \multicolumn{2}{l}{``¡No te avergüences!'' (pág. 174)}
\end{tabular} \vspace{0.5cm}

% Ejemplo 123
\begin{tabular}{llll}
(123) & \textbf{’u} & yakkø & nø’-tsek-\textbf{’a} \\
& \textsc{\textbf{neg}.opt} & \textsc{sbr.opt} & agua-pedir-\textsc{opt.\textbf{neg}} \\
& \multicolumn{3}{l}{``Que no pida agua'' (pág. 175)}
\end{tabular} \vspace{0.5cm}

}

Esta lengua cuenta con tres marcas negativas: {\setmainfont{Charis SIL} \textit{ya, yampa, 'u}} \textcolor{MidnightBlue}{\citep{zoque}}. La forma {\setmainfont{Charis SIL} \textit{ya}} niega oraciones con aspecto incompletivo (119) y completivo (120). Las construcciones que tienen una lectura de aspecto perfecto toman {\setmainfont{Charis SIL} \textit{yampa}} (121). Finalmente, las formas imperativas se niegan con {\setmainfont{Charis SIL} \textit{'u}} (122) aunque también ocurre con construcciones optativas (123). Los verbos negados llevan, además, un sufijo con valor aspectual negativo ({\setmainfont{Charis SIL} \textit{-a, -wø}}).


\chapter{Lenguas del resto del continente americano}
\section*{Aguaruna}

\noindent El aguaruna es una lengua jívara del norte del Perú. El aguaruna se habla en las estribaciones orientales de los Andes y presenta similitudes tipológicas tanto con las lenguas amazónicas como con las andinas. 

Los rasgos tipológicos más destacados son: orden de constituyentes AOV (Activo-Objeto-Verbo), perfil nominativo-acusativo, marcación combinada de núcleo y dependiente, y una sintaxis altamente hipotáctica (encadenamiento de cláusulas).
%%%%%%%%%%%%% Abreviaturas %%%%%%%%%%%%%%%%%%%%%
\footnote{ATT: atenuativo, DECL: declarativo, DISTPAST: pasado distante, IMPFV: imperfectivo, INTS: intensivo}
%%%%%%%%%%%%%%%%%%%%%%%%%%%%%%%%%%%%%%%%%%%%%%%%
\vspace{0.5cm}

{\setmainfont{Charis SIL} 

% Ejemplo 124
\begin{tabular}{lll}
(124) & wi-ka & buuta-\textbf{tsu}-ha-i  \\
& \textsc{1sg-foc} & llorar+\textsc{impfv-\textbf{neg}-1sg-decl} \\
& \multicolumn{2}{l}{``No estaba llorando'' (pág. 325)}
\end{tabular} \vspace{0.5cm}

% Ejemplo 125
\begin{tabular}{lll}
(125) & wi-ka & yu-a-\textbf{tsu}-ha-i \\
& \textsc{1sg-foc} & comer-\textsc{impfv-\textbf{neg}-1sg-decl} \\
& \multicolumn{2}{l}{``No estoy comiendo'' (pág. 325)}
\end{tabular} \vspace{0.5cm}

% Ejemplo 126
\begin{tabular}{ll}
(126) & daka-sa-\textbf{tʃa}-tata-ha-i \\
& esperar-\textsc{att-\textbf{neg}-fut-1sg-decl} \\
& ``No esperaré'' (pág. 325)
\end{tabular} \vspace{0.5cm}

% Ejemplo 127
\begin{tabular}{ll}
(127) & waina-ka-\textbf{tʃa}-amaia-ha-i \\
& ver-\textsc{ints-\textbf{neg}-distpast-1sg-decl} \\
& ``No vi a nadie'' (pág. 325)
\end{tabular} \vspace{0.5cm}

% Ejemplo 128
\begin{tabular}{ll}
(128) & \textbf{atsu}-a-wa-i \\
& \textsc{\textbf{neg}.exist-impfv-3-decl} \\
& ``No hay nada'' (pág. 326)
\end{tabular} \vspace{0.5cm}

}

La negación se realiza por medio de dos sufijos condicionados por el tiempo verbal \textcolor{MidnightBlue}{\citep{aguaruna}}. El sufijo {\setmainfont{Charis SIL} -\textit{tsu}} aparece con el tiempo presente y el pasado remoto (124) - (125), mientras que el sufijo {\setmainfont{Charis SIL} -\textit{tʃa}} lo hace con el resto de tiempos (126) - (127). Existe también el verbo de negación existencial {\setmainfont{Charis SIL} \textit{atsu}} (128).
\section*{Awa Pit (Cuaiquer)}

\noindent El awá pit, conocido también como cuaiquer o kwaiker, es una lengua indígena de la familia barbacoana. Se habla en el sur de Colombia y el norte de Ecuador.
%%%%%%%%%%%%% Abreviaturas %%%%%%%%%%%%%%%%%%%%%
\footnote{IMPFPART: participio imperfectivo, LOCUT:marcador de persona locutor, NONLOCUT: marcador de persona no locutor, PAST: pasado, TOP: marcador de tópico}
%%%%%%%%%%%%%%%%%%%%%%%%%%%%%%%%%%%%%%%%%%%%%%%%
\vspace{0.2cm}

{\setmainfont{Charis SIL} 

% Ejemplo 129
\begin{tabular}{llll}
(129) & Santos=na & \textbf{shi} & ɨ-\textbf{ma}-y \\
& Santos=\textsc{top} & \textsc{\textbf{neg}} & ir-\textsc{\textbf{neg}-nonlocut} \\
& \multicolumn{3}{l}{``Santos no fue'' (pág. 332)}
\end{tabular} \vspace{0.2cm}

% Ejemplo 130
\begin{tabular}{lllll}
(130) & ap & gallo & \textbf{shi} & \textbf{ki}-a-zi \\
& mío & gallo & \textsc{\textbf{neg}} & ser.\textsc{\textbf{neg}-past-nonlocut} \\
& \multicolumn{4}{l}{``No era mi gallo'' (pág. 333)}
\end{tabular} \vspace{0.2cm}

% Ejemplo 131
\begin{tabular}{llll}
(131) & \textbf{shi} & ayna-mtu & \textbf{ki}-s \\
& \textsc{\textbf{neg}} & cocinar-\textsc{impfpart} & ser.\textsc{\textbf{neg}-locut} \\
& \multicolumn{3}{l}{``No estoy cocinando'' (pág. 334)}
\end{tabular} \vspace{0.2cm}

% Ejemplo 132
\begin{tabular}{llll}
(132) & ap & \textbf{shi} & ka-y \\
& mío & \textsc{\textbf{neg}} & ser:permanentemente-\textsc{nonlocut} \\
& \multicolumn{3}{l}{``No es mío'' (pág. 335)}
\end{tabular} \vspace{0.2cm}

% Ejemplo 133
\begin{tabular}{llll}
(133) & kwizha=na & alizh & \textbf{shi} \\
& perro=\textsc{top} & feroz & \textsc{\textbf{neg}(nonlocut)} \\
& \multicolumn{3}{l}{``El perro no es feroz'' (pág. 336)}
\end{tabular} \vspace{0.2cm}

}

La mayoría de las construcciones negativas implican el uso de la partícula negativa {\setmainfont{Charis SIL} \textit{shi}} y se dividen en dos tipos: negación clausal y negación no clausal \textcolor{MidnightBlue}{\citep{awa}}. La negación clausal está en (129) donde la partícula {\setmainfont{Charis SIL} \textit{shi}} ocupa una posición preverbal y dentro de la morfología verbal aparece el sufijo {\setmainfont{Charis SIL} \textit{-ma}}. Esta estrategia se utiliza tanto con verbos activos como estativos. En (130) se recurre a la cópula negativa {\setmainfont{Charis SIL} \textit{ki}} que se opone a su contraparte positiva \textit{i}. Por último, en (131) la copula negativa desempeña el papel de un verbo auxiliar al aparecer junto a una forma no finita de un verbo con valor léxico, en este caso una forma participial del verbo cocinar.

Por su parte, la negación no clausal consiste en colocar la partícula negativa {\setmainfont{Charis SIL} \textit{shi}} después del elemento que se quiera negar. Por lo tanto, en (132) lo que se quiere negar es la marca posesiva de primera persona y en (133) únicamente el atributo «feroz».

El Awa pit es una lengua de doble negación cuando se trata de la negación clausal. Esto quiere decir que necesita de dos marcadores para que la construcción sea gramatical y tenga lectura negativa. Por lo tanto, utiliza recursos morfológicos como sintácticos.
\section*{Cavineña}
\addcontentsline{toc}{section}{Cavineña}

\noindent El cavineño es una lengua indígena hablada en el norte de Bolivia en la región amazónica. Pertenece a la familia tacana y es una lengua en peligro de extinción.
%%%%%%%%%%%%% Abreviaturas %%%%%%%%%%%%%%%%%%%%%
\footnote{AFFTN: afección, DAT: dativo, DESID: desiderativo, ERG: ergativo, FM: formativo, GEN: genitivo, IMPFV: imperfectivo, NPF: prefijo nominal, PERF: perfecto, PERL: perlativo, POT: potencial, QUEST: marcador de pregunta, REC.PAST: pasado reciente, REM.PAST: pasado remoto,}
%%%%%%%%%%%%%%%%%%%%%%%%%%%%%%%%%%%%%%%%%%%%%%%%
\vspace{0.5cm}

{\setmainfont{Charis SIL} 

% Ejemplo 134 predicado verbal
\begin{tabular}{lllll}
(134) & E-ra\textsubscript{A}=tu\textsubscript{O} & [e-kwe & tata-chi]\textsubscript{O} & adeba-ya=\textbf{ama} \\
& \textsc{1sg-erg=3sg(-fm)} & \textsc{1sg-gen} & padre-\textsc{afftn} & conocer-\textsc{impfv=\textbf{neg}} \\
& \multicolumn{4}{l}{``No conozco a mi padre'' (pág. 79)}
\end{tabular} \vspace{0.5cm}

% Ejemplo 135 predicado en pregunta
\begin{tabular}{llll}
(135) & Are=mi\textsubscript{O} & bakwa=ra\textsubscript{A} & a-wa=\textbf{ama}? \\
& \textsc{quest=2sg(-fm)} & vívora=\textsc{erg} & afectar-\textsc{perf=\textbf{neg}} \\
& \multicolumn{3}{l}{``¿No es una víbora la que te mordió?'' (pág. 102)}
\end{tabular} \vspace{0.5cm}

% Ejemplo 136 adjetivo
\begin{tabular}{llll}
(136) & E-na\textsubscript{S}=e-kwe & tupu=\textbf{ama}\textsubscript{CC} & ju-kware \\
& \textsc{npf-}agua=\textsc{1sg-dat} & suficiente=\textsc{\textbf{neg}} & ser-\textsc{rem.past} \\
& \multicolumn{3}{l}{``Me quedé sin agua (lit. el agua no era suficiente para mí)'' (pág. 103)}
\end{tabular} \vspace{0.5cm}

% Ejemplo 137 núcleo de una frases nominal
\begin{tabular}{llll}
(137) & ... =tuna-ja=tu\textsubscript{O} & dutya=\textbf{ama}\textsubscript{O} & nudya-kware \\
& =\textsc{3pl-dat=3sg(-fm)} & todos=\textsc{\textbf{neg}} & hacer.entrar-\textsc{rem.past} \\
& \multicolumn{3}{l}{``(Estaban tan enojados que) no los dejaron entrar a todos'' (pág. 103)}
\end{tabular} \vspace{0.5cm}

% Ejemplo 138 de una postposición
\begin{tabular}{llll}
(138) & Iyakwa=mikwana\textsubscript{S} & e-wasi=eke=\textbf{ama} & diru-ya \\
& ahora=\textsc{2pl} & \textsc{npf-}pie=\textsc{perl=\textbf{neg}} & ir-\textsc{impfv} \\
& \multicolumn{3}{l}{``Ahora ustedes no irán a pie'' (pág. 103)}
\end{tabular} \vspace{0.5cm}

% Ejemplo 139 desiderativo sufijo
\begin{tabular}{lllll}
(139) & Jadya=tibu & i-ke\textsubscript{S} & kwa-\textbf{karama} & ju-chine \\
& tal=razón & \textsc{1sg-fm} & ir-\textsc{desid.\textbf{neg}} & ser-\textsc{rec.past}\\
& \multicolumn{4}{l}{``Por eso (porque está demasiado lejos), no quiero ir'' (pág. 323)}
\end{tabular} \vspace{0.5cm}

% Ejemplo 140 afijo adjetivos
\begin{tabular}{lll}
(140) & Ji-\textbf{dama}=dya\textsubscript{CC}=tu\textsubscript{CS} & e-ju-u \\
& bueno-\textsc{\textbf{neg}=foc=3sg(-fm)} & \textsc{pot-}ser-\textsc{pot} \\
& \multicolumn{2}{l}{``(un colador hecho a mano) podría estar defectuoso'' (pág. 95)}
\end{tabular} \vspace{0.5cm}

% Ejemplo 141 Elemento independiente
\begin{tabular}{llll}
(141) & \textbf{Aama\textsubscript{CC}}=tu\textsubscript{CS} & ju-kware & salon=kwana\textsubscript{CS}... \\
& \textbf{No.existe}=\textsc{3sg(-fm)} & ser-\textsc{rem.past} & rifle=\textsc{pl} \\
& \multicolumn{3}{l}{``(Cuando era joven) no había rifles (solo escopetas)} \\
& \multicolumn{3}{l}{(lit. los rifles no existían)'' (pág. 141)}
\end{tabular} \vspace{0.5cm}

{\small
% Ejemplo 142 aparentemente no
\begin{tabular}{llll}
(142) & \textbf{Jipakwana}=ekwana-ja & radio\textsubscript{S} & ani-ya \\
& \textbf{aparentemente.no}=\textsc{1pl-dat} & radio.ondacorta & quedar.bien-\textsc{impfv}\\
& \multicolumn{3}{l}{``Parece que no tendremos esa radio} \\
& \multicolumn{3}{l}{(lit. una radio de onda corta aparentemente no nos sentará bien)'' (pág. 104)}
\end{tabular} \vspace{0.5cm}}

% Ejemplo 143 imperativos 
\begin{tabular}{lllll}
(143) & Mi-ke\textsubscript{S} & ani-kwe! Mi-ke\textsubscript{S} & je-\textbf{ume}! \\
& \textsc{2sg-fm} & sentarse-\textsc{imp.sg} & \textsc{2sg-fm} & venir-\textsc{imp.sg.\textbf{neg}} \\
& \multicolumn{4}{l}{``Tú quédate (lit. siéntate), no vengas'' (pág. 104 )}
\end{tabular} \vspace{0.5cm}

}

Cuenta con al menos 7 morfemas reconocidos para manifestar la negación \textcolor{MidnightBlue}{\citep{cavin}}. El primer morfema es el clítico {\setmainfont{Charis SIL} \textit{=ama}} que puede adherirse a tecnicamente cualquier constituyente. Puede negar el predicado verbal en construcciones declaraticas (134) o interrogaticas (135), un adjetivo (136), el núcleo de una frase nominal (137) e incluso una frase postposicional (138).

El sufijo verbal desiderativo {\setmainfont{Charis SIL} \textit{-karama}} (139). El sufijo {\setmainfont{Charis SIL} \textit{-dame}} para adjetivos (140). El elemento independiente {\setmainfont{Charis SIL} \textit{aama}} «no existe» (141). La particula de primera posición {\setmainfont{Charis SIL} \textit{jipakwana}} «aparentemente no» (142). Interjecciones negativas {\setmainfont{Charis SIL} \textit{aijama}} «no existe en absoluto», {\setmainfont{Charis SIL} \textit{juwaaba}} «el hablante no sabe» o {\setmainfont{Charis SIL} \textit{pajuani}} «el hablante no está de acuerdo». Finalmente, los afijos negativos para imperativos {\setmainfont{Charis SIL} -ume} o {\setmainfont{Charis SIL} \textit{ne- ... -ume}} (143).

Como puede notarse, esta lengua cuenta con un amplio repertorio de recursos para la marcación de la negación. Ocasionalmente también puede usarse clítico{\setmainfont{Charis SIL} \textit{ni=}} «ni siquiera» como un refuerzo negativo.
\section*{Chimariko}
\addcontentsline{toc}{section}{Chimariko}

\noindent El chimariko es una lengua actualmente extinta. En su momento fue una lengua aislada que se hablaba en el Condado de Trinity en la zona noroeste de California, Estados Unidos. Lengua sintética sufijante y de marcación en el núcleo.
%%%%%%%%%%%%% Abreviaturas %%%%%%%%%%%%%%%%%%%%%
\footnote{A: agente, ASP: aspecto, DEF: definido, DER: derivacional, LOC: locativo, MOD: modal, PROG: progresivo}
%%%%%%%%%%%%%%%%%%%%%%%%%%%%%%%%%%%%%%%%%%%%%%%%
\vspace{0.5cm}

{\setmainfont{Charis SIL} 

% Ejemplo 144
\noindent \begin{tabular}{lll}    
(144) & ya-\textbf{x-}akʰo\textbf{-na}-xan-ˀi & m-akʰo-ta-xan-tinda \\
& \textsc{1pl.a-\textbf{neg}-}matar-\textsc{\textbf{neg}-fut-asp} & \textsc{2sg-}matar-\textsc{der-fut-prog} \\
& \multicolumn{2}{l}{``No los mataremos, él te matará'' (pág. 177)}
\end{tabular} \vspace{0.5cm}

% Ejemplo 145
\noindent \begin{tabular}{lll}
(145) & ˀawa-ida-če & \textbf{x-}owo\textbf{-na}-t \\
& casa-\textsc{pos-loc} & \textsc{\textbf{neg}-}estar-\textsc{\textbf{neg}-asp} \\
& \multicolumn{2}{l}{``Ella no está en casa'' (pág. 177)}
\end{tabular} \vspace{0.5cm}

% Ejemplo 146
\noindent \begin{tabular}{ll}
(146) & q’e-h\textbf{-kuna}-coˀol \\
& morir-\textsc{3-\textbf{neg}-mod} \\
& ``Tal vez él no muera'' (pág. 177)
\end{tabular} \vspace{0.5cm}

% Ejemplo 147
\noindent \begin{tabular}{lll}
(147) & nunuˀ & n-e-mičit\textbf{-kuna} \\
& \textsc{x} & \textsc{imp.sg-1p-}patear-\textsc{\textbf{neg}} \\
& \multicolumn{2}{l}{``No me patees'' (pág. 179)}
\end{tabular} \vspace{0.5cm}

{\small
% Ejemplo 148
\noindent \begin{tabular}{lllllll}
(148) & h-inoˀy-ta & hi-suma & n-itix & xalall-op & n-akʰohoshu & \textbf{k’una} \\
& \textsc{3-}soportar-\textsc{asp} & \textsc{pos}-cara & \textsc{imp.sg-}limpiar & bebé-\textsc{def} & \textsc{imp.sg-}cortar & \textsc{\textbf{neg}} \\
& \multicolumn{6}{l}{``Ella lo soporta, límpiale la cara, (de) ese bebé, no le cortes (el ombligo)'' (pág. 179)}
\end{tabular} \vspace{0.5cm}}

}

Presenta tres estrategias diferentes para la negación \textcolor{MidnightBlue}{\citep{chimariko}}. La primera estrategia corresponde al uso del circunflejo verbal {\setmainfont{Charis SIL} \textit{x-...-na}} el cual solo aparece en construcciones con prefijos pronominales (144) y (145). La segunda estrategia consiste en el uso del prefijo {\setmainfont{Charis SIL} \textit{-kuna / -k'una / -ˀna}} que ocurre con toda clase de predicados verbales y nominales (146) e incluso en imperativos (147). La tercera y última estrategia es la partícula {\setmainfont{Charis SIL} \textit{kuna / k'una}} para la negación de imperativos (148).
\section*{Choctaw}
\addcontentsline{toc}{section}{Choctaw}

\noindent Lengua de la familia maskogui o muskogi hablada en el sudoeste de Estado Unidos en Oklahoma, Lusiana y Tennessee.
%%%%%%%%%%%%% Abreviaturas %%%%%%%%%%%%%%%%%%%%%
\footnote{DPAST: pasado distante, DS: sujeto diferente, NM: nominativo, PART: participio, PT: past, SS: mismo sujeto, TNS: tiempo predeterminado}
%%%%%%%%%%%%%%%%%%%%%%%%%%%%%%%%%%%%%%%%%%%%%%%%
\vspace{0.5cm}

{\setmainfont{Charis SIL} 

% Ejemplo 149
\begin{tabular}{llll}
(149) & John-at & shókha' & abi-\textbf{kiiyo}-tok \\
& John-\textsc{nm} & cerdo & martar-\textsc{\textbf{neg}-pt} \\
& \multicolumn{3}{l}{``John no mató al cerdo'' (pág. 322)}
\end{tabular} \vspace{0.5cm}

% Ejemplo 150
\begin{tabular}{llll}
(150) & John-at & shókha' & ab-aachi-\textbf{kiiyo}-h \\
& John-\textsc{nm} & cerdo & matar-\textsc{irr-\textbf{neg}-tns} \\
& \multicolumn{3}{l}{``John no va a matar al cerdo'' (pag. 322)}
\end{tabular} \vspace{0.5cm}

% Ejemplo 151
\begin{tabular}{llll}
(151) & John-at & shókha' & abi-tok-\textbf{kiiyo}(-h) \\
& John-\textsc{nm} & cerdo & matar-\textsc{pt-\textbf{neg}(-tns)} \\
& \multicolumn{3}{l}{``John no mató al cerdo'' (pág. 322)}
\end{tabular} \vspace{0.5cm}

% Ejemplo 152
\begin{tabular}{lllll}
(152) & ibbak & achiif-ahii-yo & \textbf{kiiyo}-kmat, & ipa-\textbf{kiiyo}-biika-tok \\
& mano & lavar-\textsc{irr-part:ds} & \textsc{\textbf{neg}-irr:ss} & comer-\textsc{\textbf{neg}-}extent-\textsc{pt} \\
& \multicolumn{4}{l}{``Si no se han lavado las manos no comen'' (pág. 323)}
\end{tabular} \vspace{0.5cm}

% Ejemplo 153
\begin{tabular}{lll}
(153) & nán=asháchchi-ttook-o & \textbf{kiiyo}-h-okii \\
& pecar-\textsc{dpast-part:ds} & \textsc{\textbf{neg}-tns-}en.realidad \\
& \multicolumn{2}{l}{``Ellos no pecaron'' (pág. 323)}
\end{tabular} \vspace{0.5cm}

}

La negación en esta lengua es parte de la morfología verbal por medio del sufijo {\setmainfont{Charis SIL} \textit{-kiiyo}} \textcolor{MidnightBlue}{\citep{choctaw}}. Este puede aparecer en posición anterior al tiempo verbal (149) y (150), como en posición posterior (151) pero nunca antecediendo al modo.

En posición posterior al tiempo verbal, este sufijo muestra algunas propiedades de una palabra independiente y el verbo que lo precede suele tomar la marca participio (152) y (153).
\section*{Garífuna}


\chapter{Lenguas del resto del mundo}
\section*{Fwe}
\addcontentsline{toc}{section}{Fwe}

\noindent El fwe es una lengua bantú perteneciente a la rama occidental del subgrupo Bantu Botatwe, siendo su pariente más cercano el shanjo. En cuanto a la variación dialectal, se distinguen principalmente el fwe de Zambia y el de Namibia, con algunas diferencias fonológicas y morfológicas. Estas son áreas rurales alrededor de la localidad de Kongola.
%%%%%%%%%%%%% Abreviaturas %%%%%%%%%%%%%%%%%%%%%
\footnote{FV: vocal final, INF: infinitivo, SBJV: subjuntivo, SM: marcador de sujeto, STAT: estativo}
%%%%%%%%%%%%%%%%%%%%%%%%%%%%%%%%%%%%%%%%%%%%%%%%
\vspace{0.5cm}

{\setmainfont{Charis SIL} 

% Ejemplo 200
\begin{tabular}{ll}
(200) & \textbf{ka}-ndi-ur-\textbf{í̠} \\
& \textsc{\textbf{neg}-sm1sg}-comprar-\textsc{\textbf{neg}}\\
& ``Yo no compro'' (Namibian Fwe) (pág. 418)
\end{tabular} \vspace{0.3cm}

% Ejemplo 201
\begin{tabular}{ll}
(201) & \textbf{tà}-ndi-ur-\textbf{í̠} \\
& \textsc{\textbf{neg}-sm1sg}-comprar-\textsc{\textbf{neg}}\\
 & ``Yo no compro'' (Zambian Fwe) (pág. 418)
\end{tabular} \vspace{0.3cm}

% Ejemplo 202
\begin{tabular}{ll}
(202) & \textbf{ka}-ndi-zibá̠r-\textbf{i} \\
& \textsc{\textbf{neg}-sm1sg}-olvidar-\textsc{\textbf{neg}}\\
& ``No me olvido'' (Namibian Fwe) (pág. 418)
\end{tabular} \vspace{0.3cm}

% Ejemplo 203
\begin{tabular}{ll}
(203) & \textbf{ta}-tu-kat-ite-\textbf{í̠} \\
& \textsc{\textbf{neg}-sm1pl}-volverse.delgado-\textsc{stat-\textbf{neg}} \\
& ``No estamos delgados'' (Zambian Fwe) (pág. 420)
\end{tabular} \vspace{0.3cm}

% Ejemplo 204
\begin{tabular}{lll}
(204) & mu-\textbf{ásha}-bútuk-\textbf{i} & cáha \\
& \textsc{sm2pl-\textbf{neg}.sbjv-}correr-\textsc{\textbf{neg}} & muy \\
& \multicolumn{2}{l}{``No vayas tan rápido'' (Namibian Fwe) (pág. 421)}
\end{tabular} \vspace{0.3cm}

% Ejemplo 205
\begin{tabular}{ll}
(205) & ku-\textbf{shá}-bon-a  \\
& \textsc{inf-\textbf{neg}.inf}-ver-\textsc{fv} \\
& ``No ver'' (pág. 422)
\end{tabular} \vspace{0.3cm}

}

La negación se hace por medio de afijos verbales, auxiliares y la combinación de ambos \textcolor{MidnightBlue}{\citep{fwe}}. Los prefijos {\setmainfont{Charis SIL} \textit{ka-, ta-}} se utilizan para la negación de verbos en indicativo (200) - (203) ocupan una posición pre-inicial. Los prefijos {\setmainfont{Charis SIL} \textit{ásha-, shá-}} aparecen con verbos en subjuntivo e infinitos respectivamente (204) y (205) en posiciones post-iniciales. Adicionalmente, está la vocal sufijada {\setmainfont{Charis SIL} \textit{-i}} que aparece en ciertas construcciones acompañando a los prefijos negativos, pero nunca puede aparecer como el único marcador negativo en la construcción. 

\part{Síntesis de las estrategias}
\section*{Lenguas de México - Estrategias}

% Table generated by Excel2LaTeX from sheet 'Mecanismos'
% \setcounter{table}{0}
% \begin{table}[htbp]
%     \centering
%       \begin{tabular}{lccc}
%       \textbf{Lengua} & \textbf{Léxico independiente} & \textbf{Afijos} & \textbf{Clíticos} \\
%       \hline
%       Acateco & X     &       &  \\
%       Chatino & X     &       &  \\
%       Chichimeco & X     & X     &  \\
%       Chinanteco & X     & X     &  \\
%       Chontal de Oaxaca & X     &       &  \\
%       Chontal de Tabasco & X     &       &  \\
%       Huave &       & X     &  \\
%       Huichol &       & X     &  \\
%       Lacandón & X     &       &  \\
%       Mixe  & X     & X     &  \\
%       Mixteco & X     &       &  \\
%       Oluteco &       &       & X \\
%       Otomí & X     &       &  \\
%       Sierra Popoluca & X     &       &  \\
%       Seri  &       & X     &  \\
%       Tarahumara & X     &       &  \\
%       Tepehua & X     &       &  \\
%       Tepehuano & X     &       &  \\
%       Tlahuica &       & X     &  \\
%       Tlapaneco &       & X     &  \\
%       Totonaco & X     & X     &  \\
%       Triqui & X     &       &  \\
%       Tsotsil & X     & X     &  \\
%       Zapoteco & X     &       &  \\
%       Zoque & X     & X     &  \\
%       \hline
%       \end{tabular}
%       \caption{Lenguas de México}
%     \label{cuadro1}
%   \end{table}
  
\noindent De las 25 lenguas que corresponden a México, solo 13 de ellas recurren a léxico independiente para la marcarción de la negacion, 5 al uso de afijos exclusivamente, 6 usan tanto afijos como léxico independiene y solo 1 lengua recurre al uso de cliticos.

El hecho de que algunas lenguas recurran tanto a léxico independiente como al uso de afijos no quiere decir, en un primer momento, que sean lenguas de doble negación. El uso de estas estrategias puede alternar por el tipo de negación: clausal, de constituyentes o de imperativos por ejemplo.

% Table generated by Excel2LaTeX from sheet 'Mex solo lexico'
% \begin{table}[htbp]
%     \centering
%       \begin{tabular}{lc}
%       \multicolumn{1}{c}{\textbf{Lengua}} & \multicolumn{1}{c}{\textbf{léxico independiente}} \\
%       \hline
%       Acateco & {\setmainfont{Charis SIL} \textit{maː}} / {\setmainfont{Charis SIL} \textit{man}} / {\setmainfont{Charis SIL} \textit{k’am}} / {\setmainfont{Charis SIL} \textit{ʔamax}} \\
%       Chatino & {\setmainfont{Charis SIL} \textit{a3}} \\
%       Chontal de Oaxaca & {\setmainfont{Charis SIL} \textit{maa}} \\
%       Chontal de Tabasco & {\setmainfont{Charis SIL} \textit{mach}}/ {\setmainfont{Charis SIL} \textit{mach’an}} / {\setmainfont{Charis SIL} \textit{mame’}} / {\setmainfont{Charis SIL} \textit{machme’}} / {\setmainfont{Charis SIL} \textit{moni’}} / {\setmainfont{Charis SIL} \textit{mani’}} \\
%       Lacandón & {\setmainfont{Charis SIL} \textit{maʔ}} \\
%       Mixteco & {\setmainfont{Charis SIL} \textit{ko̱}} / {\setmainfont{Charis SIL} \textit{ta’on}} \\
%       Otomí & {\setmainfont{Charis SIL} \textit{hinɡi}} / {\setmainfont{Charis SIL} \textit{him}} / {\setmainfont{Charis SIL} \textit{hi}} / {\setmainfont{Charis SIL} \textit{hin}} \\
%       Sierra Popoluca & {\setmainfont{Charis SIL} \textit{ʔotʔoy}} / {\setmainfont{Charis SIL} \textit{dya}} \\
%       Tarahumara & {\setmainfont{Charis SIL} \textit{’ka’t͡ʃè}} / {\setmainfont{Charis SIL} \textit{ke’tâsi}} / {\setmainfont{Charis SIL} \textit{’kíti}} / {\setmainfont{Charis SIL} \textit{ke}} \\
%       Tepehua & {\setmainfont{Charis SIL} \textit{jaantu}} \\
%       Tepehuano & {\setmainfont{Charis SIL} \textit{cham}} tu’ / {\setmainfont{Charis SIL} \textit{cham}} \\
%       Triqui & {\setmainfont{Charis SIL} \textit{ne³}} / {\setmainfont{Charis SIL} \textit{nuveé⁴}} / {\setmainfont{Charis SIL} \textit{se²}} \\
%       Zapoteco & {\setmainfont{Charis SIL} \textit{gàgé}} / {\setmainfont{Charis SIL} \textit{àgé}} / {\setmainfont{Charis SIL} \textit{bìtò}} / {\setmainfont{Charis SIL} \textit{bì}} \\
%       \hline
%       \end{tabular}%
%       \caption{Lenguas que codifican la negación en léxico o partículas independientes}
%     \label{cuadro2}%
%   \end{table}
  
  % Table generated by Excel2LaTeX from sheet 'Mex solo lexico'
% \begin{table}[htbp]
%     \centering
%       \begin{tabular}{lc}
%       \multicolumn{1}{c}{\textbf{Lengua}} & \textbf{morfema} \\
%       Huave & {\setmainfont{Charis SIL} \textit{ⁿɡo-}} / -{\setmainfont{Charis SIL} \textit{hiⁿd}} / {\setmainfont{Charis SIL} \textit{ni}}- \\
%       Huichol & -{\setmainfont{Charis SIL} \textit{ka}} / -{\setmainfont{Charis SIL} \textit{mawe}} \\
%       Seri  & -{\setmainfont{Charis SIL} \textit{m}} \\
%       Tlahuica & {\setmainfont{Charis SIL} \textit{tét}}- / {\setmainfont{Charis SIL} \textit{té}}- / {\setmainfont{Charis SIL} \textit{nó}}- \\
%       Tlapaneco & {\setmainfont{Charis SIL} \textit{ta¹ga³}}- \\
%       \end{tabular}%
%       \caption{Lenguas que condifcan la negación en afijos}
%     \label{tab:addlabel}%
%   \end{table}%
  


\printbibliography

\end{document}
