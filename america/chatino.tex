\section*{Chatino de Zacatepec}

\noindent El Chatino es una lengua mexicana que pertenece a la familia lingüística oto-mangue. Se clasifica dentro de la subfamilia Zapotecana y tiene diversas variedades dialectales que difieren en ciertos aspectos, como la pronunciación, el vocabulario y la gramática. El Chatino de Zacatepec se habla en la pequeña comunidad de San Marcos Zacatepec en el estado de Oaxaca (Villard, 2009). \vspace{1cm}

{\setmainfont{Doulos SIL}
    % Ejemplo 8
    \begin{tabular}{rllll}
        \multicolumn{1}{l}{(8)} & \textbf{a3}                                        & lakan3              & natan3 & ndolo7o3  \\
                                & \textsc{neg}                                       & hab.ser.\textsc{1s} & gente  & hab.saber \\
                                & \multicolumn{4}{l}{“No soy un profesor” (pág. 91)}                                            \\
    \end{tabular}
    \vspace{0.5cm}

    % Ejemplo 9
    \begin{tabular}{rll}
        \multicolumn{1}{l}{(9)} & a3                                             & tilyon30          \\
                                & \textsc{neg}                                   & gordo.\textsc{1s} \\
                                & \multicolumn{2}{l}{“No estoy gordo” (pág. 91)}                     \\
    \end{tabular}
    \vspace{0.5cm}

    % Ejemplo 10
    \begin{tabular}{rlllll}
        \multicolumn{1}{l}{(10)} & kwa31                                                & a3           & ki7ya3            & 7a21         & kyo3   \\
                                 & ya                                                   & \textsc{neg} & \textsc{pot}.caer & \textsc{enf} & lluvia \\
                                 & \multicolumn{5}{l}{“Ya, no lloverá mucho” (pág. 91)}                                                            \\
    \end{tabular}
} \vspace{1cm}

Villard (2009) describe esta lengua como variedad muy conservadora del chatino ya que conserva las sílabas penúltimas de las raíces bisilábicas, a diferencia de la mayoría de las otras variedades. Presenta un inventario tonal inusualmente grande de 8 categorías tonales y un sistema aspectual verbal muy complejo, intercambiabilidad de rasgos gramaticales y una fonología con sonidos poco comunes tipológicamente. El orden básico de palabras es VSO, pero se encuentran otros órdenes con motivaciones pragmáticas específicas.

La negación en el chatino de Zacatepec se expresa a través del elemento libre a3, que siempre precede al predicado que niega. La forma \textit{a3} puede encontrarse al inicio o en medio de la oración. Puede negar predicados verbales (8) o no verbales (9). Se aplica a eventos que han ocurrido (aspecto completivo), que ocurren habitualmente (aspecto habitual), o que ocurrirán en el futuro (aspecto potencial) (10).

No parece haber otras partículas de negación mencionadas en la gramática además de la forma \textit{a3}.
