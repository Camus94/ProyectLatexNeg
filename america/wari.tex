\section*{Wari'}
\addcontentsline{toc}{section}{Wari'}

\noindent El wari', también conocido como pacaás novo, es una lengua de la familia chapacura que se habla en Rondônia, Brasil. Es una lengua tonal y cuenta con un complejo sistema de clasificadores nominales.
%%%%%%%%%%%%% Abreviaturas %%%%%%%%%%%%%%%%%%%%%
\footnote{COLL: colectivo, PF: perfectivo, N: neutro, REFL: reflexivo,RM:PAST: pasado remoto, SM:sujeto masculino, }
%%%%%%%%%%%%%%%%%%%%%%%%%%%%%%%%%%%%%%%%%%%%%%%%
\vspace{0.5cm}

{\setmainfont{Charis SIL} 

% ejemplo 189
\begin{tabular}{llllll}
(189) & ara & minp & \textbf{mao} & hwe-in & naran \\
& hacer & forzar & \textsc{\textbf{neg}} & \textsc{2p-3n} & luz \\
& \multicolumn{5}{l}{``Sube un poco la luz (lit. ‘No estás forzando la luz)'' (pág. 36)}
\end{tabular} \vspace{0.5cm}

% Ejemplo 190
\begin{tabular}{llllll}
(190) & mana' & \textbf{mao} & xequequem & 'oro narima' \\
& enojarse & \textsc{\textbf{neg}} & \textsc{refl:3pf} & \textsc{coll} & mujer \\
& \multicolumn{5}{l}{``Las mujeres no estaban enojadas entre ellas'' (pág. 38)}
\end{tabular} \vspace{0.5cm}

% Ejemplo 191
\begin{tabular}{lllll}
(191) & querec & \textbf{'a} & tocwa & wari' \\
& ver & \textsc{\textbf{neg}:s} & \textsc{pass:3sm} & persona \\
& \multicolumn{4}{l}{``La persona no fue vista'' (pág. 37)}
\end{tabular} \vspace{0.5cm}

% Ejemplo 192
\begin{tabular}{lllll}
(192) & 'awi & \textbf{'ara} & ca & pije' \\
& bueno & \textsc{\textbf{neg}:p} & \textsc{3sm} & niño \\
& \multicolumn{4}{l}{``El niño no es bueno'' (pág. 37)}
\end{tabular} \vspace{0.5cm}

% Ejemplo 193
\begin{tabular}{lllllll}
(193) & mija & \textbf{'ara} & \textbf{mao} & ne & carawa & pane \\
& mucho & \textsc{\textbf{neg}:p} & \textsc{neg} & \textsc{3n} & animal & rem:past \\
& \multicolumn{6}{l}{``Había muchísima comida''}
\end{tabular} \vspace{0.5cm}

}

\textcolor{MidnightBlue}{\citet{wari}} reporta dos tipos de modificadores verbales negativos: {\setmainfont{Charis SIL} \textit{mao}} (189) y (190) y {\setmainfont{Charis SIL} \textit{á/ára}} (191) y (192). Ambos elementos se colocan después de la palabra que quieren negar. Un caso muy llamativo es que si coexisten dos de estas marcas en una costrucción se obtine una lectura fuertemente positiva (193)