\section*{Chinanteco de Lealao}

\noindent La lengua chinanteca pertenece a la familia lingüística otomange y es hablada y es hablada por el pueblo chinanteco en el Estado de Oaxaca, en México. Es una de las lengua más habladas en la región de la Sierra Norte de Oaxaca y se estima que cuenta con unos 90 000 hablantes. Algunas de las variantes más importantes son el chinanteco de de Valle Nacional, el chinanteco de Usila, el chinanteco de Quiotepec y el chinanteco de Sam Juan Lealao. Este útimo está ubicado en la parte noreste del estado de Oaxaca en el distrito de Choapan. \vspace{1cm}

{\setmainfont{Doulos SIL}
    % Ejemplo 12
    \begin{tabular}{lll}
        (12) & ʔa\textsuperscript{L}2ʔe\textsuperscript{M}     & maʔ\textsuperscript{L}-líʔ\textsuperscript{L}i \\
             & \textsc{neg}                                    & \textsc{trm}-recordar.\textsc{1s}              \\
             & \multicolumn{2}{l}{``No me acuerdo'' (pág. 32)}
    \end{tabular} \vspace{0.5cm}

    % Ejemplo 13
    \begin{tabular}{lllll}
        (13) & duʔ\textsuperscript{M}                                           & ʔa\textsuperscript{L}-ʔi\textsuperscript{L}-kuʔ\textsuperscript{M}i & miː\textsuperscript{L} & diáʔ\textsuperscript{L} \\
             & para                                                             & \textsc{neg}-\textsc{inten}-comer.\textsc{c}3                       & 3\textsc{refl}         & \textsc{pl}             \\
             & \multicolumn{4}{l}{``... para que no se los comiera'' (pág. 32)}
    \end{tabular} \vspace{0.5cm}

    % Ejemplo 14
    \begin{tabular}{lllllll}
        (14) &
    \end{tabular}
}