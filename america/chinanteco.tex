\section*{Chinanteco de Lealao}

\noindent La lengua chinanteca pertenece a la familia lingüística otomange y es hablada y es hablada por el pueblo chinanteco en el Estado de Oaxaca, en México. Es una de las lengua más habladas en la región de la Sierra Norte de Oaxaca y se estima que cuenta con unos 90 000 hablantes. Algunas de las variantes más importantes son el chinanteco de de Valle Nacional, el chinanteco de Usila, el chinanteco de Quiotepec y el chinanteco de Sam Juan Lealao. Este útimo está ubicado en la parte noreste del estado de Oaxaca en el distrito de Choapan. \vspace{1cm}

{\setmainfont{Doulos SIL}
    % Ejemplo 12
    \begin{tabular}{lll}
        (12) & ʔaLʔeM                                        & maʔL-líʔLi                        \\
             & \textsc{neg}                                  & \textsc{trm}-recordar.\textsc{1s} \\
             & \multicolumn{2}{l}{"No me acuerdo" (pág. 32)}
    \end{tabular}

}