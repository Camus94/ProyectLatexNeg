\section*{Chinanteco de Lealao}

\noindent La lengua chinanteca pertenece a la familia lingüística otomange y es hablada y es hablada por el pueblo chinanteco en el Estado de Oaxaca, en México. Es una de las lengua más habladas en la región de la Sierra Norte de Oaxaca y se estima que cuenta con unos 90 000 hablantes. Algunas de las variantes más importantes son el chinanteco de de Valle Nacional, el chinanteco de Usila, el chinanteco de Quiotepec y el chinanteco de Sam Juan Lealao. Este útimo está ubicado en la parte noreste del estado de Oaxaca en el distrito de Choapan. \vspace{1cm}

{\setmainfont{Doulos SIL}
    % Ejemplo 12
    \begin{tabular}{lll}
        (12) & ʔa\textsuperscript{L}2ʔe\textsuperscript{M}     & maʔ\textsuperscript{L}-líʔ\textsuperscript{L}i \\
             & \textsc{neg}                                    & \textsc{trm}-recordar.\textsc{1s}              \\
             & \multicolumn{2}{l}{``No me acuerdo'' (pág. 32)}
    \end{tabular} \vspace{0.5cm}

    % Ejemplo 13
    \begin{tabular}{lllll}
        (13) & duʔ\textsuperscript{M}                                           & ʔa\textsuperscript{L}-ʔi\textsuperscript{L}-kuʔ\textsuperscript{M}i & miː\textsuperscript{L} & diáʔ\textsuperscript{L} \\
             & para                                                             & \textsc{neg}-\textsc{inten}-comer.\textsc{c}3                       & 3\textsc{refl}         & \textsc{pl}             \\
             & \multicolumn{4}{l}{``... para que no se los comiera'' (pág. 32)}
    \end{tabular} \vspace{0.5cm}

    % Ejemplo 14
    \begin{tabular}{lllllll}
        (14) & ʔa\textsuperscript{L}-ʔi̧\textsuperscript{M} & zïː\textsuperscript{M} & nï\textsuperscript{M} & zïː\textsuperscript{M}nuː\textsuperscript{M} & ba\textsuperscript{H} & nï\textsuperscript{M} \\
        & \textsc{neg-deit}:ese (animado) & perro & \textsc{pausa} & coyote & \textsc{af} & \textsc{pausa} \\
        & \multicolumn{6}{l}{``Ese no es un perro, es un coyote'' (pág. 32)}
    \end{tabular} \vspace*{0.5cm}
        
%	Ejemplo 15
\begin{tabular}{llllllll}
	(15) & ʔa\textsuperscript{L}-ha̧\textsuperscript{M}	& ʔi\textsuperscript{M} & ké\textsuperscript{L} & ʔi\textsuperscript{M} & kiú̧ʔ\textsuperscript{H}u\textsuperscript{M} & ba\textsuperscript{H} & nï\textsuperscript{M} \\
	& \textsc{neg.deit}:ese (inanimado) & \textsc{rel} & mío & \textsc{rel} & de.2s & af & \textsc{af} \\
	& \multicolumn{7}{l}{``Ese no es mío, es tuyo más bien'' (pág. 33)}	
\end{tabular} \vspace{1cm}
}

Existen dos estrategias principales para expresar negación: 1) el prefijo {\setmainfont{Doulos SIL}ʔaL-}, y 2) la palabra negativa {\setmainfont{Doulos SIL}ʔaLʔeM}. La distribución de estas estrategias depende del tipo de negación y construcción: La palabra negativa funciona como predicado intransitivo negando la oración. Se «usa cuando la negación es completa o cierta» (Rupp, 1989, p. 32). El prefijo «se reserva para situaciones negativas menos ciertas o no cumplidas», como en el intentivo. Con los deícticos animado e inanimado, el prefijo niega la identidad de un referente nominal. En oraciones relativas se utiliza el prefijo en lugar de la palabra negativa.