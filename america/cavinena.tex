\section*{Cavineña}

\noindent El cavineño es una lengua indígena hablada en el norte de Bolivia en la región amazónica. Pertenece a la familia tacana y es una lengua en peligro de extinción.
%%%%%%%%%%%%% Abreviaturas %%%%%%%%%%%%%%%%%%%%%
\footnote{AFFTN: afección, DAT: dativo, DESID: desiderativo, ERG: ergativo, FM: formativo, GEN: genitivo, IMPFV: imperfectivo, NPF: prefijo nominal, PERF: perfecto, PERL: perlativo, POT: potencial, QUEST: marcador de pregunta, REC.PAST: pasado reciente, REM.PAST: pasado remoto,}
%%%%%%%%%%%%%%%%%%%%%%%%%%%%%%%%%%%%%%%%%%%%%%%%
\vspace{0.5cm}

{\setmainfont{Charis SIL} 

% Ejemplo 134 predicado verbal
\begin{tabular}{lllll}
(134) & E-ra\textsubscript{A}=tu\textsubscript{O} & [e-kwe & tata-chi]\textsubscript{O} & adeba-ya=\textbf{ama} \\
& \textsc{1sg-erg=3sg(-fm)} & \textsc{1sg-gen} & padre-\textsc{afftn} & conocer-\textsc{impfv=\textbf{neg}} \\
& \multicolumn{4}{l}{``No conozco a mi padre'' (pág. 79)}
\end{tabular} \vspace{0.5cm}

% Ejemplo 135 predicado en pregunta
\begin{tabular}{llll}
(135) & Are=mi\textsubscript{O} & bakwa=ra\textsubscript{A} & a-wa=\textbf{ama}? \\
& \textsc{quest=2sg(-fm)} & vívora=\textsc{erg} & afectar-\textsc{perf=\textbf{neg}} \\
& \multicolumn{3}{l}{``¿No es una víbora la que te mordió?'' (pág. 102)}
\end{tabular} \vspace{0.5cm}

% Ejemplo 136 adjetivo
\begin{tabular}{llll}
(136) & E-na\textsubscript{S}=e-kwe & tupu=\textbf{ama}\textsubscript{CC} & ju-kware \\
& \textsc{npf-}agua=\textsc{1sg-dat} & suficiente=\textsc{\textbf{neg}} & ser-\textsc{rem.past} \\
& \multicolumn{3}{l}{``Me quedé sin agua (lit. el agua no era suficiente para mí)'' (pág. 103)}
\end{tabular} \vspace{0.5cm}

% Ejemplo 137 núcleo de una frases nominal
\begin{tabular}{llll}
(137) & ... =tuna-ja=tu\textsubscript{O} & dutya=\textbf{ama}\textsubscript{O} & nudya-kware \\
& =\textsc{3pl-dat=3sg(-fm)} & todos=\textsc{\textbf{neg}} & hacer.entrar-\textsc{rem.past} \\
& \multicolumn{3}{l}{``(Estaban tan enojados que) no los dejaron entrar a todos'' (pág. 103)}
\end{tabular} \vspace{0.5cm}

% Ejemplo 138 de una postposición
\begin{tabular}{llll}
(138) & Iyakwa=mikwana\textsubscript{S} & e-wasi=eke=\textbf{ama} & diru-ya \\
& ahora=\textsc{2pl} & \textsc{npf-}pie=\textsc{perl=\textbf{neg}} & ir-\textsc{impfv} \\
& \multicolumn{3}{l}{``Ahora ustedes no irán a pie'' (pág. 103)}
\end{tabular} \vspace{0.5cm}

% Ejemplo 139 desiderativo sufijo
\begin{tabular}{lllll}
(139) & Jadya=tibu & i-ke\textsubscript{S} & kwa-\textbf{karama} & ju-chine \\
& tal=razón & \textsc{1sg-fm} & ir-\textsc{desid.\textbf{neg}} & ser-\textsc{rec.past}\\
& \multicolumn{4}{l}{``Por eso (porque está demasiado lejos), no quiero ir'' (pág. 323)}
\end{tabular} \vspace{0.5cm}

% Ejemplo 140 afijo adjetivos
\begin{tabular}{lll}
(140) & Ji-\textbf{dama}=dya\textsubscript{CC}=tu\textsubscript{CS} & e-ju-u \\
& bueno-\textsc{\textbf{neg}=foc=3sg(-fm)} & \textsc{pot-}ser-\textsc{pot} \\
& \multicolumn{2}{l}{``(un colador hecho a mano) podría estar defectuoso'' (pág. 95)}
\end{tabular} \vspace{0.5cm}

% Ejemplo 141 Elemento independiente
\begin{tabular}{llll}
(141) & \textbf{Aama\textsubscript{CC}}=tu\textsubscript{CS} & ju-kware & salon=kwana\textsubscript{CS}... \\
& \textbf{No.existe}=\textsc{3sg(-fm)} & ser-\textsc{rem.past} & rifle=\textsc{pl} \\
& \multicolumn{3}{l}{``(Cuando era joven) no había rifles (solo escopetas)} \\
& \multicolumn{3}{l}{(lit. los rifles no existían)'' (pág. 141)}
\end{tabular} \vspace{0.5cm}

% Ejemplo 142 aparentemente no
\begin{tabular}{llll}
(142) & \textbf{Jipakwana}=ekwana-ja & radio\textsubscript{S} & ani-ya \\
& \textbf{aparentemente.no}=\textsc{1pl-dat} & radio.ondacorta & quedar.bien-\textsc{impfv}\\
& \multicolumn{3}{l}{``Parece que no tendremos esa radio} \\
& \multicolumn{3}{l}{(lit. una radio de onda corta aparentemente no nos sentará bien)'' (pág. 104)}
\end{tabular} \vspace{0.5cm}

% Ejemplo 143 imperativos 
\begin{tabular}{lllll}
(143) & Mi-ke\textsubscript{S} & ani-kwe! Mi-ke\textsubscript{S} & je-\textbf{ume}! \\
& \textsc{2sg-fm} & sentarse-\textsc{imp.sg} & \textsc{2sg-fm} & venir-\textsc{imp.sg.\textbf{neg}} \\
& \multicolumn{4}{l}{``Tú quédate (lit. siéntate), no vengas'' (pág. 104 )}
\end{tabular} \vspace{0.5cm}

}

Cuenta con al menos 7 morfemas reconocidos para manifestar la negación \textcolor{MidnightBlue}{\citep{cavin}}. El primer morfema es el clítico {\setmainfont{Charis SIL} \textit{=ama}} que puede adherirse a tecnicamente cualquier constituyente. Puede negar el predicado verbal en construcciones declaraticas (134) o interrogaticas (135), un adjetivo (136), el núcleo de una frase nominal (137) e incluso una frase postposicional (138).

El sufijo verbal desiderativo {\setmainfont{Charis SIL} \textit{-karama}} (139). El sufijo {\setmainfont{Charis SIL} \textit{-dame}} para adjetivos (140). El elemento independiente {\setmainfont{Charis SIL} \textit{aama}} «no existe» (141). La particula de primera posición {\setmainfont{Charis SIL} \textit{jipakwana}} «aparentemente no» (142). Interjecciones negativas {\setmainfont{Charis SIL} \textit{aijama}} «no existe en absoluto», {\setmainfont{Charis SIL} \textit{juwaaba}} «el hablante no sabe» o {\setmainfont{Charis SIL} \textit{pajuani}} «el hablante no está de acuerdo». Finalmente, los afijos negativos para imperativos {\setmainfont{Charis SIL} -ume} o {\setmainfont{Charis SIL} \textit{ne- ... -ume}} (143).

Como puede notarse, esta lengua cuenta con un amplio repertorio de recursos para la marcación de la negación. Ocasionalmente también puede usarse clítico{\setmainfont{Charis SIL} \textit{ni=}} «ni siquiera» como un refuerzo negativo.