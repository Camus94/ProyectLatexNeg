\section*{Ngäbére}
\addcontentsline{toc}{section}{Ngäbére}

\noindent La lengua ngäbére o guaimí es una lengua de la familia chibchana hablada en Panamá y Costa Rica.
%%%%%%%%%%%%% Abreviaturas %%%%%%%%%%%%%%%%%%%%%
\footnote{CMPL: completivo, DAT: dativo, EXIST: existencial, NOM: nominativo, P.REC: pasado reciente, P.REM: pasado remoto, RFL: reflexivo, PRS: presente}
%%%%%%%%%%%%%%%%%%%%%%%%%%%%%%%%%%%%%%%%%%%%%%%%
\vspace{0.5cm}

{\setmainfont{Charis SIL} 

% Ejemplo 163
\begin{tabular}{llllllll}
(163) & kwi & ie & nu & kugwe & \textbf{ñaka} & ja & ru-ri \\
& gallina & \textsc{dat} & perro & sonido & \textsc{\textbf{neg}} & \textsc{rfl} & oir-\textsc{p.rec} \\
& \multicolumn{7}{l}{``La gallina no oyó los ladridos del perro'' (pág. 5)}
\end{tabular} \vspace{0.5cm}

% Ejemplo 164
\begin{tabular}{lllllll}
(164) & mä & \textbf{ñagare} & riga & ti & tä & ñe-re \\
& \textsc{2sg} & \textsc{\textbf{neg}} & ir-\textsc{irr} & \textsc{1sg} & esta.prs & decir-\textsc{prs} \\
& \multicolumn{6}{l}{``No se vaya, le estoy diciendo'' (pág. 5)}
\end{tabular} \vspace{0.5cm}

% Ejemplo 165
\begin{tabular}{llllll}
(165) & danguin & gwe & \textbf{ñukwä} & ngwian & bi-ni-na \\
& jefe & \textsc{nom} & \textsc{\textbf{neg}} & plata & dar-\textsc{p.rem-cmpl} \\
& \multicolumn{5}{l}{``El jefe ya no dio plata'' (pág. 5)}
\end{tabular} \vspace{0.5cm}

% Ejemplo 166
\begin{tabular}{llllll}
(166) & ti & \textbf{ñukwäre} & ngw-en-däri & jükrä & íe \\
& \textsc{1sg} & \textsc{\textbf{neg}.exist} & preguntar-\textsc{prs-}preguntar & todos & \textsc{dat} \\
& \multicolumn{5}{l}{``Yo no le he preguntado nada a ninguno'' (pág. 5)}
\end{tabular} \vspace{0.5cm}

% Ejemplo 167
\begin{tabular}{llllllll}
(167) & mä & \textbf{ña} & ti & kämig-a & jire & kucho & biti \\
& \textsc{2sg} & \textsc{\textbf{neg}} & \textsc{1sg} & matar-\textsc{irr} & siempre & cuchillo & con \\
& \multicolumn{7}{l}{``No me maté para siempre con el cuchillo'' (pág. 5)}
\end{tabular} \vspace{0.5cm}

% Ejemplo 168
\begin{tabular}{llllll}
(168) & ¿nire & kra & köga-ni? & \textbf{ñukwä} & kü-rü \\
& quién & chacra & comprar-\textsc{p.rem} & \textsc{\textbf{neg}} & comprar-\textsc{p.rec3} \\
& \multicolumn{5}{l}{``¿Quién compró la chacra? Nadie la compró'' (pág. 5)}
\end{tabular} \vspace{0.5cm}

}

\textcolor{MidnightBlue}{\citet{ngabere}} reporta un único marcador negativo para esta lengua: {\setmainfont{Charis SIL} \textit{ñaka}} (163)y sus alomorfos {\setmainfont{Charis SIL} \textit{ñagare}} (164), {\setmainfont{Charis SIL} \textit{ñukwä}} (165), {\setmainfont{Charis SIL} \textit{ñukwäre}} (166) y {\setmainfont{Charis SIL} \textit{ña(n)}} (167). La distinción entre una forma u otra dependen de características como el énfasis, si se trata de negación existencial o, como en el caso de {\setmainfont{Charis SIL} \textit{ñukwä}}, si puede funcionar o no como pronombre indefinido negativo (168).