\section*{Maricopa}
\addcontentsline{toc}{section}{Maricopa}

\noindent El maricopa es una lengua de la familia hokan de la rama yuman. Se habla en Norte América específicamente en la región de Arizona, Estados Unidos.
%%%%%%%%%%%%% Abreviaturas %%%%%%%%%%%%%%%%%%%%%
\footnote{REAL: realis, SJ: sufijo de sujeto}
%%%%%%%%%%%%%%%%%%%%%%%%%%%%%%%%%%%%%%%%%%%%%%%%
\vspace{0.5cm}

{\setmainfont{Charis SIL} 

% Ejemplo 151
\begin{tabular}{llll}
(151) & chii-sh & hahan-ly & \textbf{aly}=dik-\textbf{ma}-k \\
& pez-\textsc{sj} & río-dentro & \textsc{\textbf{neg}}-encontrase-\textsc{\textbf{neg}-real}\\
& \multicolumn{3}{l}{``No hay peces en el río'' (pág. 72)}
\end{tabular} \vspace{0.5cm}

% Ejemplo 152
\begin{tabular}{ll}
(152) & \textbf{waly}='-tpuy-\textbf{ma}-k \\
& \textsc{\textbf{neg}}-yo-matar-\textsc{\textbf{neg}-real} \\
& ``No lo maté'' (pág. 72)
\end{tabular} \vspace{0.5cm}

% Ejemplo 153
\begin{tabular}{lll}
(153) & 'iipaa-sh & \textbf{waly}='-do-\textbf{ma}-k \\
& hombre-\textsc{sj} & \textsc{\textbf{neg}}-yo-ser-\textsc{\textbf{neg}-real} \\
& \multicolumn{2}{l}{``No soy un hombre'' (pág. 72)}
\end{tabular} \vspace{0.5cm}

% Ejemplo 154
\begin{tabular}{lll}
(154) & Heather-sh & \textbf{waly}=yuu-\textbf{m}-haay-k \\
& Heather-\textsc{sj} & \textsc{\textbf{neg}-}ver-\textsc{\textbf{neg}-}aún-\textsc{real} \\
& \multicolumn{2}{l}{``Heather aún no lo ha visto'' (pág. 72)}
\end{tabular} \vspace{0.5cm}

% Ejemplo 155
\begin{tabular}{lll}
(155) & \textbf{aly}='-iipaa-\textbf{ma}-sh & (duu-m) \\
& \textsc{\textbf{neg}}-hombre-\textsc{\textbf{neg}-sj} & ser-\textsc{real} \\
& \multicolumn{2}{l}{``Ella no es un hombre'' (pág. 73)}
\end{tabular} \vspace{0.5cm}

% Ejemplo 156
\begin{tabular}{ll}
(156) & m-ntay-\textbf{ma}-sh \\
& 2-madre-\textsc{\textbf{neg}-sj} \\
& ``Esa no es tu madre'' (pág. 73)
\end{tabular} \vspace{0.5cm}

}

Esta lengua necesita marcar el verbo de manera simultánea con un proclítico {\setmainfont{Charis SIL} \textit{waly=}} o {\setmainfont{Charis SIL} \textit{aly=}} y con el sufijo {\setmainfont{Charis SIL} \textit{-ma}} para indicar la negación (\textcolor{MidnightBlue}{\citep{maricopa}}) (151), (152) y (153). La vocal del sufijo se pierde ante la vocal de otro sufijo no final (154).

La negación de predicados nominales se puede hacer con las mismas estrategias (155), pero es común que el porclítico se omita (156).