\section*{Sabanê}
\addcontentsline{toc}{section}{Sabanê}

\noindent El sabanê es una lengua indígena brasileña hablada en el estado de Rondônia, Brasil. Pertenece a la familia lingüística Nambikuara. Es una lengua tonal con dos tonos de nivel: alto y bajo. Morfológicamente es una lengua aglutinante y de fusión que utiliza sufijos, prefijos e infijos para expresar varias categorías gramaticales.
%%%%%%%%%%%%% Abreviaturas %%%%%%%%%%%%%%%%%%%%%
\footnote{ASSR: asertivo, NEUT: neutral, OBJ: objeto, POSS: posesivo, PRES: presente, VS: sufijo verbal}
%%%%%%%%%%%%%%%%%%%%%%%%%%%%%%%%%%%%%%%%%%%%%%%%
\vspace{0.5cm}

{\setmainfont{Charis SIL} 

% Ejemplo 173
\begin{tabular}{ll}
(173) & ay–i–\textbf{mina}–tapanal–i \\
& ir-\textsc{vs-\textbf{neg}-fut-neut-assr} \\
& ``Ella/Él no va'' (pág. 133)
\end{tabular} \vspace{0.5cm}

% Ejemplo 174
\begin{tabular}{ll}
(174) & d–apipa-sowa–na–\textbf{mina}-al-i \\
& \textsc{1poss-mano-}estar.mojado-\textsc{vs-\textbf{neg}-pres.neut-assr} \\
& ``Mi mano no está mojada'' (pág. 134)
\end{tabular} \vspace{0.5cm}

% Ejemplo 175
\begin{tabular}{ll}
(175) & t–isun–i–\textbf{misina}-al-i \\
& \textsc{1obj-}estar.enojado-\textsc{vs-\textbf{neg}-pres.neut-assr} \\
& ``No estoy enojado'' (pág. 133)
\end{tabular} \vspace{0.5cm}

% Ejemplo 176
\begin{tabular}{ll}
(176) & d–apipa.ta-t–ip–i–\textbf{misina}-al-i\\
& \textsc{1poss-}pulgar-\textsc{1obj-}correr-\textsc{vs-\textbf{neg}-pres.neut-assr} \\
& ``Mi pulgar es corto'' (pág. 133)
\end{tabular} \vspace{0.5cm}

% Ejemplo 177
\begin{tabular}{ll}
(177) & taw–i–\textbf{mina} \\
& cortar-\textsc{vs-\textbf{neg}} \\
& ``No lo cortes'' (pág. 135)
\end{tabular} \vspace{0.5cm}

}

Para expresar negación en esta lengua, se utiliza el sufijo {\setmainfont{Charis SIL} \textit{-mi(si)na}} que al final del tema verbal \textcolor{MidnightBlue}{\citep{sabane}}. Lo más común es que este sufijo se realice fonéticamente como -{\setmainfont{Charis SIL} \textit{mina}} (173) y (174), pese a que en algunas ocasiones, de manera imprevisible, ocurre su forma larga {\setmainfont{Charis SIL} \textit{-misina}} (175) y (176). Otra particularidad es que cuando la partícula negativa se añade a un morfema que inicia con vocal, se produce un fenómeno de eliminación de la vocal final de dicha partícula, en el caso de que la vocal sea la misma, se fusionan como en los ejemplos anteriores. Para la negación de imperativos el sufijo se coloca igualmente en el tema verbal (177).