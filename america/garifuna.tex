\section*{Garífuna}
\addcontentsline{toc}{section}{Garífuna}

\noindent El garífuna, una lengua arahuaca, se habla en la costa Caribe de Honduras, Guatemala, Belice y Nicaragua por el pueblo garífuna.
%%%%%%%%%%%%% Abreviaturas %%%%%%%%%%%%%%%%%%%%%
\footnote{AOR: aoristo, F: femenino, NOM: nominalizador, O: construcción objetiva, ORGN: origen, PROG: progresivo}
%%%%%%%%%%%%%%%%%%%%%%%%%%%%%%%%%%%%%%%%%%%%%%%%
\vspace{0.4cm} 

{\setmainfont{Charis SIL} 

% Ejemplo 154
\begin{tabular}{ll}
(154) & \textbf{m}-arih-in-ti-na \\
& \textsc{\textbf{neg}}-ver-\textsc{nom-aor-1sg}\\
& ``No veo'' (pág. 141)
\end{tabular} \vspace{0.2cm}

% Ejemplo 155
\begin{tabular}{lll}
(155) & \textbf{m}-arih-in & n-umuti-bu \\
& \textsc{\textbf{neg}}-ver-\textsc{nom} & \textsc{1sg-aor.o-2sg} \\
& \multicolumn{2}{l}{``No te veo'' (pág. 141)}
\end{tabular} \vspace{0.2cm}

% Ejemplo 156
\begin{tabular}{ll}
(156) & \textbf{m}-arumuga-ba \\
& \textsc{\textbf{neg}}-dormir-\textsc{imp} \\
& ``No duerma'' (pág. 146)
\end{tabular} \vspace{0.2cm}

% Ejemplo 157
\begin{tabular}{llll}
(157) & \textbf{mama} & indura-na & nuguya \\
& \textsc{\textbf{neg}} & Honduras-\textsc{orgn} & \textsc{1sg.f} \\
& \multicolumn{3}{l}{``No soy hondureña'' (pág. 143)}
\end{tabular} \vspace{0.2cm}

% Ejemplo 158
\begin{tabular}{lll}
(158) & \textbf{mama} & n-arumug \\
& \textsc{\textbf{neg}} & \textsc{1sg}-dormir-\textsc{nom} \\
& \multicolumn{2}{l}{``No voy a dormir'' (pág. 145)}
\end{tabular} \vspace{0.2cm}

% Ejemplo 159
\begin{tabular}{lll}
(129) & \textbf{mama} & t-arihu-ña-dina \\
& \textsc{\textbf{neg}} & \textsc{2sg.f-}ver-\textsc{prog-1sg} \\
& \multicolumn{2}{l}{``Ella no me está viendo'' (pág. 146)}
\end{tabular} \vspace{0.3cm}

}

La negación se manifiesta de forma sintética para las cláusulas, mientras que adopta una forma analítica para los constituyentes cuando se niegan \textcolor{MidnightBlue}{\citep{garifuna}}. La primera forma utiliza el prefijo {\setmainfont{Charis SIL} \textit{m(a)-}} adherido al lexema verbal, funciona en formas simples (154), compuestas (155) e imperativos (156). La segunda forma de negación recurre a la palabra independiente {\setmainfont{Charis SIL} \textit{mama}} (157).

La negación clausal en esta lengua está ligada a verbos con marca aspectual, si esta no aparece, el uso del prefijo resulta en una construcción agramatical. Cuando la marca de aspecto no está presente, se recurre a la negación con {\setmainfont{Charis SIL} \textit{mama}} (158). La excepción a esto corresponde al aspecto progresivo, ya que estas construcciones siempre se negarán con {\setmainfont{Charis SIL} \textit{mama}} (159).