\section*{Wappo}
\addcontentsline{toc}{section}{Wappo}

\noindent El wappo es una lengua indígena de Norteamérica que fue hablada tradicionalmente en el centro-norte de California. Actualmente está clasificada como una lengua aislada.
%%%%%%%%%%%%% Abreviaturas %%%%%%%%%%%%%%%%%%%%%
\footnote{CAUS: causativo, DEM: demostrativo, DUR: durativo, NOM: nominativo, PST: pasado, STAT: estativo}
%%%%%%%%%%%%%%%%%%%%%%%%%%%%%%%%%%%%%%%%%%%%%%%%
\vspace{0.5cm}

{\setmainfont{Charis SIL} 

% Ejemplo 183
\begin{tabular}{lll}
(183) & ah & chac-sě-\textbf{lahkhiʔ} \\
& \textsc{1sg:nom} & frío-\textsc{dur-\textbf{neg}} \\
& \multicolumn{2}{l}{``No tengo frío'' (pág 79)}
\end{tabular} \vspace{0.5cm}

% Ejemplo 184
\begin{tabular}{llll}
(184) & ce & k'ew-i & tuč-kh-\textbf{lahkhiʔ} \\
& \textsc{dem} & hombre-\textsc{nom} & grande-\textsc{stat-\textbf{neg}} \\
& \multicolumn{3}{l}{``Aquél hombre no es grande'' (pág. 79)}
\end{tabular} \vspace{0.5cm}

% Ejemplo 185
\begin{tabular}{llll}
(185) & ah & may & naw-ta-\textbf{lahkhiʔ} \\
& \textsc{1sg:nom} & quién & ver-\textsc{pst-\textbf{neg}} \\
& \multicolumn{3}{l}{``No vi a nadie'' (pág. 79)}
\end{tabular} \vspace{0.5cm}

% Ejemplo 186
\begin{tabular}{llllll}
(186) & ce & hol & ah & te & k'eč-is-ta-\textbf{lahkhiʔ} \\
& \textsc{dem} & madera & \textsc{1sg:nom} & \textsc{3sg} & cortar-\textsc{caus-pst-\textbf{neg}} \\
& \multicolumn{5}{l}{``La madera, yo no hice que él la cortara'' (pág. 4)}
\end{tabular} \vspace{0.5cm}

% Ejemplo 187
\begin{tabular}{ll}
(187) & cay'i-\textbf{lahkhiʔ} \\
& decir:\textsc{imp-\textbf{neg}} \\
& ``No digas (eso)'' (pág. 78)
\end{tabular} \vspace{0.5cm}

% Ejemplo 188
\begin{tabular}{ll}
(188) & čutehel-\textbf{lahkhiʔ} \\
& olvidar:\textsc{imp-\textbf{neg}}\\
& ``No lo olvides'' (pág. 78)
\end{tabular} \vspace{0.5cm}

}

Esta lengua tiene morfología verbal específica para para cada valor de tiempo, aspecto y modo, incluida la negación \textcolor{MidnightBlue}{\citep{wappo}}. El sufijo {\setmainfont{Charis SIL} \textit{-lahkhiʔ}} es el encargo de marcar la negación tanto en oraciones declarativas (183) - (186) como imperativas (187) y (188).