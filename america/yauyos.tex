\section*{Yauyos Quechua}
\addcontentsline{toc}{section}{Yauyos Quechua}

\noindent El yauyos es una variedad del quechua hablada en la provincia de Yauyos, Lima, en Perú.
%%%%%%%%%%%%% Abreviaturas %%%%%%%%%%%%%%%%%%%%%
\footnote{ADD: aditivo, ACC: acusativo CISL: cislocativo, DEM: demostrativo, EVC: evidencial conjetural, EVD: evidencial, INCEP: inceptivo, OBJ: objeto, PROH: prohibitivo, PST: pasado, SUBIS: subordinador sujeto idéntico, TOP: tópico}
%%%%%%%%%%%%%%%%%%%%%%%%%%%%%%%%%%%%%%%%%%%%%%%%
\vspace{0.1cm}

{\setmainfont{Charis SIL}

% Ejemplo 194
\noindent \begin{tabular}{llll}
(194) & chay-tri & mana & suya-wa-rqa=\textbf{chu} \\
& \textsc{dem-evc} & \textsc{\textbf{neg}} & esperar-\textsc{1.obj-pst=\textbf{neg}} \\
& \multicolumn{3}{l}{``Por eso ella no me ha esperado'' (pág. 289)}
\end{tabular} \vspace{0.1cm}

% Ejemplo 195
\noindent \begin{tabular}{lll}
(195) & kaspi-n-pis & ka-n=\textbf{chu} \\
& palo-\textsc{3-add} & ser-\textsc{3=\textbf{neg}}\\
& \multicolumn{2}{l}{``Ella no tiene un palo'' (pág. 289)}
\end{tabular} \vspace{0.1cm}

% Ejemplo 196
\noindent \begin{tabular}{lllll}
(196) & \textbf{ama} & mancha-ri-y=\textbf{chu} & \textbf{ama} & qawa-y=\textbf{chu} \\
& \textsc{proh} & asustarse-\textsc{incep-imp=\textbf{neg}} & \textsc{proh} & mirar-\textsc{imp=\textbf{neg}} \\
& \multicolumn{4}{l}{``¡No te asustes! ¡No mires!'' (pág. 290)}
\end{tabular} \vspace{0.1cm}

% Ejemplo 197
\noindent \begin{tabular}{llllll}
(197) & \textbf{ama} & kuti-mu-nki=\textbf{chu} & qam-qa & isturbu-m & ka-ya-nki \\
& \textsc{proh} & regresar-\textsc{cisl-2=\textbf{neg}} & tú-\textsc{top} & molestia-\textsc{evd} & ser-\textsc{prog-2} \\
& \multicolumn{5}{l}{``¡No vuelvas! ¡Eres un estorbo!'' (pág. 290)}
\end{tabular} \vspace{0.1cm}

% Ejemplo 198
\noindent \begin{tabular}{lll}
(198) & \textbf{ama} & wañu-chun=\textbf{chu} \\
& \textsc{proh} & morir-\textsc{injunc=\textbf{neg}} \\
& \multicolumn{2}{l}{``No la dejes morir'' (pág. 290)}
\end{tabular} \vspace{0.1cm}

{\small
% Ejemplo 199
\noindent \begin{tabular}{llllll}
(199) & \textbf{mana} & qali & ka-pti-n-qa & ñuqanchik-pis & taqlla-kta \\
& \textsc{\textbf{neg}} & hombre & ser-\textsc{subds-3-top} & nosotros-\textsc{add} & arado-\textsc{acc} \\
& hapi-shpa & qaluwa-nchik \\
& agarrar-\textsc{subis} & remover.tierra-\textsc{1pl} \\
& \multicolumn{5}{l}{``Cuando no hay hombres, tomamos el arado y removemos la tierra'' (pág. 291)}
\end{tabular} \vspace{0.2cm}}
}

La negación se indica mediante el clítico {\setmainfont{Charis SIL} \textit{=chu}} en combinación con las partículas negativas {\setmainfont{Charis SIL} \textit{mana}}, {\setmainfont{Charis SIL} \textit{ama}} o {\setmainfont{Charis SIL} \textit{ni}}, o junto con el sufijo {\setmainfont{Charis SIL} \textit{–pis}} (marca de aditivo ADD) \textcolor{MidnightBlue}{\citep{yauyos}}. El clítico se une al fragmento de la oración que es foco de la negación.

En (194) tenemos dos marcas con valor negativo, lo cual parece indicar que esta es una lengua de doble negación. Pero en (195) no queda muy claro si esto se cumple, pues si bien hay dos marcas, el sufijo {\setmainfont{Charis SIL} \textit{}} parece ser más una forma de licencia para que la negación pueda aparecer. El clítico {\setmainfont{Charis SIL} \textit{=chu}} aparece con {\setmainfont{Charis SIL} \textit{ama}} en prohibiciones (196), en imperativos (197), así como en admonitivos (198). El clítico {\setmainfont{Charis SIL} \textit{=chu}} no aparece en cláusulas subordinadas. En estas, la negación se indica únicamente con la partícula {\setmainfont{Charis SIL} \textit{mana}} (199).