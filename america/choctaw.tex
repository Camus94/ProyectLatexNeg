\section*{Choctaw}

\noindent Lengua de la familia maskogui o muskogi hablada en el sudoeste de Estado Unidos en Oklahoma, Lusiana y Tennessee.
%%%%%%%%%%%%% Abreviaturas %%%%%%%%%%%%%%%%%%%%%
\footnote{DPAST: pasado distante, DS: sujeto diferente, NM: nominativo, PART: participio, PT: past, SS: mismo sujeto, TNS: tiempo predeterminado}
%%%%%%%%%%%%%%%%%%%%%%%%%%%%%%%%%%%%%%%%%%%%%%%%
\vspace{0.5cm}

{\setmainfont{Charis SIL} 

% Ejemplo 149
\begin{tabular}{llll}
(149) & John-at & shókha' & abi-\textbf{kiiyo}-tok \\
& John-\textsc{nm} & cerdo & martar-\textsc{\textbf{neg}-pt} \\
& \multicolumn{3}{l}{``John no mató al cerdo'' (pág. 322)}
\end{tabular} \vspace{0.5cm}

% Ejemplo 150
\begin{tabular}{llll}
(150) & John-at & shókha' & ab-aachi-\textbf{kiiyo}-h \\
& John-\textsc{nm} & cerdo & matar-\textsc{irr-\textbf{neg}-tns} \\
& \multicolumn{3}{l}{``John no va a matar al cerdo'' (pag. 322)}
\end{tabular} \vspace{0.5cm}

% Ejemplo 151
\begin{tabular}{llll}
(151) & John-at & shókha' & abi-tok-\textbf{kiiyo}(-h) \\
& John-\textsc{nm} & cerdo & matar-\textsc{pt-\textbf{neg}(-tns)} \\
& \multicolumn{3}{l}{``John no mató al cerdo'' (pág. 322)}
\end{tabular} \vspace{0.5cm}

% Ejemplo 152
\begin{tabular}{lllll}
(152) & ibbak & achiif-ahii-yo & \textbf{kiiyo}-kmat, & ipa-\textbf{kiiyo}-biika-tok \\
& mano & lavar-\textsc{irr-part:ds} & \textsc{\textbf{neg}-irr:ss} & comer-\textsc{\textbf{neg}-}extent-\textsc{pt} \\
& \multicolumn{4}{l}{``Si no se han lavado las manos no comen'' (pág. 323)}
\end{tabular} \vspace{0.5cm}

% Ejemplo 153
\begin{tabular}{lll}
(153) & nán=asháchchi-ttook-o & \textbf{kiiyo}-h-okii \\
& pecar-\textsc{dpast-part:ds} & \textsc{\textbf{neg}-tns-}en.realidad \\
& \multicolumn{2}{l}{``Ellos no pecaron'' (pág. 323)}
\end{tabular} \vspace{0.5cm}

}

La negación en esta lengua es parte de la morfología verbal por medio del sufijo {\setmainfont{Charis SIL} \textit{-kiiyo}} \textcolor{MidnightBlue}{\citep{choctaw}}. Este puede aparecer en posición anterior al tiempo verbal (149) y (150), como en posición posterior (151) pero nunca antecediendo al modo.

En posición posterior al tiempo verbal, este sufijo muestra algunas propiedades de una palabra idenpendiente y el verbo que lo precede suele tomar la marca participio (152) y (153).