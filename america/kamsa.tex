\section*{Kamsá}
\addcontentsline{toc}{section}{Kamsá}

\noindent Kamsá, Camsá o Sibundoy, es una lengua indígena hablada en el sur de Colombia. Ya que no se ha podido determinar parentesco alguno con otras lenguas de la región se le considera un lengua aislada.
%%%%%%%%%%%%% Abreviaturas %%%%%%%%%%%%%%%%%%%%%
\footnote{CLF: clasificador, DIS: (pasado) distante, EVAL: evaluativo, PROG: progresivo, REC: TOP: topico, VBLZ: verbalizador}
%%%%%%%%%%%%%%%%%%%%%%%%%%%%%%%%%%%%%%%%%%%%%%%%
\vspace{0.5cm}

{\setmainfont{Charis SIL}

% Ejemplo 140
\begin{tabular}{lllll}
(140) & ats̈=na & [\textbf{ndoñ} & chka & ke-ts-\textbf{at}-aman] \\
& \textsc{1sg=top} & \textsc{\textbf{neg}} & así & \textsc{irr-prog-\textbf{neg}-}ser \\
& \multicolumn{4}{l}{``Yo no soy así'' (pág. 132)}
\end{tabular} \vspace{0.5cm}

% Ejemplo 141
\begin{tabular}{lll}
(141) & \textbf{ndoñ} & ke-s̈-\textbf{at}-amënts̈na \\
& \textsc{\textbf{neg}} & \textsc{irr-1sg-\textbf{neg}-}estar.cansado \\
& \multicolumn{2}{l}{``No estoy cansado'' (pág. 132)}
\end{tabular} \vspace{0.5cm}

% Ejemplo 142
\begin{tabular}{lllll}
(142) & aka-jem & ko-ch-\textbf{at}-oben & ats̈ & j-tsakmenán \\
& \textsc{2sg-eval} & \textsc{2sg-fut-\textbf{neg}-}ser.posible & \textsc{1sg} & \textsc{vblz-}seguir \\
& \multicolumn{4}{l}{``Tú no puedes seguirme/alcanzarme'' (pág. 133)}
\end{tabular} \vspace{0.5cm}

% Ejemplo 143
\begin{tabular}{llll}
(143) & \textbf{ndoñ} & tsës̈ey-be & ye-\textbf{n}-j-oshma \\
& \textsc{\textbf{neg}} & amarillo-\textsc{clf} & \textsc{3sg.dis-\textbf{neg}-vblz-}poner.huevos \\
& \multicolumn{3}{l}{``No puso huevos amarillos'' (pág. 134)}
\end{tabular} \vspace{0.5cm}

% Ejemplo 144
\begin{tabular}{llllll}
(144) & lo.mismo & \textbf{ndoñe} & benache & \textbf{ndoñe} & i-mu-\textbf{nd}-en-abwache \\
& lo.mismo & \textsc{\textbf{neg}} & rastro & \textsc{\textbf{neg}} & \textsc{dis-3pl-\textbf{neg}-rec-} visitar \\
& \multicolumn{5}{l}{``Como no había rastro, no visitaron ahí'' (pág. 133)}
\end{tabular} \vspace{0.5cm}

}

La lengua Kamsá tiene una forma particular de expresar la negación en los verbos. Utiliza la palabra {\setmainfont{Charis SIL} \textit{ndoñ(e)}} antepuesta al verbo para negarlo. Regularmente, cuando un verbo se niega, también lleva dos prefijos: {\setmainfont{Charis SIL} \textit{ke-}}, que denota irrealidad, y {\setmainfont{Charis SIL} \textit{at-}}, que es el prefijo negativo como tal (140) y (141).  

Aunque estos dos prefijos son comunes, no son obligatorios para la negación, pudiéndose utilizar únicamente {\setmainfont{Charis SIL} \textit{ndoñ(e)}} antepuesta \textcolor{MidnightBlue}{\citep{kamsa}} o incluso solo el prefijo (142).

Otra particularidad es que en ciertos tiempos verbales, como el pasado lejano, se emplean prefijos negativos distintos: {\setmainfont{Charis SIL} \textit{n-}} (143) o {\setmainfont{Charis SIL} \textit{nd-}} (144), que tienen en común con otros prefijos del Kamsá el ser homófonos, es decir, tener la misma pronunciación o fonética (por ejemplo, la partícula evidencial y el presente habitual).