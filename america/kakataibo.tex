\section*{Kakataibo}
\addcontentsline{toc}{section}{Kakataibo}

\noindent La lengua kakataibo es hablada en Sudamérica y pertenece a la familia de lenguas pano. Se encuentra en la amazonía oriental central de Perú.
Es conocida tradicionalmente como cashibo-cacataibo, pero los propios hablantes prefieren llamarla kakataibo.
%%%%%%%%%%%%% Abreviaturas %%%%%%%%%%%%%%%%%%%%%
\footnote{>: seguimiento de referencia (dependiente>principal), A: sujeto transitivo, ABS: absolutivo, COM(S): comitativo orientado al sujeto, COMPL.NEG: \textit{complaining negator}, GEN: genitivo, IMPF: imperfectivo, NAR: narrativo, NOM: nominalizador, O: objeto transitivo, P: persona gramatical, REC: recíproco, S: sujeto intransitivo}
%%%%%%%%%%%%%%%%%%%%%%%%%%%%%%%%%%%%%%%%%%%%%%%%
\vspace{0.5cm}

{\setmainfont{Charis SIL}

% Ejemplo 135
\begin{tabular}{llllll}
(135) & 'aishbi & ka & [nu=n & imi]=\textbf{ma} & 'ikën \\
& pero(\textsc{s/a>s}) & \textsc{nar.3p} & \textsc{1pl=gen} & sangre.\textsc{abs=\textbf{neg}} & ser.3p \\
& \multicolumn{5}{l}{``Pero no son nuestra sangre'' (pág. 537)}
\end{tabular} \vspace{0.3cm}

% Ejemplo 136
\begin{tabular}{llll}
(136) & upit=\textbf{ma} & ñu & 'unan-ti \\
& bueno=\textsc{\textbf{neg}} & cosa.\textsc{abs} & saber-\textsc{nom} \\
& \multicolumn{3}{l}{``No es bueno saber cosas'' (pág. 537)}
\end{tabular} \vspace{0.3cm}

% Ejemplo 137
\begin{tabular}{lllll}
(137) & a & kana & 'ë=x & kwëën-i=\textbf{ma} \\
& eso.\textsc{abs} & \textsc{nar.1sg} & \textsc{1sg=s} & querer-\textsc{impf-\textbf{neg}} \\
& \multicolumn{4}{l}{``Eso no lo quiero'' (pág. 538)}
\end{tabular} \vspace{0.3cm}

% Ejemplo 138
\begin{tabular}{lllll}
(138) & bërí & ka & uni & ain \\
& hoy & \textsc{nar.3p} & gente.\textsc{abs} & 3pos \\
& chai=bë & ain & kuku-bë & 'inan-anan-i-\textbf{mín} \\
& cuñado-\textsc{com(s)} & 3pos & suegro-\textsc{com(s)} & dar-\textsc{rec-impf-\textbf{compl.neg}.3p} \\
& \multicolumn{4}{l}{``Hoy en día la gente no se da (cosas) entre sus cuñados''} \\
& \multicolumn{4}{l}{y sus suegros (y esto me da rabia)'' (pág. 452)}
\end{tabular} \vspace{0.3cm}

% Ejemplo 139
\begin{tabular}{llll}
(139) & 'ë=x & kana & 'ux-i-\textbf{mán} \\
& \textsc{1sg=s} & \textsc{nar.1sg} & dormir-\textsc{impf-\textbf{compl}.\textbf{neg}.1/2p} \\
& \multicolumn{3}{l}{``No debería dormir, aunque me lo pidan'' (pág. 453)}
\end{tabular} \vspace{0.5cm}

}

\textcolor{MidnightBlue}{\citet{kakataibo}} identifica dos tipos morfemas destinados a la negación: el clítico {\setmainfont{Charis SIL} \textit{=ma}} y el par de sufijos {\setmainfont{Charis SIL} \textit{-mín / -mán}}. El clítico no tiene una posición definida dentro de la cláusula ya que puede adherirse a cualquier tipo de palabra: un sustantivo (135), un adjetivo (136) o un verbo (137).

Los sufijos {\setmainfont{Charis SIL} \textit{-mín / -mán}} obedecen a la distinción de persona gramatical: {\setmainfont{Charis SIL} \textit{-mín}} para tercera persona (138), mientras que {\setmainfont{Charis SIL} \textit{-mán}} es para primera y segunda persona (139).