\section*{Aguaruna}

\noindent El aguaruna es una lengua jívara del norte del Perú. El aguaruna se habla en las estribaciones orientales de los Andes y presenta similitudes tipológicas tanto con las lenguas amazónicas como con las andinas. 

Los rasgos tipológicos más destacados son: orden de constituyentes AOV (Activo-Objeto-Verbo), perfil nominativo-acusativo, marcación combinada de núcleo y dependiente, y una sintaxis altamente hipotáctica (encadenamiento de cláusulas). \vspace{0.5cm}

{\setmainfont{Charis SIL} 

% Ejemplo N
\begin{tabular}{lll}
() & wi-ka & buuta-\textbf{tsu}-ha-i  \\
& \textsc{1sg-foc} & llorar+\textsc{impfv-\textbf{neg}-1sg-decl} \\
& \multicolumn{2}{l}{``No estaba llorando'' (pág. 325)}
\end{tabular} \vspace{0.5cm}

% Ejemplo N2
\begin{tabular}{lll}
() & wi-ka & yu-a-\textbf{tsu}-ha-i \\
& \textsc{1sg-foc} & comer-\textsc{impfv-\textbf{neg}-1sg-decl} \\
& \multicolumn{2}{l}{``No estoy comiendo'' (pág. 325)}
\end{tabular} \vspace{0.5cm}

% Ejemplo N3
\begin{tabular}{ll}
() & daka-sa-\textbf{tʃa}-tata-ha-i \\
& esperar-\textsc{att-\textbf{neg}-fut-1sg-decl} \\
& ``No esperaré'' (pág. 325)
\end{tabular} \vspace{0.5cm}

% Ejemplo N4
\begin{tabular}{ll}
() & waina-ka-\textbf{tʃa}-amaia-ha-i \\
& ver-\textsc{ints-\textbf{neg}-psdist-1sg-decl} \\
& ``No vi a nadie'' (pág. 325)
\end{tabular} \vspace{0.5cm}

% Ejemplo N5
\begin{tabular}{ll}
() & \textbf{atsu}-a-wa-i \\
& \textsc{\textbf{neg}.exist-impfv-3-decl} \\
& ``No hay nada'' (pág. 326)
\end{tabular} \vspace{0.5cm}

}

La negación se realiza por medio de dos sufijos condicionados por el tiempo verbal \textcolor{MidnightBlue}{\citep{aguaruna}}. El sufijo {\setmainfont{Charis SIL} -\textit{tsu}} aparece con el tiempo presente y el pasado remoto (), mientras que el sufijo {\setmainfont{Charis SIL} -\textit{tʃa}} lo hace con el resto de tiempos (). Existe también el verbo de negación existencial {\setmainfont{Charis SIL} \textit{atsu}} ().