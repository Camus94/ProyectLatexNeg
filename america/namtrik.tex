\section*{Namtrik de Totoró}
\addcontentsline{toc}{section}{Namtrik de Totoró}

\noindent La lengua Namtrik o guambiano es una lengua de la familia Barbacoa hablada en el suroccidente de Colombia sobre la cordillera de los Andes.
%%%%%%%%%%%%% Abreviaturas %%%%%%%%%%%%%%%%%%%%%
\footnote{COP: cópula, DIST: distal, EGO: egofórico, EGO.EXP: ego experimentador, EPEN: epéntesis, NOEGO: no egofórico, PROS: progresivo, TOP: tópico, VNF: verbo no finito}
%%%%%%%%%%%%%%%%%%%%%%%%%%%%%%%%%%%%%%%%%%%%%%%%
\vspace{0.5cm}

{\setmainfont{Charis SIL} 

% Ejemplo 157
\begin{tabular}{llll}
(157) & mɨilɨ-pe & caso & mar-\textbf{mɨ}-an \\
& algunos & caso & hacer-\textsc{\textbf{neg}-noego} \\
& \multicolumn{3}{l}{``Algunos no hacen caso'' (pág. 525)}
\end{tabular} \vspace{0.5cm}

% Ejemplo 158
\begin{tabular}{lllll}
(158) & ɨ-nɨ-pe & cristiano & unɨ & kɨ-\textbf{mɨ}-an \\
& \textsc{dist-3-top} & cristiano & niño & \textsc{cop-\textbf{neg}-noego} \\
& \multicolumn{4}{l}{``que no era un niño cristiano'' (pág. 525)}
\end{tabular} \vspace{0.5cm}

% Ejemplo 159
\begin{tabular}{lllll}
(159) & \textbf{ka} & ku-ap & tso-\textbf{mɨ}-an & nɨ-pe \\
& \textsc{\textbf{neg}} & morir-\textsc{vnf} & acostado-\textsc{\textbf{neg}-noego} & \textsc{3-top} \\
& \multicolumn{4}{l}{``Ella no está enferma'' (pág. 530)}
\end{tabular} \vspace{0.5cm}

% Ejemplo 160
\begin{tabular}{lll}
(160) & \textbf{ka} & ch-i-t-an \\
& \textsc{\textbf{neg}} & decir-\textsc{epen-ego.exp-noego}\\
& \multicolumn{2}{l}{``entonces decía que no'' (pág. 531)}
\end{tabular} \vspace{0.5cm}

% Ejemplo 161
\begin{tabular}{llll}
(161) & na-pe & kisha & ash-\textbf{mɨ}-or \\
& \textsc{1-top} & llorar & ver-\textsc{\textbf{neg}-ego.sg} \\
& \multicolumn{3}{l}{``Yo no quieri llorar'' (pág.541)}
\end{tabular} \vspace{0.5cm}

% Ejemplo 162
\begin{tabular}{lll}
(162) & kutrɨ-ntr-ap & \textbf{mɨ}-an \\
& levantarse-\textsc{pros-vnf} & \textsc{\textbf{neg}-noego} \\
& \multicolumn{2}{l}{``No se va a levantar'' (pág.528)}
\end{tabular} \vspace{0.5cm}

}

Esta lengua posee tres estrategias morfosintácticas para la negación: el prefijo {\setmainfont{Charis SIL} \textit{-mɨ}} para una negación simple (157) y (158), la partícula {\setmainfont{Charis SIL} \textit{ka}} en combinación con el prefijo anterior (159) o apareciendo como complemento verbal (160), además de un auxiliar {\setmainfont{Charis SIL} \textit{ash}} «ver» para una negación desiderativa egofórica (161) \textcolor{MidnightBlue}{\citep{namtrik}}. Existe un caso particular con el aspecto progresivo donde {\setmainfont{Charis SIL} \textit{mɨ}} funciona como un auxiliar y ya no como un prefijo (162).