\section*{Mapuche}
\addcontentsline{toc}{section}{Mapuche}

\noindent El mapuche o mapudungún es una lengua aislada que se habla en Chile y Argentina por la etnia del mismo nombre.
%%%%%%%%%%%%% Abreviaturas %%%%%%%%%%%%%%%%%%%%%
\footnote{ADJ: adjetivizador, BEN: benefactivo, CA: causativo, DS: sujeto dativo, EDO: objeto directo externo, IDO: objeto directo interno, IND: indicativo, INST: instrumental, NRLD: no realizado, PVN: sustantivo verbal simple, S: sujeto, ST: estativo, TH: allá (thither) }
%%%%%%%%%%%%%%%%%%%%%%%%%%%%%%%%%%%%%%%%%%%%%%%%
\vspace{0.5cm}

{\setmainfont{Charis SIL} 

% Ejemplo 145
\begin{tabular}{ll}
(145) & la-le-\textbf{la}-y \\
& morir-\textsc{st-\textbf{neg}-ind-3}\\
& ``Él/ella no está muerto'' (pág.243)
\end{tabular} \vspace{0.5cm}

% Ejemplo 146
\begin{tabular}{lllll}
(146) & pepí & wiri-\textbf{la}-n & rumé & wütre-le-n-mew \\
& ser.capaz & escribir-\textsc{\textbf{neg}-ind1sg} & muy/mucho & frío-\textsc{st-pvn-inst} \\
& \multicolumn{4}{l}{``No puedo escribir porque hace mucho frío'' (pág. 64)}
\end{tabular} \vspace{0.5cm}

% Ejemplo 147
\begin{tabular}{llll}
(147) & lanɡ-üm-\textbf{ki}-fi-l-nɡe & tüfa-chi & üñüm \\
& morir-\textsc{ca-\textbf{neg}-edo-cond-imp2s} & este-\textsc{adj} & pájaro \\
& \multicolumn{3}{l}{``No mates ese pájaro'' (pág. 243)}
\end{tabular} \vspace{0.5cm}

{\footnotesize
% Ejemplo 148
\begin{tabular}{llllll}
(148) & petú & kudu-\textbf{nu}-l-m-i & ye-l-me-a-e-n & fürkü & ko \\
& aún & acostarse-\textsc{\textbf{neg}-cond-2-s} & llevar-\textsc{ben-th-nrld-ido-ind1sg-ds} & frío & agua \\
& \multicolumn{5}{l}{``Si aún no te vas a dormir, debes traerme agua fría'' (pág. 244)}
\end{tabular} \vspace{0.5cm}}

% Ejemplo 149
\begin{tabular}{llll}
(149) & fey & amu-\textbf{nu}-l-e, & amu-\textbf{la}-ya-n \\
& él & ir-\textsc{\textbf{neg}-cond-3} & ir-\textsc{\textbf{neg}-nrld-ind1sg} \\
& \multicolumn{3}{l}{``Si él no va, yo tampoco iré'' (pág. 179)}
\end{tabular} \vspace{0.5cm}

% Ejemplo 150
\begin{tabular}{llll}
(150) & fey-ti & ruka & \textbf{nu} \\
& esa.\textsc{det} & casa & \textsc{\textbf{neg}} \\
& \multicolumn{3}{l}{``Esa no es una casa'' (pág. 244)}
\end{tabular} \vspace{0.5cm}

}

Esta lengua cuenta con tres afijos negativos que forman parte de la morfología verbal \textcolor{MidnightBlue}{\citep{mapuche}}: {\setmainfont{Charis SIL} \textit{-la, -ki, -nu}}. El sufijo {\setmainfont{Charis SIL} \textit{-la}} se usa en formas de indicativo (145) y (146). Para la negación de imperativos se utiliza el sufijo {\setmainfont{Charis SIL} \textit{-ki}} (147) y para las formas en condicional se usa {\setmainfont{Charis SIL} \textit{-nu}} (148) y (149), aunque también puede ocuparse como una forma libre en función de nexo negativo (cópula negativa) de una construcción nominal (150).