\section*{Awa Pit (Cuaiquer)}
\addcontentsline{toc}{section}{Awa Pit}

\noindent El awá pit, conocido también como cuaiquer o kwaiker, es una lengua indígena de la familia barbacoana. Se habla en el sur de Colombia y el norte de Ecuador.
%%%%%%%%%%%%% Abreviaturas %%%%%%%%%%%%%%%%%%%%%
\footnote{IMPFPART: participio imperfectivo, LOCUT:marcador de persona locutor, NONLOCUT: marcador de persona no locutor, PAST: pasado, TOP: marcador de tópico}
%%%%%%%%%%%%%%%%%%%%%%%%%%%%%%%%%%%%%%%%%%%%%%%%
\vspace{0.2cm}

{\setmainfont{Charis SIL} 

% Ejemplo 129
\begin{tabular}{llll}
(129) & Santos=na & \textbf{shi} & ɨ-\textbf{ma}-y \\
& Santos=\textsc{top} & \textsc{\textbf{neg}} & ir-\textsc{\textbf{neg}-nonlocut} \\
& \multicolumn{3}{l}{``Santos no fue'' (pág. 332)}
\end{tabular} \vspace{0.2cm}

% Ejemplo 130
\begin{tabular}{lllll}
(130) & ap & gallo & \textbf{shi} & \textbf{ki}-a-zi \\
& mío & gallo & \textsc{\textbf{neg}} & ser.\textsc{\textbf{neg}-past-nonlocut} \\
& \multicolumn{4}{l}{``No era mi gallo'' (pág. 333)}
\end{tabular} \vspace{0.2cm}

% Ejemplo 131
\begin{tabular}{llll}
(131) & \textbf{shi} & ayna-mtu & \textbf{ki}-s \\
& \textsc{\textbf{neg}} & cocinar-\textsc{impfpart} & ser.\textsc{\textbf{neg}-locut} \\
& \multicolumn{3}{l}{``No estoy cocinando'' (pág. 334)}
\end{tabular} \vspace{0.2cm}

% Ejemplo 132
\begin{tabular}{llll}
(132) & ap & \textbf{shi} & ka-y \\
& mío & \textsc{\textbf{neg}} & ser:permanentemente-\textsc{nonlocut} \\
& \multicolumn{3}{l}{``No es mío'' (pág. 335)}
\end{tabular} \vspace{0.2cm}

% Ejemplo 133
\begin{tabular}{llll}
(133) & kwizha=na & alizh & \textbf{shi} \\
& perro=\textsc{top} & feroz & \textsc{\textbf{neg}(nonlocut)} \\
& \multicolumn{3}{l}{``El perro no es feroz'' (pág. 336)}
\end{tabular} \vspace{0.2cm}

}

La mayoría de las construcciones negativas implican el uso de la partícula negativa {\setmainfont{Charis SIL} \textit{shi}} y se dividen en dos tipos: negación clausal y negación no clausal \textcolor{MidnightBlue}{\citep{awa}}. La negación clausal está en (129) donde la partícula {\setmainfont{Charis SIL} \textit{shi}} ocupa una posición preverbal y dentro de la morfología verbal aparece el sufijo {\setmainfont{Charis SIL} \textit{-ma}}. Esta estrategia se utiliza tanto con verbos activos como estativos. En (130) se recurre a la cópula negativa {\setmainfont{Charis SIL} \textit{ki}} que se opone a su contraparte positiva \textit{i}. Por último, en (131) la copula negativa desempeña el papel de un verbo auxiliar al aparecer junto a una forma no finita de un verbo con valor léxico, en este caso una forma participial del verbo cocinar.

Por su parte, la negación no clausal consiste en colocar la partícula negativa {\setmainfont{Charis SIL} \textit{shi}} después del elemento que se quiera negar. Por lo tanto, en (132) lo que se quiere negar es la marca posesiva de primera persona y en (133) únicamente el atributo «feroz».

El Awa pit es una lengua de doble negación cuando se trata de la negación clausal. Esto quiere decir que necesita de dos marcadores para que la construcción sea gramatical y tenga lectura negativa. Por lo tanto, utiliza recursos morfológicos como sintácticos.