\section*{Pima bajo}
\addcontentsline{toc}{section}{Pima bajo}

\noindent El pima bajo es na lengua indígena hablada en el suroeste de Estados Unidos y noroeste de México. Es una lengua aglutinante y de marcación en el núcleo.
%%%%%%%%%%%%% Abreviaturas %%%%%%%%%%%%%%%%%%%%%
\footnote{CONT: continuo/progresivo, NSUJ: no sujeto, OBJ: objeto, PROS: prospectivo, SUJ: sujeto, VET: vetativo}
%%%%%%%%%%%%%%%%%%%%%%%%%%%%%%%%%%%%%%%%%%%%%%%%
\vspace{0.5cm}

{\setmainfont{Charis SIL} 

% Ejemplo 169
\begin{tabular}{llll}
(169) & aan & \textbf{im} & hias-a \\
& \textsc{1sg.suj} & \textsc{\textbf{neg}} & enterrar-\textsc{pros} \\
& \multicolumn{3}{l}{``No lo voy a enterrar'' (pág. 94)}
\end{tabular} \vspace{0.5cm}

% Ejemplo 170
\begin{tabular}{llllll}
(170) & Huaan & \textbf{im} & aag-im & ik & viv \\
& Juan & \textsc{\textbf{neg}} & desear-\textsc{cont} & \textsc{det-obj} & cigarro \\
& \multicolumn{5}{l}{``Juan no quiere ese cigarro'' (pág. 94)}
\end{tabular} \vspace{0.5cm}

% Ejemplo 171
\begin{tabular}{llll}
(171) & aap & \textbf{kova} & nook-an! \\
& \textsc{2sg.suj} & \textsc{\textbf{vet}} & hablar-\textsc{irr} \\
& \multicolumn{3}{l}{``No me hables'' (pág. 95)}
\end{tabular} \vspace{0.5cm}

% ejemplo 172
\begin{tabular}{llll}
(172) & \textbf{kova} & in-kɨi-in, & gogos!  \\
& \textsc{\textbf{vet}} & \textsc{1sg.nsuj-}morder-\textsc{imp} & perro \\
& \multicolumn{3}{l}{``No me muerdas perro'' (pág.) 95 }
\end{tabular} \vspace{0.5cm}

}

Esta lengua distingue entre negación simple y negación enfática o vetativa \textcolor{MidnightBlue}{\citep{pimbajo}}. Para el primer caso utiliza el adverbio negativo {\setmainfont{Charis SIL} \textit{(p)im}}. Este ha sufrido un proceso de reducción fonológica que ha afectado su forma histórica, {\setmainfont{Charis SIL} \textit{pim}}, hasta su forma sincrónica actual {\setmainfont{Charis SIL} \textit{im}} (169) y (170). Todo adverbio en esta lengua se coloca por lo general antes del verbo. El segundo adverbio es {\setmainfont{Charis SIL} \textit{kova}} y se utiliza donde la negación es enfática o vetativa (171) y (172).

La negación de un imperativo —una orden directa— debe entenderse como una prohibición, de ahí que algunas lenguas recurran a una marca diferente a la utilizada en la negación clausal