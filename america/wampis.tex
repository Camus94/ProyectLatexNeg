\section*{Wampis}
\addcontentsline{toc}{section}{Wampis}

\noindent El wampis es una lengua indígena que pertenece a la familia lingüística jíbara y es hablada en la Amazonía peruana, específicamente en las cuencas de los ríos Santiago y Morona en la región de Loreto. Tiene una estructura aglutinante, con sufijos que expresan distinciones gramaticales. Una característica clave de esta lengua es la intensa incorporación nominal, con la posibilidad de incorporar hasta tres sustantivos dentro del verbo. 
%%%%%%%%%%%%% Abreviaturas %%%%%%%%%%%%%%%%%%%%%
\footnote{ADD: aditivo, APPL: aplicativo, COP: cópula, DECL: declarativo, DIST.PT: pasado distante, HORT: hortativo, IPFV: imperfectivo, NMLZ: nominalizador, PT: pasado, Q: marcador de pregunta, REM.PT: pasado remoto, SS: mismo sujeto, TR: transitivizador}
%%%%%%%%%%%%%%%%%%%%%%%%%%%%%%%%%%%%%%%%%%%%%%%%
\vspace{0.5cm}

{\setmainfont{Charis SIL} 

% Ejemplo 178
\begin{tabular}{lll}
(178) & ʃuara-\textbf{t͡ʃau} & á-ia-ji \\
& enemigo-\textsc{\textbf{neg.nmlz}} & \textsc{cop-rem.pt-3.pt+decl} \\
& \multicolumn{2}{l}{``Ellos no eran enemigos'' (pág.747)}
\end{tabular} \vspace{0.3cm}

% Ejemplo 179
\begin{tabular}{ll}
(179) & ʃuara-\textbf{t͡ʃau}=aiti \\
& persona-\textsc{\textbf{neg.nmlz}=cop.3=decl}\\
& ``No es una persona'' (pág. 748)
\end{tabular} \vspace{0.3cm}

% Ejemplo 180
\begin{tabular}{lll}
(180) & wiʃi-ki-ru-a-mau & a-\textbf{t͡ʃa}-mi \\
& reír-\textsc{tr-appl-ipfv-nmlz} & \textsc{cop-\textbf{neg}-hort} \\
& \multicolumn{2}{l}{``Que no se rían de nosotros'' (pág. 639)}
\end{tabular} \vspace{0.3cm}

% Ejemplo 181
\begin{tabular}{lll}
(181) & Auaruna & ha-\textbf{t͡ʃa}-mia-ji \\
& Awajun & morir-\textsc{\textbf{neg}-dist.pt-3.pt+decl} \\
& \multicolumn{2}{l}{``El Awajun no murió'' (pág. 638)}
\end{tabular} \vspace{0.3cm}

% Ejemplo 182
\begin{tabular}{lllll}
(182) & hɨmpɨ=ʃa & uru-ka & a-sã & nankama-\textbf{t͡ʃa}-mia? \\
& colibrí=\textsc{add} & cómo-\textsc{q} & \textsc{cop-sub/3sg.ss} & pasar-\textsc{\textbf{neg}-dist.pt} \\
& \multicolumn{4}{l}{``¿Por qué no apareció el colibrí?'' (pág. 742)}
\end{tabular} \vspace{0.4cm}

}

La forma de indicar la negación varía según el mensaje que se quiera transmitir y el contexto morfosintáctico. Una oración negativa se suele acompañar con la cópula {\setmainfont{Charis SIL} \textit{a}} y puede marcarse de dos maneras distintas \textcolor{MidnightBlue}{\citep{wampis}}: [1] utilizando el nominalizador negativo {\setmainfont{Charis SIL} \textit{-t͡ʃau}}, el cual se añade como complemento copulativo (178) incluso si la cópula se comporta como clítico (179); o [2] empleando el morfema verbal negativo {\setmainfont{Charis SIL} \textit{-t͡ʃa}}, que se adjunta al propio verbo copulativo (180). Aunque este sufijo también ocurre sin la cópula dando una construcción negativa gramatical (181) o sin adherirse directamente a ella (182). La elección entre una u otra forma depende tanto del significado que se desee expresar como del entorno sintáctico en el que se encuentre.