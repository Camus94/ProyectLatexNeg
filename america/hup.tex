\section*{Hup}
\addcontentsline{toc}{section}{Hup}

\noindent La lengua hup es hablada por pueblos indígenas que habitan en las fronteras entre Colombia y el estado brasileño de Amazonas. Pertenece a la familia Hupda.
%%%%%%%%%%%%% Abreviaturas %%%%%%%%%%%%%%%%%%%%%
\footnote{OBJ: objeto, R: refuerzo, TEL: télico, énfasis contrastivo}
%%%%%%%%%%%%%%%%%%%%%%%%%%%%%%%%%%%%%%%%%%%%%%%%
\vspace{0.5cm}

{\setmainfont{Charis SIL} 

% Ejemplo 130
\begin{tabular}{llll}
(130) & maŋɡǎ & hɨd-ǎn & təw-\textbf{nɨh} \\
& Margarita & \textsc{3pl-obj} & regañar-\textsc{neg} \\
& \multicolumn{3}{l}{``Margarita aún no los ha regañado'' (pág. 726)}
\end{tabular} \vspace{0.5cm}

% Ejemplo 131
\begin{tabular}{lll}
(131) & tuk-\textbf{nɨh} & ʔam? \\
& querer-\textsc{neg} & \textsc{2sg} \\
& \multicolumn{2}{l}{``¿(Tú) no quieres?'' (pág. 727)}
\end{tabular} \vspace{0.5cm}

% Ejemplo 132
\begin{tabular}{lll}
(132) & tæ̃ʔnɔhɔ̃-\textbf{nɨh}=yɨʔ & nih \\
& reír-\textsc{neg=tel} & ser.\textsc{imp} \\
& \multicolumn{2}{l}{``No rías'' (pág. 727)}
\end{tabular} \vspace{0.5cm}

% Ejemplo 133
\begin{tabular}{lllll}
(133) & pɨhɨt & \textbf{næ} & ʔayup=tǎt & hɔ̃-\textbf{nɨh} \\
& plátano & \textsc{neg.r} & una=fruta & maduro-\textsc{neg} \\
& \multicolumn{4}{l}{``Ningún plátano está maduro'' (pág. 736 )}
\end{tabular} \vspace{0.5cm}

% Ejemplo 134
\begin{tabular}{llll}
(134) & [tiyǐʔ & pǒg] & \textbf{pā̌} \\
& hombre & grande & \textsc{neg} \\
& \multicolumn{3}{l}{``No hay un hombre grande'' (pág. 738)}
\end{tabular} \vspace{0.5cm}

}

El hup recurre a estrategias sintácticas como morfológicas para indicar la negación \textcolor{MidnightBlue}{\citep{hup}}. Esta lengua prefiere una estrategia morfológica para la negación de las cláusulas. En la mayoría de estos casos —si no es que todos— implica un único marcador negativo {\setmainfont{Charis SIL} \textit{-nɨh}} el cual ocurre como un sufijo en la base verbal (130). Este mismo sufijo se utiliza cláusulas interrogativas (131) e imperativas (132). El  Hup  utiliza  también una partícula  negativa  adicional {\setmainfont{Charis SIL} \textit{næ}} para  marcar  un refuerzo o énfasis en la negación (133)


Una segunda estrategia distinta implica la partícula negativa {\setmainfont{Charis SIL} \textit{pā̌}}. Esta estrategia se usa exclusivamente para expresar la negación de una entidad nominal, específicamente relacionada con la negación de su existencia o presencia (134).