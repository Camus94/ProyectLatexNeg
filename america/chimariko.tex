\section*{Chimariko}

\noindent El chimariko es una lengua actualmente extinta. En su momento fue una lengua aislada que se hablaba en el Condado de Trinity en la zona noroeste de California, Estados Unidos. Lengua sintética sufijante y de marcación en el núcleo.
%%%%%%%%%%%%% Abreviaturas %%%%%%%%%%%%%%%%%%%%%
\footnote{A: agente, ASP: aspecto, DEF: definido, DER: derivacional, LOC: locativo, MOD: modal, PROG: progresivo}
%%%%%%%%%%%%%%%%%%%%%%%%%%%%%%%%%%%%%%%%%%%%%%%%
\vspace{0.5cm}

{\setmainfont{Charis SIL} 

% Ejemplo 144
\begin{tabular}{lll}    
(144) & ya-\textbf{x-}akʰo\textbf{-na}-xan-ˀi & m-akʰo-ta-xan-tinda \\
& \textsc{1pl.a-\textbf{neg}-}matar-\textsc{\textbf{neg}-fut-asp} & \textsc{2sg-}matar-\textsc{der-fut-prog} \\
& \multicolumn{2}{l}{``No los mataremos, él te matará'' (pág. 177)}
\end{tabular} \vspace{0.5cm}

% Ejemplo 145
\begin{tabular}{lll}
(145) & ˀawa-ida-če & \textbf{x-}owo\textbf{-na}-t \\
& casa-\textsc{pos-loc} & \textsc{\textbf{neg}-}estar-\textsc{\textbf{neg}-asp} \\
& \multicolumn{2}{l}{``Ella no está en casa'' (pág. 177)}
\end{tabular} \vspace{0.5cm}

% Ejemplo 146
\begin{tabular}{ll}
(146) & q’e-h\textbf{-kuna}-coˀol \\
& morir-\textsc{3-\textbf{neg}-mod} \\
& ``Tal vez él no muera'' (pág. 177)
\end{tabular} \vspace{0.5cm}

% Ejemplo 147
\begin{tabular}{lll}
(147) & nunuˀ & n-e-mičit\textbf{-kuna} \\
& \textsc{x} & \textsc{imp.sg-1p-}patear-\textsc{\textbf{neg}} \\
& \multicolumn{2}{l}{``No me patees'' (pág. 179)}
\end{tabular} \vspace{0.5cm}

% Ejemplo 148
\begin{tabular}{lllllll}
(148) & h-inoˀy-ta & hi-suma & n-itix & xalall-op & n-akʰohoshu & \textbf{k’una} \\
& \textsc{3-}soportar-\textsc{asp} & \textsc{pos}-cara & \textsc{imp.sg-}limpiar & bebé-\textsc{def} & \textsc{imp.sg-}cortar & \textsc{\textbf{neg}} \\
& \multicolumn{6}{l}{``Ella lo soporta, límpiale la cara, (de) ese bebé, no le cortes (el ombligo)'' (pág. 179)}
\end{tabular} \vspace{0.5cm}

}

Presenta tres estrategias diferentes para la negación \textcolor{MidnightBlue}{\citep{chimariko}}. La primera estrategia corresponde al uso del circumflejo verbal {\setmainfont{Charis SIL} \textit{x-...-na}} el cual solo aparece en construcciones con prefijos pronominales (144) y (145). La segunda estrategia consiste en el uso del prefijo {\setmainfont{Charis SIL} \textit{-kuna / -k'una / -ˀna}} que ocurre con toda clase de predicados verbales y nominales (146) e incluso en imperativos (147). La tercera y última estrategia es la particula {\setmainfont{Charis SIL} \textit{kuna / k'una}} para la negación de imperativos (148).