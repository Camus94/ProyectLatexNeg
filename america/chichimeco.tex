\section*{Chichimeco Jonaz}

\noindent Chichimeco y chichimeca son términos utilizados para referirse tanto a un conjunto de lenguas o variedades lingüísticas, así como para denominar a un grupo de pueblos indígenas nómadas que habitaban en el centro y norte de México. El Chichimeco Jonaz pertenece a la subfamilia otopame de la familia otomangue. Esta variante se localiza en el municipio de San Luis de la Paz, en el noreste del estado de Guanajuato. \vspace{1cm}

{\setmainfont{Doulos SIL}
    % Ejemplo 11
    \begin{tabular}{rlrr}
        \multicolumn{1}{l}{(11)} & PRES       & \multicolumn{1}{l}{\textsc{ṹβ̃ã e-pã́s-βʷ}}           & \multicolumn{1}{l}{“Le pone los zapatos”}          \\
                                 & PAS.REM    & \multicolumn{1}{l}{\textsc{ṹβ̃ã u-pã́s-βʷ}}           & \multicolumn{1}{l}{“Le puso los zapatos”}          \\
                                 & PAS.REC    & \multicolumn{1}{l}{\textsc{ṹβ̃ã ku-pã́s-βʷ}}          & \multicolumn{1}{l}{“Le puso los zapatos”}          \\
                                 & PAS INM    & \multicolumn{1}{l}{\textsc{ṹβ̃ã su-pã́s-βʷ}}          & \multicolumn{1}{l}{“Le puso los zapatos”}          \\
                                 & FUT        & \multicolumn{1}{l}{\textsc{ṹβ̃ã a-pã́s-βʷ}}           & \multicolumn{1}{l}{“Le va a poner los zapatos”}    \\
                                 &            &                                                     &                                                    \\
                                 & PRES       & \multicolumn{1}{l}{\textsc{ṹβ̃ã sa-pã́s-umé}}         & \multicolumn{1}{l}{“No le pone los zapatos”}       \\
                                 & PAS.REM    & \multicolumn{1}{l}{\textsc{ṹβ̃ã sa-pã́s-umé}}         & \multicolumn{1}{l}{“No le puso los zapatos”}       \\
                                 & PAS.REC    & \multicolumn{1}{l}{\textsc{ṹβ̃ã su-pã́s-umé}}         & \multicolumn{1}{l}{“No le puso los zapatos”}       \\
                                 & PAS.INM    & \multicolumn{1}{l}{\textsc{ṹβ̃ã sa-pã́s-umé}}         & \multicolumn{1}{l}{“No le puso los zapatos”}       \\
                                 & FUT        & \multicolumn{1}{l}{\textsc{siʔá̤n3 ṹβ̃ã sa- pã́s-umé}} & \multicolumn{1}{l}{“No le va a poner los zapatos”} \\
                                 &            &                                                     &                                                    \\
                                 & PRES       & \multicolumn{1}{l}{\textsc{ṹβ̃ã ra-pã́s-βʷ}}          & \multicolumn{1}{l}{“Le pondría los zapatos”}       \\
                                 & PAS.REM    & \multicolumn{1}{l}{\textsc{ṹβ̃ã ma-pã́s-βʷ}}          & \multicolumn{1}{l}{“Le habría puesto los zapatos”} \\
                                 & PAS.REC    & \multicolumn{1}{l}{\textsc{ṹβ̃ã ma-pã́s-βʷ}}          & \multicolumn{1}{l}{“Le habría puesto los zapatos”} \\
                                 & PAS.INM    & \multicolumn{1}{l}{\textsc{ṹβ̃ã ma-pã́s-βʷ}}          & \multicolumn{1}{l}{“Le habría puesto los zapatos”} \\
                                 & FUT        & \multicolumn{1}{l}{\textsc{ṹβ̃ã a-βã́s-βʷ}}           & \multicolumn{1}{l}{“Le pondría los zapatos”}       \\
                                 & (pág. 145) &                                                     &                                                    \\
    \end{tabular}
} \vspace{1cm}

Lizárraga Navarro (2018) nos dice que tipológicamente es una lengua aglutinante con características polisintéticas. Presenta una gran riqueza morfológica de naturaleza concatenativa y no concatenativa. En la morfología verbal hay una compleja expresión de persona y número, a través de distintos paradigmas de morfemas que incluyen prefijos, sufijos, cambios internos (mutaciones consonánticas, cambios vocálicos), alternancias tonales y alternancias verbales.

El morfema \textit{-umé} tiene la función de marcar la negación en el verbo. Aparece junto con el prefijo sa- en las formas negativas del modo afirmativo y su presencia parece causar la elisión del sufijo de objeto {\setmainfont{Doulos SIL}-βw}. Esta distribución \textit{—sa-} + \textit{-umé—} parece estar restringida a las formas negativas del modo afirmativo en presente y pasado (remoto, reciente, inmediato) ya que en las formas negativas de futuro se requiere el uso la partícula {\setmainfont{Doulos SIL}siʔá̤n.}
