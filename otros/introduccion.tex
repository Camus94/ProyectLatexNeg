\section*{Introducción}
\addcontentsline{toc}{section}{Introducción}

\noindent La negación es un fenómeno que se encuentra en todas las lenguas, siendo un fenómeno lingüístico, cognitivo y lógico a la vez \textcolor{MidnightBlue}{\citep{lawler}}. Es probablemente uno de los elementos que más dota de complejidad e identidad al ser humano ya que permite algunas de las propiedades más básicas de nuestros sistemas de pensamiento: muestra tanto la complejidad de nuestro intelecto para concebir lo opuesto y contrario a lo afirmado, como la capacidad del lenguaje para expresar esta oposición de manera estructurada con fines comunicativos \textcolor{MidnightBlue}{\citep{horn,hornKato}}. 

Debido a su importancia en la comunicación humana y en los sistemas, lingüísticos el fenómeno de la negación ha sido ampliamente estudiado por diversos expertos. Sin embargo, su comprensión teórica aún presenta desafíos, ya que interviene en varios niveles de análisis lingüístico, desde la morfología hasta la sintaxis, pasando por aspectos semánticos y pragmáticos. 

Es por ello que el presente trabajo se centrará en analizar el fenómeno gramatical de la negación desde una perspectiva tipológica en el campo de la lingüística. Por medio de ejemplos de diversas lenguas con sus diferentes estrategias y estructuras negativas se buscará identificar patrones comunes y peculiaridades, con el fín de proporcionar una visión general y sistemática del fenómeno.

La relevancia de este trabajo reside entonces en ser un pequeño catálogo de ejemplos que servirán como referencia para futuras investigaciones sobre este fenómeno y un punto de partida para el análisis comparativo de las diferentes estrategias negativas utilizadas en distintas lenguas. 

La metodología utilizada en este trabajo se basa en la recolección de datos provenientes de fuentes primarias y secundarias, como gramáticas, estudios lingüísticos y análisis contrastivos. Se han tomado 58 lenguas de diferentes familias lingüísticas y se han recopilado una serie de ejemplos representativos de cada una de ellas.

En la entrada de cada una de estas lenguas, se muestran los ejemplos encontrados y se hace un breve comentario sobre ellos. Los ejemplos se centran principalmente en la negación estándar —a veces llamada negación clausal—, aunque ocasionalmente se presentan otros tipos de negación como prohibitivos, existenciales o negación de constituyentes.

Fenómenos más específicos como la doble negación, la cópula negativa o la negación asimétrica no se abordarán en este estudio, ya que requieren un análisis más detallado y profundo. Si bien es posible que aparezcan algunos casos de estos fenómenos en los ejemplos recopilados, su discusión y análisis no se lleva a cabo.

La estructura del trabajo se organiza de la siguiente manera: la primera parte consiste en una revisión muy puntal sobre los presupuestos teóricos sobre la negación. En la segunda parte se encuentran todos los ejemplos encontrados clasificados por lenguas de México, lenguas del continente americano y lenguas del resto del mundo. En una tercera parte se muestran de manera sintética las estrategias de negación encontradas en las distintas lenguas y se comentan las similitudes y diferencias entre ellas. Finalmente, en la cuarta parte se presenta una conclusión general y algunas sugerencias para futuras investigaciones en este campo.

En resumen, este trabajo busca proporcionar una visión general y sistemática del fenómeno de la negación en diferentes lenguas. Proporcionando un catálogo de ejemplos que despierte el interés investigativo y que sirva como base para análisis comparativos más profundos.

\subsection*{Sobre la definición de negación}
\addcontentsline{toc}{subsection}{Sobre la definición de negación}

\noindent El fenómeno de la negación suele conceptualizarse en las lenguas humanas como una «categoría funcional» ya que carece de contenido semántico y se utiliza para establecer relaciones entre elementos o como un «operador» sintáctico \textcolor{MidnightBlue}{\citep{lawler,CasasNavarro2005}}. Una de sus características principales es que tiene un «ámbito», esto quiere decir que siempre modifica algún nivel de la cláusula, ya sea el núcleo o acción en sí misma, los participantes, o toda la cláusula.

Al nivel que modifica se le conoce como «foco». Según el tipo de operador, se distinguen tres categorías \textcolor{MidnightBlue}{\citep{Valin}}: [1] operadores nucleares (niegan/enfocan la acción o evento), [2] operadores del core (niegan/enfocan a los participantes), y [3] operadores clausales (niegan/enfocan toda la cláusula). La negación se considera primordialmente un operador nuclear, ya que inicialmente afecta y modifica solo al núcleo o acción, sin referirse a los participantes.

Este nivel de negación sobre el evento en sí mismo es conocido como \textit{negación estándar}. «Esta es la manera más básica que las lenguas tienen para invertir el valor de las cláusulas verbales principales declarativas» \textcolor{MidnightBlue}{\citep{negStandar}}. Sin embargo, no todas las construcciones negativas tendrán este ámbito de operación.

Desde un punto de vista tipológico, la negación se manifiesta en su mayoría a través de mecanismos gramaticales explícitos, específicamente morfemas \textcolor{MidnightBlue}{\citep{morfemas}}: afijos, clíticos y léxico o partículas independientes. Aunque también se conoce y teoriza sobre otros mecanismos poco comunes en las lenguas del mundo.

Por lo tanto, en este trabajo se entenderá por negación una operación lógica con realización formal dentro de las lenguas que invierte el valor de verdad de una proposición o de algún elemento dentro de ella y que muestra una compleja interacción con muchos aspectos de la gramática como lo son el significado y la estructura de la cláusula.