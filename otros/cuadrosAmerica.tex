\section*{Lenguas del resto del continente americano - Estrategias}
\addcontentsline{toc}{section}{Lenguas del resto de américa}

% Tabla 6
\begin{table}[htbp]
\centering
\begin{tabular}{lccc}
\textbf{Lengua} & \textbf{Léxico independiente} & \textbf{Afijos} & \textbf{Clíticos} \\
\hline
Aguaruna &       & X     &  \\
Awa Pit & X     &       &  \\
Cavineña & X     & X     & X \\
Chimariko & X     & X     &  \\
Choctaw &       & X     &  \\
Garífuna & X     & X     &  \\
Hup   & X     & X     &  \\
Kakataibo &       & X     & X \\
Kamsá & X     & X     &  \\
Mapuche & X     & X     &  \\
Maricopa &       & X     & X \\
Namtrik de Totoró &  & X &  \\
Ngäbére & X     &       &  \\
Pima bajo & X     &       &  \\
Sabanê &       & X     &  \\
Wampis &       & X     &  \\
Wappo &       & X     &  \\
Wari' & X     &       &  \\
Yauyos Quechua & X     &       & X \\
\hline
\end{tabular}
\caption{Lenguas del continente americano}
\label{cuadro6}
\end{table}

\noindent De las 19 lenguas del continente americano solo 3 utilizan como mecanismo único para marcar la negación partículas o léxico independiente, 6 recurren solo a afijos, 6 tanto a léxico como a afijos, 2 a afijos y a clíticos, 1 lengua a léxico y a clíticos y el caso sorprendente de la lengua cavineña que utiliza los tres mecanismos.

En este grupo de lenguas existe una distribución más equilibrada entre los mecanismos, solo por poca diferencia el uso de afijos se muestra como preferente.

% Tabla7
\begin{table}[htbp]
\centering
\begin{tabular}{lr}
\textbf{Lengua} & \textbf{Léxico independiente} \\
\hline
Ngäbére & {\setmainfont{Charis SIL} \textit{ñaka}} / {\setmainfont{Charis SIL} \textit{ñagare}} / {\setmainfont{Charis SIL} \textit{ñukwä}} / {\setmainfont{Charis SIL} \textit{ña(n)}} \\
Pima bajo & {\setmainfont{Charis SIL} \textit{im}} / {\setmainfont{Charis SIL} \textit{kova}} \\
Wari' & {\setmainfont{Charis SIL} \textit{mao}} / {\setmainfont{Charis SIL} \textit{á}} / {\setmainfont{Charis SIL} \textit{ára}} \\
\hline
\end{tabular}
\caption{Negación codificada en léxico independiente}
\label{cuadro7}
\end{table}

% Tabla 8
\begin{table}[htbp]
\centering
\begin{tabular}{lr}
\textbf{Lengua} & \multicolumn{1}{c}{\textbf{Afijos}} \\
\hline
Aguaruna & -{\setmainfont{Charis SIL} \textit{tsu}} / -{\setmainfont{Charis SIL} \textit{tʃa}} \\
Choctaw & -{\setmainfont{Charis SIL} \textit{kiiyo}} \\
Namtrik & -{\setmainfont{Charis SIL} \textit{mɨ}} \\
Sabanê & -{\setmainfont{Charis SIL} \textit{mina}} / -{\setmainfont{Charis SIL} \textit{misina}} \\
Wampis & -{\setmainfont{Charis SIL} \textit{t͡ʃa}} / -{\setmainfont{Charis SIL} \textit{t͡ʃau}} \\
Wappo & -{\setmainfont{Charis SIL} \textit{lahkhiʔ}} \\
\hline
\end{tabular}
\caption{Lenguas que codifican la negación en afijos}
\label{cuadro8}
\end{table}

% Tabla 9
\begin{table}[htbp]
\centering
\begin{tabular}{lcr}
\multicolumn{1}{c}{\textbf{Lengua}} & \textbf{Léxico independiente} & \multicolumn{1}{c}{\textbf{Afijos}} \\
\hline
Awa Pit & {\setmainfont{Charis SIL} \textit{shi}} / {\setmainfont{Charis SIL} \textit{ki}} & -{\setmainfont{Charis SIL} \textit{ma}} \\
Chimariko & {\setmainfont{Charis SIL} \textit{kuna}} / {\setmainfont{Charis SIL} \textit{k’una}} & {\setmainfont{Charis SIL} \textit{x-...-na}} / -{\setmainfont{Charis SIL} \textit{kuna}} / -{\setmainfont{Charis SIL} \textit{k’una}} / -{\setmainfont{Charis SIL} \textit{ˀna}} \\
Garífuna & {\setmainfont{Charis SIL} \textit{mama}}  & {\setmainfont{Charis SIL} \textit{m(a)}}-  \\
Hup   & {\setmainfont{Charis SIL} \textit{næ}} / {\setmainfont{Charis SIL} \textit{pā̌}} & -{\setmainfont{Charis SIL} \textit{nɨh}} \\
Kamsá & {\setmainfont{Charis SIL} \textit{ndoñ(e)}} & {\setmainfont{Charis SIL} \textit{ke}}- / {\setmainfont{Charis SIL} \textit{at}}- / {\setmainfont{Charis SIL} \textit{n}}- / {\setmainfont{Charis SIL} \textit{nd}}- \\
Mapuche & {\setmainfont{Charis SIL} \textit{un}}    & -{\setmainfont{Charis SIL} \textit{un}} / -{\setmainfont{Charis SIL} \textit{la}} / -{\setmainfont{Charis SIL} \textit{ki}} \\
\hline
\end{tabular}
\caption{Negación codificada en léxico y afijos}
\label{cuadro9}
\end{table}

% Tabla 10
\begin{table}[htbp]
\centering
\begin{tabular}{lcc}
\multicolumn{1}{c}{\textbf{Lengua}} & \textbf{Afijos} & \textbf{Clíticos} \\
\hline
Kakataibo & -{\setmainfont{Charis SIL} \textit{mín}} / -{\setmainfont{Charis SIL} \textit{mán}} & ={\setmainfont{Charis SIL} \textit{ma}} \\
Maricopa & -{\setmainfont{Charis SIL} \textit{ma}}   & {\setmainfont{Charis SIL} \textit{waly}}= / {\setmainfont{Charis SIL} \textit{aly}}= \\
\hline
\end{tabular}
\caption{Negación codificada en afijos y clíticos}
\label{cuadro10}
\end{table}
  
% Tabla 11
\begin{table}[htbp]
\centering
\begin{tabular}{lccc}
\multicolumn{1}{c}{\textbf{Lengua}} & \textbf{Léxico independiente} & \textbf{Afijos} & \textbf{Clíticos} \\
\hline
Cavineña & {\setmainfont{Charis SIL} \textit{jipakwana}} / {\setmainfont{Charis SIL} \textit{juwaaba}} / {\setmainfont{Charis SIL} \textit{pajuani}} & -{\setmainfont{Charis SIL} \textit{karama}} / -{\setmainfont{Charis SIL} \textit{dame}} / {\setmainfont{Charis SIL} \textit{ne- ... -ume}} & ={\setmainfont{Charis SIL} \textit{ama}} \\
\hline
\end{tabular}
\caption{Negación codificada en las tres estrategias}
\label{cuadro11}
\end{table}

% Tabla 12
\begin{table}[htbp]
\centering
\begin{tabular}{lcr}
\multicolumn{1}{c}{\textbf{Lengua}} & \textbf{Léxico independiente} & \textbf{Clíticos} \\
\hline
Yauyos Quechua & {\setmainfont{Charis SIL} \textit{mana}} / {\setmainfont{Charis SIL} \textit{ama}} / {\setmainfont{Charis SIL} \textit{ni}} & ={\setmainfont{Charis SIL} \textit{chu}} \\
\hline
\end{tabular}
\caption{Negación codificada en léxico y clíticos}
\label{cuadro12}
\end{table}
  
  
  