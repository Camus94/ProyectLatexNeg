\section*{Lenguas del resto del mundo}
\addcontentsline{toc}{section}{Lenguas del resto del mundo}

% Table 13
\begin{table}[htbp]
\centering
\begin{tabular}{lccc}
\multicolumn{1}{c}{\textbf{Lengua}} & \textbf{Léxico independiente} & \textbf{Afijos} & \textbf{Clíticos} \\
\hline
Fwe   &       & X     &  \\
Gyeli & X     & X     &  \\
Jamsay &       & X     &  \\
Japhug & X     & X     &  \\
Kalamang &       & X     & X \\
Komnzo & X     &       & X \\
Madurese & X     &       & X \\
Mauwake & X     &       &  \\
Moloko & X     &       &  \\
Mongsen &       & X     &  \\
Palula & X     &       &  \\
Papuan Malay & X     &       &  \\
Pichi & X     &       &  \\
Toqabaqita & X     &       &  \\
\hline
\end{tabular}
\caption{Lenguas del resto del mundo}
\label{cuadro13}
\end{table}

De las 14 lenguas de este apartado, 6 recurren a léxico independiente para codificar la negación, solo 3 al uso de afijos, 2 al uso de afijos y clíticos, una 1 única lengua que recurre a clíticos y afijos, 2 al uso de léxico o partículas junto con afijos.

% Tabla 14
\begin{table}[htbp]
\centering
\begin{tabular}{lc}
\multicolumn{1}{c}{\textbf{Lengua}} & \textbf{Léxico independiente} \\
\hline
Mauwake & {\setmainfont{Charis SIL} \textit{me}} / {\setmainfont{Charis SIL} \textit{weetak}} / {\setmainfont{Charis SIL} \textit{wia}} / {\setmainfont{Charis SIL} \textit{marew}} \\
Moloko & {\setmainfont{Charis SIL} \textit{baj}} / {\setmainfont{Charis SIL} \textit{asabay}} / {\setmainfont{Charis SIL} \textit{fabay}} / {\setmainfont{Charis SIL} \textit{kəlo}} {\setmainfont{Charis SIL} \textit{bay}} / {\setmainfont{Charis SIL} \textit{təta}} {\setmainfont{Charis SIL} \textit{bay}} \\
Palula & {\setmainfont{Charis SIL} \textit{na}} \\
Papuan Malay & {\setmainfont{Charis SIL} \textit{tida}} / {\setmainfont{Charis SIL} \textit{tra}} / {\setmainfont{Charis SIL} \textit{bukang}} \\
Pichi & {\setmainfont{Charis SIL} \textit{nó}} / {\setmainfont{Charis SIL} \textit{nɛ́a}} / {\setmainfont{Charis SIL} \textit{nóto}} \\
Toqabaqita & {\setmainfont{Charis SIL} \textit{kesi}} / {\setmainfont{Charis SIL} \textit{aqi}} / {\setmainfont{Charis SIL} \textit{kwasi}} \\
\hline
\end{tabular}
\caption{Negación codificada en partículas o léxico independiente}
\label{cuadro14}
\end{table}

% Tabla 15
\begin{table}[htbp]
\centering
\begin{tabular}{lc}
\multicolumn{1}{c}{\textbf{Lengua}} & \textbf{Afijos} \\
\hline
Fwe   & {\setmainfont{Charis SIL} \textit{ka}}- / {\setmainfont{Charis SIL} \textit{ta}}- / {\setmainfont{Charis SIL} \textit{ásha}}- / {\setmainfont{Charis SIL} \textit{shá}}- / -{\setmainfont{Charis SIL} \textit{i}} \\
Jamsay & -{\setmainfont{Charis SIL} \textit{lì}} / -{\setmainfont{Charis SIL} \textit{lú}} / -{\setmainfont{Charis SIL} \textit{l}} / -{\setmainfont{Charis SIL} \textit{ɡo’}}  \\
Mongsen & {\setmainfont{Charis SIL} \textit{mə̀}}- / -{\setmainfont{Charis SIL} \textit{la}} \\
\hline
\end{tabular}
\caption{Negación codificada en afijos}
\label{cuadro15}
\end{table}

% Tabla 16
\begin{table}[htbp]
\centering
\begin{tabular}{lcc}
\multicolumn{1}{c}{\textbf{Lengua}} & \textbf{Léxico independiente} \textbf{Clíticos} \\
\hline
Komnzo & matak & =m / =mär \\
Madurese & {\setmainfont{Charis SIL} \textit{(n)ja’}}  & {\setmainfont{Charis SIL} \textit{ta’}}= / {\setmainfont{Charis SIL} \textit{lo’}}= \\
\hline
\end{tabular}
\caption{Negación codificada en léxico y clíticos}
\label{cuadro16}
\end{table}

% Tabla 17
\begin{table}[htbp]
\centering
\begin{tabular}{lcc}
\multicolumn{1}{c}{\textbf{Lengua}} & \textbf{Léxico independiente} & \textbf{Afijos} \\
\hline
Gyeli & {\setmainfont{Charis SIL} \textit{sàlɛ}} ́/ {\setmainfont{Charis SIL} \textit{pálɛ}} / {\setmainfont{Charis SIL} \textit{kálɛ̀}} / {\setmainfont{Charis SIL} \textit{dúù}} / {\setmainfont{Charis SIL} \textit{kí}} / {\setmainfont{Charis SIL} \textit{tí}} & -{\setmainfont{Charis SIL} \textit{lɛ}} \\
Japhug & {\setmainfont{Charis SIL} \textit{maʁ}} / {\setmainfont{Charis SIL} \textit{me}} & {\setmainfont{Charis SIL} \textit{mɤ}}- / {\setmainfont{Charis SIL} \textit{mɯ}}- / {\setmainfont{Charis SIL} \textit{ma}}- / {\setmainfont{Charis SIL} \textit{mɯ́j}}- \\
\hline
\end{tabular}
\caption{Negación codificada en léxico y afijos}
\label{cuadro17}
\end{table}

% Tabla 18
\begin{table}[htbp]
\centering
\begin{tabular}{lcc}
\multicolumn{1}{c}{\textbf{Lengua}} & \textbf{Afijos} & \textbf{Clíticos} \\
\hline
Kalamang & -{\setmainfont{Charis SIL} \textit{mun}}  & ={\setmainfont{Charis SIL} \textit{nin}} / ={\setmainfont{Charis SIL} \textit{in}} \\
\hline
\end{tabular}
\caption{Negación codificada en afijos y cliticos}
\label{cuadro18}
\end{table}