\section*{Negación estándar}
\addcontentsline{toc}{section}{Negación estándar}

\noindent El término «negación estándar» fue utilizado por primera vez por \textcolor{MidnightBlue}{\citet{Payne85}} y popularizado posteriormente por \textcolor{MidnightBlue}{\citet{Miestamo2005}} para referirse a la manera más básica y general que existe para negar predicados verbales en las distintas lenguas del mundo. \textcolor{MidnightBlue}{\citet{Auwera2020}} la definen como «la negación no enfática de un verbo léxico principal en una cláusula principal declarativa» y cualquier construcción que sea negativa pero no se ajuste exactamente a estos criterios se le denomina «negación no-estándar».

La negación estándar consiste —en términos muy generales— en una sencilla operación a nivel semántico donde el valor de verdad de una proposición se invierte. Y, al partir de esta aparente sencillez, se espera que su realización formal en las lenguas también lo sea. Es decir, se espera la presencia de un único marcador negativo y una misma forma de expresión de la proposición negativa a la positiva en términos estructurales \textcolor{MidnightBlue}{\citep{Auwera2020}}. Sin embargo, el fenómeno de la negación es mucho más complejo y no siempre se ajusta a esta expectativas.

\subsection*{Número de marcadores}
\addcontentsline{toc}{subsection}{Número de marcadores}

\noindent Lo más común tipológicamente es, en efecto, que la negación se exprese por medio de un solo marcador; alrededor de un 80\% de las lenguas del mundo cumplen esto \textcolor{MidnightBlue}{\citep{VanAlsenoy2014,morfemas,Vossen2016}}, pero hay lenguas que utilizan más de un marcador negativo en sus construcciones, siendo el tipo más común el dos marcadores. La presencia de dos marcadores negativos puede ser opcional u obligatoria (en este caso se le denomina «doble negación»), mientras que la presencia de tres o más siempre será opcional.

El francés (I) es famoso por sus dos marcadores negativos {\setmainfont{Charis SIL} \textit{ne}} y {\setmainfont{Charis SIL} \textit{pas}}. En algún punto de su historia ambos marcadores eran obligatorios, pero actualmente el primer negador {\setmainfont{Charis SIL} \textit{ne}} es opcional. \vspace{0.5cm}

% cSpell:disable
{\setmainfont{Charis SIL} 

\begin{tabular}{llllll}
(I) & Marie & \textbf{ne} & le & voit & \textbf{pas} \\
& María & \textsc{\textbf{neg1}} & a él & ver & \textsc{\textbf{neg2}} \\
& \multicolumn{5}{l}{``María no lo ve''}
\end{tabular} \vspace{0.5cm}

% cSpell:enable
}

Los siguientes ejemplos muestran construcciones con más de dos marcadores. La lengua Kanyok (II) tiene tres marcadores \textcolor{MidnightBlue}{\citep[pág. 263]{Devos2013}} para las construcciones negativas, pero solo los primeros dos son obligatorios: \vspace{0.5cm}

% cSpell:disable
{\setmainfont{Charis SIL} 

\begin{tabular}{lll}
(II) & \textbf{ka}-tù-tùm-in-\textbf{oh}' & \textbf{bènd} \\
& \textsc{\textbf{neg1}-1pl-}enviar-\textsc{tam-\textbf{neg2}} & \textsc{\textbf{neg3}} \\
& \multicolumn{2}{l}{``No lo hemos enviado''}
\end{tabular} \vspace{0.5cm}}
% cSpell:enable

La lengua lewo (III) posee cuatro marcadores \textcolor{MidnightBlue}{\citep[pág. 405]{Early1994}} \vspace{0.5cm}:

{\setmainfont{Charis SIL} 
% cSpell:disable
\begin{tabular}{lllll}
(III) & \textbf{pe}-\textbf{re} & a-pim & \textbf{re} & \textbf{poli}  \\
& \textsc{\textbf{neg1}-\textbf{neg2}} & \textsc{3pl-}venir.\textsc{real} & \textsc{\textbf{neg3}} & \textsc{\textbf{neg4}} \\
& \multicolumn{4}{l}{``Ellos no vinieron''}
\end{tabular} \vspace{0.5cm}}
% cSpell:enable

Y en la lengua bantawa pueden encontrarse construcciones con cinco negadores \textcolor{MidnightBlue}{\citep[pág. 271]{Doornenbal2009}}: \vspace{0.5cm}

{\setmainfont{Charis SIL} 
% cSpell:disable
\begin{tabular}{lll}
(IV) & \textbf{i}-ciŋ-\textbf{nin} & set-\textbf{nin}-Ø-\textbf{nin}-ci-\textbf{n} \\
& \textsc{\textbf{neg1}}-colgarse-\textsc{\textbf{neg2}} & matar-\textsc{\textbf{neg3}-prog-\textbf{neg4}-refl-\textbf{neg5}} \\
& \multicolumn{2}{l}{``Él no se mató ahorcándose''}
\end{tabular} \vspace{0.5cm}}
% cSpell:enable

La explicación para la presencia de más de un marcador suele ser la necesidad de «reforzar» al único marcador existente, en alguna etapa concreta de una lengua, que ha entrado en un proceso de «debilitamiento» y que por lo tanto resulta insuficiente para indicar negación. Esto es parte de un proceso más grande llamado \textbf{Ciclo de Jespersen}, nombre dado por \textcolor{MidnightBlue}{\citet{Dahl1979}} en referencia al trabajo de \textcolor{MidnightBlue}{\citet{Jespersen1917}}.

Ahora bien, el hecho de que una lengua utilice como estrategia para manifestar la negación un único marcador, no quiere decir que no existan fenómenos relacionados con ese mismo marcador. Es posible que exista un repertorio de marcadores junto con sus variaciones y que cada uno de ellos se manifieste bajo determinadas condiciones. La idea más acepta es, como indican \textcolor{MidnightBlue}{\citet{Auwera2020}}, que estas variaciones se deben a la presencia de marcadores de tiempo, aspecto y modo. \vspace{0.5cm}

{\setmainfont{Charis SIL} 
% cSpell:disable
\begin{tabular}{llll}
(V) & a. & \textbf{tera} & i=N-poroh-e \\
& & \textsc{\textbf{neg.real}} & \textsc{3m.sg=irr-}despejar.tierra-\textsc{irr.i} \\
& & \multicolumn{2}{l}{``Él no está limpiando la tierra''} \\
& & & \\
& b. & \textbf{ha}=me & pi=tsot-se-na-i=ro \\
& & \textsc{\textbf{neg.irr}=deo} & \textsc{2sg=}sorber-\textsc{clf-mal.rep-real.i=3nm.o} \\
& & \multicolumn{2}{l}{``No deberías sorberlo''} \\
\end{tabular} \vspace{1cm}
% cSpell:enable
}

En la descripción que hace \textcolor{MidnightBlue}{\citet[pág. 182, 195]{Michael2014}} sobre la lengua nanti, muestra cómo las marcas de modo determinan la presencia de un tipo específico de marcador negativo. Se utiliza {\setmainfont{Charis SIL} \textit{tera}} para el modo realis (Va) y {\setmainfont{Charis SIL} \textit{ha}} para el modo irrealis (Vb).

\subsection*{Simetría y asimetría}
\addcontentsline{toc}{subsection}{Simetría y asimetría}

\noindent Se habla de negación simétrica cuando la expresión de una proposición negativa es idéntica tanto en construcción como a nivel paradigmático a la expresión de una proposición positiva o afirmativa \textcolor{MidnightBlue}{\citep{Miestamo2005}}. Esto significa que, en muchas lenguas, las construcciones negativas solo difieren a las positivas en la presencia del marcador negativo y el paradigma verbal se mantiene igual, tal es el caso, por ejemplo, del español (VIa-b) y el alemán (VIc-d). \vspace{0.5cm}

{\setmainfont{Charis SIL}
% cSpell:disable
\begin{tabular}{lll}
(VI) & a. & María encontró flores \\
& b. & María \textbf{no} encontró flores \\
& & \\
& c. & Marie liebt ihn \\
& d. & Marie liebt ihn \textbf{nicht} \\
\end{tabular} \vspace{0.5cm}}
% cSpell:enable

El fenómeno contrario, la asimetría, se entiende entonces como un cambio estructural y paradigmático en la expresión de las proposiciones negativas. La lengua kariña (VI) ejemplifica este tipo de negación \textcolor{MidnightBlue}{\citep{Mosonyi2000}}. Es asimetría estructural porque no solo difiere en la presencia del marcador negativo y es paradigmática debido al cambio en el paradigma verbal. \vspace{0.5cm}

{\setmainfont{Charis SIL} 
% cSpell:disable
\begin{tabular}{llll}
(VII) & a. & m-oonaae \\
& & \textsc{2sg}-cultivar.\textsc{pres} \\
& & ``Tú cultivas'' \\
& & & \\
& b. & oona-\textbf{ja} & maana \\
& & cultivar-\textsc{\textbf{neg}} & \textsc{cop.2sg} \\
& & ``Tú no cultivas'' \\
\end{tabular} \vspace{0.5cm}} 
% cSpell:enable

La asimetría no es un fenómeno raro. De 179 lenguas que \textcolor{MidnightBlue}{\citet{Miestamo2005}} revisa en su estudio, encuentra asimetría construccional en el 46\% de ellas y 30\% presenta asimetría paradigmática.

\subsection*{Formas de expresión y naturaleza del marcador}
\addcontentsline{toc}{subsection}{Formas de expresión y naturaleza del marcador}

\noindent  El estatus sintáctico de los marcadores negativos se divide en dos grupos: libres y ligados. Dentro del primer grupo se recoge un repertorio de posibilidades conformado generalmente por partículas, adverbios, verbos y verbos auxiliares. Mientras que el segundo grupo se conforma de afijos y clíticos. La distinción y caracterización de cada uno de estos elementos es un tema de discusión dentro de la lingüística donde no existe un acuerdo definitivo para el tratamiento de estas unidades.

\subsection*{Negación con marcadores sintácticamente independientes}
\addcontentsline{toc}{subsection}{Negación con marcadores sintácticamente independientes}

\subsubsection*{Partículas y adverbios}
\addcontentsline{toc}{subsubsection}{Particulas y adverbios}

% \noindent Las distintas clases gramaticales se han definido tradicionalmente por medio de criterios semánticos, sintácticos y formales \textcolor{MidnightBlue}{\citep{asencio1981,CaleroVaquera1986}}:

% \begin{enumerate}
%     \item El criterio semántico hace referencia al significado lógico y fundamental de las unidades, de ahi, por ejemplo: sustancia (sustantivo), acción (verbo) y relaciones (preposición, conjunción y, a veces, el adverbio).
%     \item El criterio sintáctico alude tanto a la colocación de los elementos dentro de la cláusula como a su función dentro de esta.
%     \item Finalmente, el criterio formal o morfosintáctico divide los elementos en variables e invariables. Los primeros, a su vez, se clasifican en función de los accidentes gramaticales que pueden presentar (genero, número, persona, etc.)
% \end{enumerate}

\noindent El término «partícula» se utiliza para referirse de manera muy general a todos los elementos independientes dentro de una oración que no sean nombres (sustantivos, adjetivos, pronombres) y verbos. Dentro de esta categoría tan general se encuentran agrupados elementos de naturaleza heterogénea, pero que, con un poco de atención, pueden reagruparse en otras categorías mucho más reconocibles como lo son conjunciones, adposiciones y adverbios, dejando asi el término «partícula» solo para aquellos elementos que no pueden reducirse y agruparse con otras categorías.

Una partícula  es un elemento que, reducido a lo más mínimo, no cuenta con valor léxico en el plano semántico, es independiente en lo sintáctico y no presenta cambios en su forma al interactuar con otros elementos dentro de las cláusulas, es decir, es invariable.

Estas características hacen de las partículas el medio prototípico para la expresión de la negación. En efecto, los trabajos de \textcolor{MidnightBlue}{\citet{morfemas}} y \textcolor{MidnightBlue}{\citet{Vossen2016}} muestran que los marcadores sintácticamente libres son más frecuentes que los ligados (58\% ante 42\% en una muestra de 944 lenguas) donde el 91\% de los marcadores libres son partículas, dejando a los adverbios y los verbos auxiliares como estrategias poco comunes para la negación.

En noruego (VIII) y en sueco (IX) tenemos ejemplos del uso de partículas negativas ya que son elementos sin contenido semántico, invariables y que solo desempeñan la función de negar el predicado: \vspace{0.5cm}

% cSpell:disable
{\setmainfont{Charis SIL} 

\begin{tabular}{llllll}
(VIII) & Jens & skjønte & \textbf{ikke} & dette & spørmalet \\
& Juan & entender & \textsc{\textbf{neg}} & esa & pregunta \\
& \multicolumn{5}{l}{``Juan no entendió esa pregunta'' \qquad \textcolor{MidnightBlue}{\citep[pág. 7]{Taraldsen1985}}}
\end{tabular} \vspace{0.5cm}

\begin{tabular}{lllll}
(IX) & Jan & köpte & \textbf{inte} & boken \\
& Juan & compró & \textsc{\textbf{neg}} & libros \\
& \multicolumn{4}{l}{``Juan no compró libros'' \qquad \textcolor{MidnightBlue}{\citep[pág. 45]{Holmberg1995}}}
\end{tabular} \vspace{0.5cm}


} % cSpell:enable

Ya se ha mencionado que la negación no cuenta con una valor semántico en sí debido a que es un operador. Por lo tanto la manifestación de la negación por medio de un adverbio no sería la forma más esperada porque estos generalmente se clasifican siguiendo criterios según su significado: de lugar, de tiempo, de modo, de cantidad, etc. Aunque es muy frecuente que a los marcadores sintácticamente independientes se les identifique bajo este término aun sin que estos cuenten con algún valor léxico asociado. Tal es el caso del español, donde nuestro marcador negativo {\setmainfont{Charis SIL} \textit{no}} se ha clasificado histórica y tradicionalmente dentro de los adverbios.

Se habla de adverbios negativos cuando, además de negar, el marcador agrega algún matiz de significado al predicado y, opcionalmente, demande la presencia de otro elemento regido por él debido a la naturaleza mixta del adverbio como elemento con valor relacional y con contenido semántico \textcolor{MidnightBlue}{\citep{GiliGaya1973,PavonLucero2003}}.

La lengua cavineña posee un adverbio negativo al codificar la negación con un valor de posibilidad para el predicado\vspace{0.5cm}:

{\setmainfont{Charis SIL} 

% Ejemplo X aparentemente no
\begin{tabular}{llll}
(X) & \textbf{Jipakwana}=ekwana-ja & radio\textsubscript{S} & ani-ya \\
& \textbf{aparentemente.no}=\textsc{1pl-dat} & radio.ondacorta & quedar.bien-\textsc{impfv}\\
& \multicolumn{3}{l}{(lit. una radio de onda corta aparentemente no nos sentará bien)} \\
& \multicolumn{3}{l}{``Parece que no tendremos esa radio'' \qquad \textcolor{MidnightBlue}{\citep[pág. 104]{cavin}}}
\end{tabular} \vspace{0.5cm}}

\subsubsection*{Verbos auxiliares}
\addcontentsline{toc}{subsubsection}{Verbos auxiliares}

\noindent Otra forma de expresar la negación es por medio del uso de verbos auxiliares. En estos casos la marca negativa ocurre junto con categorías verbales básicas como la persona gramatical, el numero, el tiempo, el aspecto y el modo, mientras que el verbo principal permanece en una forma invariante \textcolor{MidnightBlue}{\citep{Kim2000}}. \vspace{0.5cm}

%cSpell:disable
{\setmainfont{Charis SIL} 


\begin{tabular}{llll}
(XI) & minä & \textbf{e-n} & puhu-isi \\
& yo-\textsc{nom} & \textsc{\textbf{neg-1sg}} & hablar-\textsc{cond} \\
& \multicolumn{3}{l}{``Yo no hablaría''} \qquad \textcolor{MidnightBlue}{\citep[pag. 376]{Mitchell1991}}
\end{tabular} \vspace{0.5cm}

\begin{tabular}{lllll}
(XII) & bi & dukuwūn-ma & \textbf{ə-cə̄-w} & duku-ra \\
& yo & carta-\textsc{acc} & \textsc{\textbf{neg-pst-1sg}} & escribir-\textsc{part} \\
& \multicolumn{4}{l}{``Yo no escribí una carta'' \qquad \textcolor{MidnightBlue}{\citep[pág. 213]{Payne85}}}
\end{tabular} \vspace{0.5cm}

} %cSpell:enable

El finés (XI) posee este tipo de negación. Aunque la marca negativa es propiamente {\setmainfont{Charis SIL} \textit{e}}, forma una unidad con la marca de persona gramatical dando lugar así al auxiliar {\setmainfont{Charis SIL} \textit{en}}, mientras que el verbo principal {\setmainfont{Charis SIL} \textit{puhuisi}} aparece en la forma invariante del condicional. Lo mismo ocurre con la lengua evenki (XII): el auxiliar {\setmainfont{Charis SIL} \textit{əcə̄w}} contiene la información de tiempo y persona, además de la negación, de manera independiente al verbo {\setmainfont{Charis SIL} \textit{dukura}} que aparece en otra forma invariable, la de participio.

Esta forma de negación suele confundirse con la negación de tipo morfológico, pero no hay que perder de vista que en este otro tipo de negación la marca es un afijo —un elemento ligado— que forma parte de la morfología verbal.

\subsubsection*{Posición del marcador}
\addcontentsline{toc}{subsubsection}{Posición del marcador}


\subsection*{Negación con marcadores ligados}
\addcontentsline{toc}{subsection}{Negación con marcadores ligados o dependientes}