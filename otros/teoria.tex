\section*{Negación estándar}
\addcontentsline{toc}{section}{Negación estándar}

\noindent El término «negación estándar» fue utilizado por primera vez por \textcolor{MidnightBlue}{\citet{Payne85}} y popularizado posteriormente por \textcolor{MidnightBlue}{\citet{Miestamo2005}} para referirse a la manera más básica y general que existe para negar predicados verbales en las distintas lenguas del mundo. \textcolor{MidnightBlue}{\citet{Auwera2020}} la definen como «la negación no enfática de un verbo léxico principal en una cláusula principal declarativa» y cualquier construcción que sea negativa pero no se ajuste exactamente a estos criterios se le denomina «negación no-estándar».

La negación estándar consiste —en términos muy generales— en una sencilla operación a nivel semántico donde el valor de verdad de una proposición se invierte. Y, al partir de esta aparente sencillez, se espera que su realización formal en las lenguas también lo sea. Es decir, se espera la presencia de un único marcador negativo y una misma forma de expresión de la proposición negativa a la positiva en términos estructurales \textcolor{MidnightBlue}{\citep{Auwera2020}}. Sin embargo, el fenómeno de la negación es mucho más complejo y no siempre se ajusta a esta expectativas.

\subsection*{Número de marcadores}
\addcontentsline{toc}{subsection}{Número de marcadores}

\noindent Lo más común tipológicamente es, en efecto, que la negación se exprese por medio de un solo marcador; alrededor de un 80\% de las lenguas del mundo cumplen esto \textcolor{MidnightBlue}{\citep{VanAlsenoy2014,morfemas,Vossen2016}}, pero hay lenguas que utilizan más de un marcador negativo en sus construcciones, siendo el tipo más común el dos marcadores. La presencia de dos marcadores negativos puede ser opcional u obligatoria (en este caso se le denomina «doble negación»), mientras que la presencia de tres o más siempre será opcional.

El francés (I) es famoso por sus dos marcadores negativos {\setmainfont{Charis SIL} \textit{ne}} y {\setmainfont{Charis SIL} \textit{pas}}. En algún punto de su historia ambos marcadores eran obligatorios, pero actualmente el primer negador {\setmainfont{Charis SIL} \textit{ne}} es opcional. \vspace{0.5cm}

% cSpell:disable
{\setmainfont{Charis SIL} 

\begin{tabular}{llllll}
(I) & Marie & \textbf{ne} & le & voit & \textbf{pas} \\
& María & \textsc{\textbf{neg1}} & a él & ver & \textsc{\textbf{neg2}} \\
& \multicolumn{5}{l}{``María no lo ve''}
\end{tabular} \vspace{0.5cm}

% cSpell:enable
}

Los siguientes ejemplos muestran construcciones con más de dos marcadores. La lengua Kanyok (II) tiene tres marcadores \textcolor{MidnightBlue}{\citep[pág. 263]{Devos2013}} para las construcciones negativas, pero solo los primeros dos son obligatorios: \vspace{0.5cm}

% cSpell:disable
{\setmainfont{Charis SIL} 

\begin{tabular}{lll}
(II) & \textbf{ka}-tù-tùm-in-\textbf{oh}' & \textbf{bènd} \\
& \textsc{\textbf{neg1}-1pl-}enviar-\textsc{tam-\textbf{neg2}} & \textsc{\textbf{neg3}} \\
& \multicolumn{2}{l}{``No lo hemos enviado''}
\end{tabular} \vspace{0.5cm}}
% cSpell:enable

La lengua lewo (III) posee cuatro marcadores \textcolor{MidnightBlue}{\citep[pág. 405]{Early1994}} \vspace{0.5cm}:

{\setmainfont{Charis SIL} 
% cSpell:disable
\begin{tabular}{lllll}
(III) & \textbf{pe}-\textbf{re} & a-pim & \textbf{re} & \textbf{poli}  \\
& \textsc{\textbf{neg1}-\textbf{neg2}} & \textsc{3pl-}venir.\textsc{real} & \textsc{\textbf{neg3}} & \textsc{\textbf{neg4}} \\
& \multicolumn{4}{l}{``Ellos no vinieron''}
\end{tabular} \vspace{0.5cm}}
% cSpell:enable

Y en la lengua bantawa pueden encontrarse construcciones con cinco negadores \textcolor{MidnightBlue}{\citep[pág. 271]{Doornenbal2009}}: \vspace{0.5cm}

{\setmainfont{Charis SIL} 
% cSpell:disable
\begin{tabular}{lll}
(IV) & \textbf{i}-ciŋ-\textbf{nin} & set-\textbf{nin}-Ø-\textbf{nin}-ci-\textbf{n} \\
& \textsc{\textbf{neg1}}-colgarse-\textsc{\textbf{neg2}} & matar-\textsc{\textbf{neg3}-prog-\textbf{neg4}-refl-\textbf{neg5}} \\
& \multicolumn{2}{l}{``Él no se mató ahorcándose''}
\end{tabular} \vspace{0.5cm}}
% cSpell:enable

La explicación para la presencia de más de un marcador suele ser la necesidad de «reforzar» al único marcador existente, en alguna etapa concreta de una lengua, que ha entrado en un proceso de «debilitamiento» y que por lo tanto resulta insuficiente para indicar negación. Esto es parte de un proceso más grande llamado \textbf{Ciclo de Jespersen}, nombre dado por \textcolor{MidnightBlue}{\citet{Dahl1979}} en referencia al trabajo de \textcolor{MidnightBlue}{\citet{Jespersen1917}}.

Ahora bien, el hecho de que una lengua utilice como estrategia para manifestar la negación un único marcador, no quiere decir que no existan fenómenos relacionados con ese mismo marcador. Es posible que exista un repertorio de marcadores junto con sus variaciones y que cada uno de ellos se manifieste bajo determinadas condiciones. La idea más acepta es, como indican \textcolor{MidnightBlue}{\citet{Auwera2020}}, que estas variaciones se deben a la presencia de marcadores de tiempo, aspecto y modo. \vspace{0.5cm}

{\setmainfont{Charis SIL} 
% cSpell:disable
\begin{tabular}{llll}
(V) & a. & \textbf{tera} & i=N-poroh-e \\
& & \textsc{\textbf{neg.real}} & \textsc{3m.sg=irr-}despejar.tierra-\textsc{irr.i} \\
& & \multicolumn{2}{l}{``Él no está limpiando la tierra''} \\
& & & \\
& b. & \textbf{ha}=me & pi=tsot-se-na-i=ro \\
& & \textsc{\textbf{neg.irr}=deo} & \textsc{2sg=}sorber-\textsc{clf-mal.rep-real.i=3nm.o} \\
& & \multicolumn{2}{l}{``No deberías sorberlo''} \\
\end{tabular} \vspace{1cm}
% cSpell:enable
}

En la descripción que hace \textcolor{MidnightBlue}{\citet[pág. 182, 195]{Michael2014}} sobre la lengua nanti, muestra cómo las marcas de modo determinan la presencia de un tipo específico de marcador negativo. Se utiliza {\setmainfont{Charis SIL} \textit{tera}} para el modo realis (Va) y {\setmainfont{Charis SIL} \textit{ha}} para el modo irrealis (Vb).

\subsection*{Simetría y asimetría}
\addcontentsline{toc}{subsection}{Simetría y asimetría}

\noindent Se habla de negación simétrica cuando la expresión de una proposición negativa es idéntica tanto en construcción como a nivel paradigmático a la expresión de una proposición positiva o afirmativa \textcolor{MidnightBlue}{\citep{Miestamo2005}}. Esto significa que, en muchas lenguas, las construcciones negativas solo difieren a las positivas en la presencia del marcador negativo y el paradigma verbal se mantiene igual, tal es el caso, por ejemplo, del español (VIa-b) y el alemán (VIc-d). \vspace{0.5cm}

{\setmainfont{Charis SIL}
% cSpell:disable
\begin{tabular}{lll}
(VI) & a. & María encontró flores \\
& b. & María \textbf{no} encontró flores \\
& & \\
& c. & Marie liebt ihn \\
& d. & Marie liebt ihn \textbf{nicht} \\
\end{tabular} \vspace{0.5cm}}
% cSpell:enable

El fenómeno contrario, la asimetría, se entiende entonces como un cambio estructural y paradigmático en la expresión de las proposiciones negativas. La lengua kariña (VI) ejemplifica este tipo de negación \textcolor{MidnightBlue}{\citep{Mosonyi2000}}. Es asimetría estructural porque no solo difiere en la presencia del marcador negativo y es paradigmática debido al cambio en el paradigma verbal. \vspace{0.5cm}

{\setmainfont{Charis SIL} 
% cSpell:disable
\begin{tabular}{llll}
(VII) & a. & m-oonaae \\
& & \textsc{2sg}-cultivar.\textsc{pres} \\
& & ``Tú cultivas'' \\
& & & \\
& b. & oona-\textbf{ja} & maana \\
& & cultivar-\textsc{\textbf{neg}} & \textsc{cop.2sg} \\
& & ``Tú no cultivas'' \\
\end{tabular} \vspace{0.5cm}}
% cSpell:enable

La asimetría no es un fenómeno raro. De 179 lenguas que \textcolor{MidnightBlue}{\citet{Miestamo2005}} revisa en su estudio, encuentra asimetría construccional en el 46\% de ellas y 30\% presenta asimetría paradigmática.

\subsection*{Naturaleza del marcador}
\addcontentsline{toc}{subsection}{Naturaleza del marcador}