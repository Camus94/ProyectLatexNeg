\section*{Negación estándar}

\noindent El término «negación estándar» fue utilizado por primera vez por \textcolor{MidnightBlue}{\citet{Payne85}} y popularizado posteriormente por \textcolor{MidnightBlue}{\citet{Miestamo2005}} para referirse a la manera más básica y general que existe para negar predicados verbales en las distintas lenguas del mundo. \textcolor{MidnightBlue}{\citet{Auwera2020}} la definen como «la negación no enfática de un verbo léxico principal en una cláusula principal declarativa» y cualquier construcción que sea negativa pero no se ajuste exactamente a estos criterios se le denomina «negación no-estándar».

La negación estándar consiste —en términos muy generales— en una sencilla operación a nivel semántico donde el valor de verdad de una proposición se invierte. Y, al partir de esta aparente sencillez, se espera que su realización formal en las lenguas también lo sea. Es decir, se espera la presencia de un único marcador negativo y una misma forma de expresión de la proposición negativa a la positiva en términos estructurales \textcolor{MidnightBlue}{\citep{Auwera2020}}. Sin embargo, el fenómeno de la negación es mucho más complejo y no siempre se ajusta a esta expectativas.

Lo más común tipológicamente es, en efecto, que la negación se exprese por medio de un solo marcador; alrededor de un 80\% de las lenguas del mundo cumplen esto \textcolor{MidnightBlue}{\citep{VanAlsenoy2014,morfemas,Vossen2016}}, pero hay lenguas que utilizan más de un marcador negativo en sus construcciones, siendo el tipo más común el dos marcadores. La presencia de dos marcadores negativos puede ser opcional u obligatoria (en este caso se le denomina «doble negación»), mientras que la presencia de tres o más siempre será opcional.

La explicación para la presencia de más de un marcador suele ser la necesidad de «reforzar» al único marcador existente, en alguna etapa concreta de una lengua, que ha entrado en un proceso de «debilitamiento» y que por lo tanto resulta insuficiente para indicar negación. Esto es parte de un proceso más grande llamado \textbf{Ciclo de Jespersen}, nombre dado por \textcolor{MidnightBlue}{\citet{Dahl1979}} en referencia al trabajo de \textcolor{MidnightBlue}{\citet{Jespersen1917}}.

Ahora bien, el hecho de que una lengua utilice como estrategia para manifestar la negación un único marcador, no quiere decir que no existan fenómenos relacionados con ese mismo marcador. Es posible que exista un repertorio de marcadores junto con sus variaciones y que cada uno de ellos se manifieste bajo determinadas condiciones. La idea más acepta es, como indican \textcolor{MidnightBlue}{\citet{Auwera2020}}, que estas variaciones se deben a la presencia de marcadores de tiempo, aspecto y modo.
