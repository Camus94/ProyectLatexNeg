\section*{Abreviaturas genenerales para las glosas}
\addcontentsline{toc}{section}{Lista de abreviaturas}

\noindent En la siguiente relación se encuentran las abreviaturas comunes a todas las gramáticas utilizadas para este trabajo. Las glosas particulares se especifican de manera individual por cada lengua en el pie de página.


\begin{table}[htbp]
    % \centering
      \begin{tabular}{ll}
      1     & Primera persona \\
      2     & Segunda persona \\
      3     & Tercera persona \\
      \textsc{cl}.   & Clítico \\
      \textsc{cond}. & Condicional \\
      \textsc{dem}.  & Demostrativo \\
      \textsc{det}.  & Determinante \\
      \textsc{dim}.  & Diminutivo \\
      \textsc{dur}.  & Durativo \\
      \textsc{exist}. & Existencial \\
      \textsc{foc}.  & Foco \\
      \textsc{fut}.  & Futuro \\
      \textsc{imp}.  & Imperativo \\
      \textsc{inf}.  & Infinitivo \\
      \textsc{irr}.  & Irrealis \\
      \textsc{neg}.  & Negación \\
      \textsc{obj}.  & Objeto \\
      \textsc{opt}.  & Optativo \\
      \textsc{perf}. & Perfecto \\
      \textsc{pl}.   & Plural \\
      \textsc{pos}.  & Posesivo \\
      \textsc{posd}. & Poseedor \\
      \textsc{prep}. & Preposición \\
      \textsc{pres}. & Presente \\
      \textsc{pro}.  & Pronombre \\
      \textsc{psd}.  & Pasado \\
      \textsc{rl}.   & Realis \\
      \textsc{sg}.   & Singular \\
      \textsc{x}.    & Valor desconocido o no relevante en el ejemplo \\
      \end{tabular}
    \label{glosas}
  \end{table}