\section*{Lenguas de México - Estrategias}

% Table generated by Excel2LaTeX from sheet 'Mecanismos'
% \setcounter{table}{0}
% \begin{table}[htbp]
%     \centering
%       \begin{tabular}{lccc}
%       \textbf{Lengua} & \textbf{Léxico independiente} & \textbf{Afijos} & \textbf{Clíticos} \\
%       \hline
%       Acateco & X     &       &  \\
%       Chatino & X     &       &  \\
%       Chichimeco & X     & X     &  \\
%       Chinanteco & X     & X     &  \\
%       Chontal de Oaxaca & X     &       &  \\
%       Chontal de Tabasco & X     &       &  \\
%       Huave &       & X     &  \\
%       Huichol &       & X     &  \\
%       Lacandón & X     &       &  \\
%       Mixe  & X     & X     &  \\
%       Mixteco & X     &       &  \\
%       Oluteco &       &       & X \\
%       Otomí & X     &       &  \\
%       Sierra Popoluca & X     &       &  \\
%       Seri  &       & X     &  \\
%       Tarahumara & X     &       &  \\
%       Tepehua & X     &       &  \\
%       Tepehuano & X     &       &  \\
%       Tlahuica &       & X     &  \\
%       Tlapaneco &       & X     &  \\
%       Totonaco & X     & X     &  \\
%       Triqui & X     &       &  \\
%       Tsotsil & X     & X     &  \\
%       Zapoteco & X     &       &  \\
%       Zoque & X     & X     &  \\
%       \hline
%       \end{tabular}
%       \caption{Lenguas de México}
%     \label{cuadro1}
%   \end{table}
  
\noindent De las 25 lenguas que corresponden a México, solo 13 de ellas recurren a léxico independiente para la marcarción de la negacion, 5 al uso de afijos exclusivamente, 6 usan tanto afijos como léxico independiene y solo 1 lengua recurre al uso de cliticos.

El hecho de que algunas lenguas recurran tanto a léxico independiente como al uso de afijos no quiere decir, en un primer momento, que sean lenguas de doble negación. El uso de estas estrategias puede alternar por el tipo de negación: clausal, de constituyentes o de imperativos por ejemplo.

% Table generated by Excel2LaTeX from sheet 'Mex solo lexico'
% \begin{table}[htbp]
%     \centering
%       \begin{tabular}{lc}
%       \multicolumn{1}{c}{\textbf{Lengua}} & \multicolumn{1}{c}{\textbf{léxico independiente}} \\
%       \hline
%       Acateco & {\setmainfont{Charis SIL} \textit{maː}} / {\setmainfont{Charis SIL} \textit{man}} / {\setmainfont{Charis SIL} \textit{k’am}} / {\setmainfont{Charis SIL} \textit{ʔamax}} \\
%       Chatino & {\setmainfont{Charis SIL} \textit{a3}} \\
%       Chontal de Oaxaca & {\setmainfont{Charis SIL} \textit{maa}} \\
%       Chontal de Tabasco & {\setmainfont{Charis SIL} \textit{mach}}/ {\setmainfont{Charis SIL} \textit{mach’an}} / {\setmainfont{Charis SIL} \textit{mame’}} / {\setmainfont{Charis SIL} \textit{machme’}} / {\setmainfont{Charis SIL} \textit{moni’}} / {\setmainfont{Charis SIL} \textit{mani’}} \\
%       Lacandón & {\setmainfont{Charis SIL} \textit{maʔ}} \\
%       Mixteco & {\setmainfont{Charis SIL} \textit{ko̱}} / {\setmainfont{Charis SIL} \textit{ta’on}} \\
%       Otomí & {\setmainfont{Charis SIL} \textit{hinɡi}} / {\setmainfont{Charis SIL} \textit{him}} / {\setmainfont{Charis SIL} \textit{hi}} / {\setmainfont{Charis SIL} \textit{hin}} \\
%       Sierra Popoluca & {\setmainfont{Charis SIL} \textit{ʔotʔoy}} / {\setmainfont{Charis SIL} \textit{dya}} \\
%       Tarahumara & {\setmainfont{Charis SIL} \textit{’ka’t͡ʃè}} / {\setmainfont{Charis SIL} \textit{ke’tâsi}} / {\setmainfont{Charis SIL} \textit{’kíti}} / {\setmainfont{Charis SIL} \textit{ke}} \\
%       Tepehua & {\setmainfont{Charis SIL} \textit{jaantu}} \\
%       Tepehuano & {\setmainfont{Charis SIL} \textit{cham}} tu’ / {\setmainfont{Charis SIL} \textit{cham}} \\
%       Triqui & {\setmainfont{Charis SIL} \textit{ne³}} / {\setmainfont{Charis SIL} \textit{nuveé⁴}} / {\setmainfont{Charis SIL} \textit{se²}} \\
%       Zapoteco & {\setmainfont{Charis SIL} \textit{gàgé}} / {\setmainfont{Charis SIL} \textit{àgé}} / {\setmainfont{Charis SIL} \textit{bìtò}} / {\setmainfont{Charis SIL} \textit{bì}} \\
%       \hline
%       \end{tabular}%
%       \caption{Lenguas que codifican la negación en léxico o partículas independientes}
%     \label{cuadro2}%
%   \end{table}
  
  % Table generated by Excel2LaTeX from sheet 'Mex solo lexico'
% \begin{table}[htbp]
%     \centering
%       \begin{tabular}{lc}
%       \multicolumn{1}{c}{\textbf{Lengua}} & \textbf{morfema} \\
%       Huave & {\setmainfont{Charis SIL} \textit{ⁿɡo-}} / -{\setmainfont{Charis SIL} \textit{hiⁿd}} / {\setmainfont{Charis SIL} \textit{ni}}- \\
%       Huichol & -{\setmainfont{Charis SIL} \textit{ka}} / -{\setmainfont{Charis SIL} \textit{mawe}} \\
%       Seri  & -{\setmainfont{Charis SIL} \textit{m}} \\
%       Tlahuica & {\setmainfont{Charis SIL} \textit{tét}}- / {\setmainfont{Charis SIL} \textit{té}}- / {\setmainfont{Charis SIL} \textit{nó}}- \\
%       Tlapaneco & {\setmainfont{Charis SIL} \textit{ta¹ga³}}- \\
%       \end{tabular}%
%       \caption{Lenguas que condifcan la negación en afijos}
%     \label{tab:addlabel}%
%   \end{table}%
  