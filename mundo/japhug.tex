\section*{Japhug}
\addcontentsline{toc}{section}{Japhug}

\noindent El japhug es una lengua tibetano-birmana hablada en la provincia china de Sichuan. Pertenece al subgrupo rgyalrongui dentro de las lenguas qiang. Es una lengua tonal con cuatro tonos fonológicos.
%%%%%%%%%%%%% Abreviaturas %%%%%%%%%%%%%%%%%%%%%
\footnote{AOR: aoristo, CAUS: causativo, DEM: demostrativo, DU: dual, EMPH: enfático, INDEF: indefinido, FACT: factual no pasado, LNK: enlace (linker), LOC: locativo, POSS: posesivo, PROXM: proximal, PST: pasado, SENS: sensorial, TR: transitivo, TRAL: translocativo}
%%%%%%%%%%%%%%%%%%%%%%%%%%%%%%%%%%%%%%%%%%%%%%%%
\vspace{0.5cm}

{\setmainfont{Charis SIL} 

% Ejemplo 217
\noindent \begin{tabular}{llll}
(217) & jɯfɕɯr & jɯ-\textbf{mɤ}-ɕ-tɤ-tʰu-t-a & ʑo \\
& ayer & \textsc{proxm-\textbf{neg}-tral-aor-}preguntar-\textsc{pst:tr-1sg} & \textsc{emph} \\
& \multicolumn{3}{l}{``Ayer casi no fui a preguntar sobre eso'' (pág. 514)}
\end{tabular} \vspace{0.5cm}

% Ejemplo 218
\noindent \begin{tabular}{lllllll}
(218) & tɤ-mu & nɯ & ɣɯjpa & kɯre & mɤɕtʂa & \textbf{mɯ}-nɯ-si \\
& \textsc{indef.poss-madre} & \textsc{dem} & este.año &\textsc{dem.loc} & hasta & \textsc{\textbf{neg}-aor-}morir \\
& \multicolumn{6}{l}{``La anciana no murió hasta este año'' (pág. 514)}
\end{tabular} \vspace{0.5cm}

% Ejemplo 219
\noindent \begin{tabular}{llll}
(219) & ʑa & \textbf{ma}-tɤ-nɯna-tɕi & qʰe \\
& pronto & \textsc{\textbf{neg}-imp-}descansar-\textsc{1du} & \textsc{lnk}\\
& \multicolumn{3}{l}{``(Para recuperar el tiempo perdido), no dejemos de (trabajar)} \\ 
& \multicolumn{3}{l}{temprano (hoy)'' (pág. 1130)}
\end{tabular} \vspace{0.5cm}

% Ejemplo 220
\noindent \begin{tabular}{lllll}
(220) & \textbf{mɯ́j}-tɯ-mbɣom & ri & tʰa & kɯ-z-maqʰu-tɕi \\
& \textsc{\textbf{neg}:sens-2-}tener.prisa & \textsc{lnk} & tarde & \textsc{2→1caus-}llegar.tarde:\textsc{fact-1du} \\
& \multicolumn{4}{l}{``(si) no te apuras, llegarás tarde'' (pág. 553)}
\end{tabular} \vspace{0.5cm}

% Ejemplo 221
\noindent \begin{tabular}{lll}
(221) & pɯ-mto-t-a & \textbf{maʁ} \\
& \textsc{aor-}ver-\textsc{pst:tr-1-sg} & \textsc{cop.\textbf{neg}:fact} \\
& \multicolumn{2}{l}{``No lo he visto''}
\end{tabular} \vspace{0.5cm}

}


En la lengua Japhug, la negación se manifiesta mayoritariamente a través de prefijos negativos antepuestos al verbo \textcolor{MidnightBlue}{\citep{japhug}}. Además del uso de prefijos negativos, también existe una construcción negativa perifrástica que emplea un auxiliar negativo al final de la oración. Esta construcción perifrástica es requerida para expresar casos específicos como la doble negación.

Cuatro son los prefijos negativos reconocidos en esta lengua: {\setmainfont{Charis SIL} \textit{mɤ}}- (217), {\setmainfont{Charis SIL} \textit{mɯ}}- (218), {\setmainfont{Charis SIL} \textit{ma}}- (219) y {\setmainfont{Charis SIL} \textit{mɯ́j}}- (220). La negación perifrástica se construye con la cópula negativa {\setmainfont{Charis SIL} \textit{maʁ}} (221) con o el verbo existencial negativo {\setmainfont{Charis SIL} \textit{me}} en posición posverbal.