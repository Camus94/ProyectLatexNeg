\section*{Moloko}
\addcontentsline{toc}{section}{Moloko}

\noindent El moloko es una lengua adamawa-ubangi hablada por unas 200,000 personas en el centro de Chad.
%%%%%%%%%%%%% Abreviaturas %%%%%%%%%%%%%%%%%%%%%
\footnote{CL: clase (verbal), IFV: imperfectivo, DO: objeto directo, PFV: perfectivo, POT: potencial, S: singular}
%%%%%%%%%%%%%%%%%%%%%%%%%%%%%%%%%%%%%%%%%%%%%%%%
\vspace{0.5cm}

{\setmainfont{Charis SIL} 

% Ejemplo 242
\begin{tabular}{lll}
(242) & à-l=ala & \textbf{baj} \\
& \textsc{3s+pfv-}ir=hacia & \textsc{\textbf{neg}} \\
& \multicolumn{2}{l}{``Él/Ella no vino'' (pág. 312)}
\end{tabular} \vspace{0.5cm}

% Ejemplo 243
\begin{tabular}{lll}
(243) & nóo-lo & \textbf{asabay} \\
& \textsc{1s+pot-}ir & otra.vez+\textsc{\textbf{neg}} \\
& \multicolumn{2}{l}{``No iré de nuevo'' (pág. 314)}
\end{tabular} \vspace{0.5cm}

% Ejemplo 244
\begin{tabular}{llll}
(244) & né-g-e & na & \textbf{fabay} \\
& \textsc{1s+ifv-}hacer-\textsc{cl} & \textsc{3s.do} & todavía+\textsc{\textbf{neg}} \\
& \multicolumn{3}{l}{``No lo he hecho todavía'' (pág. 314)}
\end{tabular} \vspace{0.5cm}

% Ejemplo 245
\begin{tabular}{lllll}
(245) & nə-mənjar & ndahan & \textbf{kəlo} & \textbf{bay} \\
& \textsc{1s-}ver & \textsc{3s} & antes & \textsc{\textbf{neg}} \\
& \multicolumn{4}{l}{``No la había visto antes'' (pág. 214)}
\end{tabular} \vspace{0.5cm}

% Ejemplo 246
\begin{tabular}{llll}
(246) & káa-z=ala & \textbf{təta} & \textbf{bay} \\
& \textsc{2s+pot-}llevar=hacia & tener.habilidad & \textsc{\textbf{neg}} \\
& \multicolumn{3}{l}{``No puedes traerlo'' (pág. 214)}
\end{tabular} \vspace{0.5cm}

}

En esta lengua, para indicar la negación clausal así como la de constituyentes se utiliza la partícula negativa {\setmainfont{Charis SIL} \textit{baj}} (242) o un compuesto con esta; {\setmainfont{Charis SIL} \textit{asabay}} (243), {\setmainfont{Charis SIL} \textit{fabay}} (244), {\setmainfont{Charis SIL} \textit{kəlo bay}} (245) y {\setmainfont{Charis SIL} \textit{təta bay}} (246), los cuales se ubican al final de la cláusula o elemento nominal que se desea negar \textcolor{MidnightBlue}{\citep{moloko}}.