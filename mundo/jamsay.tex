\section*{Jamsay}
\addcontentsline{toc}{section}{Jamsay}

\noindent El jamsay es una lengua mandé del grupo mandinga que se habla principalmente en Mali. Es una de las pocas lenguas mandé de carácter tonal. 
%%%%%%%%%%%%% Abreviaturas %%%%%%%%%%%%%%%%%%%%%
\footnote{DAT: dativo, IMPF: imperfectivo, L: tono bajo (low) LOGO: logofórico, NONH: no humano, P: poseedor, PERF: perfectivo, POSS: poseedor S: sujeto}
%%%%%%%%%%%%%%%%%%%%%%%%%%%%%%%%%%%%%%%%%%%%%%%%
\vspace{0.5cm}

{\setmainfont{Charis SIL} 

% Ejemplo 212
\begin{tabular}{ll}
(212) & yàː-\textbf{lì}-∅ \\
& ir-\textsc{perf.\textbf{neg}-3sgs} \\
& ``Él/Ella no fue'' (pág. 368)
\end{tabular} \vspace{0.5cm}

% Ejemplo 213
\begin{tabular}{llllll}
(213) & ɛ̀nɛ́ & mà & úró & ɛ̀jù-\textbf{lá}-∅ wá \\
& \textsc{logo.p} & \textsc{poss} & casa & bueno.\textsc{\textbf{neg}.3sgs} \\
& \multicolumn{5}{l}{``Dijo que su (propia) casa no sirve'' (pág. 235)}
\end{tabular} \vspace{0.5cm}

% Ejemplo 214
\begin{tabular}{ll}
(214) & ɛ̀ː-\textbf{lú}-m \\
& ver-\textsc{perf.\textbf{neg}-1sgs}\\
& ``No lo vi'' (pág. 368)
\end{tabular} \vspace{0.5cm}

% Ejemplo 215
\begin{tabular}{llll}
(215) & ɔ́ɣɔ́rɔ́ & kò-rú & yɔ̀wɔ̀-\textbf{l}-á \\
& rápidamente & \textsc{nonh-dat} & aceptar-\textsc{\textbf{neg}.perf-3p1s} \\
& \multicolumn{3}{l}{``No lo aceptaron rápidamente (el arado)'' (pág. 368)}
\end{tabular} \vspace{0.5cm}

% Ejemplo 216
\begin{tabular}{llll}
(216) & èjù-nɔ̀wⁿɔ̀ & àː-\textbf{ɡóː}-∅ & fúː \\
& campo.\textsc{l}-carne.\textsc{l} & atrapar-\textsc{\textbf{neg}.impf-p.pl.nonh} & todos \\
& \multicolumn{3}{l}{``todo/cualquier (tipo de) animal que (la trampa) no atrapa'' (pág. 229)}
\end{tabular} \vspace{0.5cm}

}

En los sistemas aspectuales tanto perfectivos como imperfectivos existe una fuerte tendencia a simplificar el relativamente complejo sistema de categorías aspectuales para la negación \textcolor{MidnightBlue}{\citep{jamsay}}, reduciéndolas solo a dos grandes categorías. [1] Para el aspecto perfectivo:  el prefijo {\setmainfont{Charis SIL} \textit{-lì}} para primera y segunda persona plural y tercera singular (212), en algunos casos {\setmainfont{Charis SIL} \textit{-lá}} para tercera (213), {\setmainfont{Charis SIL} \textit{-lú}} para primera y segunda persona singular (214) y {\setmainfont{Charis SIL} \textit{-l}} para tercera persona plural (215). [2] Para el aspecto imperfectivo:  El prefijo {\setmainfont{Charis SIL} \textit{-ɡo'}} (216).

Esta lengua exhibe un complejidad enorme en toda categoría aspectual y existen numerosas variaciones sobre la marcación de la negación, las cuales escapan al alcance de este trabajo. Los morfemas mostrados en los ejemplos son solo la parte introductoria de un mundo mucho infinitamente más grande.