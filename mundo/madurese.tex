\section*{Madurese}
\addcontentsline{toc}{section}{Madurese}

\noindent El madurés es la variedad de la lengua javanesa que se habla en la isla de Madura, situada frente a la costa noreste de Java, en Indonesia.
%%%%%%%%%%%%% Abreviaturas %%%%%%%%%%%%%%%%%%%%%
\footnote{AV: voz activa, DEF: definido, NOM: nominalización, RED: reduplicación}
%%%%%%%%%%%%%%%%%%%%%%%%%%%%%%%%%%%%%%%%%%%%%%%%
\vspace{0.5cm}

{\setmainfont{Charis SIL} 

% Ejemplo 231
\begin{tabular}{llll}
(231) & Hasan & \textbf{ta'}=mokol & Bambang \\
& Hasan & \textsc{neg=av}.golpear & Bambang \\
& \multicolumn{3}{l}{``Hasan no golpeó a Bambang'' (pág. 237)}
\end{tabular} \vspace{0.5cm}

% Ejemplo 232
\begin{tabular}{llllll}
(232) & Red-mored-da & \textbf{ta'}=bisa & maca & buku & reya \\
& \textsc{red-}estudiante-\textsc{def} & \textsc{\textbf{neg}}=poder & \textsc{av.}leer & libro & este \\
& \multicolumn{5}{l}{``Los estudiantes no pueden leer este libro'' (pág. 274)}
\end{tabular} \vspace{0.5cm}

% Ejemplo 233
\begin{tabular}{llll}
(233) & Deni & \textbf{lo'}=ngarte & jawab-an \\
& Deni & \textsc{\textbf{neg}=}entender & respuesta.\textsc{nom} \\
& \multicolumn{3}{l}{``Deni no entiende la respuesta'' (pág. 274)} 
\end{tabular} \vspace{0.5cm}

% Ejemplo 234
\begin{tabular}{lllll}
(234) & Ali & \textbf{lo'}=kodu & mokol & ale'-eng \\
& Ali & \textsc{\textbf{neg}}=deber(obligación) & \textsc{av.}golpear & hermano.menor-\textsc{def} \\
& \multicolumn{4}{l}{``Ali no debe golpear a su hermanito'' (pág. 274)}
\end{tabular} \vspace{0.5cm}

% Ejemplo 235
\begin{tabular}{lllll}
(235) & \textbf{Ja'} & entar & dha’ & Jakarta! \\
& \textsc{\textbf{neg}} & ir & hacia & Jakarta \\
& \multicolumn{4}{l}{``No vayas a Jakarta'' (pág. 274)}
\end{tabular} \vspace{0.5cm}

% Ejemplo 236
\begin{tabular}{lll}
(236) & \textbf{Ja}' & dhak-gendhak! \\
& \textsc{\textbf{neg}} & \textsc{red}-ser.arrogante \\
& \multicolumn{2}{l}{``No seas arrogante'' (pág. 274)}
\end{tabular} \vspace{0.5cm}

}

\textcolor{MidnightBlue}{\citet{madu}} señala que los verbos se niegan mediante clíticos preverbales. El clitico {\setmainfont{Charis SIL} \textit{ta'=}} se considera la forma estandar y es usado en el Este de la isla de Madura (231) y (232), mientras que la forma {\setmainfont{Charis SIL} \textit{lo'=}} se usa en el Oeste pero se considera variación dialectal (233) y (234).

Por su parte, los imperativos se niegan con la partícula independiente {\setmainfont{Charis SIL} \textit{(n)ja'}} antepuesta al verbo (235) y (236).