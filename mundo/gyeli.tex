\section*{Gyeli}
\addcontentsline{toc}{section}{Gyeli}

\noindent El gyeli es una lengua bantú hablada por unas pocas miles de personas en el sur de Camerún, cerca de la frontera con Gabón y Guinea Ecuatorial. Pertenece al grupo A de las lenguas bantúes.
%%%%%%%%%%%%% Abreviaturas %%%%%%%%%%%%%%%%%%%%%
\footnote{{\setmainfont{Charis SIL} \textit{∅}}: sustantivo sin prefijo, COM: comitativo, LOC: locativo, PRS: presente, PST: pasado, R: realis, RECIP: recíproco}
%%%%%%%%%%%%%%%%%%%%%%%%%%%%%%%%%%%%%%%%%%%%%%%%
\vspace{0.3cm}

{\setmainfont{Charis SIL} 

% Ejemplo 206
\begin{tabular}{lll}
(206) & ba-H & dyû-\textbf{lɛ} \\
& \textsc{2-prs} & matar-\textsc{\textbf{neg}} \\
& \multicolumn{2}{l}{``Ellos no matan'' (pág. 369)}
\end{tabular} \vspace{0.3cm}

% Ejemplo 207
\begin{tabular}{llll}
(207) & ba & \textbf{sàlɛ́} & dyú(w)-ala \\
& 2 & \textsc{\textbf{neg}.pst} & matar-\textsc{recip}\\
& \multicolumn{3}{l}{``Ellos no se mataron'' (pág. 370)}
\end{tabular} \vspace{0.3cm}

% Ejemplo 208
\begin{tabular}{lllllllll}
(208) & [mɛ̀ɛ̀ & \textbf{kálɛ̀} & ná & bɛ̀ & nà] & jí & ɛ́ & vâ \\
& \textsc{1sg.fut} & \textsc{\textbf{neg}.fut} & aun & ser & \textsc{com} & \textsc{∅7}.lugar & \textsc{loc} & aquí \\
& \multicolumn{8}{l}{``Ya no tendré un lugar aquí'' (pág. 418)}
\end{tabular} \vspace{0.3cm}

% Ejemplo 209
\begin{tabular}{llll}
(209) & be-H & \textbf{dúù}-H & vũ̀ ũ̀ \\
& \textsc{2pl-prs} & must.\textsc{\textbf{neg}-r} & preocuparse \\
& \multicolumn{3}{l}{``Ustedes no deberían preocuparse'' (pág. 424)}
\end{tabular} \vspace{0.3cm}

% Ejemplo 210
\begin{tabular}{llllll}
(210) & mɛ & \textbf{kí} & bɛ̀ & nà & tsídí \\
& \textsc{1sg} & \textsc{\textbf{neg}} & ser & \textsc{com} & \textsc{∅1.}carne \\
& \multicolumn{5}{l}{``No tengo carne'' (pág. 451)}
\end{tabular} \vspace{0.3cm}

% Ejemplo 211
\begin{tabular}{llllllll}
(211) & a & múà & nà & bábɛ̀ & \textbf{tí} & wúmbɛ & wɛ̀ \\
& 1 & ser & \textsc{com} & \textsc{∅7.}enfermo & \textsc{\textbf{neg}} & querer-\textsc{r} & morir \\
& \multicolumn{7}{l}{``Estaba enfermo y no quería morir'' (pág. 553)}
\end{tabular} \vspace{0.5cm}

}

La lengua Gyeli emplea distintos marcadores y estrategias para la negación, según las categorías de tiempo y modo \textcolor{MidnightBlue}{\citep{gyeli}} teniendo en total 5 morfemas, tres para la negación estandar: [1] el sufijo {\setmainfont{Charis SIL} \textit{-lɛ}} para el presente (206), [2] los auxiliares {\setmainfont{Charis SIL} \textit{sàlɛ́/pálɛ}} para los tiempos de pasado (207), [3] el auxiliar {\setmainfont{Charis SIL} \textit{kálɛ̀}} para el futuro (208) y dos para le negación clausal no estandar: [4] el semi-auxiliar {\setmainfont{Charis SIL} \textit{dúù}} en construcciones de subjuntivo y presente (209), [5] el auxiliar {\setmainfont{Charis SIL} \textit{kí/tí}} para el presente (210), imperativos e infinitivos (211). Tanto en el pasado reciente como en el pasado remoto los verbos auxiliares negativos {\setmainfont{Charis SIL} \textit{sàlɛ́}} y {\setmainfont{Charis SIL} \textit{pálɛ́}} parecen ser intercambiables libremente.