\section*{Fwe}

\noindent El fwe es una lengua bantú erteneciente a la rama occidental del subgrupo Bantu Botatwe, siendo su pariente más cercano el shanjo. En cuanto a la variación dialectal, se distinguen principalmente el fwe de Zambia y el de Namibia, con algunas diferencias fonológicas y morfológicas. Estas son áreas rurales alrededor de la localidad de Kongola. \vspace{0.5cm}

{\setmainfont{Charis SIL} 

% Ejemplo N
\begin{tabular}{ll}
() & \textbf{ka}-ndi-ur-\textbf{í̠} \\
& \textsc{\textbf{neg}-1sg.suj}-comprar-\textsc{\textbf{neg}}\\
& ``Yo no compro'' (Namibian Fwe) (pág. 418)
\end{tabular} \vspace{0.3cm}

% Ejemplo N2
\begin{tabular}{ll}
() & \textbf{tà}-ndi-ur-\textbf{í̠} \\
& \textsc{\textbf{neg}-1sg.suj}-comprar-\textsc{\textbf{neg}}\\
 & ``Yo no compro'' (Zambian Fwe) (pág. 418)
\end{tabular} \vspace{0.3cm}

% Ejemplo N3
\begin{tabular}{ll}
() & \textbf{ka}-ndi-zibá̠r-\textbf{i} \\
& \textsc{\textbf{neg}-1sg.suj}-olvidar-\textsc{\textbf{neg}}\\
& ``No me olvido'' (Namibian Fwe) (pág. 418)
\end{tabular} \vspace{0.3cm}

% Ejemplo N4
\begin{tabular}{ll}
() & \textbf{ta}-tu-kat-ite-\textbf{í̠} \\
& \textsc{\textbf{neg}-1pl.suj}-volverse.delgado-\textsc{est-\textbf{neg}} \\
& ``No estamos delgados'' (Zambian Fwe) (pág. 420)
\end{tabular} \vspace{0.3cm}

% Ejemplo N5
\begin{tabular}{lll}
() & mu-\textbf{ásha}-bútuk-\textbf{i} & cáha \\
& \textsc{2pl.suj-\textbf{neg}.sbjv-}correr-\textsc{\textbf{neg}} & muy \\
& \multicolumn{2}{l}{``No vayas tan rápido'' (Namibian Fwe) (pág. 421)}
\end{tabular} \vspace{0.3cm}

% Ejemplo N6
\begin{tabular}{ll}
() & ku-\textbf{shá}-bon-a  \\
& \textsc{inf-\textbf{neg}.inf}-ver-\textsc{vf} \\
& ``No ver'' (pág. 422)
\end{tabular} \vspace{0.3cm}

}

La negación se hace por medio de afijos verbales, auxiliares y la combinación de ambos \textcolor{MidnightBlue}{\citep{fwe}}. Los prefijos {\setmainfont{Charis SIL} \textit{ka-, ta-}} se utilizan para la negación de verbos en indicativo () y ocupan una posición preinicial. Los prefijos {\setmainfont{Charis SIL} \textit{ásha-, shá-}} aparecen con verbos en subjuntivo e infinitos respectivamente () en posiciones postininiciales. Adicionalmente, está la vocal sufijada {\setmainfont{Charis SIL} \textit{-i}} que aparece en ciertas construcciones acompañando a los prefijos negativos, pero nunca puede aparecer como el único marcador negativo en la construcción(). 