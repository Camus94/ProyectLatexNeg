\section*{Fwe}
\addcontentsline{toc}{section}{Fwe}

\noindent El fwe es una lengua bantú perteneciente a la rama occidental del subgrupo Bantu Botatwe, siendo su pariente más cercano el shanjo. En cuanto a la variación dialectal, se distinguen principalmente el fwe de Zambia y el de Namibia, con algunas diferencias fonológicas y morfológicas. Estas son áreas rurales alrededor de la localidad de Kongola.
%%%%%%%%%%%%% Abreviaturas %%%%%%%%%%%%%%%%%%%%%
\footnote{FV: vocal final, INF: infinitivo, SBJV: subjuntivo, SM: marcador de sujeto, STAT: estativo}
%%%%%%%%%%%%%%%%%%%%%%%%%%%%%%%%%%%%%%%%%%%%%%%%
\vspace{0.5cm}

{\setmainfont{Charis SIL} 

% Ejemplo 200
\begin{tabular}{ll}
(200) & \textbf{ka}-ndi-ur-\textbf{í̠} \\
& \textsc{\textbf{neg}-sm1sg}-comprar-\textsc{\textbf{neg}}\\
& ``Yo no compro'' (Namibian Fwe) (pág. 418)
\end{tabular} \vspace{0.3cm}

% Ejemplo 201
\begin{tabular}{ll}
(201) & \textbf{tà}-ndi-ur-\textbf{í̠} \\
& \textsc{\textbf{neg}-sm1sg}-comprar-\textsc{\textbf{neg}}\\
 & ``Yo no compro'' (Zambian Fwe) (pág. 418)
\end{tabular} \vspace{0.3cm}

% Ejemplo 202
\begin{tabular}{ll}
(202) & \textbf{ka}-ndi-zibá̠r-\textbf{i} \\
& \textsc{\textbf{neg}-sm1sg}-olvidar-\textsc{\textbf{neg}}\\
& ``No me olvido'' (Namibian Fwe) (pág. 418)
\end{tabular} \vspace{0.3cm}

% Ejemplo 203
\begin{tabular}{ll}
(203) & \textbf{ta}-tu-kat-ite-\textbf{í̠} \\
& \textsc{\textbf{neg}-sm1pl}-volverse.delgado-\textsc{stat-\textbf{neg}} \\
& ``No estamos delgados'' (Zambian Fwe) (pág. 420)
\end{tabular} \vspace{0.3cm}

% Ejemplo 204
\begin{tabular}{lll}
(204) & mu-\textbf{ásha}-bútuk-\textbf{i} & cáha \\
& \textsc{sm2pl-\textbf{neg}.sbjv-}correr-\textsc{\textbf{neg}} & muy \\
& \multicolumn{2}{l}{``No vayas tan rápido'' (Namibian Fwe) (pág. 421)}
\end{tabular} \vspace{0.3cm}

% Ejemplo 205
\begin{tabular}{ll}
(205) & ku-\textbf{shá}-bon-a  \\
& \textsc{inf-\textbf{neg}.inf}-ver-\textsc{fv} \\
& ``No ver'' (pág. 422)
\end{tabular} \vspace{0.3cm}

}

La negación se hace por medio de afijos verbales, auxiliares y la combinación de ambos \textcolor{MidnightBlue}{\citep{fwe}}. Los prefijos {\setmainfont{Charis SIL} \textit{ka-, ta-}} se utilizan para la negación de verbos en indicativo (200) - (203) ocupan una posición pre-inicial. Los prefijos {\setmainfont{Charis SIL} \textit{ásha-, shá-}} aparecen con verbos en subjuntivo e infinitos respectivamente (204) y (205) en posiciones post-iniciales. Adicionalmente, está la vocal sufijada {\setmainfont{Charis SIL} \textit{-i}} que aparece en ciertas construcciones acompañando a los prefijos negativos, pero nunca puede aparecer como el único marcador negativo en la construcción. 