\section*{Palula}
\addcontentsline{toc}{section}{Palula}

\noindent El palula es una lengua dardica que se habla en los valles de Biori y Patan, en el norte de Pakistán. Pertenece a la rama shina del grupo dardico occidental.
%%%%%%%%%%%%%%%%%%%%%%%%%% ABREVIATURAS %%%%%%%%%%%%%%%%%%%%
\footnote{ADJ: adjetivizador, ERG: ergativo, MSG: masculino singular, NOM: nominativo,  PFV: perfectivo, PPTC: participio perfectivo, PROX: proximal, PRS: presente, REFL: reflexivo}
%%%%%%%%%%%%%%%%%%%%%%%%%%%%%%%%%%%%%%%%%%%%%%%%
\vspace{0.5cm}

{\setmainfont{Charis SIL} 

% Ejemplo 252
\begin{tabular}{llllll}
(252) & amzarái & muṛ-u=bhaáu & insaán & \textbf{na} & kha-áan-u \\
& león & morir.\textsc{pptc-msg-adj} & ser.humano & \textsc{\textbf{neg}} & comer-\textsc{prs-msg} \\
& \multicolumn{5}{l}{``Un león no se come a un ser humano que ha muerto'' (pág. 411)}
\end{tabular} \vspace{0.5cm}

% Ejemplo 253
\begin{tabular}{llll}
(253) & phoó & \textbf{na} & wháat-u \\
& niño & \textsc{\textbf{neg}} & bajar.\textsc{pfv-msg} \\
& \multicolumn{3}{l}{``El niño no volvió a bajar'' (pág. 411)}
\end{tabular} \vspace{0.5cm}

% Ejemplo 254
\begin{tabular}{llllll}
(254) & asím & tu & \textbf{na} & bulaḍíl-u & hín-u \\
& \textsc{1pl.erg} & \textsc{2sg.nom} & \textsc{\textbf{neg}} & llamar.\textsc{pfv-msg} & ser.\textsc{prs-msg} \\
& \multicolumn{5}{l}{``No te hemos llamado'' (pág. 412)}
\end{tabular} \vspace{0.5cm}

% Ejemplo 255
\begin{tabular}{lllllll}
(255) & teeṇíi & kuṇaák & anú & qísum & \textbf{na} & bhanǰé \\
& \textsc{refl} & hijo & \textsc{prox.msg.nom} & de.tal.forma & \textsc{\textbf{neg}} & golpear.\textsc{imp.sg} \\
& \multicolumn{6}{l}{``No golpees a tu propio hijo de esta manera'' (pág. 419)}
\end{tabular} \vspace{0.5cm}

% Ejemplo 256
\begin{tabular}{lllllll}
(256) & ée & iṇc̣ & ma & típa & \textbf{na} & kha \\
& oh & oso & \textsc{1sg.nom} & ahora & \textsc{\textbf{neg}} & comer.\textsc{imp.sg} \\
& \multicolumn{6}{l}{``Oh, oso, no me comas ahora'' (pág. 419)}
\end{tabular} \vspace{0.5cm}

}

La principal estrategia para la negación es la partícula preverbal {\setmainfont{Charis SIL} \textit{na}} \textcolor{MidnightBlue}{\citep{palula}}. Esta partícula se ocupa de la negación básica, es decir, de la negación clausal (252), (253) y (254). Para la negación de imperativos la lengua recurre a la misma estrategia: la forma verbal imperativa precesida por la partícula negativa (255) y (256).