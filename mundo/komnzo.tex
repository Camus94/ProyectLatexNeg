\section*{Komnzo}
\addcontentsline{toc}{section}{Komnzo}

\noindent El komnzo es una lengua papú hablada por dos aldeas de la provincia de Sandaun en Papúa Nueva Guinea. Pertenece a la familia del Sepik Oriental-Madang dentro del phylum Trans-Nueva Guinea. 
%%%%%%%%%%%%% Abreviaturas %%%%%%%%%%%%%%%%%%%%%
\footnote{B: prefijos beta, DU: dual, ERG: ergativo, EXT: raíz verbal extendida, FEM: femenino, INDF: indefinido, IO: objeto indirecto, M: medio, ND: no dual, NPST: no pasado, PFV: perfectivo, POSS: posesivo, PROP: propietario, PST: pasado, REDUP: reduplicación, SBJ: sujeto, VC: cambio de valencia, VENT: venitivo}
%%%%%%%%%%%%%%%%%%%%%%%%%%%%%%%%%%%%%%%%%%%%%%%%
\vspace{0.5cm}

{\setmainfont{Charis SIL} 

% Ejemplo 227
\noindent \begin{tabular}{llllll}
(227) & \textbf{keke} & kwa & ra & nä & zrän/zin/th \\
& \textsc{\textbf{neg}} & \textsc{fut} & qué & \textsc{indf} & \textsc{2/3pl:sbj>3sg.fem:io:irr:pfv:vent/dejar} \\
& \multicolumn{5}{l}{``Ellos no dejarán nada para ella'' (pág. 311)}
\end{tabular} \vspace{0.5cm}

% Ejemplo 228
\noindent \begin{tabular}{lllll}
(228) & be & \textbf{kma=m} & ŋazi=karä & k-a-thafrak-w-é \\
& \textsc{2sg.erg} & \textsc{\textbf{pot=appr}} & coco=\textsc{prop} & \textsc{m.b-vc-}mezclar.\textsc{ext-nd-2sg.imp} \\
& \multicolumn{4}{l}{``No debes mezclarlo con coco'' (pág. 251)}
\end{tabular} \vspace{0.5cm}

{\small
% Ejemplo 229
\noindent \begin{tabular}{llllll}
(229) & sitau=ane & ŋare & mane & e/r/na & minu \\
& sitau=\textsc{poss.sg} & mujer & which & 2/3\textsc{du:sbj:pst:ipfv}/ser & mujer.esteril \\ 
& e/rn/a & nge & \textbf{matak} \\
& 2/3\textsc{du:sbj:pst:ipfv}/ser & hijo & \textbf{nada} \\
& \multicolumn{5}{l}{``En cuanto a las dos esposas de Sitau, eran mujeres estériles y sin hijos''} \\
& (pág.91)
\end{tabular} \vspace{0.5cm}}

% ejemplo 230
\noindent \begin{tabular}{lllll}
(230) & tüfr=\textbf{mär} & kafar-kafar & n/rä/ & komnzo \\
& mucho=\textsc{\textbf{priv}} & \textsc{redup-}grande & \textsc{1pl:sbj:npst:ipfv}/ser & únicamente \\
& etha=nzo \\
& pocos=\textsc{únicamente} \\
& \multicolumn{4}{l}{``No somos muchos ancianos... solo unos pocos'' (pág. 92)}
\end{tabular} \vspace{0.5cm}

}

La negación se expresa de forma perifrástica con el negador {\setmainfont{Charis SIL} \textit{keke}} en posición preverbal \textcolor{MidnightBlue}{\citep{komzo}} (227). La construcción prohibitiva consiste en la partícula potencial (POT) {\setmainfont{Charis SIL} \textit{kma}} junto con el clítico aprehensivo (APPR) {\setmainfont{Charis SIL} \textit{=m}}, y el verbo en imperativo (228). 

La negación a nivel de constituyente se puede expresar de varias formas. El cuantificador {\setmainfont{Charis SIL} \textit{matak}} «nada» se usa para expresar no existencia (229), generalmente en una cláusula copulativa. También se puede usar en una predicación no verbal. Alternativamente, cualquier sintagma nominal puede negarse usando el marcador de caso privativo (PRIV){\setmainfont{Charis SIL} \textit{=mär}} (230).