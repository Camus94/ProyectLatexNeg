\section*{Papuan Malay}
\addcontentsline{toc}{section}{Papuan Malay}

\noindent 
El papuan malayo o melanesio malayo es una lengua malaya creolizada que surgió por contacto entre varias lenguas papúes y el malayo, hablándose hoy ampliamente como lingua franca en partes de las provincias indonesias de Papúa y Papúa Occidental.
%%%%%%%%%%%%%%%%%%%%%%%%%% ABREVIATURAS %%%%%%%%%%%%%%%%%%%%
\footnote{EXIST: existencial, REL: relativizador}
%%%%%%%%%%%%%%%%%%%%%%%%%%%%%%%%%%%%%%%%%%%%%%%%
\vspace{0.5cm}

{\setmainfont{Charis SIL} 

% Ejemplo 257
\noindent \begin{tabular}{lllllll}
(257) & de & \textbf{tra} & datang ... & de & \textbf{tida} & datang \\
& \textsc{3sg} & \textsc{\textbf{neg}} & venir & \textsc{3sg} & \textsc{\textbf{neg}} & venir \\
& \multicolumn{6}{l}{``Ella no vino... ella no vino'' (pág. 520)}
\end{tabular} \vspace{0.5cm}

% Ejemplo 258
\noindent \begin{tabular}{lllll}
(258) & \textbf{tra} & ada & kamar & mandi \\
& \textsc{\textbf{neg}} & \textsc{exist} & cuarto & baño \\
& \multicolumn{4}{l}{``No hay bañós'' (pág. 520)}
\end{tabular} \vspace{0.5cm}

% Ejemplo 259
\noindent \begin{tabular}{lllll}
(259) & tong & \textbf{tra} & ke & kampung \\
& \textsc{1pl} & \textsc{\textbf{neg}} & hacia & villa \\
& \multicolumn{4}{l}{``No (vamos) al pueblo'' (pág. 521)}
\end{tabular} \vspace{0.5cm}

% Ejemplo 260
\noindent \begin{tabular}{llllll}
(260) & sa & \textbf{bukang} & orang & yang & seraka \\
& \textsc{1sg} & \textsc{\textbf{neg}} & persona & \textsc{rel} & ser.codicioso \\
& \multicolumn{5}{l}{``No soy una persona codiciosa'' (pág. 522)}
\end{tabular} \vspace{0.5cm}

% Ejemplo 261
\noindent \begin{tabular}{llllllll}
(261) & \textbf{bukang} & dong & maing, & dong & taguling & di & pecek \\
& \textsc{\textbf{neg}} & \textsc{3pl} & jugar & \textsc{3pl} & ser.arrastrado & en & lodo \\
& \multicolumn{7}{l}{``(la situación) no fue (que) jugaran (al fútbol, sino)} \\ & \multicolumn{7}{l}{que fueron arrastrados en el lodo'' (pág. 523)}
\end{tabular} \vspace{0.5cm}

% Ejemplo 262
\noindent «Respuesta a la pregunta [acerca de un accidente:]¿qué hizo? (¿Estaba) borracho?»

\noindent \begin{tabular}{llll}
(262) & \textbf{bukang} & dia & balap \\
& \textsc{\textbf{neg}} & \textsc{3sg} & carrera \\
& \multicolumn{3}{l}{``no, (pasó porque) él estaba corriendo (en su moto)''}
\end{tabular} \vspace{0.5cm}

}

\textcolor{MidnightBlue}{\citet{papuan}} afirma que las cláusulas negativas se construyen con los adverbios negativos {\setmainfont{Charis SIL} \textit{tida/tra}} y {\setmainfont{Charis SIL} \textit{bukang}}. El negador {\setmainfont{Charis SIL} \textit{tida/tra}} se utiliza para expresar la negación de cláusulas verbales (257), existenciales (258) y no verbales con predicado preposicional (259). 

El adverbio {\setmainfont{Charis SIL} \textit{bukang}} se ocupa para negar cláusulas no verbales (260), para construír una negación contrastiva (261) y para dar una respuesta negativa a una pregunta (262).