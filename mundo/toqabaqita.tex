\section*{Toqabaqita}
\addcontentsline{toc}{section}{Toqabaqita}

\noindent El toqabaqita es una lengua moribunda de Bolivia hablada actualmente por apenas unas decenas de personas mayores en la localidad de Lliquimuni, departamento de La Paz. Pertenece a la familia lingüística uru-chipaya.
%%%%%%%%%%%%%%%%%%%%%%%%%% ABREVIATURAS %%%%%%%%%%%%%%%%%%%%
\footnote{DU: dual, NEGV: verbo negativo, NFUT: no futuro, OBJ: objeto, PRF: perfecto}
%%%%%%%%%%%%%%%%%%%%%%%%%%%%%%%%%%%%%%%%%%%%%%%%
\vspace{0.5cm}

{\setmainfont{Charis SIL} 

% Ejemplo 268
\begin{tabular}{llll}
(268) & keeroqa & \textbf{kesi} & fula \\
& \textsc{3du} & \textsc{3du.\textbf{neg}} & llegar \\
& \multicolumn{3}{l}{``Los dos no llegaron'' (pág. 734)}
\end{tabular} \vspace{0.5cm}

% ejemplo 269
\begin{tabular}{llll}
(269) & A: Qo & riki-a & naqa? \\
& \textsc{2sg.nfut} & ver-\textsc{3sg.obj} & \textsc{prf} \\
& B: Qe=\textbf{aqi} \\
& \textsc{3sg.nfut=\textbf{negv}} (no.ser) \\
& \multicolumn{2}{l}{A: ``¿Lo viste?''} \\
& B: ``no'' (pág. 734)
\end{tabular} \vspace{0.5cm}

% Ejemplo 270
\begin{tabular}{lllllll}
(270) & nau & qe & \textbf{aqi} & \textbf{kwasi} & rongo & qoe \\
& \textsc{1sg} & \textsc{3sg.nfut} & \textsc{\textbf{negv}} & \textsc{1sg.\textbf{neg}} & escuchar & \textsc{2sg} \\
& \multicolumn{6}{l}{``No te escuché''}
\end{tabular} \vspace{0.5cm}

}

Toqabaqita cuenta con una variedad limitada de elementos para expresar la negación. Se utilizan principalmente dos tipos: marcadores de sujeto negativos y un verbo negativo \textcolor{MidnightBlue}{\citep{toqabaqita}}. Hay tres estrategias básicas de negación en Toqabaqita: [1] una construcción negativa simple que utiliza solo marcadores de sujeto negativos {\setmainfont{Charis SIL} \textit{kesi}} (268), [2] un verbo léxico negativo que expresa la falta de algo, como {\setmainfont{Charis SIL} \textit{aqi},} (269), y [3] una construcción negativa doble que consta de una cláusula de evento negativo precedida por una «minicláusula» con el verbo negativo (270). 