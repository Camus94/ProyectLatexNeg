\section*{Kalamang}
\addcontentsline{toc}{section}{Kalamang}

\noindent El kalamang es una lengua papú hablada en las Tierras Altas de Papúa Nueva Guinea. Pertenece a la familia lingüística trans-nueva guinea. Tipológicamente, es una lengua tonal aglutinante con un elaborado sistema de clasificadores nominales.
%%%%%%%%%%%%% Abreviaturas %%%%%%%%%%%%%%%%%%%%%
\footnote{EMPH: enfático, OBJ: objeto, PROH: prohibitivo, PROX: proximal, TOP: tópico}
%%%%%%%%%%%%%%%%%%%%%%%%%%%%%%%%%%%%%%%%%%%%%%%%
\vspace{0.5cm}

{\setmainfont{Charis SIL} 

% Ejemplo 222
\begin{tabular}{lll}
(222) & ma & sem=\textbf{nin} \\
& \textsc{3sg} & estar.asustado=\textsc{\textbf{neg}} \\
& \multicolumn{2}{l}{``Ella no está asustado'' (pág. 305)}
\end{tabular} \vspace{0.5cm}

% Ejemplo 223
\begin{tabular}{llll}
(223) & ma & yuon=at & konat=\textbf{nin} \\
& \textsc{3sg} & sol=\textsc{obj} & ver=\textsc{\textbf{neg}} \\
& \multicolumn{3}{l}{``Él no vio el sol'' (pág. 305)}
\end{tabular} \vspace{0.5cm}

% Ejemplo 224
\begin{tabular}{llll}
(224) & wa & me & mang=\textbf{nin} \\
& \textsc{prox} & \textsc{top} & amargo=\textsc{\textbf{neg}} \\
& \multicolumn{3}{l}{``Esto no está amargo'' (pág. 305)}
\end{tabular} \vspace{0.5cm}

% Ejemplo 225
\begin{tabular}{lllllll}
(225) & an & me & ka & an & ∅=\textbf{nin} & o \\
& \textsc{1sg} & \textsc{top} & \textsc{2sg} & \textsc{1sg} & dar=\textsc{\textbf{neg}} & \textsc{emph} \\
& \multicolumn{6}{l}{``A mí, a mí no me diste'' (pág. 306)}
\end{tabular} \vspace{0.5cm}

% Ejemplo 226
\begin{tabular}{lll}
(226) & ka-\textbf{mun} & koyal=\textbf{in} \\
& \textsc{2sg-proh} & molestar=\textsc{\textbf{neg}} \\
& \multicolumn{2}{l}{``No molestes'' (pág. 308)}
\end{tabular} \vspace{0.5cm}

}

\textcolor{MidnightBlue}{\citet{vaisser}} indica que la negación estándar se hace mediante el clítico negador {\setmainfont{Charis SIL} \textit{=nin}}, el cual niega cláusulas principales verbales declarativas (222). Este clítico puede negar verbos transitivos (223) como intransitivos (224). Las construcciones que predican el acto de «dar», pero que se realizan con un morfema cero, pueden negarse también con el clítico (225).

Los prohibitivos se construyen con el sufijo pronominal {\setmainfont{Charis SIL} \textit{-mun}} y el clítico {\setmainfont{Charis SIL} \textit{=in}} en el verbo (226).