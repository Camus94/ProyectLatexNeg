\section*{Mauwake}
\addcontentsline{toc}{section}{Mauwake}

\noindent El mauwake es una lengua papú hablada por unas 700 personas en una sola aldea llamada Boikin, en la provincia de Sandaun de Papúa Nueva Guinea. Pertenece a la familia del Sepik Oriental-Madang.
%%%%%%%%%%%%% Abreviaturas %%%%%%%%%%%%%%%%%%%%%
\footnote{CF: foco contrastivo, GEN: genitivo, NMZ: nominalizador, P: frase, PA: pasado, S:singular, UNM: pronombre no notificado (unmarked)}
%%%%%%%%%%%%%%%%%%%%%%%%%%%%%%%%%%%%%%%%%%%%%%%%
\vspace{0.5cm}

{\setmainfont{Charis SIL} 

% Ejemplo 237
\begin{tabular}{llllll}
(237) & I & iinan & aasa & me & kuuf-a-mik \\
& \textsc{1p.unm} & cielo & canoa & \textsc{neg} & ver-\textsc{pa-1/3p} \\
& \multicolumn{5}{l}{``No vimos los aviones'' (pág. 284)}
\end{tabular} \vspace{0.5cm}

% Ejemplo 238
\begin{tabular}{lllll}
(238) & O & somek & mua & weetak \\
& \textsc{3s.unm} & canción & hombre & \textsc{neg} \\
& \multicolumn{4}{l}{``Él no es un maestro'' (pág. 285)}
\end{tabular} \vspace{0.5cm}

% Ejemplo 239
\begin{tabular}{llll}
(239) & Yo & opora & wia \\
& \textsc{1p.unm} & hablar & \textsc{neg} \\
& \multicolumn{3}{l}{``No tengo nada qué decir'' (pág. 286)}
\end{tabular} \vspace{0.5cm}

% Ejemplo 240
\begin{tabular}{lllll}
(240) & Awuliak & fain & afila & marew \\
&  papa.dulce & esta & dulce & \textsc{neg} \\
& \multicolumn{4}{l}{``Esta papa no es dulce'' (pág.285)}
\end{tabular} \vspace{0.5cm}

% Ejemplo 241
\begin{tabular}{lllllll}
(241) & Ona & muuka & me & sesek-owa=ke & me & ma-e-k \\
& \textsc{3s.gen} & hijo & \textsc{neg} & enviar-\textsc{nmz=cf} & \textsc{neg} & decir-\textsc{pa-3s} \\
& \multicolumn{6}{l}{``Él no dijo que no enviaría (lit: decir sobre no enviar) a su hijo'' pág. 291}
\end{tabular} \vspace{0.5cm}

}

Esta lengua cuenta con cuatro negadores distintos: {\setmainfont{Charis SIL} \textit{me}} (237), {\setmainfont{Charis SIL} \textit{weetak}} (238), {\setmainfont{Charis SIL} \textit{wia}} (239) y {\setmainfont{Charis SIL} \textit{marew}} (240) los cuales tienen funciones que se superponen parcialmente entre sí \textcolor{MidnightBlue}{\citep{mauwake}}. En los cuatro ejemplos anteriores, los marcadores negativos son técnicamente intercambiables entre sí y este cambio solo produciría pequeños matices en el significado.

Una característica particular en mauwake es que la doble negación resulta en la cancelación de la negación en lugar de intensificarla (241).