\section*{Pichi}
\addcontentsline{toc}{section}{Pichi}

\noindent El pichi (o pichinglis) es una lengua criolla basada principalmente en el inglés que se habla en Nigeria, Camerún y Ecuatorial Guinea.
%%%%%%%%%%%%%%%%%%%%%%%%%% ABREVIATURAS %%%%%%%%%%%%%%%%%%%%
\footnote{FOC: foco, IPFV: imperfectivo, PRF: perfecto, PST: pasado, SBJ: sujeto, SBJV: subjuntivo}
%%%%%%%%%%%%%%%%%%%%%%%%%%%%%%%%%%%%%%%%%%%%%%%%
\vspace{0.5cm}

{\setmainfont{Charis SIL} 

% Ejemplo 263
\begin{tabular}{lllllll}
(263) & dɛn & \textbf{nó} & de & gí & \textbf{nó} & nátín \\
& \textsc{3pl} & \textsc{\textbf{neg}} & \textsc{ipfv} & dar & \textsc{\textbf{neg}} & nada \\
& \multicolumn{6}{l}{``Ellos no dan nada'' (pág. 202)}
\end{tabular} \vspace{0.5cm}

% Ejemplo 264
\begin{tabular}{lllllll}
(264) & a & \textbf{nó} & bin & fít & ték & motó \\
& \textsc{1sg.sbj} & \textsc{\textbf{neg}} & \textsc{pst} & poder & tomar & carro \\
& \multicolumn{6}{l}{``No pude tomar el carro'' (pág. 203)}
\end{tabular} \vspace{0.5cm}

% Ejemplo 265
\begin{tabular}{llllllll}
(265) & mék & yu & \textbf{nó} & kán & a & las & cinco \\
& \textsc{sbjv} & \textsc{2sg} & \textsc{\textbf{neg}} & venir & a & las & cinco \\
& \multicolumn{7}{l}{``No vengas a las cinco (en punto)'' (pág. 203)}
\end{tabular} \vspace{0.5cm}

% Ejemplo 266
\begin{tabular}{lllll}
(266) & e & \textbf{nɛ́a} & bɔ́n & pikín \\
& \textsc{3sg.sbj} & \textsc{\textbf{neg}.prf} & dar.a.luz & niño \\
& \multicolumn{4}{l}{``Aún no ha dado a luz a un niñó'' (pág. 203)}
\end{tabular} \vspace{0.5cm}

% ejemplo 267
\begin{tabular}{llllll}
(267) & \textbf{nóto} & ɔ́l & húman & fít & máred \\
& \textsc{\textbf{neg}.foc} & todas & mujer & poder & casarse \\
& \multicolumn{5}{l}{``No todas las mujeres pueden casarse'' (pág. 210)}
\end{tabular} \vspace{0.5cm}

}

El sistema de negación en pichi se basa principalmente en el uso del negador general {\setmainfont{Charis SIL} \textit{nó}} , que desempeña un papel tanto en la negación verbal como en la negación de sintagmas nominales, actuando como una partícula negativa \textcolor{MidnightBlue}{\citep{pichi}}. En la negación de cláusulas (263) y (264) se observa la concordancia negativa; cuando el verbo está negado, los sintagmas nominales no específicos también pueden estar precedidos por nó. La negación de imperativos sigue la misma estrategia (265).

La negación del tiempo/aspecto perfecto es con el adverbio {\setmainfont{Charis SIL} \textit{nɛ́a}} (266). La negación de constituyentes puede hacerse de la misma manera recurriendo a {\setmainfont{Charis SIL} \textit{nó}}, pero suele usar en su lugar una versión de negativa de marcador de foco {\setmainfont{Charis SIL} \textit{nóto}} (267).