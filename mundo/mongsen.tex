\section*{Mongsen}
\addcontentsline{toc}{section}{Mongsen}

\noindent El mongsen o mongsen ao es una lengua tibeto-birmana que se habla en los estados de Nagaland y Assam en el noreste de la India, así como en zonas adyacentes de Myanmar. Pertenece al subgrupo del Naga dentro de las lenguas kuki-chin.
%%%%%%%%%%%%%%%%%%%%%%%%%% ABREVIATURAS %%%%%%%%%%%%%%%%%%%%
\footnote{COND: condicional, DEC: declarativo, INST: instrumental, LOC: locativo, NRL: no relacional, PROX: demostrativo próximo PST: pasado, QPTCL: partícula interrogativa}
%%%%%%%%%%%%%%%%%%%%%%%%%%%%%%%%%%%%%%%%%%%%%%%%
\vspace{0.5cm}

{\setmainfont{Charis SIL} 

% Ejemplo 247
\begin{tabular}{llllll}
(247) & \textbf{mə̀}-tʃhà-pàla & a-ki & ku & tsəŋlaʔ & la-ì-ùʔ \\
& \textsc{\textbf{neg}-}hacer-\textsc{cond} & \textsc{nrl-}casa & \textsc{loc} & rayo & golpear-\textsc{irr-dec} \\
& \multicolumn{5}{l}{``si no lo haces, entonces un rayo caerá sobre la casa'' (pág. 292)}
\end{tabular} \vspace{0.5cm}

% Ejemplo 248
\begin{tabular}{lll}
(248) & nì & \textbf{mə̀}-tʃàʔ-ì \\
& \textsc{1sg} & \textsc{\textbf{neg}}-consumir-\textsc{irr} \\
& \multicolumn{2}{l}{``No comeré'' (pág. 353)}
\end{tabular} \vspace{0.5cm}

% Ejemplo 249
\begin{tabular}{llll}
(249) & təɹà & nə & \textbf{mə̀}-athà-\textbf{la} \\
& poco & \textsc{inst} & \textsc{\textbf{neg}}-caer-\textsc{\textbf{neg}.pst} \\
& \multicolumn{3}{l}{``Él falló un poco al caer'' (pág. 341)}
\end{tabular} \vspace{0.5cm}

% Ejemplo 250
\begin{tabular}{ll}
(250) & \textbf{mə̀}-tʃuŋ-\textbf{la} \\
& \textsc{\textbf{neg}}-comer.carne-\textsc{\textbf{neg}.pst} \\
& ``No comí'' (pág. 354)
\end{tabular} \vspace{0.5cm}

% Ejemplo 251
\begin{tabular}{lllll}
(251) & pi & mətət & sə̀ & \textbf{mə̀}-mətət-\textbf{la} \\
& \textsc{prox} & saber & \textsc{qptcl} & \textsc{\textbf{neg}}-saber-\textsc{\textbf{neg}.pst} \\
& \multicolumn{4}{l}{``¿Lo sabes o no?''} (pág. 342)
\end{tabular} \vspace{0.5cm}

}

\textcolor{MidnightBlue}{\citet{Coupe}} registra dos afijos para las construcciones negativas: {\setmainfont{Charis SIL} \textit{mə̀-}} y {\setmainfont{Charis SIL} \textit{-la}}. El marcador negativo por defecto es el prefijo {\setmainfont{Charis SIL} \textit{mə̀-}} (247) y (248). En construcciones de tiempo pasado aparece en combinación con el sufijo {\setmainfont{Charis SIL} \textit{-la}} (249) y (250) formando así un especie de circunflejo. Este también aparece en construcciones disyuntivas de tipo interrogativo (251).