\section*{Tlapaneco (me’phaa de Zilacayotitlán)}

\noindent El me’phaa es una lengua perteneciente a la familia lingüística otomangue. Forma parte de la familia Tlapaneco-Subtiaba del grupo Tlapaneco-Mangueano. La comunidad de Zilacayotitlán se encuentra en el estado de Guerrero, México en las regiones de la montaña y la costa chica. 
% Está ubicada a 13 kilómetros de la cabecera municipal del municipio de Atlamajalcingo del Monte.

El me'phaa es una lengua con verbos al inicio de la oración en la cual las relaciones gramaticales se indican en el núcleo. Es una lengua altamente flexiva que se caracteriza por tener elementos morfológicos adheridos al tema verbal que indican información sintáctica. Estos elementos incluyen flexión de persona, animacidad, polaridad y cambio de voz. Sin embargo, hasta el momento no se han documentado afijos derivativos en esta lengua. \vspace{0.5cm}

{\setmainfont{Charis SIL} 

% Ejemplo 98
\begin{tabular}{ll}
(98) & ta¹ga³-ta³-ɡa¹yaa³²-’\\
& \textsc{neg.comp-2sg}-correr.\textsc{2sg-pah-sg}\\
& ``No corriste'' (pág. 36)
\end{tabular} \vspace{0.5cm}

% Ejemplo 99
\begin{tabular}{ll}
(99) & ta¹ga³-gra¹k[aa²]-uun³ \\
& \textsc{neg.comp}-caerse-\textsc{1sg} \\
& ``No me caí'' (pág. 66)
\end{tabular} \vspace{0.5cm}

}

La negación se marca por medio de prefijos y está convinada con el aspecto completivo por lo tanto la negación es parte de la flexión verbal \textcolor{MidnightBlue}{\citep{Tlapaneco}}. El prefijo utilizado es {\setmainfont{Charis SIL} \textit{ta¹ga³-}} (98) - (99.)