\section*{Tlapaneco (me’phaa de Zilacayotitlán)}
\addcontentsline{toc}{section}{Tlapaneco}

\noindent El me’phaa es una lengua perteneciente a la familia lingüística otomangue. Forma parte de la familia Tlapaneco-Subtiaba del grupo Tlapaneco-Mangueano. La comunidad de Zilacayotitlán se encuentra en el estado de Guerrero, México en las regiones de la montaña y la costa chica.
%%%%%%%%%%%%% Abreviaturas %%%%%%%%%%%%%%%%%%%%%
\footnote{COM: completivo, PAH: participante del acto de habla}
%%%%%%%%%%%%%%%%%%%%%%%%%%%%%%%%%%%%%%%%%%%%%%%%
\vspace{0.5cm}

{\setmainfont{Charis SIL} 

% Ejemplo 98
\begin{tabular}{ll}
(98) & \textbf{ta¹ga³}-ta³-ɡa¹yaa³²-’\\
& \textsc{\textbf{neg}.com-2sg}-correr.\textsc{2sg-pah-sg}\\
& ``No corriste'' (pág. 36)
\end{tabular} \vspace{0.5cm}

% Ejemplo 99
\begin{tabular}{ll}
(99) & \textbf{ta¹ga³}-gra¹k[aa²]-uun³ \\
& \textsc{\textbf{neg}.com}-caerse-\textsc{1sg} \\
& ``No me caí'' (pág. 66)
\end{tabular} \vspace{0.5cm}

}

La negación se marca por medio de prefijos y está combinada con el aspecto completivo por lo tanto la negación es parte de la flexión verbal \textcolor{MidnightBlue}{\citep{Tlapaneco}}. El prefijo utilizado es {\setmainfont{Charis SIL} \textit{ta¹ga³-}} (98) - (99.)