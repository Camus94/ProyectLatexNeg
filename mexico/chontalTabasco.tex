\section*{Chontal de Tabasco}
\addcontentsline{toc}{section}{Chontal de Tabasco}

\noindent El chontal de San Carlos Macuspana es una lengua que pertenece a la familia lingüística maya, específicamente al grupo cholano-tzeltalano. Se habla en el municipio de Macuspana, en el estado de Tabasco, México.

Morfológicamente, es una lengua con marcación de núcleo. Esto significa que marca las relaciones gramaticales de sujeto y objeto en el verbo a través de afijos que tradicionalmente se conocen como Set A para el sistema ergativo y Set B para el sistema absolutivo.
%%%%%%%%%%%%%%%%%%%%%%%% Abreviaturas %%%%%%%%%%%%%%%%%%
\footnote{A: ergativo (juego A), B: absolutivo (juego B), EST: estativo, CLNUM: clasificador numeral, PI: plural inclusivo}
%%%%%%%%%%%%%%%%%%%%%%%%%%%%%%%%%%%%%%%%%%%%%%%%%%%%%%%
\vspace{0.5cm}

{\setmainfont{Charis SIL} 

% Ejemplo 21
\begin{tabular}{llllll}
(21) & \textbf{mach} & u-chäp-ka n& ni & un-p'e & kwa' \\
& \textsc{\textbf{neg}} & \textsc{a3}-cuece-\textsc{est} & \textsc{det} & uno-\textsc{clnum} & cosa \\
& \multicolumn{5}{l}{``No se cuece nada (pág. 73)''} \\
\end{tabular} \vspace{0.3cm}

% Ejemplo 22
\begin{tabular}{lllll}
(22) & si & \textbf{mach'an} & kä-yok  & patan=t'oko'\\
& si & \textsc{\textbf{neg}} & \textsc{a1}-chico & trabajo=\textsc{pi}\\
& \multicolumn{4}{l}{``Si no tenemos nuestro trabajito (pág. 73)''} \\
\end{tabular} \vspace{0.3cm}

% Ejemplo 23
\begin{tabular}{lllll}
(23) & \textbf{mame'} & x-ik-et & ti' & k'ak' \\
& \textsc{\textbf{neg}} & ir-\textsc{opt-b2} & boca & fuego\\
& \multicolumn{4}{l}{``No vayas a la orilla del fuego (pág. 74)''} \\
\end{tabular} \vspace{0.3cm}

% Ejemplo 24
\begin{tabular}{lllll}
(24) & \textbf{machme'} & x-ik-et & ti' & k'ak' \\
& \textsc{\textbf{neg}} & ir-\textsc{opt-b2} & boca & fuego\\
& \multicolumn{4}{l}{``No vayas a la orilla del fuego (pág. 74)''} \\
\end{tabular} \vspace{0.3cm}

% Ejemplo 25
\begin{tabular}{llll}
(25) & \textbf{moni'} & u-k'än -ka-ø & xan' \\
& \textsc{\textbf{neg}} & \textsc{a3}-usar-\textsc{est-b3} & gusano \\
& \multicolumn{3}{l}{``Ya no se usa gusano (pág. 42)''}\\
\end{tabular} \vspace{0.3cm}

% Ejemplo 26
\begin{tabular}{llll}
(26) & \textbf{mani'} & u-k'än -ka-ø & bojte' \\
& \textsc{\textbf{neg}} & \textsc{a3}-usar-\textsc{est-b3} & palo encestado \\
 & \multicolumn{3}{l}{``Ya no se usa árbol encestado''} \\
\end{tabular} \vspace{0.5cm}
}

La negación se expresa mediante partículas que se anteponen al verbo, como ocurre en otras lenguas mayas. \textcolor{MidnightBlue}{\citet{ChontalTabasco}} señala señala la existencia de 4 de estas partículas: \textit{mach} «no» (21), \textit{mach'an / ma'an} «no hay» (22), negación de advertencia (23 - 24) \textit{machme' / mame} «no sea que, no vaya a ser que» y negación enfática (25 - 26) \textit{moni'/mani'} «ya no, no es».