\section*{Huichol (Wixárika)}

\noindent El wixárika, también conocido como huichol, pertenece a la rama meridional de la amplia familia lingüística uto-azteca, una de las más extensas de América en número de lenguas y hablantes. Se habla en comunidades indígenas huicholas asentadas en la Sierra Madre Occidental de México. \vspace{0.5cm}
%%%%%%%%%%%%%%%%%% Abreviaturas %%%%%%%%%%%%%%%%%%%%%%%%%%%%%%%%
\footnote{AS1: primera asercion, IPFV: imperfectivo, NARR: narrativo, SBJ: sujeto}
%%%%%%%%%%%%%%%%%%%%%%%%%%%%%%%%%%%%%%%%%%%%%%%%%%%%%%%%%%%%%%%%
{\setmainfont{Charis SIL} 

% Ejemplo 32
\begin{tabular}{llll}
(32) & 'ikɨ & ki & pi-\textbf{ka}-hekwa \\
& \textsc{dem} & casa & \textsc{as1-\textbf{neg}-ser.nueva} \\
& \multicolumn{3}{l}{``Esta casa no es nueva'' (pág. 103)}
\end{tabular} \vspace{0.5cm}

% Ejmplo 33
\begin{tabular}{lll}
(33) & ne-'iwa-ma & pi-\textbf{ka}-'ane-kai \\
& \textsc{1sg}-hermano-\textsc{pl} & de.esta.manera-\textsc{\textbf{neg}}-ser-\textsc{ipfv} \\
& \multicolumn{2}{l}{``Mis hermanos no era así'' (pág. 110)}
\end{tabular} \vspace{0.3cm}

% Ejemplo 34
\begin{tabular}{lll}
(34) & kwitɨ & \textbf{ka}-pan-ku-ke-ni \\
& rápido & \textsc{\textbf{neg}-x-x}-levantarse-\textsc{narr} \\
& \multicolumn{2}{l}{``No se para rápido'' (pág. 141)}
\end{tabular} \vspace{0.3cm}

% Ejemplo 35
\begin{tabular}{lll}
(35) & hauki & pu-\textbf{mawe} \\
& no.saber & \textsc{as1-\textbf{neg}.exist} \\
& \multicolumn{2}{l}{``No sé, no está...'' (pág. 112)}
\end{tabular} \vspace{0.3cm}

% ejemplo 36
\begin{tabular}{lllll}
(36) & kumu & ne-maine-hepaɨ & 'ukara-tsi & pu-\textbf{mawe}-kai \\
& como & \textsc{1sg.sbj}-decir-como & mujer-\textsc{pl} & \textsc{as1-\textbf{neg}.exist-ipfv} \\
& \multicolumn{4}{l}{``... como estoy diciendo, no había mujeres'' (pág. 114)}
\end{tabular} \vspace{0.3cm}

}

El wixárika muestra una marcada tendencia al polisintetismo \textcolor{MidnightBlue}{\citep{Huichol}}, rasgo muy poco común en el resto de lenguas de la misma familia uto-azteca, que suelen ser más bien aglutinantes o aislantes. En este aspecto se asemeja más a lenguas como el náhuatl y el cora. Otra particularidad es que carece de una separación tajante entre cláusulas intransitivas y transitivas, exhibiendo más bien un comportamiento escalar entre dichas categorías. Asimismo, utiliza diversos mecanismos morfosintácticos para matizar el grado de transitividad en la expresión de distintos tipos de eventos.

La negación es parte de la morfología verbal por medio del prefijo {\setmainfont{Charis SIL} \textit{ka}-} (32) (33) (34). Existe también una negación existencial bajo la forma verbal supletiva {\setmainfont{Charis SIL} \textit{mawe}} (35) (36).