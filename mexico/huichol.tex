\section*{Huichol (Wixárika)}
\addcontentsline{toc}{section}{Huichol}

\noindent El wixárika, también conocido como huichol, pertenece a la rama meridional de la amplia familia lingüística uto-azteca, una de las más extensas de América en número de lenguas y hablantes. Se habla en comunidades indígenas huicholas asentadas en la Sierra Madre Occidental de México. \vspace{0.5cm}
%%%%%%%%%%%%%%%%%% Abreviaturas %%%%%%%%%%%%%%%%%%%%%%%%%%%%%%%%
\footnote{AS1: primera asercion, IPFV: imperfectivo, NARR: narrativo, SBJ: sujeto}
%%%%%%%%%%%%%%%%%%%%%%%%%%%%%%%%%%%%%%%%%%%%%%%%%%%%%%%%%%%%%%%%
{\setmainfont{Charis SIL} 

% Ejemplo 32
\begin{tabular}{llll}
(32) & 'ikɨ & ki & pi-\textbf{ka}-hekwa \\
& \textsc{dem} & casa & \textsc{as1-\textbf{neg}-ser.nueva} \\
& \multicolumn{3}{l}{``Esta casa no es nueva'' (pág. 103)}
\end{tabular} \vspace{0.5cm}

% Ejmplo 33
\begin{tabular}{lll}
(33) & ne-'iwa-ma & pi-\textbf{ka}-'ane-kai \\
& \textsc{1sg}-hermano-\textsc{pl} & de.esta.manera-\textsc{\textbf{neg}}-ser-\textsc{ipfv} \\
& \multicolumn{2}{l}{``Mis hermanos no era así'' (pág. 110)}
\end{tabular} \vspace{0.3cm}

% Ejemplo 34
\begin{tabular}{lll}
(34) & kwitɨ & \textbf{ka}-pan-ku-ke-ni \\
& rápido & \textsc{\textbf{neg}-x-x}-levantarse-\textsc{narr} \\
& \multicolumn{2}{l}{``No se para rápido'' (pág. 141)}
\end{tabular} \vspace{0.3cm}

% Ejemplo 35
\begin{tabular}{lll}
(35) & hauki & pu-\textbf{mawe} \\
& no.saber & \textsc{as1-\textbf{neg}.exist} \\
& \multicolumn{2}{l}{``No sé, no está...'' (pág. 112)}
\end{tabular} \vspace{0.3cm}

% ejemplo 36
\begin{tabular}{lllll}
(36) & kumu & ne-maine-hepaɨ & 'ukara-tsi & pu-\textbf{mawe}-kai \\
& como & \textsc{1sg.sbj}-decir-como & mujer-\textsc{pl} & \textsc{as1-\textbf{neg}.exist-ipfv} \\
& \multicolumn{4}{l}{``... como estoy diciendo, no había mujeres'' (pág. 114)}
\end{tabular} \vspace{0.3cm}

}

\textcolor{MidnightBlue}{\citep{Huichol}}

La negación es parte de la morfología verbal por medio del prefijo {\setmainfont{Charis SIL} \textit{ka}-} (32), (33) y (34). Existe también una negación existencial bajo la forma verbal supletiva {\setmainfont{Charis SIL} \textit{mawe}} (35) y (36).