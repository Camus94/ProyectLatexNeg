\section*{Totonaco de Tuxtla}

\noindent  El totonaco es parte de la familia de lenguas totonaco-tepehua, que se habla en los Estados de Veracruz Y Puebla. Las variantes principales son el totonaco de la sierra (central del sur) y el totonaco de la costa. Es una lengua tonal, lo que significa que se utiliza el tono para diferenciar e indicar diversos valorores gramaticales. Es una lengua aglutinante, ya que las palabras pueden estar formadas por múltiples elementos que se añaden para expresar diferentes significados.
