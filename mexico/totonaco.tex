\section*{Totonaco de Tuxtla}
\addcontentsline{toc}{section}{Totonaco de Tuxtla}

\noindent  El totonaco es parte de la familia de lenguas totonaco-tepehua, que se habla en los Estados de Veracruz Y Puebla. Las variantes principales son el totonaco de la sierra (central del sur) y el totonaco de la costa. Es una lengua tonal, lo que significa que se utiliza el tono para diferenciar e indicar diversos valorores gramaticales. Es una lengua aglutinante, ya que las palabras pueden estar formadas por múltiples elementos que se añaden para expresar diferentes significados.
%%%%%%%%%%%%% Abreviaturas %%%%%%%%%%%%%%%%%%%%%
\footnote{INCOM: incompletivo, JF: junta fonológica, O: objeto, PASD: pasado, PFTO: perfecto, S: sujeto}
%%%%%%%%%%%%%%%%%%%%%%%%%%%%%%%%%%%%%%%%%%%%%%%%
\vspace{0.5cm}

{\setmainfont{Charis SIL} 

% Ejemplo 100
\begin{tabular}{ll}
(100) & \textbf{ni}-xa-k-xkuli-y \\
& \textsc{\textbf{neg}-pasd-s1sg}-fumar-\textsc{incom} \\
& ``Yo no fumaba'' (pág. 78)
\end{tabular} \vspace{0.5cm}

% Ejemplo 101
\begin{tabular}{lll}
(101) & \textbf{nitu} & xa-k-wi \\
& \textsc{\textbf{neg}} & \textsc{pasd-s1sg}-estar.sentado \\
& \multicolumn{2}{l}{``Yo no estaba sentado''} (pág. 78)
\end{tabular} \vspace{0.5cm}

% Ejemplo 102
\begin{tabular}{lll}
(102) & \textbf{ni}-chixku & kit \\
& \textsc{\textbf{neg}}-hombre & yo \\
& \multicolumn{2}{l}{``Yo no soy hombre''} (pág. 78)
\end{tabular} \vspace{0.5cm}

% Ejemplo 103
\begin{tabular}{lll}
(103) & \textbf{nitu} & k-laqapas-∅ \\
& \textsc{\textbf{neg}} & \textsc{s1sg}-conocer[\textsc{o3sg}]-\textsc{incom}\\
& \multicolumn{2}{l}{``Yo no conozco nada''} (pág. 79)
\end{tabular} \vspace{0.5cm}

% Ejemplo 104
\begin{tabular}{llll}
(104) & \textbf{nitu} & k-maxki-qoo-nit=i & liwat \\
& \textsc{\textbf{neg}} & \textsc{s1sg}-dar[\textsc{o3sg}]-\textsc{3pl-pfto=jf} & comida \\
& \multicolumn{3}{l}{``No les he dado comida a ellos''} (pág. 79)
\end{tabular} \vspace{0.5cm}

}

\textcolor{MidnightBlue}{\citet{Totonaco}} describe dos marcas para la negación en totonaco: {\setmainfont{Charis SIL} \textit{ni-, nitu'}}. La primera es un prefijo que precede a la marca de tiempo/modo (100), mientras que la segunda es una partícula que precede a todo el verbo (101). Los predicados nominales se niegan únicamente con {\setmainfont{Charis SIL} \textit{ni-}} (102). {\setmainfont{Charis SIL} \textit{nitu'}} se utiliza con verbos transitivos y ditransitivos, ocurre con el aspecto incompletivo y con objetos de de tercera persona singular (103) - (104). En el resto de los casos, como con verbos dinámicos, se recurre a {\setmainfont{Charis SIL} \textit{ni-}}.