\section*{Tepehua de Huehuetla}

\noindent El tepehua es una lengua indígena poco documentada que se habla en algunos pueblos de los estados mexicanos de Hidalgo, Puebla y Veracruz. Pertenece a la familia lingüística totonaca, que tiene dos ramas: el tepehua y el totonaco.

El tepehua tiene tres variedades: el tepehua de Huehuetla, el de Pisaflores y el de Tlachichilco. La variedad de Huehuetla es la que se describe en el documento y se habla en el pueblo de Huehuetla en Hidalgo, así como en algunas comunidades cercanas. Se estima que hay alrededor de 8.000 hablantes de tepehua en total. \vspace{0.5cm}

{\setmainfont{Charis SIL} 

% Ejemplo 81
\begin{tabular}{llllll}
(81) & maa & \textbf{jaantu} & laa-y & 7alin & s-7asqat'a-7an \\
& \textsc{evi} & \textsc{\textbf{neg}} & poder-\textsc{imp} & haber & \textsc{3pos}-hijo-\textsc{pl.pos} \\
& \multicolumn{5}{l}{``Él/ella no puede tener hijos'' (pág. 578)}
\end{tabular} \vspace{0.5cm}

% Ejemplo 82
\begin{tabular}{lll}
(82) & \textbf{jaantu} & lapanak \\
& \textsc{\textbf{neg}} & persona \\
& \multicolumn{2}{l}{``Él no era una persona (humano)'' (pág. 580)}
\end{tabular} \vspace{0.5cm}

% Ejemplo 83
\begin{tabular}{lllllll}
(83) & 7ix-jun-niita & juu & lapanak & maa & \textbf{jaantu} & lhuu \\
& \textsc{psd}-ser-\textsc{pftv} & \textsc{det} & persona & \textsc{evi} & \textsc{\textbf{neg}} & muchos \\
& \multicolumn{6}{l}{``La multitud no era muy numerosa'' (pág. 580)}
\end{tabular} \vspace{0.5cm}

% Ejemplo 84
\begin{tabular}{lllll}
(84) & \textbf{jaantu} & k-lakask'in & nii & 7a-miilhpa-t'i \\
& \textsc{\textbf{neg}} & \textsc{1suj}-querer & \textsc{sb} & \textsc{irr}-cantar-\textsc{2sg.suj.pftv}\\
& \multicolumn{4}{l}{``No quiero que tu cantes'' (pág. 582)}
\end{tabular} \vspace{0.5cm}

% Ejemplo 85
\begin{tabular}{lll}
(85) & \textbf{jaantu} & tu7u7 \\
& \textsc{\textbf{neg}} & algo \\
& \multicolumn{2}{l}{``Nada'' (pág.583)}
\end{tabular} \vspace{0.5cm}

% Ejemplo 86
\begin{tabular}{lll}
(86) & \textbf{jaantu} & laqlhuu \\
& \textsc{\textbf{neg}} & caro \\
& \multicolumn{2}{l}{``Barato, no caro'' (pág. 584)}
\end{tabular} \vspace{0.5cm}

}

En esta lengua, la partícula independiente {\setmainfont{Charis SIL} \textit{jaantu}} es utilizada para negar tanto cláusulas como frases \textcolor{MidnightBlue}{\citep{Tepehua}}. Parece no tener ningún tipo de restricción y basta con su colocación antes del elemento al que negará (81) - (86).