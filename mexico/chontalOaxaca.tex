\section*{Chontal de san Pedro Huamelula, Sierra baja de Oaxaca}

\noindent El chontal de Oaxaca pertenece a la familia lingüística tequistlateca o tequistlatecana, que tiene algunos cientos de hablantes en la sierra y costas del estado de Oaxaca, en México.

Se trata de una lengua seriamente amenazada y en grave peligro de extinción. Según datos de 1990, tenía alrededor de 900 hablantes de la variedad de la sierra baja. Pero para 2011 se estima que su número se ha reducido drásticamente a tan sólo 100 personas, todos ellos adultos mayores de 70 años. \vspace{0.5cm}

{\setmainfont{Doulos SIL}
% Ejemplo 16
\begin{tabular}{llll}
	(16) & maa=ya'& tes & 'oo-daɡu-o' \\
	& \textsc{neg}=\textsc{1s.agt} & qué & amar-\textsc{sjntv.s-2s.pac} \\
	& \multicolumn{3}{l}{``yo no te quiero'' (pág. 53)}
\end{tabular} \vspace{0.5cm}

% Ejemplo 17
\begin{tabular}{lllll}
	(17) & maa & o-tayɡi & kwa & naa=sa \\
	& \textsc{neg} & \textsc{2s.pos}-palabra & dice & \textsc{ref=dem} \\
	& \multicolumn{4}{l}{``tú no sabes hablar'' (pág. 54)}
\end{tabular} \vspace{0.5cm}

% Ejemplo 18
\begin{tabular}{llllll} %6
	(18) & l-i-pekwe’ & naa=sa & maa & xolay-’uy & may-pa \\
	& \textsc{det-3s.pos}-esposo & \textsc{ref=dem} & \textsc{neg} & existir-\textsc{dur.s} & ir-\textsc{pfv.s} \\
\end{tabular} \vspace{0.5cm}

%Ejemplo 19
\begin{tabular}{lll}
	(19) & maa & čo-lyu-pa \\
	& \textsc{neg} & levantarse-\textsc{cloc-pfv.s} \\
	& \multicolumn{2}{l}{``No se levantó'' (pág.67)}
\end{tabular} \vspace{0.5cm}

%Ejemplo 20
\begin{tabular}{lll}
	(20) & maa & šaxko-pa \\
	& \textsc{neg} & encontrar-\textsc{pfv.s} \\
	& \multicolumn{2}{l}{``No la encontró'' (pág. 70)}
\end{tabular} \vspace{0.5cm}
}

Tipológicamente, el chontal de Oaxaca es una lengua sintética \textcolor{MidnightBlue}{\citep{ChontalOaxaca}}, es decir que emplea bastantes afijos y morfemas que se agregan a las palabras para expresar diversas modificaciones y significados gramaticales. Su morfología es principalmente prefijante en la clase nominal, mientras que en la flexión verbal predomina más bien los sufijos. El orden de palabras es bastante flexible.

En cuanto a la negación, utiliza la forma independiente \textit{maa} a la cual pueden adherirse clíticos (16). No parece existir alguna restricción con respecto a su posición dentro de una oración.
