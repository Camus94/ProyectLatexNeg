\section*{Chontal de san Pedro Huamelula, Sierra baja de Oaxaca}
\addcontentsline{toc}{section}{Chontal de san Pedro Huamelula}

\noindent El chontal de Oaxaca pertenece a la familia lingüística tequistlateca o tequistlatecana, que tiene algunos cientos de hablantes en la sierra y costas del estado de Oaxaca, en México.

Tipológicamente, es una lengua sintética, es decir que emplea bastantes afijos y morfemas que se agregan a las palabras para expresar diversas modificaciones y significados gramaticales. Su morfología es principalmente prefijante en la clase nominal, mientras que en la flexión verbal predomina más bien los sufijos, además, el orden de palabras es bastante flexible. %%%%%%%%%%%%%%%%%%%% Abreviaturas %%%%%%%%%%%
\footnote{1S-3P: marcadores de persona, AGT: agentivo, SJNTV: subjuntivo, PAC: paciente, REF: referente conocido, DUR: durativo, PFV: perfectivo, CLOC: cislocativo}
%%%%%%%%%%%%%%%%%%%%%%%%%%%%%%%
\vspace{0.5cm}

{\setmainfont{Charis SIL}

% Ejemplo 16
\begin{tabular}{llll}
(16) & \textbf{maa}=ya'& tes & 'oo-daɡu-o' \\
& \textsc{\textbf{neg}}=\textsc{1s.agt} & qué & amar-\textsc{sjntv.sg-2s.pac} \\
& \multicolumn{3}{l}{``yo no te quiero'' (pág. 53)}
\end{tabular} \vspace{0.5cm}

% Ejemplo 17
\begin{tabular}{lllll}
(17) & \textbf{maa} & o-tayɡi & kwa & naa=sa \\
& \textsc{\textbf{neg}} & \textsc{2s.pos}-palabra & dice & \textsc{ref=dem} \\
& \multicolumn{4}{l}{``tú no sabes hablar'' (pág. 54)}
\end{tabular} \vspace{0.5cm}

% Ejemplo 18
\begin{tabular}{llllll}
(18) & l-i-pekwe’ & naa=sa & \textbf{maa} & xolay-’uy & may-pa \\
& \textsc{det-3s.pos}-esposo & \textsc{ref=dem} & \textsc{\textbf{neg}} & existir-\textsc{dur.sg} & ir-\textsc{pfv.sg} \\
& \multicolumn{5}{l}{``Su esposo de ella no estaba, se había ido'' (pág. 65)}
\end{tabular} \vspace{0.5cm}

%Ejemplo 19
\begin{tabular}{lll}
(19) & \textbf{maa} & čo-lyu-pa \\
& \textsc{\textbf{neg}} & levantarse-\textsc{cloc-pfv.sg} \\
& \multicolumn{2}{l}{``No se levantó'' (pág.67)}
\end{tabular} \vspace{0.5cm}

%Ejemplo 20
\begin{tabular}{lll}
(20) & \textbf{maa} & šaxko-pa \\
& \textsc{\textbf{neg}} & encontrar-\textsc{pfv.sg} \\
& \multicolumn{2}{l}{``No la encontró'' (pág. 70)}
\end{tabular} \vspace{0.5cm}

}

En cuanto a la negación, utiliza la forma independiente \textit{maa} a la cual pueden adherirse clíticos (16) \textcolor{MidnightBlue}{\citep{ChontalOaxaca}}. No parece existir alguna restricción con respecto a su posición dentro de una oración.