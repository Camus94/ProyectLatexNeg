\section*{Tarahumara (Rarámuri)}
\addcontentsline{toc}{section}{Tarahumara}

\noindent El choguita rarámuri es una variedad del rarámuri que es parte de la familia de lenguas yuto-aztecas. Se trata de una lengua altamente sintética y aglutinante con un sistema morfológico complejo, que presenta rasgos típicos de las lenguas yuto-aztecas como la predominancia de sufijación, marcación a nivel del núcleo (\textit{head-marking}), incorporación nominal y composición. Tiene un sistema prosódico de palabra complejo con acento, tonema y restricciones en la ventana de asignación de acento, siendo muy inusual tipológicamente su ventana inicial de tres sílabas para la asignación de acento.
%%%%%%%%%%%%% Abreviaturas %%%%%%%%%%%%%%%%%%%%%
\footnote{CL: partícula de final de cláusula, COP: Cópula, DUB: dubitativo, IMPF: imperfectivo, PASS: pasivo, PROH: prohibitivo, VBLZ: verbalizador}
%%%%%%%%%%%%%%%%%%%%%%%%%%%%%%%%%%%%%%%%%%%%%%%%
\vspace{0.5cm}

{\setmainfont{Charis SIL} 

% Ejemplo 76
\begin{tabular}{llllll}
(76) & \textbf{ke}, & \textbf{ke} & me & o'báta & a'lé \\
& \textsc{\textbf{neg}} & \textsc{\textbf{neg}} & casi & feroz.\textsc{pl} & \textsc{dub}\\
& \multicolumn{5}{l}{``No, casi no son bravos'' (pág. 515)}
\end{tabular} \vspace{0.3cm}

% Ejemplo 77
\begin{tabular}{llllllll}
(77) & \textbf{'ka't͡ʃè} & kai'nâ-ma & a'lé & \textbf{ke} & naʔ'pô-suwa & 'ká & ba \\
& \textsc{\textbf{neg}} & cosecha-\textsc{fut.sg} & \textsc{dub} & \textsc{\textbf{neg}} & maleza-\textsc{cond.pass} & \textsc{cop.irr} & \textsc{cl} \\
& \multicolumn{7}{l}{``No habrá ninguna cosecha si no se realiza el deshierbe'' (pág.519)}
\end{tabular} \vspace{0.3cm}

% Ejemplo 78
\begin{tabular}{lllll}
(78) & \textbf{'ka't͡ʃè}=ko & waʔlu-'bê & buʔu-'rú-i & t͡ʃa'bèi=ko  \\
& \textsc{\textbf{neg}}.porque=\textsc{emph} & grande-más & camino-\textsc{vblz-impf} & antes=\textsc{emph} \\
& \multicolumn{4}{l}{``Porque antes no había camino grande'' (pág. 520)}
\end{tabular} \vspace{0.3cm}

% Ejemplo 79
\begin{tabular}{llllll}
(79) & \textbf{ke'tâsi} & pe & \textbf{ke} & 't͡ʃó & na'wà-li \\
& \textsc{\textbf{neg}} & apenas & \textsc{\textbf{neg}} & ya & llegar-\textsc{psd} \\
& \multicolumn{5}{l}{``No, todavía no llega'' (pág. 516)}
\end{tabular} \vspace{0.3cm}

% Ejemplo 80
\begin{tabular}{lll}
(80) & \textbf{'kíti} & na'là-ka \\
& \textsc{\textbf{neg} (proh)} & llorar-\textsc{imp.sg}\\
& \multicolumn{2}{l}{``¡No llores!'' (pág. 528)}
\end{tabular} \vspace{0.5cm}

}

En esta lengua existen una serie de formas libres con valores negativos \textcolor{MidnightBlue}{\citep{tarahumara}}. La forma {\setmainfont{Charis SIL} \textit{ke}} funciona tanco como una negeación exclamativa como el marcador de negación clausal (76). La forma {\setmainfont{Charis SIL} \textit{'ka't͡ʃè}} es el marcador de negación clausal por excelencia (77), y en algunos casos sirve para introducir cláusulas explicativas con valor negativo (78). la forma {\setmainfont{Charis SIL} \textit{ke'tâsi}} también funciona como negador clausal y como una interjección (79). Finalmente, la forma {\setmainfont{Charis SIL} \textit{'kíti}} es un imperativo negativo, al que algunos autores suelen llamar un \textit{prohibitivo} (80).