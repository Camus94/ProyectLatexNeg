\section*{Triqui de Copala}
\addcontentsline{toc}{section}{Triqui de Copala}

\noindent El triqui es una lengua de la familia oto-mangue del grupo mixtecano. Se divide en tres variedades principales: Copala, San Andrés Chicahuaxtla y San Martín Itunyoso, las cuales son bastante diferentes en sonidos y significado de las palabras. También existen diferencias en las clases de palabras, como pronombres y conjunciones. Las tres variantes se encuentran en el Estado de Oaxaca, México.
%%%%%%%%%%%%% Abreviaturas %%%%%%%%%%%%%%%%%%%%%
\footnote{( - ): El autor indica que {\setmainfont{Charis SIL} \textit{ma'3}} es una partícula que expresa modo y que esta siempre acompaña a construcciones negativas}
%%%%%%%%%%%%%%%%%%%%%%%%%%%%%%%%%%%%%%%%%%%%%%%%
\vspace{0.4cm}

{\setmainfont{Charis SIL} 

% Ejemplo 105
\begin{tabular}{llllll}
(105) & \textbf{ne³} & Guaá⁴ & me³ & so'³ & ma'³ \\
& \textsc{\textbf{neg}} & Juan & ser & él & ( - ) \\
& \multicolumn{5}{l}{``No es Juan'' (pág. 74)}
\end{tabular} \vspace{0.2cm}

% Ejemplo 106
\begin{tabular}{llllll}
(106) & \textbf{nuveé⁴} & Guaá⁴ & me³ & so'³ & ma'³ \\
& \textsc{\textbf{neg}} & Juan & ser & él & ( - ) \\
& \multicolumn{5}{l}{``No es Juan'' (pág. 74)}
\end{tabular} \vspace{0.2cm}

% Ejemplo 107
\begin{tabular}{llllll}
(107) & \textbf{nuveé⁴} & Guaá⁴ & ca'anj³² & Macáá⁵ & ma'³ \\
& \textsc{\textbf{neg}} & Juan & fue & México & ( - ) \\
& \multicolumn{5}{l}{``No fue Juan el que fue a México'' (pág. 73)} 
\end{tabular} \vspace{0.2cm}

% Ejemplo 108
\begin{tabular}{llllll}
(108) & \textbf{ne³} & náán⁵ & cha³na̱¹ & yatzíj⁵ & ma'³ \\
& \textsc{\textbf{neg}} & lavar & mujer &  ropa & ( - ) \\
& \multicolumn{5}{l}{``La mujer no lava la ropa'' (pág. 114)}
\end{tabular} \vspace{0.2cm}

% Ejemplo 109
\begin{tabular}{lllll}
(109) & \textbf{ne³} & cane̱² & so'³ & ma'³ \\
& \textsc{\textbf{neg}} & bañarse & él & ( - ) \\
& \multicolumn{4}{l}{``Él no se bañó'' (pág. 114)}
\end{tabular} \vspace{0.2cm}

% Ejemplo 110
\begin{tabular}{lllll}
(110) & \textbf{se²} & cane³² & so'³ & ma'³ \\
& \textsc{\textbf{neg}} & bañarse.\textsc{fut} & él & ( - ) \\
& \multicolumn{4}{l}{``Él no se va a bañar'' (pág. 115)}
\end{tabular} \vspace{0.3cm}

}

En esta lengua, existen dos partículas negativas para formar una frase nominal negativa \textcolor{MidnightBlue}{\citep{Triqui}}. Las partículas son {\setmainfont{Charis SIL} \textit{ne³, nuveé⁴}}. Cuando las construcciones cuentan con la cópula {\setmainfont{Charis SIL} \textit{me³}} los hablantes utilizan ambas partículas en una aparente variación libre (105) y (106). Se utiliza {\setmainfont{Charis SIL} \textit{nuveé⁴}} cuando se quiere negar la veracidad de lo dicho por otro (107). Por su parte, la negación clausal recurre al adverbio {\setmainfont{Charis SIL} \textit{ne³}}, tanto como para tiempo presente (108), como para pasado (109). En futuro se utiliza {\setmainfont{Charis SIL} \textit{se²}} (110). Las construcciones negativas suelen ser acompañadas de una partícula que da énfasis a la negación.