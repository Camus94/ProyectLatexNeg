\section*{Acateco de la frontera sur}

\noindent El Akateko es una lengua indígena mexicana que se habla en la frontera sur del país, principalmente en el estado de Chiapas. Es una lengua maya que pertenece al subgrupo k'anjobalano junto con otras lenguas como el K'anjobal, el Chuj y el Tojolabal.

Es una lengua de marcación ergativa-absolutiva al igual que el resto de lenguas mayas. Posee un sistema explícito de categorización léxica tanto para humanos, animales, plantas y productos de madera; incluso posee clasificadores numerales.%%%%%%%%%% Abreviaturas %%%%%%%%%%%
\footnote{A: caso ergativo y morfema posesivo, B: caso absolutivo, COMP: completivo, PRED: predicado no verbal, CN: clasificador nominal, INCOMP: incompletivo, REFl: reflexivo, PROGR: progresivo} \vspace{0.3cm}

{\noindent \setmainfont{Charis SIL}
% Ejemplo 1
\begin{tabular}{lllll}
(1) & \textbf{maː} & ø-aw-ʔab'=k'al & y-ʔek' & tiempo \\
& \textsc{\textbf{neg}.comp} & \textsc{b3-a2sg}-sentir\textsc{=dur} & \textsc{a3}-pasar & tiempo \\
& \multicolumn{4}{l}{“No sentiste cómo pasó el tiempo” (pág. 62)}
\end{tabular} \vspace{0.20cm}

% Ejemplo 2
\noindent \begin{tabular}{llllll}
(2) & \textbf{man} & mimam-ø-ox & teʔ & naː & tiʔ \\
 & \textsc{\textbf{neg}.pred} & grande-\textsc{b3-irr} & \textsc{cn}:madera & casa & esta \\
 & \multicolumn{5}{l}{“Esta casa no es grande” (pág. 81)}
\end{tabular} \vspace{0.20cm}

% Ejemplo 3
\noindent \begin{tabular}{lllllll}
(3) & nax & weto & \textbf{k'am} & ci-ø-toː & nax & yekal \\
& \textsc{nc}:hombre & Beto & \textsc{\textbf{neg}.incomp} & \textsc{incomp-b3}-ir & \textsc{pro}:hombre & mañana \\
& \multicolumn{6}{l}{“Beto no se va a ir mañana” (pág. 73)}                         
\end{tabular} \vspace{0.20cm}

% Ejemplo 4
\noindent \begin{tabular}{llll}
(4) & \textbf{k'am}-ø & b'ey & c-in-toː=an \\
& \textsc{\textbf{neg}.exist-b3} & \textsc{prep}:donde & \textsc{incomp-b1.sg}-ir=\textsc{cl.1sg} \\
& \multicolumn{3}{l}{“No hay lugar donde me vaya” (Pág. 73)}  
\end{tabular} \vspace{0.20cm}

% Ejemplo 5
{\footnotesize
\noindent \begin{tabular}{llllllllll}
(5) & \textbf{ʔamax} & mitaʔ & ø-y-ʔaʔ & meter & s-b'a & nax & y-in & xun & ȼetal \\
& \textsc{\textbf{neg}.foc}  & acaso & \textsc{b3-a3}-poner & meter & \textsc{a3-refl} & \textsc{pro}:hombre & \textsc{a3-prep}:en & una & cosa \\
& \multicolumn{9}{l}{“Él no se puede meter en esta cosa” (pág. 75)}                                                         
\end{tabular} \vspace{0.20cm}
}

% Ejemplo 6
\noindent \begin{tabular}{llllll}
(6) & \textbf{man} & ø-w-ʔoːtax & max & čekel & š-ø-xul=an \\
& \textsc{\textbf{neg}.irr} & \textsc{b3-a1.sg}-saber & quién & quién & \textsc{comp-b3}-venir=\textsc{cl1s} \\
& \multicolumn{5}{l}{“No sé quién vino” (pág. 181)}                       
\end{tabular} \vspace{0.20cm}

% Ejemplo 7
\noindent \begin{tabular}{lllll}
(7) & \textbf{man} & lalan-ø-ox & s-wey & nax \\
& \textsc{\textbf{neg}.progr} & \textsc{progr-b3-irr} & \textsc{a3}-dormir & \textsc{pro}:hombre \\
& \multicolumn{4}{l}{“(Él) no está durmiendo” (pág. 149)}        
\end{tabular} \vspace{0.25cm}

}

La marca de negación en esta lengua precede siempre al verbo y es aplicable a distintas nociones como la existencia y a características pragmáticas como la focalización de algún elemento dentro de la construcción \textcolor{MidnightBlue}{\citep{acateco}}.