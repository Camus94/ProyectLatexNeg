\section*{Seri}
\addcontentsline{toc}{section}{Seri}

\noindent La lengua seri es una lengua indígena que se habla en el noroeste de México. Es una lengua aislada que constituye su propia familia lingüística. Posee una enorme cantidad de alomorfía dentro de la morfología verbal. Es decir, aún después de aplicar las reglas fonológicas existentes, muchas veces quedan dos, tres, cuatro o más alomorfos supletivos para ciertos morfemas, condicionados por una variedad de factores. Es especialmente frecuente encontrar alomorfía condicionada por la transitividad superficial de la cláusula.

En cuanto a los sustantivos, estos no están marcados para el caso gramatical. Hay numerosos artículos definidos en seri que históricamente se derivan de verbos y que indican la posición o dirección de movimiento del ítem nominal.
%%%%%%%%%%%%% Abreviaturas %%%%%%%%%%%%%%%%%%%%%
\footnote{D: detransitivizador, DECL: declarativo, INTERR: interrogativo, EMPH: enfático, NOM: nominalizador, S: sujeto}
%%%%%%%%%%%%%%%%%%%%%%%%%%%%%%%%%%%%%%%%%%%%%%%%
\vspace{0.5cm}

{\setmainfont{Charis SIL} 

% Ejemplo 71
\begin{tabular}{ll}
(71) & t-\textbf{m}-afp \\
& \textsc{rl}-\textsc{\textbf{neg}}-llegar \\
& ``¿No llegó él?'' (pág. 21)
\end{tabular} \vspace{0.5cm}

% Ejemplo 72
\begin{tabular}{lll}
(73) & ik-oː-ʔit & iʔ-t-\textbf{m}-amšo-ʔo \\
& \textsc{inf-d}-comer & \textsc{1sg.s-rl-\textbf{neg}-}querer-\textsc{x} \\
& \multicolumn{2}{l}{``NO quiero comer'' (pág. 21)}
\end{tabular} \vspace{0.5cm}

% Ejemplo 73
\begin{tabular}{ll}
(73) & im-χó-\textbf{m}-aː \\
& hacia-\textsc{emph-\textbf{neg}}-moverse \\
& ``Él no viene'' (pág. 22)
\end{tabular} \vspace{0.5cm}

% Ejemplo 74
\begin{tabular}{lllll}
(74) & piest & ʔant & s-\textbf{m}-iːx & k-e-ya  \\
& fiesta & tierra & \textsc{irr-\textbf{neg}}-ubicar & \textsc{nom}-decir/\textsc{d-interr} \\
& \multicolumn{4}{l}{``¿No habrá una fiesta?'' (pág. 24)}
\end{tabular} \vspace{0.5cm}

% Ejemplo 75
\begin{tabular}{lll}
(75) & ʔim-íp-aːɬ & i-\textbf{m}-ataχ-iʔa \\
& \textsc{1sg.obj-irr}-acompañar & \textsc{nom-\textbf{neg}-}ir-\textsc{decl}\\
& \multicolumn{2}{l}{``Él no fue conmigo'' (pág. 25)}
\end{tabular} \vspace{0.5cm}

}

El seri es una lengua de estructura Sujeto-Objeto-Verbo (\textsc{svo}) \textcolor{MidnightBlue}{\citep{Seri}}. Utiliza una serie de prefijos dentro de la frase verbal para expresar diversas nociones como persona y número del sujeto u objeto. Entre estos prefijos tenemos tenemos el prefijo negativo {\setmainfont{Charis SIL} \textit{m}-}. Al parecer es el prefijo más cercano a la raíz verbal (71) - (75).