\section*{Zoque de San Miguel Chimalapa}

\noindent El zoque es una de las siete agrupaciones linguísticas que conforman a la familia mixe-zoque y pertenece a la rama zoqueana. Se habla en los Estados mexicanos de Oaxaca y Chiapas. El zoque de Chimalapa se habla en el municipio de San Miguel Chimalapa al Este del Estado de Oaxaca.

Se trata de una lengua ergativa para la primera y tercera persona, mientras que para la segunda persona muestra un comportamiento nominativo-acusativo. Es polisentética y con marcación en el núcleo.\footnote{S.I: sujeto independiente, AP: antipasivo, PL.PAH: plural de participante del acto de habla, ICP: incompletivo, A: agente, CMP.I: completivo independiente, S.D: sujeto dependiente, S: sujeto, ICP.D: incompletivo dependiente, OPT: Optativo, SBR: subordinador}  \vspace{0.5cm}

{\setmainfont{Charis SIL}

% Ejemplo 119
\begin{tabular}{lll}
(119) & \textbf{ya} & tø=køx-’oy-tam-\textbf{a }\\
& \textsc{\textbf{neg}} & \textsc{1s.i}=comer-\textsc{ap-pl.pah-icp.\textbf{neg}} \\
& \multicolumn{2}{l}{``No comemos'' (pág. 167)} 
\end{tabular} \vspace{0.5cm}

% Ejemplo 120
\begin{tabular}{llll}
(120) & tøx & \textbf{ya} & tø=min-\textbf{wø} \\
& \textsc{1pro} & \textsc{\textbf{neg}} & \textsc{1s.i}=venir-\textsc{cmp.i.\textbf{neg}} \\
& \multicolumn{3}{l}{``Yo no vine'' (pág. 169)}
\end{tabular} \vspace{0.5cm}

% Ejemplo 121
\begin{tabular}{llll}
(121) & pe & \textbf{yampa} & Ø=yets-\textbf{a} \\
& pero & \textsc{\textbf{neg}} & \textsc{3s.d}=llegar-\textsc{perf.pres.\textbf{neg}}\\
& \multicolumn{3}{l}{``Pero no ha llegado'' (pág. 171)}
\end{tabular} \vspace{0.5cm}

% Ejemplo 122
\begin{tabular}{lll}
(122) & \textbf{’u} & ’øm=tsaxø-\textbf{wø} \\
& \textsc{\textbf{neg}.imp} & \textsc{2s}=avergonzar-\textsc{icp.d} \\
& \multicolumn{2}{l}{``¡No te avergüences!'' (pág. 174)}
\end{tabular} \vspace{0.5cm}

% Ejemplo 123
\begin{tabular}{llll}
(123) & \textbf{’u} & yakkø & nø’-tsek-\textbf{’a} \\
& \textsc{\textbf{neg}.opt} & \textsc{sbr.opt} & agua-pedir-\textsc{opt.\textbf{neg}} \\
& \multicolumn{3}{l}{``Que no pida agua'' (pág. 175)}
\end{tabular} \vspace{0.5cm}

}

Esta lengua cuenta con tres marcas negativas: {\setmainfont{Charis SIL} \textit{ya, yampa, 'u}} \textcolor{MidnightBlue}{\citep{zoque}}. La forma {\setmainfont{Charis SIL} \textit{ya}} niega oraciones con aspecto incompletivo (119) y completivo (120). Las construcciones que tienen una lectura de aspecto perfecto toman {\setmainfont{Charis SIL} \textit{yampa}} (121). Finalmente, las formas imperativas se niegan con {\setmainfont{Charis SIL} \textit{'u}} (122) aunque también ocurre con construcciones optativas (123). Los verbos negados llevan, además, un sufijo con valor aspectual negativo ({\setmainfont{Charis SIL} \textit{-a, -wø}}).