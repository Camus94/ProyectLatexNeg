\section*{Mixe de Ayutla}
\addcontentsline{toc}{section}{Mixe de Ayutla}

\noindent La lengua mixe de Ayutla pertenece a la rama mixe de la familia lingüística mixe-zoque. Se habla principalmente en la región montañosa del estado de Oaxaca, México, en una pequeña comunidad llamada San Pedro y San Pablo Ayutla. 

Es una lengua polisintética con marcación en el núcleo y sistema inverso. Cuenta con construcciones seriales de verbos y permite la incorporación tanto de argumentos como de no argumentos. La mayor parte de la morfología recae en el verbo. No hay marcación de caso en los sustantivos, pero persisten remanentes de marcadores locativos que se han fusionado con otros morfemas. 
%%%%%%%%%%%%%%%%%%%%% Abreviaturas %%%%%%%%%%%%%%%%%%%%%
\footnote{A: sujeto de verbo transitivo, ASSOC: asociativio, CMPLZ: complementaddor, DEP: (aspecto) dependiente, PERF: perfecto, S: sujeto de verbo intransitivo, VBLZ: verbalizador}
%%%%%%%%%%%%%%%%%%%%%
\vspace{0.5cm}
{\setmainfont{Charis SIL} 

% Ejemplo 41
\begin{tabular}{lllll}
(41) & \textbf{ka't} & ëëts & xë'n & n-tun-t \\
& \textsc{\textbf{neg}} & \textsc{1pl} & como & \textsc{1a-hacer-pl.dep} \\
& \multicolumn{4}{l}{``No hicimos nada'' (pág. 448)}
\end{tabular} \vspace{0.3cm}

% Ejemplo 42
\begin{tabular}{llllll}
(42) & jëts & ja+tu'uk & kääjp & \textbf{ka't} & n-uk-ex-ät-n \\
& y & otro & pueblo & \textsc{\textbf{neg}} & \textsc{1a-x-ver-vblz-perf.dep} \\
& \multicolumn{5}{l}{``Y otros pueblos que no conozco'' (pág. 448)}
\end{tabular} \vspace{0.3cm}

% Ejemplo 43
\begin{tabular}{llllll}
(43) & kuu & anä'äjk & \textbf{ka't} & tëjk & t-ex-päät-y \\
& \textsc{cmplz} & joven.gente & \textsc{\textbf{neg}} & casa & \textsc{3a-ver}-encontrar-\textsc{dep} \\
& \multicolumn{5}{l}{``(Se dice) que los jovenes no valoran la casa'' (pág. 448)}
\end{tabular} \vspace{0.3cm}

% Ejemplo 44
\begin{tabular}{lll}
(44) & \textbf{ni}-pëën & y-\textbf{ka}-tän-y \\
& \textsc{\textbf{neg}}-quien & \textsc{3s-\textbf{neg}}-estarse-\textsc{dep} \\
& \multicolumn{2}{l}{``Nadie estaba'' (pág. 448)}
\end{tabular} \vspace{0.5cm}

% Ejemplo 45
\begin{tabular}{llll}
(45) & \textbf{ni}-tii & jä'äy & t-\textbf{ka}-mëët \\
& \textsc{\textbf{neg}}-qué & gente & \textsc{3a-\textbf{neg}-assoc} \\
& \multicolumn{3}{l}{``Las personas no tienen nada'' (pág. 448)}
\end{tabular} \vspace{0.5cm}

}

La negación se expresa por medio de la partícula independiente {\setmainfont{Charis SIL} \textit{ka't}}, la cual siempre precede al verbo. Su uso implica la aparición de una «marca de dependencia aspectual» \textcolor{MidnightBlue}{\citep{mixe}} en el verbo de manera obligatoria (41) (42) (43). Asimismo, existe también el prefijo {\setmainfont{Charis SIL} \textit{ni}-} que se une a palabras interrogaticas (44) (45) para otorgar un sentido negativo. El uso de este prefijo desencadena una marcación obligatoria en el verbo por medio del prefijo {\setmainfont{Charis SIL} \textit{ka}-}.