\section*{Oluteco}

\noindent El oluteco es una lengua indígena que pertenece a la familia mixezoque y se habla en la comunidad de Oluta, Veracruz, México. Representa una variedad conservadora del proto-mixezoque por lo que su estudio permite reconstruir la proto-lengua.

Es una lengua ergativa con marcación en el núcleo. Distingue tres personas gramaticales en singular y cuatro en plural. Los pronombres personales son proclíticos que anteceden al verbo. Tiene un sistema muy complejo de aspecto con distinciones entre completivo, incompletivo e irrealis. El aspecto interactúa con el sistema de inverso propio de lenguas ergativas.
%%%%%%%%%%%%%%%%%%%%%%%%%%% Abreviaturas %%%%%%%%%%%%%%%%%%%%%%%
\footnote{A: marcador de persona del juego \textsc{a}, B: marcador de persona del juego \textsc{b}, AN: clítico para animados, APL3: aplicativo asociativo y comitativo, EV: evidencial, C: marcador de persona del juego c, CAUS: causativo,  INCD incompletivo de dependientes, INV: inverso, INCI.I: incompletivo de independientes intransitivo, INCI.T: incompletivo de independientes transitivo, NMZR: nominalizador, RLTVZR: relativizador}
%%%%%%%%%%%%%%%%%%%%%%%%%%%%%%%%%%%%%%%%%%%%%%%%%%%%%%%%%%%%%%%%
\vspace{0.2cm}

{\setmainfont{Charis SIL}

{\footnotesize
% Ejemplo 53
\noindent \begin{tabular}{lllllll}
(53) & naʔkxej=xü=k & tuk & ta=\textbf{kaː}=moːyʔ-e & jamaj=k & muːxi-nak & kay-e \\
& cuando=\textsc{ev=an} & uno & \textsc{c3\textbf{neg}}=dar=\textsc{incd} & aquel=\textsc{an} & pájaro-\textsc{dim} & comer-\textsc{nmzr} \\
& \multicolumn{6}{l}{``Cuando uno no le da comida a aquel pajarito'' (pág. 351)}
\end{tabular} \vspace{0.2cm}
}

% Ejemplo 54
\noindent \begin{tabular}{lllll}
(54) & \textbf{kaː} =seme & ʔit-ü-pa=k & meːnyu & ʔi=tükaw \\
& \textsc{\textbf{neg}}=mucho & existir=\textsc{inv-inci.i=an} & dinero & \textsc{a3(posd)}=papá \\
& \multicolumn{4}{l}{``No tiene mucho dinero su papá de él'' (pág. 354)}
\end{tabular} \vspace{0.2cm}

% Ejemplo 55
\noindent \begin{tabular}{llll}
(55) & jaʔmej & mi=\textbf{kaː}=müː-nükx-ü-pa & jaːmu \\
& así & \textsc{b2=\textbf{neg}=apl3}-ir-\textsc{inv-inci.i} & viento \\
& \multicolumn{3}{l}{``De esa manera no te lleva el viento'' (pág. 361)}
\end{tabular} \vspace{0.2cm}

% Ejemplo 56
\noindent \begin{tabular}{lll}
(56) & ʔi=\textbf{kaː}=yak-pot-pe=k & tüpx-i \\
& \textsc{a3=\textbf{neg}=caus}-reventar-\textsc{inci.t=an} & torcer-\textsc{nmzr}\\
& \multicolumn{2}{l}{``No revienta la reata'' (pág. 362)}
\end{tabular} \vspace{0.2cm}

% Ejemplo 57
\noindent \begin{tabular}{lll}
(57) & ta & tax=\textbf{kaː}=tzum-pa \\
& \textsc{cond} & \textsc{c1(local)=\textbf{neg}}=amarrar-inci.i \\
& \multicolumn{2}{l}{``Si no te amarro'' (pág. 365)}
\end{tabular} \vspace{0.2cm}

% Ejemplo 58
{\small
\noindent \begin{tabular}{lllllll}
(58) & pero & jamaj & jaykak & ʔi=\textbf{kaː}=ʔix+kap-pe & jaʔ & pün-ʔa=jeʔ \\
& pero & aquel & gente & \textsc{a3=\textbf{neg}=}conocer-\textsc{inci.t} & él & quien-rltvzr=ese \\
& \multicolumn{6}{l}{``Pero aquella gente no sabía quien era ese'' (pág. 377)}
\end{tabular} \vspace{0.25cm}
}

}

El verbo en oluteco es una palabra polimorfémica \textcolor{MidnightBlue}{\citep{Oluteco}} y contiene la información de los argumentos centrales de la oración. La negación es por medio de un proclítico de la forma {\setmainfont{Charis SIL} \textit{kaː}} y ocupa la penúltima posición más alejada de la raíz verbal en presencia de otros clíticos y afijos.