\section*{Huave de San Mateo del Mar}

\noindent El huave de San Mateo del Mar es una lengua indígena que se habla en el estado de Oaxaca, México. Es una lengua que constituye por sí misma una familia lingüística, es decir, no está relacionada genéticamente con ninguna otra lengua.

Tiene un alineamiento nominativo-acusativo en cuanto al sujeto, mientras que para el objeto muestra un alineamiento neutro excepto en la marca de plural que concuerda con el objeto. En la flexión verbal se distingue morfológicamente el aspecto (completivo, incompletivo, puntual), el tiempo (presente, futuro) y la persona. 

Para cambiar la valencia verbal, el huave tiene construcciones pasivas, reflexivas, recíprocas y causativas marcadas mediante afijos. Los modificadores como conceptos de propiedad anteceden al sustantivo dentro de la frase nominal, mientras que las cláusulas relativas tienen una posición postnominal.\vspace{0.2cm}

{\setmainfont{Charis SIL}

% Ejemplo 27
\begin{tabular}{lllll}
 (27) & s'ik & \textbf{ⁿɡo}-nambiy & a & puty\\
& yo & \textsc{\textbf{neg}}-matar & \textsc{det} & perro \\
& \multicolumn{4}{l}{``Yo no maté al perro (pág. 54)''}
\end{tabular}\vspace{0.15cm}

% Ejemplo 28
\begin{tabular}{lll}
(28) & \textbf{ⁿɡo}-membyol & s'ik \\
& \textsc{\textbf{neg}}-ayudar.2 & yo \\
& \multicolumn{2}{l}{``No me ayudas (pág. 55)''} \\
\end{tabular} \vspace{0.15cm}

% Ejemplo 29
\begin{tabular}{lllllll}
(29) & aj & ka & kuchil & ka & \textbf{ⁿɡo}-ma-a-jior & u-xinɡ \\
& \textsc{det} & \textsc{det} & cuchillo & \textsc{det} & \textsc{\textbf{neg}-sb-vt}-tener & \textsc{pos.1}-nariz \\
& \multicolumn{6}{l}{``Este cuchillo no tiene filo (pág. 124)''} \\
\end{tabular} \vspace{0.15cm}

% Ejemplo 30
\begin{tabular}{lll}
(30) & \textbf{ⁿɡo}-na-ⁿɡal & \textbf{ni}-kʷa-\textbf{hiⁿd} \\
& \textsc{\textbf{neg}}-1-comprar & \textsc{pro.\textbf{neg}}-qué-\textsc{\textbf{neg}.disl} \\
& \multicolumn{2}{l}{``No compré nada (pág. 216)''} \\
\end{tabular} \vspace{0.15cm}

% Ejemplo 31
\begin{tabular}{lll}
(31) & \textbf{ni}-kʷa-\textbf{hiⁿd} & ta-ⁿɡal-as \\
& \textsc{\textbf{neg}}-qué-\textsc{\textbf{neg}.disl} & \textsc{comp}-comprar-\textsc{comp.1} \\
& \multicolumn{2}{l}{``Nada compré (pág. 2016)''} \\
\end{tabular} \vspace{0.2cm}

}

En cuanto a la negación, utiliza el prefijo {\setmainfont{Charis SIL} \textit{ⁿɡo}-} como parte de la morfología verbal (27)(28)(29). \textcolor{MidnightBlue}{\citet{Huave}} describe una serie de pronombres negativos que pueden funcionar como negadores dentro de la cláusula siempre y cuando se encuentren en posición preverbal (31). En el caso de que estos pronombres negativos ocupen una posición postverbal, es obligatorio que el verbo lleve el prefijo de negación (30). En ambos casos, el pronombre negativo va siempre acompañado del sufijo {\setmainfont{Charis SIL} -\textit{hiⁿd}}, considerado una forma dislocada con valor negativo.