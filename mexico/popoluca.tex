\section*{Sierra Popoluca}
\addcontentsline{toc}{section}{Sierra Popoluca}

\noindent El popoluca es una lengua mixe-zoqueana hablada en el sur del estado de Veracruz, México, por aproximadamente 28,000 personas. Pertenece a la rama mixeana de la familia mixe-zoqueana.
%%%%%%%%%%%%% Abreviaturas %%%%%%%%%%%%%%%%%%%%%
\footnote{ABS: absolutivo, INC: incompletivo, LOC\textsubscript{\emph{applic}}: instrumento, IPSR: poseedor inclusivo de primera persona, OPT: optativo, PLU\textsubscript{\emph{nonsap}}: plural del participante fuera del acto de habla, PLu\textsubscript{\emph{sap}}: plural del participante del acto de habla}
%%%%%%%%%%%%%%%%%%%%%%%%%%%%%%%%%%%%%%%%%%%%%%%% 
\vspace{0.5cm}

{\setmainfont{Charis SIL} 

% Ejemplo 65
\begin{tabular}{llll}
(65) & \textbf{ʔotʔoy} & ø+mɨɨch-kaʔ-taʔm-ɨʔ & woonyi \\
& \textsc{\textbf{neg}} & \textsc{3abs}+jugar-\textsc{loc\textsubscript{\emph{applic}}-pl-imp} & niña \\
& \multicolumn{3}{l}{``No juegues (juegos) con las niñas'' (pág.478)}
\end{tabular} \vspace{0.5cm}

% Ejemplo 66
\begin{tabular}{lll}
(66) & \textbf{ʔotʔoy} & ʔaʔm=seet-taʔm-ɨ \\
& \textsc{\textbf{neg}} & mirar=regreso-\textsc{plu\textsubscript{\emph{sap}}-imp} \\
& \multicolumn{2}{l}{``No mires atrás''}
\end{tabular} \vspace{0.5cm}

% Ejemplo 67
\begin{tabular}{llll}
(67) & tan+manɨk & \textbf{ʔotʔoy} & ø+mɨɨch-yaj-ʔiny \\
& 1.\textsc{ipsr+}niño & \textsc{\textbf{neg}} & \textsc{3abs+}jugar-\textsc{plu\textsubscript{nonsap}-opt} \\
& \multicolumn{3}{l}{``Nuestros hijos no deberían jugar (allí)'' (pág. 478)}
\end{tabular} \vspace{0.5cm}

% Ejemplo 68
\begin{tabular}{lll}
(68) & \textbf{dya} & ʔa+ʔɨks.i=juʔy-pa \\
& \textsc{\textbf{neg}} & \textsc{abs}+grano.de.maíz=comprar-\textsc{inc} \\
& \multicolumn{2}{l}{``No compramos maíz'' (pág. 608)}
\end{tabular} \vspace{0.5cm}

% Ejemplo 69
\begin{tabular}{lll}
(69) &  \textbf{dya} & ø+kamam \\
& \textsc{\textbf{neg}} & \textsc{3abs+}duro \\
& \multicolumn{2}{l}{``No está duro'' (pág. 609)}
\end{tabular} \vspace{0.5cm}

% Ejemplo 70
\begin{tabular}{lllll}
(70) & \textbf{dya} & ʔi+kuʔt-pa & jeʔm & kaʔnpu \\
& \textsc{\textbf{neg}} & \textsc{3erg}+comer-\textsc{inc} & \textsc{dem} & huevo \\
& \multicolumn{4}{l}{``Ella no se comió el huevo'' (pág. 610)}
\end{tabular} \vspace{0.5cm}

}

La negación en esta lengua es por medio de dos partículas \textcolor{MidnightBlue}{\citep{Popoluca}}. Para el caso de imperativos (65) - (66) y el modo optativo (67) se utiliza la forma independiente {\setmainfont{Charis SIL} \textit{ʔotʔoy}}. Para el resto de construcciones se utiliza la partícula {\setmainfont{Charis SIL} \textit{dya}} (68) - (70).