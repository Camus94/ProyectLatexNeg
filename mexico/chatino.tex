\section*{Chatino de Zacatepec}
\addcontentsline{toc}{section}{Chatino de Zacatepec}

\noindent El Chatino es una lengua mexicana que pertenece a la familia lingüística oto-mangue. Se clasifica dentro de la subfamilia Zapotecana y tiene diversas variedades dialectales que difieren en ciertos aspectos, como la pronunciación, el vocabulario y la gramática. El Chatino de Zacatepec se habla en la pequeña comunidad de San Marcos Zacatepec en el Estado de Oaxaca. \vspace{1cm} %%%%%%%%%% Abreviaturas %%%%%%%%%%%
\footnote{H: habitual, P: potencial, ENF: enfático}

{\setmainfont{Charis SIL}
% cSpell:disable
% Ejemplo 8
\begin{tabular}{lllll}
(8) & \textbf{a3} & lakan3 & natan3 & ndolo7o3 \\
& \textsc{\textbf{neg}} & \textsc{h}.ser.\textsc{1sg} & gente & \textsc{h}.saber \\
& \multicolumn{4}{l}{“No soy un profesor” (pág. 91)}
\end{tabular} \vspace{0.5cm}

% Ejemplo 9
\begin{tabular}{lll}
(9) & \textbf{a3} & tilyon30 \\
& \textsc{\textbf{neg}} & gordo.\textsc{1sg} \\
& \multicolumn{2}{l}{“No estoy gordo” (pág. 91)} \\
\end{tabular} \vspace{0.5cm}

% Ejemplo 10
\begin{tabular}{llllll}
(10) & kwa31 & \textbf{a3} & ki7ya3 & 7a21 & kyo3 \\
& ya & \textsc{\textbf{neg}} & \textsc{p}.caer & \textsc{enf} & lluvia \\
& \multicolumn{5}{l}{“Ya, no lloverá mucho” (pág. 91)}                
\end{tabular} \vspace{1cm}
% Cspell:enable
} 

 La negación en el chatino de Zacatepec se expresa a través del elemento libre {\setmainfont{Charis SIL} \textit{a3}}, que siempre precede al predicado que niega \textcolor{MidnightBlue}{\citet{chatino}}. La forma {\setmainfont{Charis SIL} \textit{a3}} puede encontrarse al inicio o en medio de la oración. Puede negar predicados verbales (8) o no verbales (9). Se aplica a eventos que han ocurrido (aspecto completivo), que ocurren habitualmente (aspecto habitual), o que ocurrirán en el futuro (aspecto potencial) (10).

No parece haber otras partículas de negación mencionadas en la gramática además de la forma {\setmainfont{Charis SIL} \textit{a3}}.