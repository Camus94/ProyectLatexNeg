\section*{Otomí de San Ildefonso Tultepec}
\addcontentsline{toc}{section}{Otomí de San Ildefonso}

\noindent El otomí es un lengua de la familia Otomangue de la rama otomangue occidental. El otomí de San Ildefonso Tultepec es la variante que se utiliza en la localidad rural del mismo nombre en el sur del Estado de Querétaro, México.

Es una lengua con morfología tanto acumulativa (afijos y clíticos) como no acumulativa (mutaciones consonánticas y cambios tonales). Los procesos de derivación y flexión se realizan sobre todo mediante afijos y clíticos. Hay varianza morfológica condicionada fonológica, gramatical y léxicamente.
%%%%%%%%%%%%% Abreviaturas %%%%%%%%%%%%%%%%%%%%%
\footnote{A: ajuste morfológico, D: (forma) dependiente, DEF: Definido, IMP: imperfecto, L: (forma) libre, P: palabra adverbial (ya), TNP: tema de no presente}
%%%%%%%%%%%%%%%%%%%%%%%%%%%%%%%%%%%%%%%%%%%%%%%%
\vspace{0.5cm}

{\setmainfont{Charis SIL} 

% Ejemplo 59
\begin{tabular}{llll}
(59) & ko & \textbf{hinɡi} & ø=hand-i \\
& porque & \textsc{\textbf{neg}} & \textsc{3.pres=ver-l} \\
& \multicolumn{3}{l}{``Porque no ve'' (pag. 59)}
\end{tabular} \vspace{0.3cm}

% Ejemplo 60
\begin{tabular}{llll}
(60) & ya & \textbf{hinɡi} & ø=hand-ø-i \\
& ya & \textsc{\textbf{neg}} & \textsc{3.pres}=ver-\textsc{3obj-l} \\
& \multicolumn{3}{l}{``Ya no la ve'' (pág. 228)}
\end{tabular} \vspace{0.3cm}

% Ejemplo 61
\begin{tabular}{lllllll}
(61) & pero & nu=yu̠ & ya & \textbf{hinɡi} & ø=ne-ø=r & tsibi \\
& pero & \textsc{def=dem.i.pl} & \textsc{p} & \textsc{\textbf{neg}} & \textsc{3.pres}=querer-\textsc{3obj=sg} & lumbre \\
& \multicolumn{6}{l}{``Pero esas no quieren lumbre'' (pág. 113)}
\end{tabular} \vspace{0.3cm}

% Ejemplo 62
\begin{tabular}{lll}
(62) & \textbf{him}=bi & <d>in-ø-i \\
& \textsc{\textbf{neg}=3.psd} & <\textsc{tnp}>encontrar-\textsc{3obj-l} \\
& \multicolumn{2}{l}{``No la encontraron'' (pág. 55)}
\end{tabular} \vspace{0.3cm}

% Ejemplo 63
\begin{tabular}{llll}
(63) & \textbf{hi}=mí & 'ba̠-i & ya'pu̠ \\
& \textsc{\textbf{neg}=3imp} & pararse-\textsc{l} & lejos \\
& \multicolumn{3}{l}{``No estaba parado lejos'' (pág. 59)}
\end{tabular} \vspace{0.3cm}

% Ejemplo 64
\begin{tabular}{llll}
(64) & pwes & \textbf{hin}=da & ñünɡ-a=n'a \\
& pues & \textsc{\textbf{neg}=3irr} & comer.\textsc{a-d}=uno\\
& \multicolumn{3}{l}{``Uno no come'' (pág. 107)}
\end{tabular} \vspace{0.5cm}

}

Esta lengua utiliza como marcador de polaridad negativa la palabra {\setmainfont{Charis SIL} \textit{hinɡi}} (59) - (61). Este marcador se ajusta morfofonológicamente cuando funciona como anfitrión de un clitico \textcolor{MidnightBlue}{\citep{Otomi}}. Esta forma adoptada es {\setmainfont{Charis SIL} \textit{hi}} más una consonante nasal que toma el punto de articulación de la consonante inmediata al clítico (62) - (64).