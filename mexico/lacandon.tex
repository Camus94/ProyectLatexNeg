\section*{Lacandon}
\addcontentsline{toc}{section}{Lacandon}

\noindent La lengua lacandona pertenece a la familia lingüística maya. Es una lengua en peligro de extinción que conserva rasgos arcaicos del proto-maya que se han perdido en otros idiomas mayances modernos.

 Tiene una estructura ergativa en la morfología del verbo. Hace distinciones entre posesión alienable e inalienable mediante afijos posesivos distintos para cada categoría. Asimismo, tiene un sistema prolífico para derivar neologismos a partir de raíces mayances.

Esta lengua es hablada principalmente en el área selvática de Chiapas y en partes de la selva del norte de Guatemala, específicamente en la región histórica conocida como Las Cañadas. Sus hablantes son los pueblos lacandones, grupos mayances que lograron aislarse durante siglos en la espesura de la selva y pudieron preservar su idioma y tradiciones.

Hoy en día sólo quedan unos pocos cientos de hablantes lacandones dispersos en pequeñas comunidades. La lengua ha sufrido desplazamiento por el español y está clasificada oficialmente como «en peligro de extinción». Sin embargo, existen esfuerzos tanto de académicos como de activistas lacandones para documentarla y revitalizar su uso. \vspace{0.5cm}

{\setmainfont{Charis SIL} 
% Ejemplo 37
\begin{tabular}{ll}
(37) & \textbf{maʔ}  \\
& \textsc{\textbf{neg}} \\
& ``No'' (pág. 82)
\end{tabular} \vspace{0.5cm}

% Ejemplo 38
\begin{tabular}{llll}
(38) & tin & t'an & \textbf{maʔ} \\
& \ yo & decir & \textsc{\textbf{neg}} \\
& \multicolumn{3}{l}{``Yo diría que no'' (pág. 82)}
\end{tabular} \vspace{0.5cm}

% Ejemplo 39
\begin{tabular}{lllll}
(39) & heʔ & + & \textbf{maʔ} & → hemaʔ \\
& eso & & \textsc{\textbf{neg}} & \\
& \multicolumn{4}{l}{``Eso no'' (Pág. 82)}
\end{tabular} \vspace{0.5cm}

% Ejemplo 40
\begin{tabular}{lllll}
(40) & kooč & + & \textbf{maʔ} & → maʔkoč \\
& ancho & & \textsc{\textbf{neg}} & \\
& \multicolumn{4}{l}{``Estrecho''}
\end{tabular} \vspace{0.5cm}
}

La negación funciona de manera análoga al español \textcolor{MidnightBlue}{\citep{lacandon}}, es decir, es un elemento libre con la forma {\setmainfont{Charis SIL} \textit{maʔ}} (37) (38). También puede utilizarse para formar compuestos con otros elementos (39) (40).