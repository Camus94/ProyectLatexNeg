\section*{Tlahuica de San Juan Atzingo}
\addcontentsline{toc}{section}{Tlahuica de San Juan Atzingo}

\noindent El tlahuica es una lengua otopame hablada en San Juan Atzingo y otras comunidades del Municipio de Ocuilán, en el Estado de México. Se caracteriza por tener diferentes clases verbales indicadas por medio de pautas de prefijación en los verbos. Estos prefijos contienen información sobre la persona, el tiempo, el aspecto y el modo, así como también sobre la transitividad. Las clases verbales del tlahuica tienen motivación morfológica y semántica.
%%%%%%%%%%%%% Abreviaturas %%%%%%%%%%%%%%%%%%%%%
\footnote{GI: grupo I, GII: grupo II, TR: transitivo}
%%%%%%%%%%%%%%%%%%%%%%%%%%%%%%%%%%%%%%%%%%%%%%%%


\vspace{0.5cm}

{\setmainfont{Charis SIL} 

% Ejemplo 92
\begin{tabular}{ll}
(92) & \textbf{tét}-kuntu-ts’ája \\
& \textsc{\textbf{neg}-2pl.pres.tr.gi}-enojar \\
& ``(Ustedes) no lo hacen enojar'' (pág. 197)
\end{tabular} \vspace{0.3cm}

% Ejemplo 93
\begin{tabular}{ll}
(93) & \textbf{té}-lu-tǔhnɡi \\
& \textsc{\textbf{neg}-1sg.pres.tr.gii}-esconder \\
& ``No lo escondo'' (pág. 197)
\end{tabular} \vspace{0.3cm}

% Ejemplo 94
\begin{tabular}{lll}
(94) & kǎkɨ & \textbf{te}-lu-nantli \\
& \textsc{pro.1sg} & \textsc{\textbf{neg}-1sg.pres.tr}-ser mamá\\
& \multicolumn{2}{l}{``Yo no soy mamá'' (pág. 199)}
\end{tabular} \vspace{0.3cm}

% Ejemplo 95
\begin{tabular}{llll}
(95) & kǎkɨ & \textbf{te}-lu-lǒ & pa-t\textsuperscript{h}ó \\
& \textsc{pro.1sg} & \textsc{\textbf{neg}-1sg.pres.tr}-estar & \textsc{1sg.pos}-casa \\
& \multicolumn{3}{l}{``Yo no estoy en mi casa'' (pág. 199)}
\end{tabular} \vspace{0.3cm}

% Ejemplo 96
\begin{tabular}{ll}
(96) & \textbf{nó}-kilu-nú \\
& \textsc{\textbf{neg}-1sg.fut.tr.gi}-despertar \\
& ``No lo voy a despertar'' (pág. 204)
\end{tabular} \vspace{0.3cm}

% Ejemplo 97
\begin{tabular}{ll}
(97) & \textbf{nó}-kilu-nu-hə́ \\
& \textsc{\textbf{neg}-1sg.fut.tr.gi}-despertar-\textsc{pl} \\
& ``No lo vamos a despertar'' (pág. 204)
\end{tabular} \vspace{0.5cm}

}

\textcolor{MidnightBlue}{\citet{Tlahuica}} describe dos prefijos verbales para la negación en esta lengua: {\setmainfont{Charis SIL} \textit{té-,tét-}}. Menciona claramente que se desconoce la motivación de la aparición de uno u otro (92) - (95). Hay un cambio de los prefijos de negación que pueden predecirse debido a un cambio en el \textsc{tam}. Estos prefijos no guardan relación alguna en forma por lo cual los llama «prefijos \textit{portmanteu}» \textcolor{MidnightBlue}{\citep[pág. 200]{Tlahuica}}. En el caso del tiempo futuro el prefijo {\setmainfont{Charis SIL} \textit{nó-}} es el utilizado para la negación (96) (95).