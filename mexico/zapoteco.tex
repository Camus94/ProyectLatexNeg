\section*{Zapoteco de Zoochina}

\noindent El zapoteco es una lengua de la familia oto-mangue. El zapoteco de Zoochina se habla en la comunidad de San Jerónimo Zoochina en el municipio de San Baltasar Yatzahi el Bajo dentro del distrito de Villa Alta en el Estado de Oaxaca. Es una lengua con complejidad laríngea en el nivel fonológico y posee una serie de fenómenos que han sido denominados «morfología asociada a la predicación». \vspace{0.5cm}

{\setmainfont{Charis SIL}

% Ejemplo 114
\begin{tabular}{lll}
(114) & ya & tø=køx-’oy-tam-a \\
& \textsc{neg} & \textsc{1suji}=comer-\textsc{ap-pl.pah-incomp.neg} \\
& \multicolumn{2}{l}{``No comemos'' (pág. 167)}
\end{tabular} \vspace{0.5cm}

% Ejemplo 115
\begin{tabular}{lll}
(115) & ya & ’øn=ker-tsøk-ø \\
& \textsc{neg} & \textsc{agt}=creer-hacer-incomp.neg \\
& \multicolumn{2}{l}{``No lo creo'' (pág. 167)}
\end{tabular} \vspace{0.5cm}

% Ejemplo 116
\begin{tabular}{llll}
(116) & tøx & ya & tø=min-wø \\
& \textsc{1pro} & \textsc{neg} & \textsc{1suji}=venir-compi.neg \\
& \multicolumn{3}{l}{``Yo no vine'' (pág. 169)}
\end{tabular} \vspace{0.5cm}

% Ejemplo 117
\begin{tabular}{lllll}
(117) & ya & ’øn=mux-wø & ti & ’øy=nøji \\
& \textsc{neg} & \textsc{1agt}=saber-\textsc{compi.neg} & qué & \textsc{3posd}=nombre \\
& \multicolumn{4}{l}{``No supe cómo se llama'' (pág. 170)}
\end{tabular} \vspace{0.5cm}





}

Esta lengua cuenta con tres marcas negativas: {\setmainfont{Charis SIL} \textit{ya, yampa, 'u}} \textcolor{MidnightBlue}{\citep{zapoteco}}. La forma {\setmainfont{Charis SIL} \textit{ya}} niega oraciones con aspecto incompletivo (114) - (115) y completivo (116) - (117).