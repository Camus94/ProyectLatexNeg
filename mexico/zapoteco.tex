\section*{Zapoteco de Zoochina}

\noindent Las lenguas zapotecas pertenecen a la familia oto-mangue. El zapoteco de Zoochina se habla en la comunidad de San Jerónimo Zoochina en el municipio de San Baltasar Yatzahi el Bajo dentro del distrito de Villa Alta en el Estado de Oaxaca. 

Es una lengua tonal con complejidad laríngea en el nivel fonológico y posee una serie de fenómenos que han sido denominados «morfología asociada a la predicación». \footnote{NG: negación general, ICP: incompletivo, NOM: nominativo, O: rol de objeto sintáctico, INA: inanimado, DEF: definido, NV: negación verbal, ADV:M: adverbio de modo, CPL: completivo, CLFPRO: clasificado pronominal, FOR: formal, COPEXST: cópula existencial, ANIM: animado, INTS: intensificador} \vspace{0.5cm}

{\setmainfont{Charis SIL} 

% Ejemplo 114
\begin{tabular}{llll}
(114) & \textbf{gàgé} & dx-góˀò=á=nh & sókr=nhàˀ \\
& \textsc{\textbf{ng}} & \textsc{icp-}meter=\textsc{1sg.nom=o3ina} & azúcar=\textsc{def} \\
& \multicolumn{3}{l}{``No le pongo azúcar [al café]'' (pág. 149)}
\end{tabular} \vspace{0.5cm}

% Ejemplo 115
\begin{tabular}{lll}
(115) & \textbf{àgé}=sh-yìxhghoh=òˀ & rrêntà=nhàˀ \\
& \textsc{\textbf{ng}=icp-}pagarlo\textsc{=2sg.nom} & renta=\textsc{def} \\
& \multicolumn{2}{l}{``No pagas la renta'' (pág. 409)}
\end{tabular} \vspace{0.3cm}

% Ejemplo 116
\begin{tabular}{lllll}
(116) & \textbf{bìtò}=lghâ & be-ônh & béné=nhàˀ & dàˀ=nhàˀ \\
& \textsc{\textbf{nv}=adv:m} & \textsc{cpl-}hacer & \textsc{clfpro:for=def} & \textsc{clfpro:ina=def} \\
& \multicolumn{4}{l}{``Tal vez (ella) no hizo eso [las tortillas]'' (pág. 149)}
\end{tabular} \vspace{0.3cm}

% Ejemplo 117
\begin{tabular}{llll}
(117) & \textbf{bìtò} & zó=éˀ & zítòˀ \\
& \textsc{\textbf{nv}} & \textsc{copexst:anim=3for.nom} & lejos \\
& \multicolumn{3}{l}{``(Él) no vive lejos'' pág. 442}
\end{tabular} \vspace{0.3cm}

% Ejemplo 118
\begin{tabular}{llll}
(118) & \textbf{bì}=dx & dx-éˀn+d=áˀ & [w-sháˀ+yíchgh=áˀ]oc \\
& \textsc{\textbf{nv}=ints} & \textsc{icp-}querer=\textsc{1sg.nom} & \textsc{irr-}preocuparse=\textsc{1sg.nom} \\
& \multicolumn{3}{l}{``Ya no quiero preocuparme'' (pág. 535)}
\end{tabular} \vspace{0.3cm}

}

Esta lengua tiene dos tipos de negación: la verbal y la general \textcolor{MidnightBlue}{\citep{zapoteco}}. La negación general es de uso bastante marginal ya que es aceptada por pocos hablantes y se utiliza para negar toda una propisición (114) - (115). La forma {\setmainfont{Charis SIL} \textit{gàgé}} atiende a las restricciones de \textit{items} con inicios silábicos obligatorios de al menos dos segmentos de estructura \textsc{cv}, pero tiene una relación de variación libre con la forma {\setmainfont{Charis SIL} \textit{àgé=}} la cual no cumple con la condición de inicio obligatorio \textcolor{MidnightBlue}{\citep[pág. 51]{zapoteco}}. La negación verbal (116) - (118) es la forma aceptada como la estrategia estandar de la negación en la lengua. Recurre a la forma {\setmainfont{Charis SIL} \textit{bìtò}} y a su versión corta {\setmainfont{Charis SIL} \textit{bì=}}.

                                                
        