\section*{Tsotsil}

\noindent El tsotsil es una lengua maya del grupo tseltalano hablada en el Estado de Chiapas, México. Tiene una morfología altamente prefijal y sufijal, y reduplicación parcial.
%%%%%%%%%%%%% Abreviaturas %%%%%%%%%%%%%%%%%%%%%
\footnote{NT: Aspecto neutral (\textit{neutral aspect})}
%%%%%%%%%%%%%%%%%%%%%%%%%%%%%%%%%%%%%%%%%%%%%%%%
\vspace{0.5cm}

{\setmainfont{Charis SIL}

% Ejemplo 111
\begin{tabular}{lllll}
(111) & \textbf{mu} & vinik-\textbf{uk} & li & Petul-e \\
& \textsc{\textbf{neg}} & hombre-\textsc{\textbf{neg}} & \textsc{det} & \textsc{cl} \\
& \multicolumn{4}{l}{``Petul no es un hombre (aún)'' (pág. 13)}
\end{tabular} \vspace{0.5cm}

% Ejemplo 112
\begin{tabular}{lll}
(112) & \textbf{mu} & p'ij-\textbf{uk} \\
& \textsc{\textbf{neg}} & pequeño-\textsc{\textbf{neg}} \\
& \multicolumn{2}{l}{``Él no es pequeño'' (pág. 13)}
\end{tabular} \vspace{0.5cm}

% Ejemplo 113
\begin{tabular}{lll}
(113) & \textbf{mu} & x-7abtej \\
& \textsc{\textbf{neg}} & \textsc{nt}-trabajar \\
& \multicolumn{2}{l}{``Él no va a trabajar'' (pág. 13)}
\end{tabular} \vspace{0.5cm}

}

\textcolor{MidnightBlue}{\citet{Tzotzil}} explica que la negación se hace por medio de la partícula {\setmainfont{Charis SIL} \textit{mu}} en combinación con el sufijo {\setmainfont{Charis SIL} \textit{-uk}}. Esta combinación se cumple en predicados nominales (111) y adjetivales (112). Para la negación clausal solo se manifiesta la partícula {\setmainfont{Charis SIL} \textit{nu}} (113).