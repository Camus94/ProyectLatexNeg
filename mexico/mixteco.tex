\section*{Mixteco de San Andrés Yutatío}

\noindent La lengua mixteca es parte de la familia otomangue. Se hablaba principalmente en Oaxaca, Puebla y Guerrero. La variante estudiada pertenece al municipio de Tezoatlán, San Andrés Yutatío, Oaxaca. \vspace{0.3cm}

{\setmainfont{Doulos SIL} 

% Ejemplo 46
\begin{tabular}{lllllllllll}
(46) & kátoó & nda̱'o & de̱'e di'íi & ̱ndiko & xi & tído & ko̱ & tí'a & ta'on & xi \\
& le:gusta & mucho & [hija:de-mí] & molerá & ella & pero & \textsc{neg} & sabe & \textsc{neg} & ella \\
& \multicolumn{10}{l}{``A mi hija le gusta mucho moler, pero no sabe'' (pág. 139)}
\end{tabular} \vspace{0.2cm}

% ejemplo 47
\begin{tabular}{lllllllllll}
(47) & ko̱ & kóni & ta'on & xi & kandía & xi & ña̱ & [ni ka'in & ̱xí'ín & xí] \\
& \textsc{neg} & quiere & \textsc{neg} & él & aceptará & él & lo:que & dije-yo & a & él \\
& \multicolumn{10}{l}{``No quiere aceptar lo que le dije'' (pág.169)}
\end{tabular} \vspace{0.2cm}

% Ejemplo 48
\begin{tabular}{lllll}
(48) & k\underline{o} & díkó & ná & yá'\underline{a} \\
& \textsc{neg} & venden & ellos & chile \\
& \multicolumn{4}{l}{``Ellos no venden chile'' (pág. 187)}
\end{tabular} \vspace{0.2cm}

% Ejemplo 49
\begin{tabular}{llllll}
(49) &  k\underline{o} & díkó & ta'on & ná & yá'\underline{a} \\
& \textsc{neg} & venden & \textsc{neg} & ellos & chile \\
& \multicolumn{5}{l}{``Ellos no venden chile'', (pág. 187)}
\end{tabular} \vspace{0.2cm}

% Ejemplo 50
{\small
\begin{tabular}{llllllllllll}
 (50) & \underline{o} d\underline{u}ú & ta & Juan & ní & sa'\underline{a}n & yúk\underline{u} & koni & ta & Beto & va & n\underline{i} sa'\underline{a}n \\
 & \textsc{neg} & él (joven) & Juan & \textsc{nrlzd} & fue & monte & ayer & él (joven) & Beto & \textsc{af} & fue \\
 & \multicolumn{11}{l}{``No fue Juan quien fue al monte ayer, sino Beto'' (pág. 176)}
\end{tabular} \vspace{0.2cm}
}

% Ejemplo 51
\begin{tabular}{lllllll}
(51) & \underline{o} d\underline{u}ú & ña & nda\underline{a} & kí\underline{a}n & ka'\underline{a}n & n\underline{a} \\
& \textsc{neg} & ella (cosa) & verdadera & es-ella (cosa) & habla & él \\
& \multicolumn{6}{l}{``No es verdad lo que dice él'' (pág. 176)}
\end{tabular} \vspace{0.2cm}

% Ejemplo 52
\begin{tabular}{llllll}
(52) & iin & ích\underline{i} & k\underline{o} & vá'a & kíán \\
& un & camino & \textsc{neg} & bueno & es-él \\
& \multicolumn{5}{l}{``Es un camino malo (lit. no bueno)'' (pág.181)}
\end{tabular} \vspace{0.3cm}
}

La negación en este lengua toma diferentes formas: puede ser por medio de adverbios negativos {\setmainfont{Doulos SIL} k\underline{o} y ta'on}, los cuales suelen aparecer de manera conjunta rodeando al verbo (46) (47). De acuerdo con \textcolor{MidnightBlue}{\citet{Mixteco}}, el adverbio principal y obligatorio para la negación es {\setmainfont{Doulos SIL} k\underline{o}}, mientras que {\setmainfont{Doulos SIL} ta'on} parece dar énfasis a la negación (48) (49). Contrucción de frases sustantivas negativas por medio de la expresión {\setmainfont{Doulos SIL} \underline{o} d\underline{u}ú} que se presenta siempre al inicio de la oración (50) (51). Una frase adjetiva negativa por medio del adverbio {\setmainfont{Doulos SIL} k\underline{o}} precediendo al adjetivo que modifica (52).