\section*{Mixteco de San Andrés Yutatío}
\addcontentsline{toc}{section}{Mixteco de San Andrés Yutatío}

\noindent La lengua mixteca es parte de la familia otomangue. Se hablaba principalmente en Oaxaca, Puebla y Guerrero. La variante estudiada pertenece al municipio de Tezoatlán, San Andrés Yutatío, Oaxaca.
%%%%%%%%%%%%%%%%%%%% Abreviaturas %%%%%%%%%%%%%%%%%%%%
\footnote{NO:REAL: no realizado, AF: afirmación}
%%%%%%%%%%%%%%%%%%%
\vspace{0.2cm}

{\setmainfont{Charis SIL} 

{\small
% Ejemplo 46
\noindent \begin{tabular}{lllllllllll}
(46) & kátoó & nda̱'o & de̱'e di'íi & ̱ndiko & xi & tído & \textbf{ko̱} & tí'a & \textbf{ta'on} & xi \\
& le:gusta & mucho & [hija:de-mí] & molerá & ella & pero & \textsc{\textbf{neg}} & sabe & \textsc{\textbf{neg}} & ella \\
& \multicolumn{10}{l}{``A mi hija le gusta mucho moler, pero no sabe'' (pág. 139)}
\end{tabular} \vspace{0.2cm}}

% ejemplo 47
\noindent \begin{tabular}{lllllllllll}
(47) & \textbf{ko̱} & kóni & \textbf{ta'on} & xi & kandía & xi & ña̱ & [ni ka'in & ̱xí'ín & xí] \\
& \textsc{\textbf{neg}} & quiere & \textsc{\textbf{neg}} & él & aceptará & él & lo:que & dije-yo & a & él \\
& \multicolumn{10}{l}{``No quiere aceptar lo que le dije'' (pág.169)}
\end{tabular} \vspace{0.2cm}

% Ejemplo 48
\noindent \begin{tabular}{lllll}
(48) & \textbf{k\underline{o}} & díkó & ná & yá'\underline{a} \\
& \textsc{\textbf{neg}} & venden & ellos & chile \\
& \multicolumn{4}{l}{``Ellos no venden chile'' (pág. 187)}
\end{tabular} \vspace{0.2cm}

% Ejemplo 49
\noindent \begin{tabular}{llllll}
(49) &  \textbf{k\underline{o}} & díkó & \textbf{ta'on} & ná & yá'\underline{a} \\
& \textsc{\textbf{neg}} & venden & \textsc{\textbf{neg}} & ellos & chile \\
& \multicolumn{5}{l}{``Ellos no venden chile'', (pág. 187)}
\end{tabular} \vspace{0.2cm}

% Ejemplo 50
{\small
\noindent \begin{tabular}{llllllllllll}
 (50) & \textbf{\underline{o} d\underline{u}ú} & ta & Juan & ní & sa'\underline{a}n & yúk\underline{u} & koni & ta & Beto & va & n\underline{i} sa'\underline{a}n \\
 & \textsc{\textbf{neg}} & él & Juan & \textsc{no:real} & fue & monte & ayer & él & Beto & \textsc{af} & fue \\
 & \multicolumn{11}{l}{``No fue Juan quien fue al monte ayer, sino Beto'' (pág. 176)}
\end{tabular} \vspace{0.2cm}
}

% Ejemplo 51
\noindent \begin{tabular}{lllllll}
(51) & \textbf{\underline{o} d\underline{u}ú} & ña & nda\underline{a} & kí\underline{a}n & ka'\underline{a}n & n\underline{a} \\
& \textsc{\textbf{neg}} & ella (cosa) & verdadera & es-ella (cosa) & habla & él \\
& \multicolumn{6}{l}{``No es verdad lo que dice él'' (pág. 176)}
\end{tabular} \vspace{0.2cm}

% Ejemplo 52
\noindent \begin{tabular}{llllll}
(52) & iin & ích\underline{i} & \textbf{k\underline{o}} & vá'a & kíán \\
& un & camino & \textsc{\textbf{neg}} & bueno & es-él \\
& \multicolumn{5}{l}{``Es un camino malo (lit. no bueno)'' (pág.181)}
\end{tabular} \vspace{0.25cm}

}

La negación en este lengua toma diferentes formas: puede ser por medio de adverbios negativos {\setmainfont{Charis SIL} \textit{k\underline{o}} y \textit{ta'on}}, los cuales suelen aparecer de manera conjunta rodeando al verbo (46) (47). De acuerdo con \textcolor{MidnightBlue}{\citet{Mixteco}}, el adverbio principal y obligatorio para la negación es {\setmainfont{Charis SIL} \textit{k\underline{o}}}, mientras que {\setmainfont{Charis SIL} \textit{ta'on}} parece dar énfasis a la negación (48) (49). Contrucción de frases sustantivas negativas por medio de la expresión {\setmainfont{Charis SIL} \underline{o} \textit{d\underline{u}ú}} que se presenta siempre al inicio de la oración (50) (51). Una frase adjetiva negativa por medio del adverbio {\setmainfont{Charis SIL} \textit{k\underline{o}}} precediendo al adjetivo que modifica (52).