\section*{Tepehuano del sur}
\addcontentsline{toc}{section}{Tepehuano del sur}

\noindent El Tepehuano del Sureste es una lengua tepimán de la familia yuto-azteca que se habla en la Sierra Madre de Durango, México.
%%%%%%%%%%%%% Abreviaturas %%%%%%%%%%%%%%%%%%%%%
\footnote{ADVR: marcador adverbial, DC: cláusula dependiente, DIR: direccional, EST: estativo, PNCT: punctual, POSP: posposición, RED: reduplicación, SBJ: sujeto, SUB: subordinador}
%%%%%%%%%%%%%%%%%%%%%%%%%%%%%%%%%%%%%%%%%%%%%%%%


\vspace{0.5cm}

{\setmainfont{Charis SIL} 

% Ejemplo 87
\noindent \begin{tabular}{llllll}
(87) & \textbf{cham tu’} & tu’ & ja’tkam & ja’pi & xi’-xbulhi-k  \\
& \textsc{\textbf{neg}} & algo & persona & pero & \textsc{red:pl}-remolino-\textsc{pnct}\\
& \multicolumn{5}{l}{``Esos no eran humanos, eran remolinos'' (pág. 110)}
\end{tabular} \vspace{0.5cm}

% Ejemplo 88
\noindent \begin{tabular}{llllll}
(88) & \textbf{cham tu’} & kɨʼn & ya’ & ja’p & jim-iñ-ji \\
& \textsc{\textbf{neg}} & \textsc{posp}:con & \textsc{dir} & \textsc{dir} & ir-\textsc{1sg.sbj-dc} \\
& \multicolumn{5}{l}{``No vine para molestarte con eso'' (pág. 111)}
\end{tabular} \vspace{0.5cm}

% Ejemplo 89
\noindent \begin{tabular}{llllll}
(89) & na=ch & \textbf{cham} & agren & mui’ & tu-ma-mar-ka’ \\
& \textsc{sub=1pl.sbj} & \textsc{\textbf{neg}} & a propósito & un montón de & \textsc{dur-red:pl}-hijo-\textsc{est} \\
& \multicolumn{5}{l}{``Para no tener una gran cantidad de hijos'' (pág. 112)}
\end{tabular} \vspace{0.5cm}

% Ejemplo 90
\noindent \begin{tabular}{lllll}
(90) & dai & na=m & \textbf{cham} & pensar-ka’ \\
& solo & \textsc{sub=3pl.sbj} & \textsc{\textbf{neg}} & pesar-\textsc{est} \\
& \multicolumn{4}{l}{``Solo que no estaban pensando'' (pág. 112)}
\end{tabular} \vspace{0.5cm}

% Ejemplo 91
{\small
\noindent \begin{tabular}{llllllll}
(91) & \textbf{Cham} & mat-am & \textbf{cham} & mat-am & ma’n & na=m-jax & kai’ch \\
& \textsc{\textbf{neg}} & saber-\textsc{3pl.sbj} & \textsc{\textbf{neg}} & saber-\textsc{3pl.sbj} & único & \textsc{sub=3pl.sbj-advr} & decir \\
& \multicolumn{7}{l}{``Ellos no saben, no saben, dicen lo mismo'' (pág. 112)}
\end{tabular} \vspace{0.5cm} }

}

La marcación de la negación en esta lengua es por medio de la partícula {\setmainfont{Charis SIL} \textit{cham tu’}} (87) (88) o por medio de su forma simple {\setmainfont{Charis SIL} \textit{cham}} (89) - (91) \textcolor{MidnightBlue}{\citep{Tepehuano}}. La posición de la partícula muestra si se trata de negación clausal o frasal, ya que esta se antepone al elemento que negará.