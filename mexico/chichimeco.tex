\section*{Chichimeco Jonaz}

\noindent Chichimeco y chichimeca son términos utilizados para referirse tanto a un conjunto de lenguas o variedades lingüísticas, así como para denominar a un grupo de pueblos indígenas nómadas que habitaban en el centro y norte de México. El Chichimeco Jonaz pertenece a la subfamilia otopame de la familia otomangue. Esta variante se localiza en el municipio de San Luis de la Paz, en el noreste del estado de Guanajuato. \vspace{0.25cm}

{\setmainfont{Doulos SIL}
    % Ejemplo 11
    \begin{tabular}{llll}
        (11) & PRES       & ṹβ̃ã e-pã́s-βʷ           & “Le pone los zapatos”          \\
             & PAS.REM    & ṹβ̃ã u-pã́s-βʷ           & “Le puso los zapatos”          \\
             & PAS.REC    & ṹβ̃ã ku-pã́s-βʷ          & “Le puso los zapatos”          \\
             & PAS INM    & ṹβ̃ã su-pã́s-βʷ          & “Le puso los zapatos”          \\
             & FUT        & ṹβ̃ã a-pã́s-βʷ           & “Le va a poner los zapatos”    \\
             &            &                        &                                \\
             & PRES       & ṹβ̃ã sa-pã́s-umé         & “No le pone los zapatos”       \\
             & PAS.REM    & ṹβ̃ã sa-pã́s-umé         & “No le puso los zapatos”       \\
             & PAS.REC    & ṹβ̃ã su-pã́s-umé         & “No le puso los zapatos”       \\
             & PAS.INM    & ṹβ̃ã sa-pã́s-umé         & “No le puso los zapatos”       \\
             & FUT        & siʔá̤n3 ṹβ̃ã sa- pã́s-umé & “No le va a poner los zapatos” \\
             &            &                        &                                \\
             & PRES       & ṹβ̃ã ra-pã́s-βʷ          & “Le pondría los zapatos”       \\
             & PAS.REM    & ṹβ̃ã ma-pã́s-βʷ          & “Le habría puesto los zapatos” \\
             & PAS.REC    & ṹβ̃ã ma-pã́s-βʷ          & “Le habría puesto los zapatos” \\
             & PAS.INM    & ṹβ̃ã ma-pã́s-βʷ          & “Le habría puesto los zapatos” \\
             & FUT        & ṹβ̃ã a-βã́s-βʷ           & “Le pondría los zapatos”       \\
             & (pág. 145) &                        &                                \\
    \end{tabular}
} \vspace{0.25cm}

\textcolor{MidnightBlue}{\citet{chichimeco}} nos dice que tipológicamente es una lengua aglutinante con características polisintéticas. Presenta una gran riqueza morfológica de naturaleza concatenativa y no-concatenativa. En la morfología verbal hay una compleja expresión de persona y número, a través de distintos paradigmas de morfemas que incluyen prefijos, sufijos, cambios internos (mutaciones consonánticas, cambios vocálicos), alternancias tonales y verbales.

El morfema \textit{-umé} tiene la función de marcar la negación en el verbo. Aparece junto con el prefijo \textit{sa-} en las formas negativas del modo afirmativo y su presencia parece causar la elisión del sufijo de objeto {\setmainfont{Charis SIL}\textit{-βw}}. Esta distribución \textit{—sa-} + \textit{-umé—} parece estar restringida a las formas negativas del modo afirmativo en presente y pasado (remoto, reciente, inmediato) ya que en las formas negativas de futuro se requiere el uso la partícula {\setmainfont{Charis SIL}\textit{siʔá̤n}.}
